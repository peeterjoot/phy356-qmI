%
% Copyright � 2012 Peeter Joot.  All Rights Reserved.
% Licenced as described in the file LICENSE under the root directory of this GIT repository.
%

%\chapter{Notes and problems for Desai Chapter V}
\label{chap:desaiCh5}
%\blogpage{http://sites.google.com/site/peeterjoot/math2010/desaiCh5.pdf}
%\date{Oct 18, 2010}
%
%\section{Motivation}
%
%Chapter V notes for \citep{desai2009quantum}.
%


%
\makeoproblem{}{problem:desaiCh5:1}{\citep{desai2009quantum} pr 5.1}{
Obtain \(S_x, S_y, S_z\) for \textAndIndex{spin 1} in the representation in which \(S_z\) and \(S^2\) are diagonal.

} % problem
%
\makeanswer{problem:desaiCh5:1}{
%
For spin 1, we have
%
\begin{equation}\label{eqn:desaiCh5:100}
\begin{aligned}
S^2 = 1 (1+1) \Hbar^2 \BOne
\end{aligned}
\end{equation}
%
and are interested in the states \(\ket{1,-1}, \ket{1, 0}, and \ket{1,1}\).  If, like angular momentum, we assume that we have for \(m_s = -1,0,1\)
%
\begin{equation}\label{eqn:desaiCh5:101}
\begin{aligned}
S_z \ket{1,m_s} = m_s \Hbar \ket{1, m_s}
\end{aligned}
\end{equation}
%
and introduce a column matrix representations for the kets as follows
%
\begin{equation}\label{eqn:desaiCh5:102}
\begin{aligned}
\ket{1,1} &=
\begin{bmatrix}
1 \\
0 \\
0
\end{bmatrix} \\
\ket{1,0} &=
\begin{bmatrix}
0 \\
1 \\
0
\end{bmatrix} \\
\ket{1,-1} &=
\begin{bmatrix}
0 \\
0 \\
-1
\end{bmatrix},
\end{aligned}
\end{equation}
%
then we have, by inspection
\begin{equation}\label{eqn:desaiCh5:103}
\begin{aligned}
S_z &= \Hbar
\begin{bmatrix}
1 & 0 & 0 \\
0 & 0 & 0 \\
0 & 0 & -1
\end{bmatrix}.
\end{aligned}
\end{equation}
%
Note that, like the Pauli matrices, and unlike angular momentum, the spin states \(\ket{-1, m_s}, \ket{0, m_s}\) have not been considered.  Do those have any physical interpretation?

That question aside, we can proceed as in the text, utilizing the ladder operator commutators
%
\begin{equation}\label{eqn:desaiCh5:104}
\begin{aligned}
S_{\pm} &= S_x \pm i S_y,
\end{aligned}
\end{equation}
%
to determine the values of \(S_x\) and \(S_y\) indirectly.  We find
%
\begin{equation}\label{eqn:desaiCh5:105}
\begin{aligned}
\antisymmetric{S_{+}}{S_{-}} &= 2 \Hbar S_z \\
\antisymmetric{S_{+}}{S_{z}} &= -\Hbar S_{+} \\
\antisymmetric{S_{-}}{S_{z}} &= \Hbar S_{-}.
\end{aligned}
\end{equation}
%
Let
\begin{equation}\label{eqn:desaiCh5:106}
\begin{aligned}
S_{+} &=
\begin{bmatrix}
a & b & c \\
d & e & f \\
g & h & i
\end{bmatrix}.
\end{aligned}
\end{equation}
%
Looking for equality between \(\antisymmetric{S_{z}}{S_{+}}/\Hbar = S_{+}\), we find
\begin{equation}\label{eqn:desaiCh5:107}
\begin{aligned}
\begin{bmatrix}
0 & b & 2 c \\
-d & 0 & f \\
-2g & -h & 0
\end{bmatrix}
&=
\begin{bmatrix}
a & b & c \\
d & e & f \\
g & h & i
\end{bmatrix},
\end{aligned}
\end{equation}
%
so we must have
\begin{equation}\label{eqn:desaiCh5:108}
\begin{aligned}
S_{+} &=
\begin{bmatrix}
0 & b & 0 \\
0 & 0 & f \\
0 & 0 & 0
\end{bmatrix}.
\end{aligned}
\end{equation}
%
Furthermore, from \(\antisymmetric{S_{+}}{S_{-}} = 2 \Hbar S_z\), we find
\begin{equation}\label{eqn:desaiCh5:109}
\begin{aligned}
\begin{bmatrix}
\Abs{b}^2 & 0 & 0 \\
0 & \Abs{f}^2 - \Abs{b}^2 & 0 \\
0 & 0 & -\Abs{f}^2
\end{bmatrix}
&=
2 \Hbar^2
\begin{bmatrix}
1 & 0 & 0 \\
0 & 0 & 0 \\
0 & 0 & -1
\end{bmatrix}.
\end{aligned}
\end{equation}
%
We must have \(\Abs{b}^2 = \Abs{f}^2 = 2 \Hbar^2\).  We could probably pick any
\(b = \sqrt{2} \Hbar e^{i\phi}\), and \(f = \sqrt{2} \Hbar e^{i\theta}\), but assuming we have no reason for a non-zero phase we try
%
\begin{equation}\label{eqn:desaiCh5:110}
\begin{aligned}
S_{+}
&=
\sqrt{2} \Hbar
\begin{bmatrix}
0 & 1 & 0 \\
0 & 0 & 1 \\
0 & 0 & 0
\end{bmatrix}.
\end{aligned}
\end{equation}
%
Putting all the pieces back together, with \(S_x = (S_{+} + S_{-})/2\), and \(S_y = (S_{+} - S_{-})/2i\), we finally have
\begin{equation}\label{eqn:desaiCh5:111}
\begin{aligned}
S_x &=
\frac{\Hbar}{\sqrt{2}}
\begin{bmatrix}
0 & 1 & 0 \\
1 & 0 & 1 \\
0 & 1 & 0
\end{bmatrix} \\
S_y &=
\frac{\Hbar}{\sqrt{2} i}
\begin{bmatrix}
0 & 1 & 0 \\
-1 & 0 & 1 \\
0 & -1 & 0
\end{bmatrix} \\
S_z &=
\Hbar
\begin{bmatrix}
1 & 0 & 0 \\
0 & 0 & 0 \\
0 & 0 & -1
\end{bmatrix}.
\end{aligned}
\end{equation}
%
A quick calculation verifies that we have \(S_x^2 + S_y^2 + S_z^2 = 2 \Hbar \BOne\), as expected.
} % answer

%
\makeoproblem{}{problem:desaiCh5:2}{\citep{desai2009quantum} pr 5.2}{
%
Obtain eigensolution for operator \(A = a \sigma_y + b \sigma_z\).  Call the eigenstates \(\ket{1}\) and \(\ket{2}\), and determine the probabilities that they will correspond to \(\sigma_x = +1\).

} % problem
%
\makeanswer{problem:desaiCh5:2}{
%
The first part is straight forward, and we have
\begin{equation}\label{eqn:desaiCh5:822}
\begin{aligned}
A &= a \PauliY + b \PauliZ \\
&=
\begin{bmatrix}
b & -i a \\
ia & -b
\end{bmatrix}.
\end{aligned}
\end{equation}
%
Taking \(\Abs{A - \lambda I} = 0\) we get
%
\begin{equation}\label{eqn:desaiCh5:202}
\begin{aligned}
\lambda &= \pm \sqrt{a^2 + b^2},
\end{aligned}
\end{equation}
%
with eigenvectors proportional to
\begin{equation}\label{eqn:desaiCh5:203}
\begin{aligned}
\ket{\pm} &=
\begin{bmatrix}
i a \\
b \mp \sqrt{a^2 + b^2}
\end{bmatrix}
\end{aligned}
\end{equation}
%
The normalization constant is \(1/\sqrt{2 (a^2 + b^2) \mp 2 b \sqrt{a^2 + b^2}}\).  Now we can call these \(\ket{1}\), and \(\ket{2}\) but what does the last part of the question mean?  What is meant by \(\sigma_x = +1\)?

Asking the prof about this, he says:

``I think it means that the result of a measurement of the x component of spin is \(+1\). This corresponds to the eigenvalue of \(\sigma_x\) being \(+1\). The spin operator \(S_x\) has eigenvalue \(+\Hbar/2\)''.

Aside: Question to consider later.  Is it significant that \(\bra{1} \sigma_x \ket{1} = \bra{2} \sigma_x \ket{2} = 0\)?

So, how do we translate this into a mathematical statement?

First let us recall a couple of details.  Recall that the x spin operator has the matrix representation
\begin{equation}\label{eqn:desaiCh5:204}
\begin{aligned}
\sigma_x = \PauliX.
\end{aligned}
\end{equation}
%
This has eigenvalues \(\pm 1\), with eigenstates \((1,\pm 1)/\sqrt{2}\).  At the point when the x component spin is observed to be \(+1\), the state of the system was then
\begin{equation}\label{eqn:desaiCh5:205}
\begin{aligned}
\ket{x+} =
\inv{\sqrt{2}}
\begin{bmatrix}
1 \\
1
\end{bmatrix}
\end{aligned}
\end{equation}
%
Let us look at the ways that this state can be formed as linear combinations of our states \(\ket{1}\), and \(\ket{2}\).  That is
\begin{equation}\label{eqn:desaiCh5:206}
\begin{aligned}
\inv{\sqrt{2}}
\begin{bmatrix}
1 \\
1
\end{bmatrix}
&=
\alpha \ket{1}
+ \beta \ket{2},
\end{aligned}
\end{equation}
or
\begin{equation}\label{eqn:desaiCh5:207}
\begin{aligned}
\begin{bmatrix}
1 \\
1
\end{bmatrix}
&=
\frac{\alpha}{\sqrt{(a^2 + b^2) - b \sqrt{a^2 + b^2}}}
\begin{bmatrix}
i a \\
b - \sqrt{a^2 + b^2}
\end{bmatrix} \\
&\quad+\frac{\beta}{\sqrt{(a^2 + b^2) + b \sqrt{a^2 + b^2}}}
\begin{bmatrix}
i a \\
b + \sqrt{a^2 + b^2}
\end{bmatrix}
\end{aligned}
\end{equation}
%
Letting \(c = \sqrt{a^2 + b^2}\), this is
\begin{equation}\label{eqn:desaiCh5:208}
\begin{aligned}
\begin{bmatrix}
1 \\
1
\end{bmatrix}
&=
\frac{\alpha}{\sqrt{c^2 - b c}}
\begin{bmatrix}
i a \\
b - c
\end{bmatrix}
+\frac{\beta}{\sqrt{c^2 + b c}}
\begin{bmatrix}
i a \\
b + c
\end{bmatrix}.
\end{aligned}
\end{equation}
%
We can solve the \(\alpha\) and \(\beta\) with Cramer's rule, yielding
\begin{equation}\label{eqn:desaiCh5:842}
\begin{aligned}
\begin{vmatrix}
1 & i a \\
1 & b - c
\end{vmatrix}
&=
\frac{\beta}{\sqrt{c^2 + b c}}
\begin{vmatrix}
i a  & i a \\
b + c & b - c
\end{vmatrix} \\
\begin{vmatrix}
1 & i a \\
1 & b + c
\end{vmatrix}
&=
\frac{\alpha}{\sqrt{c^2 - b c}}
\begin{vmatrix}
i a  & i a \\
b - c & b + c
\end{vmatrix},
\end{aligned}
\end{equation}
%
or
\begin{equation}\label{eqn:desaiCh5:209}
\begin{aligned}
\alpha &= \frac{(b + c - ia)\sqrt{c^2 - b c}}{2 i a c} \\ %= \frac{(a -i(b + c))\sqrt{1 - b/c}}{2 a} \\
\beta &= \frac{(b - c - ia)\sqrt{c^2 + b c}}{-2 i a c} %= \frac{(a + i(b - c))\sqrt{1 + b/c}}{2 a}.
\end{aligned}
\end{equation}
%
It is \(\Abs{\alpha}^2\) and \(\Abs{\beta}^2\) that are probabilities, and after a bit of algebra we find that those are
%
\begin{equation}\label{eqn:desaiCh5:210}
\begin{aligned}
\Abs{\alpha}^2 = \Abs{\beta}^2 = \inv{2},
\end{aligned}
\end{equation}
%
so if the x spin of the system is measured as \(+1\), we have a \(50\%\) chance that the measured eigenvalue for the operator \(A\) would be \(\sqrt{a^2 + b^2}\) (ie: with state \(\ket{1}\).

Is that what the question was asking?  I think that I have actually got it backwards.  I think that the question was asking for the probability of finding state \(\ket{x+}\) (measuring a spin 1 value for \(\sigma_x\)) given the state \(\ket{1}\) or \(\ket{2}\).

So, suppose that we have
%
\begin{equation}\label{eqn:desaiCh5:211}
\begin{aligned}
\mu_{+} \ket{x+} + \nu_{+} \ket{x-} &= \ket{1} \\
\mu_{-} \ket{x+} + \nu_{-} \ket{x-} &= \ket{2},
\end{aligned}
\end{equation}
%
or (considering both cases simultaneously),
\begin{equation}\label{eqn:desaiCh5:862}
\begin{aligned}
\mu_{\pm}
\begin{bmatrix}
1 \\
1
\end{bmatrix}
+ \nu_{\pm}
\begin{bmatrix}
1 \\
-1
\end{bmatrix}
&=
\inv{\sqrt{ c^2 \mp b c }}
\begin{bmatrix}
i a \\
b \mp c
\end{bmatrix} \\
\implies \\
\mu_{\pm}
\begin{vmatrix}
1 & 1 \\
1 & -1
\end{vmatrix}
&=
\inv{\sqrt{ c^2 \mp b c }}
\begin{vmatrix}
i a & 1 \\
b \mp c & -1
\end{vmatrix},
\end{aligned}
\end{equation}
%
or
\begin{equation}\label{eqn:desaiCh5:212}
\begin{aligned}
\mu_{\pm} &=
\frac{ia + b \mp c}{2 \sqrt{c^2 \mp bc}} .
\end{aligned}
\end{equation}
%
Unsurprisingly, this mirrors the previous scenario and we find that we have a probability \(\Abs{\mu}^2 = 1/2\) of measuring a spin 1 value for \(\sigma_x\) when the state of the operator \(A\) has been measured as \(\pm \sqrt{a^2 + b^2}\) (ie: in the states \(\ket{1}\), or \(\ket{2}\) respectively).

No measurement of the operator \(A = a \sigma_y + b\sigma_z\) gives a biased prediction of the state of the state \(\sigma_x\).  Loosely, this seems to justify calling these operators orthogonal.  This is consistent with the geometrical antisymmetric nature of the spin components where we have \(\sigma_y \sigma_x = -\sigma_x \sigma_y\), just like two orthogonal vectors under the Clifford product.

} % answer

%
\makeoproblem{}{problem:desaiCh5:3}{\citep{desai2009quantum} pr 5.3}{
%
Obtain the expectation values of \(S_x, S_y, S_z\) for the case of a spin \(1/2\) particle with the spin pointed in the direction of a vector with azimuthal angle \(\beta\) and polar angle \(\alpha\).

} % problem
%
\makeanswer{problem:desaiCh5:3}{
%
Let us work with \(\sigma_k\) instead of \(S_k\) to eliminate the \(\Hbar/2\) factors.  Before considering the expectation values in the arbitrary spin orientation, let us consider just the expectation values for \(\sigma_k\).  Introducing a matrix representation (assumed normalized) for a reference state
%
\begin{equation}\label{eqn:desaiCh5:300}
\begin{aligned}
\ket{\psi} &=
\begin{bmatrix}
a \\
b
\end{bmatrix},
\end{aligned}
\end{equation}
%
we find
%
\begin{equation}\label{eqn:desaiCh5:301}
\begin{aligned}
\bra{\psi} \sigma_x \ket{\psi}
&=
\begin{bmatrix}
a^\conj & b^\conj
\end{bmatrix}
\PauliX
\begin{bmatrix}
a \\
b
\end{bmatrix}
= a^\conj b + b^\conj a
\\
\bra{\psi} \sigma_y \ket{\psi}
&=
\begin{bmatrix}
a^\conj & b^\conj
\end{bmatrix}
\PauliY
\begin{bmatrix}
a \\
b
\end{bmatrix}
= - i a^\conj b + i b^\conj a
\\
\bra{\psi} \sigma_x \ket{\psi}
&=
\begin{bmatrix}
a^\conj & b^\conj
\end{bmatrix}
\PauliZ
\begin{bmatrix}
a \\
b
\end{bmatrix}
= a^\conj a - b^\conj b
\end{aligned}
\end{equation}
%
Each of these expectation values are real as expected due to the Hermitian nature of \(\sigma_k\).  We also find that
%
\begin{equation}\label{eqn:desaiCh5:303}
\begin{aligned}
\sum_{k=1}^3 {\bra{\psi} \sigma_k \ket{\psi}}^2 &= (\Abs{a}^2 + \Abs{b}^2)^2 = 1
\end{aligned}
\end{equation}
%
So a vector formed with the expectation values as components is a unit vector.  This does not seem too unexpected from the section on the projection operators in the text where it was stated that \(\bra{\chi} \Bsigma \ket{\chi} = \Bp\), where \(\Bp\) was a unit vector, and this seems similar.  Let us now consider the arbitrarily oriented spin vector \(\Bsigma \cdot \Bn\), and look at its expectation value.
%
With \(\Bn\) as the rotated image of \(\zcap\) by an azimuthal angle \(\beta\), and polar angle \(\alpha\), we have
\begin{equation}\label{eqn:desaiCh5:304}
\Bn = (\sin\alpha \cos\beta,\sin\alpha \sin\beta,\cos\alpha).
\end{equation}
That is
\begin{equation}\label{eqn:desaiCh5:305}
\begin{aligned}
\Bsigma \cdot \Bn &= \sin\alpha \cos\beta \sigma_x + \sin\alpha \sin\beta \sigma_y + \cos\alpha \sigma_z.
\end{aligned}
\end{equation}
%
The \(k = x,y,y\) projections of this operator
\begin{equation}\label{eqn:desaiCh5:306}
\begin{aligned}
\inv{2} \tr { \sigma_k (\Bsigma \cdot \Bn)} \sigma_k,
\end{aligned}
\end{equation}
are just the Pauli matrices scaled by the components of \(\Bn\)
\begin{equation}\label{eqn:desaiCh5:307}
\begin{aligned}
\inv{2} \tr { \sigma_x (\Bsigma \cdot \Bn)} \sigma_x &= \sin\alpha \cos\beta \sigma_x  \\
\inv{2} \tr { \sigma_y (\Bsigma \cdot \Bn)} \sigma_y &= \sin\alpha \sin\beta \sigma_y  \\
\inv{2} \tr { \sigma_z (\Bsigma \cdot \Bn)} \sigma_z &= \cos\alpha \sigma_z,
\end{aligned}
\end{equation}
%
so our \(S_k\) expectation values are by inspection
%
\begin{equation}\label{eqn:desaiCh5:308}
\begin{aligned}
\bra{\psi} S_x \ket{\psi} &= \frac{\Hbar}{2} \sin\alpha \cos\beta ( a^\conj b + b^\conj a ) \\
\bra{\psi} S_y \ket{\psi} &= \frac{\Hbar}{2} \sin\alpha \sin\beta ( - i a^\conj b + i b^\conj a ) \\
\bra{\psi} S_z \ket{\psi} &= \frac{\Hbar}{2} \cos\alpha ( a^\conj a - b^\conj b )
\end{aligned}
\end{equation}
%
Is this correct?  While \((\Bsigma \cdot \Bn)^2 = \Bn^2 = I\) is a unit norm operator, we find that the expectation values of the coordinates of \(\Bsigma \cdot \Bn\) cannot be viewed as the coordinates of a unit vector.  Let us consider a specific case, with \(\Bn = (0,0,1)\), where the spin is oriented in the \(x,y\) plane.  That gives us
%
\begin{equation}\label{eqn:desaiCh5:309}
\begin{aligned}
\Bsigma \cdot \Bn = \sigma_z
\end{aligned}
\end{equation}
%
so the expectation values of \(S_k\) are
\begin{equation}\label{eqn:desaiCh5:310}
\begin{aligned}
\expectation{S_x} &= 0 \\
\expectation{S_y} &= 0 \\
\expectation{S_z} &= \frac{\Hbar}{2} ( a^\conj a - b^\conj b )
\end{aligned}
\end{equation}
%
Given this is seems reasonable that from \eqnref{eqn:desaiCh5:308} we find
%
\begin{equation}\label{eqn:desaiCh5:311}
\begin{aligned}
\sum_k {\bra{\psi} S_k \ket{\psi}}^2 \ne \Hbar^2/4,
\end{aligned}
\end{equation}
%
(since we do not have any reason to believe that in general \(( a^\conj a - b^\conj b )^2 = 1\) is true).

The most general statement we can make about these expectation values (an average observed value for the measurement of the operator) is that
%
\begin{equation}\label{eqn:desaiCh5:312}
\begin{aligned}
\Abs{\expectation{S_k}} \le \frac{\Hbar}{2}
\end{aligned}
\end{equation}
%
with equality for specific states and orientations only.

} % answer

%
\makeoproblem{}{problem:desaiCh5:4}{\citep{desai2009quantum} pr 5.4}{
FIXME: describe.
} % problem
%
\makeanswer{problem:desaiCh5:4}{
%
Take the azimuthal angle, \(\beta = 0\), so that the spin is in the
x-z plane at an angle \(\alpha\) with respect to the z-axis, and the unit vector is \(\Bn = (\sin\alpha, 0, \cos\alpha)\).  Write
%
\begin{equation}\label{eqn:boostCommutator:500}
\begin{aligned}
\ket{\chi_{n+}} = \ket{+\alpha}
\end{aligned}
\end{equation}
%
for this case.  Show that the probability that it is in the spin-up state in the direction \(\theta\) with respect to the z-axis is
%
\begin{equation}\label{eqn:boostCommutator:501}
\begin{aligned}
\Abs{ \braket{+\theta}{+\alpha} }^2 = \cos^2 \left( \frac{\alpha - \theta}{2} \right)
\end{aligned}
\end{equation}
%
Also obtain the expectation value of \(\Bsigma \cdot \Bn\) with respect to the state \(\ket{+\theta}\).
%
\paragraph{Solution}
%
For this orientation we have
%
\begin{equation}\label{eqn:boostCommutator:502}
\begin{aligned}
\Bsigma \cdot \Bn
&=
\sin\alpha \PauliX + \cos\alpha \PauliZ
=
\begin{bmatrix}
\cos\alpha & \sin\alpha \\
\sin\alpha & -\cos\alpha
\end{bmatrix}
\end{aligned}
\end{equation}
%
Confirmation that our eigenvalues are \(\pm 1\) is simple, and our eigenstates for the \(+1\) eigenvalue is found to be
%
\begin{equation}\label{eqn:boostCommutator:503}
\begin{aligned}
\ket{+\alpha} \propto
\begin{bmatrix}
\sin\alpha \\
1 - \cos\alpha
\end{bmatrix}
=
\begin{bmatrix}
\sin\alpha/2 \cos\alpha/2 \\
2 \sin^2 \alpha/2
\end{bmatrix}
\propto
\begin{bmatrix}
\cos \alpha/2 \\
\sin\alpha/2
\end{bmatrix}
\end{aligned}
\end{equation}
%
This last has unit norm, so we can write
\begin{equation}\label{eqn:boostCommutator:504}
\begin{aligned}
\ket{+\alpha} =
\begin{bmatrix}
\cos \alpha/2 \\
\sin\alpha/2
\end{bmatrix}
\end{aligned}
\end{equation}
%
If the state has been measured to be
%
\begin{equation}\label{eqn:boostCommutator:505}
\begin{aligned}
\ket{\phi} = 1 \ket{+\alpha} + 0 \ket{-\alpha},
\end{aligned}
\end{equation}
%
then the probability of a second measurement obtaining \(\ket{+\theta}\) is
%
\begin{equation}\label{eqn:boostCommutator:506}
\begin{aligned}
\Abs{ \braket{+\theta}{\phi} }^2
&=
\Abs{ \braket{+\theta}{+\alpha} }^2 .
\end{aligned}
\end{equation}
%
Expanding just the inner product first we have
%
\begin{equation}\label{eqn:desaiCh5:882}
\begin{aligned}
\braket{+\theta}{+\alpha}
&=
\begin{bmatrix}
C_{\theta/2} & S_{\theta/2}
\end{bmatrix}
\begin{bmatrix}
C_{\alpha/2} \\ S_{\alpha/2}
\end{bmatrix} \\
&=
S_{\theta/2} S_{\alpha/2} + C_{\theta/2} C_{\alpha/2}  \\
&= \cos\left( \frac{\theta - \alpha}{2} \right)
\end{aligned}
\end{equation}
%
So our probability of measuring spin up state \(\ket{+\theta}\) given the state was known to have been in spin up state \(\ket{+\alpha}\) is
%
\begin{equation}\label{eqn:boostCommutator:508}
\begin{aligned}
\Abs{ \braket{+\theta}{+\alpha} }^2
= \cos^2\left( \frac{\theta - \alpha}{2} \right)
\end{aligned}
\end{equation}
%
Finally, the expectation value for \(\Bsigma \cdot \Bn\) with respect to \(\ket{+\theta}\) is
\begin{equation}\label{eqn:desaiCh5:902}
\begin{aligned}
\begin{bmatrix}
C_{\theta/2} & S_{\theta/2}
\end{bmatrix}
&
\begin{bmatrix}
C_\alpha & S_\alpha \\
S_\alpha & -C_\alpha
\end{bmatrix}
\begin{bmatrix}
C_{\theta/2} \\
S_{\theta/2}
\end{bmatrix} \\
&=
\begin{bmatrix}
C_{\theta/2} & S_{\theta/2}
\end{bmatrix}
\begin{bmatrix}
C_\alpha C_{\theta/2} + S_\alpha S_{\theta/2} \\
S_\alpha C_{\theta/2} - C_\alpha S_{\theta/2}
\end{bmatrix} \\
&=
C_{\theta/2} C_\alpha C_{\theta/2} + C_{\theta/2} S_\alpha S_{\theta/2}
+ S_{\theta/2} S_\alpha C_{\theta/2} - S_{\theta/2} C_\alpha S_{\theta/2} \\
&=
C_\alpha ( C_{\theta/2}^2 -S_{\theta/2}^2 )
+ 2 S_\alpha S_{\theta/2} C_{\theta/2} \\
&=
 C_\alpha C_\theta
+ S_\alpha S_\theta \\
&= \cos( \alpha - \theta )
\end{aligned}
\end{equation}
%
Sanity checking this we observe that we have \(+1\) as desired for the \(\alpha = \theta\) case.
} % answer
\makeoproblem{}{problem:desaiCh5:5}{\citep{desai2009quantum} pr 5.5}{
\index{ensemble average}
\index{density matrix}
Consider an arbitrary density matrix, \(\rho\), for a spin \(1/2\) system.  Express each matrix element in terms of the ensemble averages \([S_i]\) where \(i = x,y,z\).
} % problem
\makeanswer{problem:desaiCh5:5}{
Let us omit the spin direction temporarily and write for the density matrix
\begin{equation}\label{eqn:desaiCh5:922}
\begin{aligned}
\rho
&=
w_{+} \ket{+}\bra{+}
+
w_{-} \ket{-}\bra{-} \\
&=
w_{+} \ket{+}\bra{+}
+
(1 - w_{+})\ket{-}\bra{-} \\
&=
\ket{-}\bra{-}
+
w_{+} (\ket{+}\bra{+} -\ket{+}\bra{+})
\end{aligned}
\end{equation}
%
For the ensemble average (no sum over repeated indices) we have
\begin{equation}\label{eqn:desaiCh5:942}
\begin{aligned}
[S] = \expectation{S}_{av} &=
w_{+} \bra{+} S \ket{+}
+
w_{-} \bra{-} S \ket{-} \\
&= \frac{\Hbar}{2}( w_{+} -w_{-} ) \\
&= \frac{\Hbar}{2}( w_{+} -(1 - w_{+}) ) \\
&=
\Hbar w_{+} - \inv{2}
\end{aligned}
\end{equation}
%
This gives us
\begin{equation}\label{eqn:desaiCh5:962}
\begin{aligned}
w_{+} = \inv{\Hbar} [S] + \inv{2}
\end{aligned}
\end{equation}
%
and our density matrix becomes
\begin{equation}\label{eqn:desaiCh5:982}
\begin{aligned}
\rho
&=
\inv{2} ( \ket{+}\bra{+} +\ket{-}\bra{-} )
+
\inv{\Hbar} [S] (\ket{+}\bra{+} -\ket{+}\bra{+}) \\
&=
\inv{2} I
+
\inv{\Hbar} [S] (\ket{+}\bra{+} -\ket{+}\bra{+}) \\
\end{aligned}
\end{equation}
%
Utilizing
%
\begin{equation}\label{eqn:desaiCh5:1002}
\begin{aligned}
\ket{x+} &=
\inv{\sqrt{2}}
\begin{bmatrix}
1 \\
1
\end{bmatrix} \\
\ket{x-} &=
\inv{\sqrt{2}}
\begin{bmatrix}
1 \\
-1
\end{bmatrix} \\
\ket{y+} &=
\inv{\sqrt{2}}
\begin{bmatrix}
1 \\
1
\end{bmatrix} \\
\ket{y-} &=
\inv{\sqrt{2}}
\begin{bmatrix}
1 \\
-i
\end{bmatrix} \\
\ket{z+} &=
\begin{bmatrix}
1 \\
0
\end{bmatrix} \\
\ket{z-} &=
\begin{bmatrix}
0 \\
1
\end{bmatrix}
\end{aligned}
\end{equation}
%
We can easily find
\begin{equation}\label{eqn:desaiCh5:1022}
\begin{aligned}
\ket{x+}\bra{x+} -\ket{x+}\bra{x+} &= \PauliX = \sigma_x \\
\ket{y+}\bra{y+} -\ket{y+}\bra{y+} &= \PauliY = \sigma_y \\
\ket{z+}\bra{z+} -\ket{z+}\bra{z+} &= \PauliZ = \sigma_z
\end{aligned}
\end{equation}
%
So we can write the density matrix in terms of any of the ensemble averages as
\begin{equation}\label{eqn:desaiCh5:1042}
\begin{aligned}
\rho =
\inv{2} I
+
\inv{\Hbar} [S_i] \sigma_i
=
\inv{2} (I + [\sigma_i] \sigma_i )
\end{aligned}
\end{equation}
%
Alternatively, defining \(\BP_i = [\sigma_i] \Be_i\), for any of the directions \(i = 1,2,3\) we can write
%
\begin{equation}\label{eqn:desaiCh5:503}
\begin{aligned}
\rho = \inv{2} (I + \Bsigma \cdot \BP_i )
\end{aligned}
\end{equation}
%
In equation (5.109) we had a similar result in terms of the polarization vector \(\BP = \bra{\alpha} \Bsigma \ket{\alpha}\), and the individual weights \(w_\alpha\), and \(w_\beta\), but we see here that this \((w_\alpha - w_\beta)\BP\) factor can be written exclusively in terms of the ensemble average.  Actually, this is also a result in the text, down in (5.113), but we see it here in a more concrete form having picked specific spin directions.
} % answer

%
\makeoproblem{}{problem:desaiCh5:6}{\citep{desai2009quantum} pr 5.6}{
%
\index{spin!time evolution}
If a Hamiltonian is given by \(\Bsigma \cdot \Bn\) where \(\Bn = (\sin\alpha\cos\beta, \sin\alpha\sin\beta, \cos\alpha)\), determine the time evolution operator as a 2 x 2 matrix.  If a state at \(t = 0\) is given by
%
\begin{equation}\label{eqn:desaiCh5:600}
\begin{aligned}
\ket{\phi(0)} =
\begin{bmatrix}
a \\
b
\end{bmatrix},
\end{aligned}
\end{equation}
%
then obtain \(\ket{\phi(t)}\).

} % problem
%
\makeanswer{problem:desaiCh5:6}{
%
Before diving into the meat of the problem, observe that a tidy factorization of the Hamiltonian is possible as a composition of rotations.  That is
%
\begin{equation}\label{eqn:desaiCh5:1062}
\begin{aligned}
H
&= \Bsigma \cdot \Bn \\
&= \sin\alpha \sigma_1 ( \cos\beta + \sigma_1 \sigma_2 \sin\beta ) + \cos\alpha \sigma_3 \\
&= \sigma_3 \left(
\cos\alpha
+ \sin\alpha \sigma_3 \sigma_1 e^{ i \sigma_3 \beta }
\right) \\
&=
\sigma_3 \exp\left( \alpha i \sigma_2
\exp\left( \beta i \sigma_3
\right)
\right)
\end{aligned}
\end{equation}
%
So we have for the time evolution operator
%
\begin{equation}\label{eqn:desaiCh5:610}
\begin{aligned}
U(\Delta t)
&=
\exp( -i \Delta t H /\Hbar )
=
\exp \left(
- \frac{\Delta t}{\Hbar} i \sigma_3 \exp\Bigl( \alpha i \sigma_2
\exp\left( \beta i \sigma_3
\right)
\Bigr)
\right).
\end{aligned}
\end{equation}
%
Does this really help?  I guess not, but it is nice and tidy.

Returning to the specifics of the problem, we note that squaring the Hamiltonian produces the identity matrix
%
\begin{equation}\label{eqn:desaiCh5:615}
\begin{aligned}
(\Bsigma \cdot \Bn)^2 &= I \Bn^2 = I.
\end{aligned}
\end{equation}
%
This allows us to exponentiate \(H\) by inspection utilizing
%
\begin{equation}\label{eqn:desaiCh5:620}
\begin{aligned}
e^{i \mu (\Bsigma \cdot \Bn) } = I \cos\mu + i (\Bsigma \cdot \Bn) \sin\mu
\end{aligned}
\end{equation}
%
Writing \(\sin\mu = S_\mu\), and \(\cos\mu = C_\mu\), we have
\begin{equation}\label{eqn:desaiCh5:625}
\begin{aligned}
\Bsigma \cdot \Bn &=
\begin{bmatrix}
C_\alpha & S_\alpha e^{-i\beta} \\
S_\alpha e^{i\beta} & -C_\alpha
\end{bmatrix},
\end{aligned}
\end{equation}
%
and thus
\begin{equation}\label{eqn:desaiCh5:630}
\begin{aligned}
U(\Delta t) = \exp( -i \Delta t H /\Hbar )
=
\begin{bmatrix}
C_{\Delta t/\Hbar} -i S_{\Delta t/\Hbar} C_\alpha & -i S_{\Delta t/\Hbar} S_\alpha e^{-i\beta} \\
-i S_{\Delta t/\Hbar} S_\alpha e^{i\beta} & C_{\Delta t/\Hbar} + i S_{\Delta t/\Hbar} C_\alpha
\end{bmatrix}.
\end{aligned}
\end{equation}
%
Note that as a sanity check we can calculate that \( U(\Delta t) U(\Delta t)^\dagger = 1\) as expected.

Now for \(\Delta t = t\), we have
\begin{equation}\label{eqn:desaiCh5:640}
\begin{aligned}
U(t,0)
\begin{bmatrix}
a \\
b
\end{bmatrix}
&=
\begin{bmatrix}
a C_{t/\Hbar} -a i S_{t/\Hbar} C_\alpha  - b i S_{t/\Hbar} S_\alpha e^{-i\beta} \\
-a i S_{t/\Hbar} S_\alpha e^{i\beta} + b C_{t/\Hbar} + b i S_{t/\Hbar} C_\alpha
\end{bmatrix}.
\end{aligned}
\end{equation}
%
It does not seem terribly illuminating to multiply this all out, but we can factor the results slightly to tidy it up.  That gives us
%
\begin{equation}\label{eqn:desaiCh5:650}
\begin{aligned}
U(t,0)
\begin{bmatrix}
a \\
b
\end{bmatrix}
&=
\cos(t/\Hbar)
\begin{bmatrix}
a \\
b
\end{bmatrix}
+
\sin(t/\Hbar) \cos\alpha
\begin{bmatrix}
-a \\
b
\end{bmatrix}
+ i
\sin(t/\Hbar) \sin\alpha
\begin{bmatrix}
b e^{-i\beta} \\
-a e^{i \beta}
\end{bmatrix}
\end{aligned}
\end{equation}
%
} % answer

%
\makeoproblem{}{problem:desaiCh5:7}{\citep{desai2009quantum} pr 5.7}{
%
Consider a system of spin \(1/2\) particles in a mixed ensemble containing a mixture of 25\% with spin given by \(\ket{z+}\) and 75\% with spin given by \(\ket{x-}\).  Determine the density matrix \(\rho\) and ensemble averages \(\expectation{\Bsigma_i}_{\text{av}}\) for \(i = x,y,z\).

} % problem
%
\makeanswer{problem:desaiCh5:7}{
%
We have
%
\begin{equation}\label{eqn:desaiCh5:1082}
\begin{aligned}
\rho
&=
\frac{1}{4} \ket{z+}\bra{z+}
+\frac{3}{4} \ket{x-}\bra{x-} \\
&=
\inv{4}
\begin{bmatrix}
1 \\
0
\end{bmatrix}
\begin{bmatrix}
1 & 0
\end{bmatrix}
+\frac{3}{4}
\inv{2}
\begin{bmatrix}
1 \\
-1
\end{bmatrix}
\begin{bmatrix}
1 & -1
\end{bmatrix} \\
&=
\inv{4} \left(
\inv{2}
\begin{bmatrix}
2 & 0 \\
0 & 0
\end{bmatrix}
+
\frac{3}{2}
\begin{bmatrix}
1 & -1 \\
-1 & 1
\end{bmatrix}
\right) \\
\end{aligned}
\end{equation}
%
Giving us
\begin{equation}\label{eqn:desaiCh5:700}
\begin{aligned}
\rho =
\inv{8}
\begin{bmatrix}
5 & -3 \\
-3 & 3
\end{bmatrix}.
\end{aligned}
\end{equation}
%
Note that we can also factor the identity out of this for
%
\begin{equation}\label{eqn:desaiCh5:1102}
\begin{aligned}
\rho
&=
\inv{2}
\begin{bmatrix}
5/4 & -3/4 \\
-3/4 & 3/4
\end{bmatrix}
\\
&=
\inv{2}\left(
I +
\begin{bmatrix}
1/4 & -3/4 \\
-3/4 & -1/4
\end{bmatrix}
\right)
\end{aligned}
\end{equation}
%
which is just:
\begin{equation}\label{eqn:desaiCh5:701}
\begin{aligned}
\rho = \inv{2} \left( I + \inv{4} \sigma_z -\frac{3}{4} \sigma_x \right)
\end{aligned}
\end{equation}
%
Recall that the ensemble average is related to the trace of the density and operator product
%
\begin{equation}\label{eqn:desaiCh5:1122}
\begin{aligned}
\tr( \rho A )
&=
\sum_\beta \bra{\beta} \rho A \ket{\beta} \\
&=
\sum_{\beta} \bra{\beta} \left( \sum_\alpha w_\alpha \ket{\alpha}\bra{\alpha} \right) A \ket{\beta} \\
&=
\sum_{\alpha, \beta} w_\alpha \braket{\beta}{\alpha}\bra{\alpha} A \ket{\beta} \\
&=
\sum_{\alpha, \beta} w_\alpha \bra{\alpha} A \ket{\beta} \braket{\beta}{\alpha}
\\
&=
\sum_{\alpha} w_\alpha \bra{\alpha} A \left( \sum_\beta \ket{\beta} \bra{\beta} \right) \ket{\alpha}
\\
&=
\sum_\alpha w_\alpha \bra{\alpha} A \ket{\alpha}
\end{aligned}
\end{equation}
%
But this, by definition of the ensemble average, is just
\begin{equation}\label{eqn:desaiCh5:710}
\begin{aligned}
\tr( \rho A )
&=
\expectation{A}_{\text{av}}.
\end{aligned}
\end{equation}
%
We can use this to compute the ensemble averages of the Pauli matrices
\begin{equation}\label{eqn:desaiCh5:1142}
\begin{aligned}
\expectation{\sigma_x}_{\text{av}} &= \tr \left(
\inv{8}
\begin{bmatrix}
5 & -3 \\
-3 & 3
\end{bmatrix}
\PauliX
\right) = -\frac{3}{4} \\
\expectation{\sigma_y}_{\text{av}} &= \tr \left(
\inv{8}
\begin{bmatrix}
5 & -3 \\
-3 & 3
\end{bmatrix}
\PauliY
\right) = 0 \\
\expectation{\sigma_z}_{\text{av}} &= \tr \left(
\inv{8}
\begin{bmatrix}
5 & -3 \\
-3 & 3
\end{bmatrix}
\PauliZ
\right) = \frac{1}{4} \\
\end{aligned}
\end{equation}
%
We can also find without the explicit matrix multiplication from \eqnref{eqn:desaiCh5:701}
%
\begin{equation}\label{eqn:desaiCh5:1162}
\begin{aligned}
\expectation{\sigma_x}_{\text{av}} &= \tr \inv{2}\left(
\sigma_x + \inv{4} \sigma_z \sigma_x -\frac{3}{4} \sigma_x^2
\right) = -\frac{3}{4} \\
\expectation{\sigma_y}_{\text{av}} &= \tr \inv{2}\left(
\sigma_y + \inv{4} \sigma_z \sigma_y -\frac{3}{4} \sigma_x \sigma_y
\right) = 0 \\
\expectation{\sigma_z}_{\text{av}} &= \tr \inv{2}\left(
\sigma_z + \inv{4} \sigma_z^2 -\frac{3}{4} \sigma_x \sigma_z
\right) = \inv{4}.
\end{aligned}
\end{equation}
%
(where to do so we observe that \(\tr \sigma_i \sigma_j = 0\) for \(i\ne j\) and \(\tr \sigma_i = 0\), and \(\tr \sigma_i^2 = 2\).)

We see that the traces of the density operator and Pauli matrix products act very much like dot products extracting out the ensemble averages, which end up very much like the magnitudes of the projections in each of the directions.

} % answer

%
\makeoproblem{}{problem:desaiCh5:8}{\citep{desai2009quantum} pr 5.8}{
%
Show that the quantity \(\Bsigma \cdot \Bp V(r) \Bsigma \cdot \Bp\), when simplified, has a term proportional to \(\BL \cdot \Bsigma\).

} % problem
%
\makeanswer{problem:desaiCh5:8}{
%
Consider the operation
%
\begin{equation}\label{eqn:desaiCh5:1182}
\begin{aligned}
\Bsigma \cdot \Bp V(r) \Psi
&=
- i \Hbar \sigma_k \partial_k V(r) \Psi \\
&=
- i \Hbar \sigma_k (\partial_k V(r)) \Psi
+ V(r) (\Bsigma \cdot \Bp ) \Psi  \\
\end{aligned}
\end{equation}
%
With \(r = \sqrt{\sum_j x_j^2}\), we have
%
\begin{equation}\label{eqn:desaiCh5:1202}
\begin{aligned}
\partial_k V(r) = \inv{2}\inv{r} 2 x_k \PD{r}{V(r)},
\end{aligned}
\end{equation}
%
which gives us the commutator
%
\begin{equation}\label{eqn:desaiCh5:800}
\begin{aligned}
\antisymmetric{ \Bsigma \cdot \Bp}{V(r)}
&=
- \frac{i \Hbar}{r} \PD{r}{V(r)} (\Bsigma \cdot \Bx)
\end{aligned}
\end{equation}
%
Insertion into the operator in question we have
%
\begin{equation}\label{eqn:desaiCh5:801}
\begin{aligned}
\Bsigma \cdot \Bp V(r) \Bsigma \cdot \Bp =
- \frac{i \Hbar}{r} \PD{r}{V(r)} (\Bsigma \cdot \Bx) (\Bsigma \cdot \Bp )
+ V(r) (\Bsigma \cdot \Bp )^2
\end{aligned}
\end{equation}
%
With decomposition of the \((\Bsigma \cdot \Bx) (\Bsigma \cdot \Bp )\) into symmetric and antisymmetric components, we should have in the second term our \(\Bsigma \cdot \BL\)
%
\begin{equation}\label{eqn:desaiCh5:802}
\begin{aligned}
(\Bsigma \cdot \Bx) (\Bsigma \cdot \Bp )
=
\inv{2} \symmetric{\Bsigma \cdot \Bx}{\Bsigma \cdot \Bp}
+\inv{2} \antisymmetric{\Bsigma \cdot \Bx}{\Bsigma \cdot \Bp}
\end{aligned}
\end{equation}
%
where we expect \(\Bsigma \cdot \BL \propto \antisymmetric{\Bsigma \cdot \Bx}{\Bsigma \cdot \Bp}\).  Alternately in components
%
\begin{equation}\label{eqn:desaiCh5:1222}
\begin{aligned}
(\Bsigma \cdot \Bx) (\Bsigma \cdot \Bp )
&=
\sigma_k x_k \sigma_j p_j \\
&=
x_k p_k I + \sum_{j\ne k} \sigma_k \sigma_j x_k p_j \\
&=
x_k p_k I + i \sum_m \epsilon_{kjm} \sigma_m x_k p_j \\
&=
I (\Bx \cdot \Bp) + i (\Bsigma \cdot \BL)
\end{aligned}
\end{equation}
} % answer
