%
% Copyright © 2012 Peeter Joot.  All Rights Reserved.
% Licenced as described in the file LICENSE under the root directory of this GIT repository.
%

\makeproblem{Parity operator (2007 PHY355H1F 1b)}{problem:qmIexamPracticeParityOperator:1}{
\index{operator!parity}
If \(\Pi\) is the parity operator, defined by \(\Pi \ket{x} = \ket{-x}\), where \(\ket{x}\) is the eigenket of the position operator \(X\) with eigenvalue \(x\)), and \(P\) is the momentum operator conjugate to \(X\), show (carefully) that \(\Pi P \Pi = -P\).

} % problem

\makeanswer{problem:qmIexamPracticeParityOperator:1}{

Consider the matrix element \(\bra{-x'} \antisymmetric{\Pi}{P} \ket{x}\).  This is

\begin{equation}\label{eqn:qmIexamPracticeParityOperator:430}
\begin{aligned}
\bra{-x'} \antisymmetric{\Pi}{P} \ket{x}
&=
\bra{-x'} \Pi P - P \Pi \ket{x} \\
&=
\bra{-x'} \Pi P \ket{x} - \bra{-x} P \Pi \ket{x} \\
&=
\bra{x'} P \ket{x} - \bra{-x} P \ket{-x} \\
&=
- i \Hbar \left(
\delta(x'-x) \PD{x}{}
-
\mathLabelBox{\delta(-x -(-x'))}{\(= \delta(x'-x) = \delta(x-x')\)}
\PD{-x}{}
\right) \\
&=
- 2 i \Hbar
\delta(x'-x) \PD{x}{} \\
&=
2 \bra{x'} P \ket{x} \\
&=
2 \bra{-x'} \Pi P \ket{x} \\
\end{aligned}
\end{equation}

We have taken advantage of the Hermitian property of \(P\) and \(\Pi\) here, and can rearrange for

\begin{equation}\label{eqn:qmIexamPractice2007Dec:1b:10}
\bra{-x'} \Pi P - P \Pi - 2 \Pi P \ket{x} = 0
\end{equation}

Since this is true for all \(\bra{-x}\) and \(\ket{x}\) we have

\begin{equation}\label{eqn:qmIexamPractice2007Dec:1b:20}
\Pi P + P \Pi = 0.
\end{equation}

Right multiplication by \(\Pi\) and rearranging we have
\begin{equation}\label{eqn:qmIexamPractice2007Dec:1b:30}
\Pi P \Pi = - P \Pi \Pi = - P.
\end{equation}

} % answer
