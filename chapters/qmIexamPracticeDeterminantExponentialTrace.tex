%
% Copyright © 2012 Peeter Joot.  All Rights Reserved.
% Licenced as described in the file LICENSE under the root directory of this GIT repository.
%
\makeproblem{Trace and determinant of an exponential operator (2008 PHY355H1F final 1d.)}{problem:qmIexamPractice:2}{
If \(A\) is an Hermitian operator, show that
\begin{equation}\label{eqn:qmIexamPractice2008Dec:1d:10}
\Det( \exp A ) = \exp ( \tr(A) )
\end{equation}
where the \textAndIndex{determinant} (\(\Det\)) of an operator is the product of all its eigenvectors.
} % problem
\makeanswer{problem:qmIexamPractice:2}{
The eigenvalues clue in the question provides the starting point.  We write the exponential in its series form
\begin{equation}\label{eqn:qmIexamPractice2008Dec:1d:20}
e^A = 1 + \sum_{k=1}^\infty \inv{k!} A^k
\end{equation}
%
Now, suppose that we have the following eigenvalue relationships for \(A\)
\begin{equation}\label{eqn:qmIexamPractice2008Dec:1d:30}
A \ket{n} = \lambda_n \ket{n}.
\end{equation}
%
From this the exponential is
\begin{equation}\label{eqn:qmIexamPractice:610}
\begin{aligned}
e^A \ket{n}
&= \ket{n} + \sum_{k=1}^\infty \inv{k!} A^k \ket{n} \\
&= \ket{n} + \sum_{k=1}^\infty \inv{k!} (\lambda_n)^k \ket{n} \\
&= e^{\lambda_n} \ket{n}.
\end{aligned}
\end{equation}
%
We see that the eigenstates of \(e^A\) are those of \(A\), with eigenvalues \(e^{\lambda_n}\).

By the definition of the determinant given we have
%
\begin{equation}\label{eqn:qmIexamPractice:630}
\begin{aligned}
\Det( e^A )
&= \prod_n e^{\lambda_n} \\
&= e^{\sum_n \lambda_n} \\
&= e^{\trace(A)}. \qedmarker
\end{aligned}
\end{equation}
%
} % answer
