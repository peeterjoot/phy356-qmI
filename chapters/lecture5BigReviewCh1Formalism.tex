%
% Copyright © 2012 Peeter Joot.  All Rights Reserved.
% Licenced as described in the file LICENSE under the root directory of this GIT repository.
%
%\QMlecture{5 --- Review --- October 12, 2010}
%
%Review.  What have we learned?

Information about systems comes from vectors and operators.  Express the vector \(\ket{\phi}\) describing the system in terms of eigenvectors \(\ket{a_n}, n \in 1,2,3,\cdots\) of some operator \(A\).
\index{operator}
\index{eigenvector}
%
\begin{equation}\label{eqn:PHY356FLecture10:51}
\ket{\phi} = \sum_n c_n \ket{a_n} = \sum_n \ket{c_n a_n}
\end{equation}

What are the coefficients \(c_n\)?  Act on both sides by \(\bra{a_m}\) to find
%
\begin{equation}\label{eqn:lecture5BigReviewCh1Formalism:71}
\begin{aligned}
\braket{a_m}{\phi}
&= \sum_n \bra{a_m} \ket{ c_n a_n } \\
&= \sum_n c_n \braket{a_m}{ a_n } \\
&= \sum_n c_n
\mathLabelBox{\braket{a_m}{a_n}}{Kronecker delta}
\\
&= \sum c_n \delta_{mn} \\
&= c_m
\end{aligned}
\end{equation}
\index{Kronecker delta}

So our coefficients are
%
\begin{equation}\label{eqn:PHY356FLecture10:51c}
c_m = \braket{a_m}{\phi}.
\end{equation}

The complete decomposition in terms of the chosen basis of \(A\) is then
%
\begin{equation}\label{eqn:PHY356FLecture10:51b}
\ket{\phi} = \sum_n \braket{a_n}{\phi} \ket{a_n}
= \left( \sum_n \ket{a_n} \bra{a_n} \right) \ket{\phi}.
\end{equation}

Note carefully the physics convention for this complex inner product.  We have linearity in the \textunderline{second} argument
\index{braket}
\begin{equation}\label{eqn:PHY356FLecture10:51e}
\braket{\psi}{a \phi} = a \braket{\psi}{\phi},
\end{equation}

whereas the normal mathematics convention is to define complex inner products as linear in the \textunderline{first} argument
%
\begin{equation}\label{eqn:PHY356FLecture10:51f}
\innerprod{a \psi}{\phi} = a \innerprod{\psi}{\phi}.
\end{equation}

We can make an analogy with 3D Euclidean inner products easily
%
\begin{equation}\label{eqn:lecture5BigReviewCh1Formalism:91}
\begin{aligned}
\Bv = \sum_i v_i \Be_i
\end{aligned}
\end{equation}
%
\begin{equation}\label{eqn:lecture5BigReviewCh1Formalism:111}
\begin{aligned}
\Be_1 \cdot \Bv = \sum_i v_i \Be_1 \cdot \Be_i = v_1
\end{aligned}
\end{equation}

\index{probability}
\index{outcome}
\index{eigenvalue}
Physical information comes from the probability for obtaining a measurement of the physical entity associated with operator \(A\).  The probability of obtaining outcome \(a_m\), an eigenvalue of \(A\), is \(\Abs{c_m}^2\)

