%
% Copyright � 2012 Peeter Joot.  All Rights Reserved.
% Licenced as described in the file LICENSE under the root directory of this GIT repository.
%

%\chapter{On commutation of exponentials}
\label{chap:exponentialCommutation}
%\blogpage{http://sites.google.com/site/peeterjoot/math2010/exponentialCommutation.pdf}
%\date{May 30, 2010}
%
%\subsection{Motivation}
%
Previously while working
\href{https://peeterjoot.wordpress.com/2010/05/23/effect-of-sinusoid-operators/}{a Liboff problem}, I wondered about what the conditions were required for exponentials to commute.  In those problems the exponential arguments were operators.  Exponentials of bivectors as in quaternion like spatial or Lorentz boosts are also good examples of (sometimes) non-commutative exponentials.  It appears likely that the key requirement is that the exponential arguments commute, but how does one show this?  Here this is explored a bit.
%
%\subsection{Guts}
%
If one could show that it was true that

\begin{align}\label{eqn:exponentialCommutation:1}
e^{x} e^{y} = e^{x + y}.
\end{align}

Then it would also imply that

\begin{align}\label{eqn:exponentialCommutation:2}
e^{x} e^{y} = e^{y} e^{x}.
\end{align}

Let us perform the school boy exercise to prove \eqnref{eqn:exponentialCommutation:1} and explore the restrictions for such a proof.  We assume a power series definition of the exponential operator, and do not assume the values \(x,y\) are numeric, instead just that they can be multiplied.  A commutative multiplication will not be assumed.

By virtue of the power series exponential definition we have

\begin{align}\label{eqn:exponentialCommutation:3}
e^{x} e^{y} =
\sum_{k=0}^\infty \inv{k!} x^k
\sum_{m=0}^\infty \inv{m!} y^m.
\end{align}

To attempt to put this into \(e^{x + y}\) form we will need to change the order that we evaluate the double sum, and here a picture \cref{fig:gridSummation} is helpful.

\imageFigure{../figures/phy356-qmI/gridSummation}{Double sum diagonal ordering}{fig:gridSummation}{0.4}

For somebody who has seen this summation trick before the picture probably says it all.  We want to iterate over all pairs \((k, m)\), and could do so in \(\{(k, 0), (k, 1), \cdots (k, \infty), k \in [0, \infty] \}\) order as in our sum.  This is all the pairs of points in the upper right hand side of the grid.  We can also cover these grid coordinates in a different order.  In particular, these can be iterated over the diagonals.  The first diagonal having the point \((0,0)\), the second with the points \(\{(0, 1), (1, 0)\}\), the third with the points \(\{(0, 2), (1, 1), (2, 0)\}\).

Observe that along each diagonal the sum of the coordinates is constant, and increases by one.  Also observe that the number of points in each diagonal is this sum.  These observations provide a natural way to index the new grid traversal.  Labeling each of these diagonals with index \(j\), and points on that subset with \(n=0,1,\cdots, j\), we can express the original loop indices \(k\) and \(m\) in terms of these new (coupled) loop indices \(j\) and \(n\) as follows

\begin{align}\label{eqn:exponentialCommutation:4}
k &= j - n \\
m &= n.
\end{align}

Our sum becomes

\begin{align}\label{eqn:exponentialCommutation:5}
e^{x} e^{y} =
\sum_{j=0}^\infty \sum_{n=0}^j
\inv{(j-n)!} x^{j-n}
\inv{n!} y^n.
\end{align}

With one small rearrangement, by introducing a \(j!\) in both the numerator and the denominator, the goal is almost reached.

\begin{align}\label{eqn:exponentialCommutation:6}
e^{x} e^{y}
=
\sum_{j=0}^\infty \inv{j!} \sum_{n=0}^j \frac{j!}{(j-n)! n!} x^{j-n} y^n
=
\sum_{j=0}^\infty \inv{j!} \sum_{n=0}^j \binom{n}{j} x^{j-n} y^n.
\end{align}

This shows where we have a requirement that \(x\) and \(y\) commute, because only in that case do we have a binomial expansion

\begin{align}\label{eqn:exponentialCommutation:6b}
(x + y)^j = \sum_{n=0}^j \binom{n}{j} x^{j-n} y^n,
\end{align}

in the interior sum.  This reduced the problem to a consideration of the implication of possible non-commutation have on the binomial expansion.  Consider the simple special case of \((x + y)^2\).  If \(x\) and \(y\) do not necessarily commute, then we have

\begin{align}\label{eqn:exponentialCommutation:7}
(x + y)^2 = x^2 + x y + y x + y^2
\end{align}

whereas the binomial expansion formula has no such allowance for non-commutative multiplication and just counts the number of times a product can occur in any ordering as in

\begin{align}\label{eqn:exponentialCommutation:8}
(x + y)^2 = x^2 + 2 x y + y^2 = x^2 + 2 y x + y^2.
\end{align}

One sees the built in requirement for commutative multiplication here.  Now this does not prove that \(e^{x} e^{y} != e^{y} e^{x}\) unconditionally if \(x\) and \(y\) do not commute, but we do see that a requirement for commutative multiplication is sufficient if we want equality of such commuted exponentials.  In particular, the end result of the Liboff calculation where we had

\begin{align}\label{eqn:exponentialCommutation:9}
e^{i \hat{f}} e^{-i \hat{f}},
\end{align}

and was assuming this to be unity even for the differential operators \(\hat{f}\) under consideration is now completely answered (since we have \((i \hat{f}) (-i \hat{f}) \psi = (-i \hat{f}) (i \hat{f}) \psi\)).
