%
% Copyright � 2012 Peeter Joot.  All Rights Reserved.
% Licenced as described in the file LICENSE under the root directory of this GIT repository.
%

%\chapter{Notes for Desai Chapter 26}
\label{chap:desaiCh26}
%\blogpage{http://sites.google.com/site/peeterjoot/math2010/desaiCh26.pdf}
%\date{Dec 9, 2010}
%
%\section{Motivation}
%
%Chapter 26 notes for \citep{desai2009quantum}.
%
%\section{Guts}
%
\section{Trig relations}
\index{trigonometric identities}

To verify equations 26.3-5 in the text it is worth noting that

\begin{equation}\label{eqn:desaiCh26:150}
\begin{aligned}
\cos(a + b)
&= \Real( e^{ia} e^{ib} ) \\
&= \Real( (\cos a + i \sin a)( \cos b + i \sin b) ) \\
&= \cos a \cos b - \sin a \sin b
\end{aligned}
\end{equation}

and
\begin{equation}\label{eqn:desaiCh26:170}
\begin{aligned}
\sin(a + b)
&= \Imag( e^{ia} e^{ib} ) \\
&= \Imag( (\cos a + i \sin a)( \cos b + i \sin b) ) \\
&= \cos a \sin b + \sin a \cos b
\end{aligned}
\end{equation}

So, for
\begin{equation}\label{eqn:desaiCh26:10}
\begin{aligned}
x &= \rho \cos\alpha \\
y &= \rho \sin\alpha
\end{aligned}
\end{equation}

the transformed coordinates are
\begin{equation}\label{eqn:desaiCh26:190}
\begin{aligned}
x'
&= \rho \cos(\alpha + \phi) \\
&= \rho (\cos \alpha \cos \phi - \sin \alpha \sin \phi) \\
&= x \cos \phi - y \sin \phi
\end{aligned}
\end{equation}

and
\begin{equation}\label{eqn:desaiCh26:210}
\begin{aligned}
y'
&= \rho \sin(\alpha + \phi) \\
&= \rho (\cos \alpha \sin \phi + \sin \alpha \cos \phi) \\
&= x \sin \phi + y \cos \phi \\
\end{aligned}
\end{equation}

This allows us to read off the rotation matrix.  Without all the messy trig, we can also derive this matrix with geometric algebra.

\begin{equation}\label{eqn:desaiCh26:230}
\begin{aligned}
\Bv'
&= e^{- \Be_1 \Be_2 \phi/2 } \Bv e^{ \Be_1 \Be_2 \phi/2 } \\
&= v_3 \Be_3 + (v_1 \Be_1 + v_2 \Be_2) e^{ \Be_1 \Be_2 \phi } \\
&= v_3 \Be_3 + (v_1 \Be_1 + v_2 \Be_2) (\cos \phi + \Be_1 \Be_2 \sin\phi) \\
&= v_3 \Be_3
+ \Be_1 (v_1 \cos\phi - v_2 \sin\phi)
+ \Be_2 (v_2 \cos\phi + v_1 \sin\phi)
\end{aligned}
\end{equation}

Here we use the Pauli-matrix like identities

\begin{equation}\label{eqn:desaiCh26:20}
\begin{aligned}
\Be_k^2 &= 1 \\
\Be_i \Be_j &= -\Be_j \Be_i,\quad i\ne j
\end{aligned}
\end{equation}

and also note that \(\Be_3\) commutes with the bivector for the \(x,y\) plane \(\Be_1 \Be_2\).  We can also read off the rotation matrix from this.

\section{Infinitesimal transformations}
\index{infinitesimal transformation}

Recall that in the problems of Chapter 5, one representation of spin one matrices were calculated \chapcite{desaiCh5}.  Since the choice of the basis vectors was arbitrary in that exercise, we ended up with a different representation.  For \(S_x, S_y, S_z\) as found in (26.20) and (26.23) we can also verify easily that we have eigenvalues \(0, \pm \Hbar\).  We can also show that our spin kets in this non-diagonal representation have the following column matrix representations:

\begin{equation}\label{eqn:desaiCh26:30}
\begin{aligned}
\ket{1,\pm 1}_x
&=
\inv{\sqrt{2}} \begin{bmatrix}
0 \\
1 \\
\pm i
\end{bmatrix} \\
\ket{1,0}_x
&=
\begin{bmatrix}
1 \\
0 \\
0
\end{bmatrix} \\
\ket{1,\pm 1}_y
&=
\inv{\sqrt{2}} \begin{bmatrix}
\pm i \\
0 \\
1
\end{bmatrix} \\
\ket{1,0}_y
&=
\begin{bmatrix}
0 \\
1 \\
0
\end{bmatrix} \\
\ket{1,\pm 1}_z
&=
\inv{\sqrt{2}} \begin{bmatrix}
1 \\
\pm i \\
0
\end{bmatrix} \\
\ket{1,0}_z
&=
\begin{bmatrix}
0 \\
0 \\
1
\end{bmatrix}
\end{aligned}
\end{equation}

\section{Verifying the commutator relations}
\index{commutator relations}

\index{spin!one}
Given the (summation convention) matrix representation for the spin one operators

\begin{equation}\label{eqn:desaiCh26:40}
(S_i)_{jk} = - i \Hbar \epsilon_{ijk},
\end{equation}

let us demonstrate the commutator relation of (26.25).

\begin{equation}\label{eqn:desaiCh26:250}
\begin{aligned}
{\antisymmetric{S_i}{S_j}}_{rs}
&=
(S_i S_j - S_j S_i)_{rs} \\
&=
\sum_t (S_i)_{rt} (S_j)_{ts} - (S_j)_{rt} (S_i)_{ts} \\
&=
(-i\Hbar)^2 \sum_t \epsilon_{irt} \epsilon_{jts} - \epsilon_{jrt} \epsilon_{its} \\
&=
-(-i\Hbar)^2 \sum_t \epsilon_{tir} \epsilon_{tjs} - \epsilon_{tjr} \epsilon_{tis} \\
\end{aligned}
\end{equation}

Now we can employ the summation rule for sums products of antisymmetric tensors over one free index (4.179)

\begin{equation}\label{eqn:desaiCh26:50}
\sum_i
\epsilon_{ijk} \epsilon_{iab}
=
\delta_{ja}
\delta_{kb}
-\delta_{jb}
\delta_{ka}.
\end{equation}

Continuing we get
\begin{equation}\label{eqn:desaiCh26:270}
\begin{aligned}
{\antisymmetric{S_i}{S_j}}_{rs}
&=
-(-i\Hbar)^2 \left(
\delta_{ij}
\delta_{rs}
-\delta_{is}
\delta_{rj}
-\delta_{ji}
\delta_{rs}
+\delta_{js}
\delta_{ri} \right) \\
&=
(-i\Hbar)^2 \left(
\delta_{is}
\delta_{jr}
-
\delta_{ir}
\delta_{js}
\right)
\\
&=
(-i\Hbar)^2 \sum_t \epsilon_{tij} \epsilon_{tsr}
\\
&=
i\Hbar \sum_t \epsilon_{tij} (S_t)_{rs}
\qedmarker
\end{aligned}
\end{equation}

\section{General infinitesimal rotation}
\index{infinitesimal rotation}

Equation (26.26) has for an infinitesimal rotation counterclockwise around the unit axis of rotation vector \(\Bn\)

\begin{equation}\label{eqn:desaiCh26:60}
\BV' = \BV + \epsilon \Bn \cross \BV.
\end{equation}

Let us derive this using the geometric algebra rotation expression for the same

\begin{equation}\label{eqn:desaiCh26:290}
\begin{aligned}
\BV'
&=
e^{-I\Bn \alpha/2}
\BV
e^{I\Bn \alpha/2} \\
&=
e^{-I\Bn \alpha/2}
\left(
(\BV \cdot \Bn)\Bn
+(\BV \wedge \Bn)\Bn
\right)
e^{I\Bn \alpha/2} \\
&=
(\BV \cdot \Bn)\Bn
+(\BV \wedge \Bn)\Bn
e^{I\Bn \alpha}
\end{aligned}
\end{equation}

We note that \(I\Bn\) and thus the exponential commutes with \(\Bn\), and the projection component in the normal direction.  Similarly \(I\Bn\) anticommutes with \((\BV \wedge \Bn) \Bn\).  This leaves us with

\begin{equation}\label{eqn:desaiCh26:310}
\begin{aligned}
\BV'
&=
(\BV \cdot \Bn)\Bn
\left(
+(\BV \wedge \Bn)\Bn
\right)
( \cos \alpha + I \Bn \sin\alpha)
\end{aligned}
\end{equation}

For \(\alpha = \epsilon \rightarrow 0\), this is

\begin{equation}\label{eqn:desaiCh26:330}
\begin{aligned}
\BV'
&=
(\BV \cdot \Bn)\Bn
+(\BV \wedge \Bn)\Bn
( 1 + I \Bn \epsilon) \\
&=
(\BV \cdot \Bn)\Bn
+(\BV \wedge \Bn)\Bn
+\epsilon I^2(\BV \cross \Bn)\Bn^2 \\
&=
\BV
+ \epsilon (\Bn \cross \BV)
\qedmarker
\end{aligned}
\end{equation}

\section{Position and angular momentum commutator}
\index{position commutator}
\index{momentum commutator}

Equation (26.71) is
\begin{equation}\label{eqn:desaiCh26:70}
\antisymmetric{x_i}{L_j} = i \Hbar \epsilon_{ijk} x_k.
\end{equation}

Let us derive this.  Recall that we have for the position-momentum commutator

\begin{equation}\label{eqn:desaiCh26:80}
\antisymmetric{x_i}{p_j} = i \Hbar \delta_{ij},
\end{equation}

and for each of the angular momentum operator components we have
\begin{equation}\label{eqn:desaiCh26:90}
L_m = \epsilon_{mab} x_a p_b.
\end{equation}

The commutator of interest is thus
\begin{equation}\label{eqn:desaiCh26:350}
\begin{aligned}
\antisymmetric{x_i}{L_j}
&=
x_i \epsilon_{jab} x_a p_b
-
\epsilon_{jab} x_a p_b x_i \\
&=
\epsilon_{jab}
x_a
\left(
x_i
p_b
-
p_b x_i \right) \\
&=
\epsilon_{jab}
x_a
i \Hbar \delta_{ib} \\
&=
i \Hbar
\epsilon_{jai}
x_a \\
&=
i \Hbar
\epsilon_{ija}
x_a
\qedmarker
\end{aligned}
\end{equation}

\section{A note on the angular momentum operator exponential sandwiches}
\index{angular momentum operator}
\index{exponential sandwiches}

In (26.73-74) we have
\begin{equation}\label{eqn:desaiCh26:100}
e^{i \epsilon L_z/\Hbar} x e^{-i \epsilon L_z/\Hbar}
= x + \frac{i \epsilon}{\Hbar} \antisymmetric{L_z}{x}
\end{equation}

Observe that
\begin{equation}\label{eqn:desaiCh26:110}
\antisymmetric{x}{\antisymmetric{L_z}{x}} = 0
\end{equation}

so from the first two terms of (10.99)

\begin{equation}\label{eqn:desaiCh26:120}
e^{A} B e^{-A}
= B + \antisymmetric{A}{B}
+\inv{2} \antisymmetric{A}{\antisymmetric{A}{B}} \cdots
\end{equation}

we get the desired result.

\section{Trace relation to the determinant}
\index{trace}
\index{determinant}

Going from (26.90) to (26.91) we appear to have a mystery identity
\begin{equation}\label{eqn:desaiCh26:130}
\det \left( \BOne + \mu \BA \right) = 1 + \mu \tr \BA
\end{equation}

According to wikipedia, under derivative of a \textAndIndex{determinant}, \citep{wiki:determinant}, this is good for small \(\mu\), and related to something called the \textAndIndex{Jacobi identity}.  Someday I should really get around to studying determinants in depth, and will take this one for granted for now.
