%
% Copyright © 2012 Peeter Joot.  All Rights Reserved.
% Licenced as described in the file LICENSE under the root directory of this GIT repository.
%
%\makeproblem{Eigenvectors of the Harmonic oscillator creation operator (2008 PHY355H1F final 1e.)}{problem:qmIexamPractice:3}{
\makeproblem{SHO creation operator (2008 PHY355H1F final 1e.)}{problem:qmIexamPractice:3}{
Prove that the only eigenvector of the Harmonic oscillator creation operator is \(\ket{\text{null}}\).
} % problem
\makeanswer{problem:qmIexamPractice:3}{
Recall that the creation (raising) operator was given by
\begin{equation}\label{eqn:qmIexamPractice2008Dec:1e:10}
a^\dagger
= \sqrt{\frac{m \omega}{2 \Hbar}} X - \frac{ i }{\sqrt{2 m \omega \Hbar} } P
= \inv{ \alpha \sqrt{2} } X - \frac{ i \alpha }{\sqrt{2} \Hbar } P,
\end{equation}
where \(\alpha = \sqrt{\Hbar/m \omega}\).  Now assume that \(a^\dagger \ket{\phi} = \lambda \ket{\phi}\) so that
\begin{equation}\label{eqn:qmIexamPractice2008Dec:1e:20}
\bra{x} a^\dagger \ket{\phi} = \bra{x} \lambda \ket{\phi}.
\end{equation}

Write \(\braket{x}{\phi} = \phi(x)\), and expand the LHS using \eqnref{eqn:qmIexamPractice2008Dec:1e:10} for
\begin{equation}\label{eqn:qmIexamPractice:650}
\begin{aligned}
\lambda \phi(x)
&= \bra{x} a^\dagger \ket{\phi}  \\
&= \bra{x} \left( \inv{ \alpha \sqrt{2} } X - \frac{ i \alpha }{\sqrt{2} \Hbar } P \right) \ket{\phi} \\
&= \frac{x \phi(x)}{ \alpha \sqrt{2} } - \frac{ i \alpha }{\sqrt{2} \Hbar } (-i\Hbar)\PD{x}{} \phi(x) \\
&= \frac{x \phi(x)}{ \alpha \sqrt{2} } - \frac{ \alpha }{\sqrt{2} } \PD{x}{\phi(x)}.
\end{aligned}
\end{equation}

As usual write \(\xi = x/\alpha\), and rearrange.  This gives us
\begin{equation}\label{eqn:qmIexamPractice2008Dec:1e:30}
\PD{\xi}{\phi} +\sqrt{2} \lambda \phi - \xi \phi = 0.
\end{equation}

Observe that this can be viewed as a homogeneous LDE of the form
\begin{equation}\label{eqn:qmIexamPractice2008Dec:1e:40}
\PD{\xi}{\phi} - \xi \phi = 0,
\end{equation}

augmented by a forcing term \(\sqrt{2}\lambda \phi\).  The homogeneous equation has the solution \(\phi = A e^{\xi^2/2}\), so for the complete equation we assume a solution
%
\begin{equation}\label{eqn:qmIexamPractice2008Dec:1e:50}
\phi(\xi) = A(\xi) e^{\xi^2/2}.
\end{equation}

Since \(\phi' = (A' + A \xi) e^{\xi^2/2}\), we produce a LDE of
%
\begin{equation}\label{eqn:qmIexamPractice:670}
\begin{aligned}
0 &= (A' + A \xi -\xi A + \sqrt{2} \lambda A ) e^{\xi^2/2} \\
&= (A' + \sqrt{2} \lambda A ) e^{\xi^2/2},
\end{aligned}
\end{equation}

or
\begin{equation}\label{eqn:qmIexamPractice2008Dec:1e:60}
0 = A' + \sqrt{2} \lambda A.
\end{equation}

This has solution \(A = B e^{-\sqrt{2} \lambda \xi}\), so our solution for \eqnref{eqn:qmIexamPractice2008Dec:1e:30} is
\begin{equation}\label{eqn:qmIexamPractice2008Dec:1e:70}
\phi(\xi) = B e^{\xi^2/2 - \sqrt{2} \lambda \xi}
= B' e^{ (\xi - \lambda \sqrt{2} )^2/2}.
\end{equation}

This wave function is an imaginary Gaussian with minimum at \(\xi = \lambda\sqrt{2}\).  It is also unnormalizable since we require \(B' = 0\) for any \(\lambda\) if \(\int \Abs{\phi}^2 < \infty\).  Since \(\braket{\xi}{\phi} = \phi(\xi) = 0\), we must also have \(\ket{\phi} = 0\), completing the exercise.

} % answer
