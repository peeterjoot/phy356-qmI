%
% Copyright © 2012 Peeter Joot.  All Rights Reserved.
% Licenced as described in the file LICENSE under the root directory of this GIT repository.
%
%
\makeproblem{Two level quantum system (2008 PHY355H1F final 2.)}{problem:qmIexamPractice2008Dec:4}{
\index{two-level system}
Consider a two-level quantum system, with basis states \(\{\ket{a}, \ket{b}\}\).  Suppose that the Hamiltonian for this system is given by
%
\begin{equation}\label{eqn:qmIexamPractice2008Dec:2:10}
H =
\frac{\Hbar \Delta}{2} (
\ket{b}\bra{b}
- \ket{a}\bra{a}
)
+ i \frac{\Hbar \Omega}{2} (
\ket{a}\bra{b}
- \ket{b}\bra{a}
)
\end{equation}

where \(\Delta\) and \(\Omega\) are real positive constants.

Find the energy eigenvalues and the normalized energy eigenvectors (expressed in terms of the \(\{\ket{a}, \ket{b}\}\) basis).

Write the time evolution operator \(U(t) = e^{-i H t/\Hbar}\) using these eigenvectors.

} % problem
%
\makeanswer{problem:qmIexamPractice2008Dec:4}{
%
The eigenvalue part of this problem is probably easier to do in matrix form.  Let
%
\begin{equation}\label{eqn:qmIexamPractice2008Dec:2:20}
\begin{aligned}
\ket{a} &=
\begin{bmatrix}
1 \\
0
\end{bmatrix} \\
\ket{b} &=
\begin{bmatrix}
0 \\
1
\end{bmatrix}.
\end{aligned}
\end{equation}

Our Hamiltonian is then
\begin{equation}\label{eqn:qmIexamPractice2008Dec:2:30}
H = \frac{\Hbar}{2}
\begin{bmatrix}
-\Delta & i \Omega \\
-i \Omega & \Delta
\end{bmatrix}.
\end{equation}

Computing \(\det{H - \lambda I} = 0\), we get
%
\begin{equation}\label{eqn:qmIexamPractice2008Dec:2:40}
\lambda = \pm \frac{\Hbar}{2} \sqrt{ \Delta^2 + \Omega^2 }.
\end{equation}

Let \(\delta = \sqrt{ \Delta^2 + \Omega^2 }\).  Our normalized eigenvectors are found to be
%
\begin{equation}\label{eqn:qmIexamPractice2008Dec:2:50}
\ket{\pm} = \inv{\sqrt{ 2 \delta (\delta \pm \Delta)} }
\begin{bmatrix}
i \Omega \\
\Delta \pm \delta
\end{bmatrix}.
\end{equation}

In terms of \(\ket{a}\) and \(\ket{b}\), we then have
%
\begin{equation}\label{eqn:qmIexamPractice2008Dec:2:60}
\ket{\pm} = \inv{\sqrt{ 2 \delta (\delta \pm \Delta)} }
\left(
i \Omega \ket{a}
+ (\Delta \pm \delta) \ket{b} \right).
\end{equation}

Note that our Hamiltonian has a simple form in this basis.  That is
%
\begin{equation}\label{eqn:qmIexamPractice2008Dec:2:70}
H = \frac{\delta \Hbar}{2} (\ket{+}\bra{+} - \ket{-}\bra{-} )
\end{equation}

Observe that once we do the diagonalization, we have a Hamiltonian that appears to have the form of a scaled projector for an open Stern-Gerlach apparatus.

Observe that the diagonalized Hamiltonian operator makes the time evolution operator's form also simple, which is, by inspection
%
\begin{equation}\label{eqn:qmIexamPractice2008Dec:2:80}
U(t) =
e^{-i t \frac{\delta}{2}} \ket{+}\bra{+}
+ e^{i t \frac{\delta}{2}} \ket{-}\bra{-}.
\end{equation}

Since we are asked for this in terms of \(\ket{a}\), and \(\ket{b}\), the projectors \(\ket{\pm}\bra{\pm}\) are required.  These are
%
\begin{equation}\label{eqn:qmIexamPractice2008Dec:690}
\begin{aligned}
\ket{\pm}\bra{\pm}
&= \inv{2 \delta (\delta \pm \Delta)}
\Bigl( i \Omega \ket{a} + (\Delta \pm \delta) \ket{b} \Bigr)
\Bigl( -i \Omega \bra{a} + (\Delta \pm \delta) \bra{b} \Bigr) \\
\end{aligned}
\end{equation}
%
\begin{equation}\label{eqn:qmIexamPractice2008Dec:2:90}
\ket{\pm}\bra{\pm}
= \inv{2 \delta (\delta \pm \Delta)}
\Bigl(
\Omega^2 \ket{a}\bra{a}
+(\delta \pm \delta)^2 \ket{b}\bra{b}
+i \Omega (\Delta \pm \delta) (
\ket{a}\bra{b}
-\ket{b}\bra{a}
)
\Bigr)
\end{equation}

Substitution into \eqnref{eqn:qmIexamPractice2008Dec:2:80} and a fair amount of algebra leads to
\begin{equation}\label{eqn:qmIexamPractice2008Dec:2:100}
\begin{aligned}
U(t) &= \cos(\delta t/2) \Bigl( \ket{a}\bra{a} + \ket{b}\bra{b} \Bigr) \\
&\quad
+ i \frac{\Omega}{\delta} \sin(\delta t/2) \Biglr{
\ket{a}\bra{a} - \ket{b}\bra{b}
-i (\ket{a}\bra{b} - \ket{b}\bra{a} )
}.
\end{aligned}
\end{equation}
Note that while a big cumbersome, we can also verify that we can recover the original Hamiltonian from \eqnref{eqn:qmIexamPractice2008Dec:2:70} and \eqnref{eqn:qmIexamPractice2008Dec:2:90}.

\paragraph{Q: (b)}
Suppose that the initial state of the system at time \(t = 0\) is \(\ket{\phi(0)}= \ket{b}\).  Find an expression for the state at some later time \(t > 0\), \(\ket{\phi(t)}\).
\paragraph{A:}
Most of the work is already done.  Computation of \(\ket{\phi(t)} = U(t) \ket{\phi(0)}\) follows from \eqnref{eqn:qmIexamPractice2008Dec:2:100}
\begin{equation}\label{eqn:qmIexamPractice2008Dec:2:110}
\ket{\phi(t)} =
\cos(\delta t/2) \ket{b}
- i \frac{\Omega}{\delta} \sin(\delta t/2) \Bigl(
\ket{b} +i \ket{a}
\Bigr).
\end{equation}

\paragraph{Q: (c)}

Suppose that an observable, specified by the operator \(X = \ket{a}\bra{b} + \ket{b}\bra{a}\), is measured for this system.  What is the probability that, at time \(t\), the result \(1\) is obtained?  Plot this probability as a function of time, showing the maximum and minimum values of the function, and the corresponding values of \(t\).

\paragraph{A:}

The language of questions like these attempt to bring some physics into the mathematics.  The phrase ``the result \(1\) is obtained'', is really a statement that the operator \(X\), after measurement is found to have the eigenstate with numeric value 1.

We can calculate the eigenvectors for this operator easily enough and find them to be \(\pm 1\).  For the positive eigenvalue we can also compute the eigenstate to be
%
\begin{equation}\label{eqn:qmIexamPractice2008Dec:2:120}
\ket{X+} = \inv{\sqrt{2}} \Bigl( \ket{a} + \ket{b} \Bigr).
\end{equation}

The question of what the probability for this measurement is then really a question asking for the computation of the amplitude
%
\begin{equation}\label{eqn:qmIexamPractice2008Dec:2:130}
\Abs{
\inv{\sqrt{2}}
\braket{
 (a + b)}{\phi(t)}}^2
\end{equation}

From \eqnref{eqn:qmIexamPractice2008Dec:2:110} we find this probability to be
%
\begin{equation}\label{eqn:qmIexamPractice2008Dec:710}
\begin{aligned}
\Abs{
\inv{\sqrt{2}}
\braket{
 (a + b)}{\phi(t)}}^2
&=
\inv{2} \left(
\left(\cos(\delta t/2) + \frac{\Omega}{\delta} \sin(\delta t/2)\right)^2
+ \frac{ \Omega^2 \sin^2(\delta t/2)}{\delta^2}
\right) \\
&=
\inv{4} \left( 1 + 3 \frac{\Omega^2}{\delta^2} + \frac{\Delta^2}{\delta^2} \cos (\delta t) + 2 \frac{ \Omega}{\delta} \sin(\delta t) \right)
\end{aligned}
\end{equation}

We have a simple superposition of two sinusoids out of phase, periodic with period \(2 \pi/\delta\).  I had attempted a rough sketch of this on paper, but will not bother scanning it here or describing it further.

\paragraph{Q: (d)}

Suppose an experimenter has control over the values of the parameters \(\Delta\) and \(\Omega\).  Explain how she might prepare the state \((\ket{a} + \ket{b})/\sqrt{2}\).

\paragraph{A:}

For this part of the question I was not sure what approach to take.  I thought perhaps this linear combination of states could be made to equal one of the energy eigenstates, and if one could prepare the system in that state, then for certain values of \(\delta\) and \(\Delta\) one would then have this desired state.

To get there I note that we can express the states \(\ket{a}\), and \(\ket{b}\) in terms of the eigenstates by inverting
%
\begin{equation}\label{eqn:qmIexamPractice2008Dec:2:150}
\begin{bmatrix}
\ket{+} \\
\ket{-} \\
\end{bmatrix}
=\inv{\sqrt{2\delta}}
\begin{bmatrix}
\frac{i \Omega}{\sqrt{\delta + \Delta}} & \sqrt{\delta + \Delta} \\
\frac{i \Omega}{\sqrt{\delta - \Delta}} & -\sqrt{\delta - \Delta}
\end{bmatrix}
\begin{bmatrix}
\ket{a} \\
\ket{b} \\
\end{bmatrix}.
\end{equation}

Skipping all the algebra one finds
%
\begin{equation}\label{eqn:qmIexamPractice2008Dec:2:160}
\begin{bmatrix}
\ket{a} \\
\ket{b} \\
\end{bmatrix}
=
\begin{bmatrix}
-i\sqrt{\delta - \Delta} & -i\sqrt{\delta + \Delta} \\
\frac{\Omega}{\sqrt{\delta - \Delta}} &
-\frac{\Omega}{\sqrt{\delta + \Delta}}
\end{bmatrix}
\begin{bmatrix}
\ket{+} \\
\ket{-} \\
\end{bmatrix}.
\end{equation}

Unfortunately, this does not seem helpful.  I find
%
\begin{equation}\label{eqn:qmIexamPractice2008Dec:2:170}
\inv{\sqrt{2}} ( \ket{a} + \ket{b} ) =
\frac{\ket{+}}{\sqrt{\delta - \Delta}}( \Omega - i (\delta - \Delta) )
-\frac{\ket{-}}{\sqrt{\delta + \Delta}}( \Omega + i (\delta + \Delta) )
\end{equation}

There is no obvious way to pick \(\Omega\) and \(\Delta\) to leave just \(\ket{+}\) or \(\ket{-}\).  When I did this on paper originally I got a different answer for this sum, but looking at it now, I can not see how I managed to get that answer (it had no factors of \(i\) in the result as the one above does).

\paragraph{A physical system for this Hamiltonian}

I wondered what physical system such a Hamiltonian would correspond to, and noted that this bore some similarity to the up vs. down states of the Ammonia atom as discussed in \citep{feynman1963flp}.  In that text the Hamiltonian is reasoned to have the form
%
\begin{equation}\label{eqn:qmIexamPractice2008Dec:2:180}
H = E_0 ( {\lvert {b} \rangle}{\langle {b} \rvert}+ {\lvert {a} \rangle}{\langle {a} \rvert})- A( {\lvert {a} \rangle}{\langle {b} \rvert}+ {\lvert {b} \rangle}{\langle {a} \rvert}).
\end{equation}

In Feynman's treatment, the Hamiltonian is just specified by giving values to \(H_{ij}\), but the expression can easily seen to be equivalent.  While these do not look equivalent on the surface, they both have the same diagonalization, which allows us to give a physical interpretation to this sort of problem (one which is recurrant in the old QMI exams).
} % answer
