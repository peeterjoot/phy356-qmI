%
% Copyright � 2012 Peeter Joot.  All Rights Reserved.
% Licenced as described in the file LICENSE under the root directory of this GIT repository.
%

%\chapter{A problem on spherical harmonics}
\index{spherical harmonics}
%\section{A problem on spherical harmonics (2010 final exam reflection)}
\label{chap:qmIexam2010}
%\blogpage{http://sites.google.com/site/peeterjoot/math2010/qmIexam2010.pdf}
%\date{Jan 9, 2011}
%
\makeproblem{A problem of spherical harmonics (2010 PHY356 final exam)}{problem:qmIexam2010:1}{
%\subsection{Motivation}
\index{spherical harmonics}
One of the PHY356 exam questions from the final I recall screwing up on, and figuring it out after the fact on the drive home.  The question actually clarified a difficulty I had, but unfortunately I had not had the good luck to perform such a question, to help figure this out before the exam.

From what I recall the question provided an initial state, with some degeneracy in \(m\), perhaps of the following form
\begin{equation}\label{eqn:qmIexam2010:10}
\ket{\phi(0)} =
\sqrt{\frac{1}{7}} \ket{ 12 }
+\sqrt{\frac{2}{7}} \ket{ 10 }
+\sqrt{\frac{4}{7}} \ket{ 20 },
\end{equation}
and a Hamiltonian of the form
\begin{equation}\label{eqn:qmIexam2010:20}
H = \alpha L_z
\end{equation}
} % problem

%From what I recall of the problem, I am going to reattempt it here now.
\makeanswer{problem:qmIexam2010:1}{
\paragraph{Evolved state}
\index{evolved state}
One part of the question was to calculate the evolved state.  Application of the time evolution operator gives us
\begin{equation}\label{eqn:qmIexam2010:30}
\ket{\phi(t)} =
e^{-i \alpha L_z t/\Hbar} \left(
\sqrt{\frac{1}{7}} \ket{ 12 }
+\sqrt{\frac{2}{7}} \ket{ 10 }
+\sqrt{\frac{4}{7}} \ket{ 20 } \right).
\end{equation}
%
Now we note that \(L_z \ket{12} = 2 \Hbar \ket{12}\), and \(L_z \ket{ l 0} = 0 \ket{l 0}\), so the exponentials reduce this nicely to just
\begin{equation}\label{eqn:qmIexam2010:40}
\ket{\phi(t)} =
\sqrt{\frac{1}{7}} e^{ -2 i \alpha t } \ket{ 12 }
+\sqrt{\frac{2}{7}} \ket{ 10 }
+\sqrt{\frac{4}{7}} \ket{ 20 }.
\end{equation}
\paragraph{Probabilities for \(L_z\) measurement outcomes}
%
I believe we were also asked what the probabilities for the outcomes of a measurement of \(L_z\) at this time would be.  Here is one place that I think that I messed up, and it is really a translation error, attempting to get from the English description of the problem to the math description of the same.  I had trouble with this process a few times in the problems, and managed to blunder through use of language like ``measure'', and ``outcome'', but do not think I really understood how these were used properly.

What are the outcomes that we measure?  We measure operators, but the result of a measurement is the eigenvalue associated with the operator.  What are the eigenvalues of the \(L_z\) operator?  These are the \(m \Hbar\) values, from the operation \(L_z \ket{l m} = m \Hbar \ket{l m}\).  So, given this initial state, there are really two outcomes that are possible, since we have two distinct eigenvalues.  These are \(2 \Hbar\) and \(0\) for \(m = 2\), and \(m= 0\) respectively.

A measurement of the ``outcome'' \(2 \Hbar\), will be the probability associated with the amplitude \(\braket{ 1 2 }{\phi(t)}\) (ie: the absolute square of this value).  That is
%
\begin{equation}\label{eqn:qmIexam2010:50}
\Abs{ \braket{ 1 2 }{\phi(t) } }^2 = \frac{1}{7}.
\end{equation}
%
Now, the only other outcome for a measurement of \(L_z\) for this state is a measurement of \(0 \Hbar\), and the probability of this is then just \(1 - \frac{1}{7} = \frac{6}{7}\).  On the exam, I think I listed probabilities for three outcomes, with values \(\frac{1}{7}, \frac{2}{7}, \frac{4}{7}\) respectively, but in retrospect that seems blatantly wrong.
%
\paragraph{Probabilities for \(\BL^2\) measurement outcomes}
%
What are the probabilities for the outcomes for a measurement of \(\BL^2\) after this?  The first question is really what are the outcomes.  That is really a question of what are the possible eigenvalues of \(\BL^2\) that can be measured at this point.  Recall that we have
%
\begin{equation}\label{eqn:qmIexam2010:60}
\BL^2 \ket{l m} = \Hbar^2 l (l + 1) \ket{l m}
\end{equation}
%
So for a state that has only \(l=1,2\) contributions before the measurement, the eigenvalues that can be observed for the \(\BL^2\) operator are respectively \(2 \Hbar^2\) and \(6 \Hbar^2\) respectively.

For the \(l=2\) case, our probability is \(4/7\), leaving \(3/7\) as the probability for measurement of the \(l=1\) (\(2 \Hbar^2\)) eigenvalue.  We can compute this two ways, and it seems worthwhile to consider both.  This first method makes use of the fact that the \(L_z\) operator leaves the state vector intact, but it also seems like a bit of a cheat.  Consider instead two possible results of measurement after the \(L_z\) observation.  When an \(L_z\) measurement of \(0 \Hbar\) is performed our state will be left with only the \(m=0\) kets.  That is
%
\begin{equation}\label{eqn:qmIexam2010:70}
\ket{\psi_a} = \inv{\sqrt{3}} \left( \ket{10} + \sqrt{2} \ket{20} \right),
\end{equation}
%
whereas, when a \(2 \Hbar\) measurement of \(L_z\) is performed our state would then only have the \(m=2\) contribution, and would be
\begin{equation}\label{eqn:qmIexam2010:80}
\ket{\psi_b} = e^{-2 i \alpha t} \ket{12 }.
\end{equation}
%
We have two possible ways of measuring the \(2 \Hbar^2\) eigenvalue for \(\BL^2\).  One is when our state was \(\ket{\psi_a}\) (, and the resulting state has a \(\ket{10}\) component, and the other is after the \(m=2\) measurement, where our state is left with a \(\ket{12}\) component.

The resulting probability is then a conditional probability result
\begin{equation}\label{eqn:qmIexam2010:n}
\frac{6}{7} \Abs{ \braket{10}{\psi_a} }^2 + \frac{1}{7} \Abs{ \braket{12 }{\psi_b}}^2 = \frac{3}{7}
\end{equation}
%
The result is the same, as expected, but this is likely a more convincing argument.

%I wasted time here trying to recall the spin rotation relations for \(L_z\).
%I believe they weare not required.
} % answer
