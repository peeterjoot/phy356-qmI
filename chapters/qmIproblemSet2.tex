%
% Copyright � 2012 Peeter Joot.  All Rights Reserved.
% Licenced as described in the file LICENSE under the root directory of this GIT repository.
%

%\chapter{PHY356 Problem Set II}
\label{chap:qmIproblemSet2}
%\blogpage{http://sites.google.com/site/peeterjoot/math2010/qmIproblemSet2.pdf}
%\date{Oct 23, 2010}
%
\makeproblem{ps II}{problem:qmIproblemSet2:1}{
%
A particle of mass \(m\) is free to move along the x-direction such that \(V(X)=0\). Express the time evolution operator \(U(t,t_0)\) defined by Eq. (2.166) using the momentum eigenstates \(\ket{p}\) with delta-function normalization. Find \(\bra{x} U(t,t_0) \ket{x'}\),  where \(\ket{x}\) and \(\ket{x'}\) are position eigenstates.  What is the physical meaning of this expression?

} % problem
%
\makeanswer{problem:qmIproblemSet2:1}{
\index{matrix element!momentum}
%
We can expand the time evolution operator in series
%
\begin{equation}\label{eqn:qmIproblemSet2:5020}
\begin{aligned}
U(t,t_0)
&= e^{-i H(t-t_0)/\Hbar} \\
&= e^{ -i P^2 (t-t_0)/ 2m \Hbar } \\
&= 1 + \sum_{k=1}^\infty \inv{k!} \left( -i \frac{P^2 (t-t_0)}{2m \Hbar} \right)^k.
\end{aligned}
\end{equation}
%
We can now evaluate the momentum matrix element \(\bra{p} U(t,t_0) \ket{p'}\), which will essentially require the value of \(\bra{p} P^{2k} \ket{p'}\).  That is
%
\begin{equation}\label{eqn:qmIproblemSet2:5040}
\begin{aligned}
\bra{p} P^{2k} \ket{p'}
&= \bra{p} P^{2k-1} P \ket{p'} \\
&= \bra{p} P^{2k-1} \ket{p'} p' \\
&= \cdots \\
&= \braket{p}{p'} (p')^{2k}.
\end{aligned}
\end{equation}
%
The momentum matrix element is therefore reduced to
%
\begin{equation}\label{eqn:qmIproblemSet2:1}
\bra{p} U(t,t_0) \ket{p'}
=
\braket{p}{p'} \exp\left( -i \frac{p^2 (t-t_0)}{2m \Hbar} \right)
= \delta(p-p') \exp\left( -i \frac{p^2 (t-t_0)}{2m \Hbar} \right)
\end{equation}
%
\paragraph{Position matrix element}
\index{matrix element!position}
For the position matrix element we have a similar sum
\begin{equation}\label{eqn:qmIproblemSet2:5060}
\bra{x} U(t,t_0) \ket{x'}
=
\braket{x}{x'}
+ \sum_{k=1}^\infty \inv{k!} \bra{x} \left( -i \frac{P^2 (t-t_0)}{2m \Hbar} \right)^k \ket{x'},
\end{equation}
%
and require \(\bra{x} P^{2k} \ket{x'}\) to continue.  That is
%
\begin{equation}\label{eqn:qmIproblemSet2:5080}
\begin{aligned}
\bra{x} P^{2k} \ket{x'}
&=
\int dx''\bra{x} P^{2k-1} \ket{x''}\bra{x''} P \ket{x'} \\
&=
\int dx''\bra{x} P^{2k-1} \ket{x''} \delta(x''-x') (-i\Hbar) \frac{d}{dx'} \\
&=
\bra{x} P^{2k-1} \ket{x'} (-i\Hbar) \frac{d}{dx'} \\
&= \cdots \\
&= \braket{x}{x'} \left( (-i\Hbar) \frac{d}{dx'} \right)^{2k}
\end{aligned}
\end{equation}
%
Our position matrix element is therefore the differential operator
%
\begin{equation}\label{eqn:qmIproblemSet2:10}
\bra{x} U(t,t_0) \ket{x'}
=
\braket{x}{x'} \exp\left( \frac{i (t-t_0)\Hbar}{2m} \frac{d^2}{d{x'}^2} \right)
=\delta(x-x') \exp\left( \frac{i (t-t_0)\Hbar}{2m} \frac{d^2}{d{x'}^2} \right)
\end{equation}
%
\paragraph{Physical interpretation of the position matrix element operator}
%
Finally, we need to determine the physical meaning of such a matrix element operator.

With the delta function that this matrix element operator includes it really only takes on a meaning with a convolution integral.  The simplest such integral would be
%
\begin{equation}\label{eqn:qmIproblemSet2:5100}
\begin{aligned}
\int dx' \bra{x} U \ket{x'} \braket{x'}{\phi_0}
&=
\bra{x} U \ket{\phi_0} \\
&=
\braket{x}{\phi(t)} \\
&=
\phi(x,t),
\end{aligned}
\end{equation}
%
or
\begin{equation}\label{eqn:qmIproblemSet2:5120}
\phi(x,t) = \int dx' \bra{x} U \ket{x'} \phi(x',0)
\end{equation}
%
The LHS has a physical meaning, and in the absolute square
%
\begin{equation}\label{eqn:qmIproblemSet2:5000}
\int_{x_0}^{x_0+ \Delta x} \Abs{\phi(x,t)}^2 dx,
\end{equation}
%
provides the probability that the particle will be found in the region \([x_0, x_0+ \Delta x]\).

If we ignore the absolute square requirement and think of the (presumed normalized) wave function \(\phi(x,t)\) more loosely as representing a probability directly, then we can in turn give a meaning to the matrix element \(\bra{x} U \ket{x'}\) for the time evolution operator.  This provides an operator valued weighting function that provides us with the probability that a particle initially at position \(x'\) will be at position \(x\) at time \(t\).  This probability is indirect since we need to absolute square and sum over a finite interval to obtain the probability of finding the particle in that interval.

Observe that the integral on the RHS of \eqnref{eqn:qmIproblemSet2:5000} is a summation over all \(x'\), so we can think of this as adding the probabilities that the particle was at each point to arrive at the total probability for finding it at the new location \(x\).  The time evolution operator matrix element provides the weighting in this conditional probability.

In \eqnref{eqn:qmIproblemSet2:10} we found that the time evolution operators matrix element is differential operator in the position representation.  In the general case this means that this probability weighting is not just numeric since the operation of the matrix element initial time wave function can produce wave functions for additional states.  In some special cases, we may find that this weighting is strictly numeric, and one such example would be the Gaussian wave packet \(\phi(x',0) = e^{-a{x'}^2}\).  Application of the differential operations would then produce polynomial weighted multiples of the original Gaussian.  In this special case we would be able to write
%
\begin{equation}\label{eqn:qmIproblemSet2:5140}
\phi(x,t) = \int dx' \bra{x} U \ket{x'} \phi(x',0) = \int dx' K(x,x',t) \phi(x',0)
\end{equation}
%
Where \(K(x,x',t)\) is a polynomial valued function (and is in fact another exponential), and now just provides a numerical weighting for the conditional probability for the particle to move from \(x'\) to \(x\) in time \(t\).  In \citep{liboff2003iqm}, this \(K(x,x',t)\) is called the Propagator function.  It is perhaps justifiable to also call our similar operator valued matrix element a Propagator.

%Based on this, I would be inclined to state that the position matrix element of the time evolution operator \(\bra{x} U \ket{x'}\) represents something akin to an operator form of probability amplitude for a particle to travel between two points.
%Consider two additional contexts where this matrix element arises.  We can find this matrix element by expanding a normalized inner product for the time evolved state
%\begin{align*}
%1
%&= \braket{\phi(t)}{\phi(t)} \\
%&= \bra{\phi(t)} U \ket{\phi_0} \\
%&= \iint dx dx' \bra{\phi(t)} \ket{x} \bra{x} U \ket{x'}\bra{x'} \ket{\phi_0} \\
%&= \iint dx dx' \phi^\conj(x,t) \bra{x} U \ket{x'} \phi(x,0).
%\end{align*}
%
%This matrix element also arises in the expectation value of the time evolution operator with respect to the initial time state.  That is
%
%\begin{align*}
%\expectation{U}
%&= \bra{\phi_0} U \ket{\phi_0} \\
%&= \braket{\phi_0}{\phi} = \int dx \phi^\conj(x,0) \phi(x,t) \\
%&= \iint dx dx' \bra{\phi_0} \ket{x} \bra{x} U \ket{x'}\bra{x'} \ket{\phi_0} \\
%&= \iint dx dx' \phi^\conj(x,0) \bra{x} U \ket{x'} \phi(x,0)
%\end{align*}
%
%\subsection{My grade}
%
%I got full marks on this assignment.  There is apparently another way to do part of the first question on the position representation, and I was instructed by the TA to see the posted solution, which is not yet available.
} % answer
