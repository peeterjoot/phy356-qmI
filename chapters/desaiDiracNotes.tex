%
% Copyright � 2012 Peeter Joot.  All Rights Reserved.
% Licenced as described in the file LICENSE under the root directory of this GIT repository.
%

%\chapter{Dirac Notation Ponderings}
\label{chap:desaiDiracNotes}
%\blogpage{http://sites.google.com/site/peeterjoot/math2010/desaiDiracNotes.pdf}
%\date{July 23, 2010}

%\section{Motivation}
I have got the textbook \citep{desai2009quantum} now for the QM I course I will be taking in the fall, and have started some light perusing.  Starting things off is the Dirac bra ket notation.  Some aspects of that notation, or the explanation in the text, are not quite obvious to me so here I try to make sense of things.

There are a pair of relations given to define the Dirac adjoint.  These are 1.26 and 1.27 respectively:
\index{Dirac!adjoint}
%
\begin{equation}\label{eqn:desaiDiracNotes:21}
\begin{aligned}
\left( A \ket{\alpha} \right)^\conj &= \bra{\alpha} A^\dagger \\
{\bra{\beta} A \ket{\alpha}}^\conj &= \bra{\alpha} A^\dagger \ket{\beta}
\end{aligned}
\end{equation}

Is there some redundancy to these definitions.  Namely is 1.27 a consequence of 1.26?

Since the ket was defined as the conjugate of the bra, we can probably rewrite 1.26 as
\index{bra}
\index{ket}
%
\begin{equation}\label{eqn:desaiDiracNotes:41}
\bra{\alpha} A^\conj = \bra{\alpha} A^\dagger
\end{equation}

The operational word here is "probably".  This seems somewhat dubious.  For example with the identity operator this would mean
\index{identity}
%
\begin{equation}\label{eqn:desaiDiracNotes:61}
\left( \ket{\alpha} \right)^\conj = \bra{\alpha},
\end{equation}

and I am unsure that this makes sense.  If one assumes that it does, then one can find that 1.26 implies 1.27, as follows.

Left ``multiplication'', by the ket \(\ket{\beta}\) gives
%
\begin{equation}\label{eqn:desaiDiracNotes:81}
\begin{aligned}
(\bra{\alpha} A^\dagger) \ket{\beta} &= (\bra{\alpha} A^\conj) \ket{\beta} \\
&= {\bra{\beta} (A \ket{\alpha})}^\conj
\end{aligned}
\end{equation}

Again the dubious operation \(\bra{\alpha}^\conj = \ket{\alpha}\) has been employed implicitly.

Also note that I have added and retained parenthesis to retain the operational direction.  Is that operational direction not important?  For example, given an operator like \(p = -i \Hbar \partial_x\), it makes a big difference whether the operator operates to the left or to the right.  In the text, this last relation is equation 1.27 once the parens are dropped, so it does appear that 1.27 is a consequence of 1.26.  This also then seems to imply that in a bra operator ket sandwich, the operator implicitly operates on the ket (to the right), while an adjoint operator implicitly operates on the bra (to the left).

Let us compare this to the simpler and more pedestrian notation found in an old fashioned book like Bohm's \citep{bohm1989qt}.  His expectation values explicitly use an integral definition, and his adjoint definition is very explicit about order of operations.  Namely
\index{expectation}
%
\begin{equation}\label{eqn:desaiDiracNotes:1}
\int \phi^\conj (A \psi)
\equiv \int \psi (A^\dagger \phi^\conj)
\end{equation}

Starting with a concrete definition like this seems a bit easier.  Suppose we also define the bra ket sandwich based on the integral as follows
%
\begin{equation}\label{eqn:desaiDiracNotes:101}
\begin{aligned}
\bra{\phi} A \ket{\psi}
&\equiv \bra{\phi} (A \ket{\psi}) \\
&\equiv \int \phi^\conj (A \psi) \\
\end{aligned}
\end{equation}

Now, we can rewrite \eqnref{eqn:desaiDiracNotes:1}, as
%
\begin{equation}\label{eqn:desaiDiracNotes:121}
\begin{aligned}
\int \phi^\conj (A \psi)   &\equiv \int \psi (A^\dagger \phi^\conj) \\
&\implies \\
\bra{\phi} (A \ket{\psi})  &= \bra{\psi^\conj} ( A^\dagger \ket{\phi^\conj} ) \\
&\implies \\
\left(\bra{\phi} (A \ket{\psi}) \right)^\conj  &= ( \bra{\phi} A^\dagger ) \ket{\psi}
\end{aligned}
\end{equation}

When starting off with the integral we see the notational requirement for non-adjoint operators to operate implicitly to the right, and the adjoint operators to operate implicitly to the left.  With that notation requirement we can drop the parens and recover 1.27.

A couple clarification goals are now complete.  The first is seeing how equation 1.26 in the text implies 1.27 (provided the plain old conjugation of a bra creates a ket).  We also have reconciled the Dirac notation with the familiar integral inner product notation, and seen two different ways that clarify the implicit operator directionality in the bra operator ket sandwiches.

Update.  Vatche, my professor for the course, also had trouble with 1.26.  He feels it ought to be
%
\begin{equation}\label{eqn:desaiDiracNotes:141}
\left( A \ket{\alpha} \right)^\dagger = \bra{\alpha} A^\dagger.
\end{equation}

Matrix notation was used to demonstrate this, since conjugation only changes the element values and does not transpose the matrix.  Use of the identity operator makes his point particularly clear.

