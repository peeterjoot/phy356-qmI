%
% Copyright © 2012 Peeter Joot.  All Rights Reserved.
% Licenced as described in the file LICENSE under the root directory of this GIT repository.
%
%\QMlecture{8 --- Making Sense of Quantum Mechanics --- November 2, 2010}
%
\subsection{Discussion}
%
\paragraph{Desai:} ``Quantum Theory is a linear theory ...''
%
We can discuss SHM without using sines and cosines or complex exponentials, say, only using polynomials, but it would be HARD to do so, and much more work.  We want the framework of Hilbert space, linear operators and all the rest to make our life easier.
%
\paragraph{Dirac:} ``All the same the Mathematics is only a tool and one should learn to hold the physical ideas on one's mind without reference to the mathematical form''
%
You have to be able to understand the concepts and apply the concepts as well as the mathematics.
%
\paragraph{Deyirmenjian:} ``Think before you compute.''
%
Joke: With his name included it is the 3Ds.  There is a lot of information included in the question, so read it carefully.
%
\paragraph{Q:} The equation \(A \ket{a_n} = a_n \ket{a_n}\) for operator \(A\), eigenvalue \(a_n\), \(n = 1,2\) and eigenvector \(\ket{a_n}\) that is identified by the eigenvalue \(a_n\) says that
%
\begin{itemize}
\item (a) measuring the physical quantity associated with \(A\) gives result \(a_n\)
\item (b) \(A\) acting on the state \(\ket{a_n}\) gives outcome \(a_n\)
\item (c) the possible outcomes of measuring the physical quantity associated with \(A\) are the eigenvalues \(a_n\)
\item (d) Quantum mechanics is hard.
\end{itemize}

\(\ket{a_n}\) is a vector in a vector space or Hilbert space identified by some quantum number \(a_n, n \in 1,2, \cdots\).

The \(a_n\) values could be expressions.  Example, Angular momentum is describe by states \(\ket{lm}, l = 0,1,2,\cdots\) and \(m = 0, \pm 1, \pm 2\)

Recall that the problem is
%
\begin{equation}\label{eqn:PHY356HNov2:10}
\begin{aligned}
\BL^2 \ket{lm} &= l(l+1) \Hbar^2 \ket{lm} \\
L_z \ket{lm} &= m \Hbar \ket{lm}
\end{aligned}
\end{equation}
%
We have respectively eigenvalues \(l(l+1)\Hbar^2\), and \(m \Hbar\).
%
\paragraph{A:} Answer is (c).  \(a_n\) is not a measurement itself.  These represent possibilities.  Contrast this to classical mechanics where time evolution is given without probabilities
\index{measurement}
%
\begin{equation}\label{eqn:PHY356HNov2:20}
\begin{aligned}
\BF_{\text{net}} &= m \Ba \\
\Bx(0), \Bx'(0) &\implies \Bx(t), \Bx'(t)
\end{aligned}
\end{equation}
%
The eigenvalues are the possible outcomes, but we only know statistically that these are the possibilities.

(a),(b) are incorrect because we do not know what the initial state is, nor what the final outcome is.  We also can not say ``gives result \(a_n\)''.  That statement is too strong!
%
\paragraph{Q:} We would not say that \(A\) acts on pure state \(\ket{a_n}\), instead.  If the state of the system is \(\ket{\psi} = \ket{a_5}\), the probability of measuring outcome \(a_5\) is
\index{pure state}
\begin{itemize}
\item (a) \(a_5\)
\item (b) \(a_5^2\)
\item (c) \(\braket{a_5}{\psi} = \braket{a_5}{a_5} = 1\).
\item (d) \(\Abs{\braket{a_5}{\psi}}^2 = \Abs{\braket{a_5}{a_5}}^2 = \Abs{1}^2 = 1\).
\end{itemize}
%
\paragraph{A:} (d) The eigenformula equation does not say anything about any specific outcome.  We want to talk about probability amplitudes.  When the system is prepared in a particular pure eigenstate, then we have a guarantee that the probability of measuring that state is unity.  We would not say (c) because the probability amplitudes are the absolute square of the complex number \(\braket{a_n}{a_n}\).
\index{eigenstate}

The probability of outcome \(a_n\), given initial state \(\ket{\Psi}\) is \(\Abs{\braket{a_n}{\Psi}}^2\).

Wave function collapse:  When you make a measurement of the physical quantity associated with \(A\), then the state of the system will be the value \(\ket{a_5}\).  The state is not the number (eigenvalue) \(a_5\).

Example: SGZ.  With a ``spin-up'' measurement in the z-direction, the state of the system is \(\ket{z+}\).  The state before the measurement, by the magnet, was \(\ket{\Psi}\).  After the measurement, the state describing the system is \(\ket{\phi} = \ket{z+}\).  The measurement outcome is \(+\frac{\Hbar}{2}\) for the spin angular momentum along the z-direction.

FIXME: SGZ picture here.

There is an interaction between the magnet and the silver atoms coming out of the oven.  Before that interaction we have a state described by \(\ket{\Psi}\).  After the measurement, we have a new state \(\ket{\phi}\).  We call this the collapse of the wave function.  In a future course (QM interpretations) the language used and interpretations associated with this language can be discussed.
%
\paragraph{Q:} Express Hermitian operator \(A\) in terms of its eigenvectors.
\index{Hermitian operator}
\paragraph{Q:} The above question is vague because
%
\begin{itemize}
\item (a) The eigenvectors may form a discrete set.
\item (b) The eigenvectors may form a continuous set.
\item (c) The eigenvectors may not form a complete set.
\item (d) The eigenvectors are not given.
\end{itemize}
%
\paragraph{A:} None of the above.  A Hermitian operator is guaranteed to have a complete set of eigenvectors.  The operator may also be both discrete and continuous (example: the complete spin wave function).
%
\index{completeness}
\paragraph{discrete:}
%
\begin{equation}\label{eqn:lecture8makingSenseOfQM:1020}
\begin{aligned}
A &= A \BOne \\
&= A \left( \sum_n \ket{a_n} \bra{a_n} \right) \\
&= \sum_n (A \ket{a_n} )\bra{a_n} \\
&= \sum_n (a_n \ket{a_n}) \bra{a_n} \\
&= \sum_n a_n \ket{a_n} \bra{a_n}
\end{aligned}
\end{equation}
%
\paragraph{continuous:}
%
\begin{equation}\label{eqn:lecture8makingSenseOfQM:1040}
\begin{aligned}
A &= A \BOne \\
&= A \left( \int d\alpha \ket{\alpha} \bra{\alpha} \right) \\
&= \int d\alpha (A \ket{\alpha} )\bra{\alpha} \\
&= \int d\alpha (\alpha \ket{\alpha}) \bra{\alpha} \\
&= \int d\alpha \alpha \ket{\alpha} \bra{\alpha}
\end{aligned}
\end{equation}
%
An example is the position eigenstate \(\ket{x}\), eigenstate of the Hermitian operator \(X\).  \(\alpha\) is a label indicating the summation.
%
\paragraph{general case with both discrete and continuous:}
%
\begin{equation}\label{eqn:lecture8makingSenseOfQM:1060}
\begin{aligned}
A &= A \BOne \\
&= A \left( \sum_n \ket{a_n} \bra{a_n} + \int d\alpha \ket{\alpha} \bra{\alpha} \right) \\
&= \sum_n \left(A \ket{a_n} \right)\bra{a_n} + \int d\alpha \left(A \ket{\alpha} \right)\bra{\alpha} \\
&= \sum_n \left(a_n \ket{a_n}\right) \bra{a_n} + \int d\alpha \left(\alpha \ket{\alpha}\right) \bra{\alpha} \\
&= \sum_n a_n \ket{a_n} \bra{a_n} + \int d\alpha \alpha \ket{\alpha} \bra{\alpha}
\end{aligned}
\end{equation}
%
\paragraph{Problem Solving}
%
\begin{itemize}
\item MODEL -- Quantum, linear vector space
\item VISUALIZE -- Operators can have discrete, continuous or both discrete and continuous eigenvectors.
\item SOLVE -- Use the identity operator.
\item CHECK -- Does the above expression give \(A \ket{a_n} = a_n \ket{a_n}\).
\end{itemize}
%
\paragraph{Check}
%
\begin{equation}\label{eqn:lecture8makingSenseOfQM:1080}
\begin{aligned}
A \ket{a_m}
&= \sum_n a_n \ket{a_n} \braket{a_n}{a_m} + \int d\alpha \alpha \ket{\alpha} \braket{\alpha}{a_n} \\
&= \sum_n a_n \ket{a_n} \delta_{nm} \\
&= a_m \ket{a_m}
\end{aligned}
\end{equation}
%
What remains to be shown, used above, is that the continuous and discrete eigenvectors are orthonormal.  He has an example vector space, not yet discussed.
%
\paragraph{Q:} what is \(\bra{\Psi_1} A \ket{\Psi_1}\), where \(A\) is a Hermitian operator, and \(\ket{\Psi_1}\) is a general state.
%
\paragraph{A:} \(\bra{\Psi_1} A \ket{\Psi_1} =\) average outcome for \textunderline{many measurements} of the physical quantity associated with \(A\) such that the system is prepared in state \(\ket{\Psi_1}\) prior to each measurement.
%
\paragraph{Q:}  What if the preparation is \(\ket{\Psi_2}\).  This is not necessarily an eigenstate of \(A\), it is some linear combination of eigenstates.  It is a general state.
\paragraph{A:}  \(\bra{\Psi_2} A \ket{\Psi_2} = \) average of the physical quantity associated with \(A\), but the preparation is \(\ket{\Psi_2}\), not \(\ket{\Psi_1}\).
%
\paragraph{Q:}  What if our initial state is a little bit of \(\ket{\Psi_1}\), and a little bit of \(\ket{\Psi_2}\), and a little bit of \(\ket{\Psi_N}\).  ie: how to describe what comes out of the oven in the Stern-Gerlach experiment.  That spin is a statistical mixture.  We could understand this as only a statistical mix.  This is a physical relevant problem.
\paragraph{A:}  To describe that statistical situation we have the following.
%
\begin{equation}\label{eqn:PHY356HNov2:30}
\begin{aligned}
\expectation{A}_{\text{average}} = \sum_j w_j \bra{\Psi_j} A \ket{\Psi_j}
\end{aligned}
\end{equation}
%
We sum up all the expectation values modified by statistical weighting factors.  These \(w_j\)'s are statistical weighting factors for a preparation associated with \(\ket{\Psi_j}\), real numbers (that sum to unity).  Note that these states \(\ket{\Psi_j}\) are not necessarily orthonormal.

With insertion of the identity operator we have
%
\begin{equation}\label{eqn:lecture8makingSenseOfQM:1100}
\begin{aligned}
\expectation{A}_{\text{average}}
&= \sum_j w_j \bra{\Psi_j} \BOne A \ket{\Psi_j} \\
&= \sum_j w_j \bra{\Psi_j} \left( \sum_n \ket{a_n} \bra{a_n} \right) A \ket{\Psi_j} \\
&= \sum_j \sum_n w_j \braket{\Psi_j}{a_n} \bra{a_n} A \ket{\Psi_j} \\
&= \sum_j \sum_n w_j \bra{a_n} A \ket{\Psi_j} \braket{\Psi_j}{a_n}  \\
&= \sum_n \bra{a_n} A \left( \sum_j w_j \ket{\Psi_j} \bra{\Psi_j} \right) \ket{a_n}  \\
\end{aligned}
\end{equation}
%
This inner bit is called the density operator \(\rho\)
%
\begin{equation}\label{eqn:PHY356HNov2:40}
\begin{aligned}
\rho &\equiv \sum_j w_j \ket{\Psi_j} \bra{\Psi_j}
\end{aligned}
\end{equation}
%
Returning to the average we have
%
\begin{equation}\label{eqn:PHY356HNov2:50}
\begin{aligned}
\expectation{A}_{\text{average}} = \sum_n \bra{a_n} A \rho \ket{a_n} \equiv \tr(A \rho)
\end{aligned}
\end{equation}
%
The trace of an operator \(A\) is
%
\begin{equation}\label{eqn:PHY356HNov2:60}
\begin{aligned}
\tr(A) = \sum_j \bra{a_j} A \ket{a_j} = \sum_j A_{jj}
\end{aligned}
\end{equation}
%
\section{Projection operator}
\index{projection operator}
%Section 5.9, ?

Returning to the last lecture.  From chapter 1, we have
%
\begin{equation}\label{eqn:PHY356Foct26:1000}
\begin{aligned}
P_n = \ket{a_n} \bra{a_n}
\end{aligned}
\end{equation}
%
is called the \textAndIndex{projection operator}.  This is physically relevant.  This takes a general state and gives you the component of that state associated with that eigenvector.  Observe
\begin{equation}\label{eqn:PHY356Foct26:210}
\begin{aligned}
P_n \ket{\phi} =
\ket{a_n} \braket{a_n}{\phi}
=
\mathLabelBox{\braket{a_n}{\phi}}{coefficient}
\ket{a_n}
\end{aligned}
\end{equation}
%
\makeexample{Projection operator for the \(\ket{z+}\) state}{example:lecture8makingSenseOfQM:1}{
%
\begin{equation}\label{eqn:PHY356Foct26:211}
\begin{aligned}
P_{z+} = \ket{z+} \bra{z+}
\end{aligned}
\end{equation}
%
We see that the density operator
%
\begin{equation}\label{eqn:PHY356HNov2:40b}
\begin{aligned}
\rho &\equiv \sum_j w_j \ket{\Psi_j} \bra{\Psi_j},
\end{aligned}
\end{equation}
%
can be written in terms of the Projection operators
%
\begin{equation}\label{eqn:lecture8makingSenseOfQM:1120}
\begin{aligned}
\ket{\Psi_j} \bra{\Psi_j} = \text{Projection operator for state} \ket{\Psi_j}
\end{aligned}
\end{equation}
%
The projection operator is like a dot product, determining the quantity of a state that lines in the direction of another state.
} % example
%
\paragraph{Q:} What is the projection operator for spin-up along the z-direction.
\paragraph{A:}
%
\begin{equation}\label{eqn:PHY356HNov2:100}
\begin{aligned}
P_{z+} = \ket{z+}\bra{z+}
\end{aligned}
\end{equation}
%
Or in matrix form with
%
\begin{equation}\label{eqn:PHY356HNov2:105}
\begin{aligned}
\bra{z+} &=
\begin{bmatrix}
1 \\
0
\end{bmatrix} \\
\bra{z-} &=
\begin{bmatrix}
0 \\
1
\end{bmatrix},
\end{aligned}
\end{equation}
%
so
\begin{equation}\label{eqn:PHY356HNov2:110}
\begin{aligned}
P_{z+} = \ket{z+}\bra{z+} =
\begin{bmatrix}
1 \\
0
\end{bmatrix}
\begin{bmatrix}
1 & 0
\end{bmatrix}
=
\begin{bmatrix}
1 & 0 \\
0 & 0
\end{bmatrix}
\end{aligned}
\end{equation}
%
\makeexample{A harder problem.}{example:lecture8makingSenseOfQM:2}{

What is \(P_\chi\), where
%
\begin{equation}\label{eqn:PHY356HNov2:120}
\begin{aligned}
\ket{\chi} =
\begin{bmatrix}
c_1 \\
c_2
\end{bmatrix}
\end{aligned}
\end{equation}
%
Note: We want normalized states, with \(\braket{\chi}{\chi} = \Abs{c_1}^2 + \Abs{c_2}^2 = 1\).

%
\paragraph{A:}
%
\begin{equation}\label{eqn:PHY356HNov2:130}
\begin{aligned}
P_{\chi} = \ket{\chi}\bra{\chi} =
\begin{bmatrix}
c_1^\conj \\
c_2^\conj
\end{bmatrix}
\begin{bmatrix}
c_1 & c_2
\end{bmatrix}
=
\begin{bmatrix}
c_1^\conj c_1 & c_1^\conj c_2 \\
c_2^\conj c_1 & c_2^\conj c_2
\end{bmatrix}
\end{aligned}
\end{equation}
%
Observe that this has the proper form of a projection operator is that the square is itself
%
\begin{equation}\label{eqn:lecture8makingSenseOfQM:1140}
\begin{aligned}
(\ket{\chi}\bra{\chi}) (\ket{\chi}\bra{\chi})
&= \ket{\chi} (\braket{\chi}{\chi} )\bra{\chi} \\
&= \ket{\chi} \bra{\chi}
\end{aligned}
\end{equation}
%
} % example

\makeexample{Projection}{example:lecture8makingSenseOfQM:3}{

Show that \(P_{\chi} = a_0 \BOne + \Ba \cdot \Bsigma\), where \(\Ba = (a_x, a_y, a_z)\) and \(\Bsigma = (\sigma_x, \sigma_y, \sigma_z)\).
%
\paragraph{A:} See Section 5.9.  Note the following about computing \((\Bsigma \cdot \Ba)^2\).
%
\begin{equation}\label{eqn:lecture8makingSenseOfQM:1160}
\begin{aligned}
(\Bsigma \cdot \Ba)^2
&=
(a_x \sigma_x
+ a_y \sigma_y
+ a_z \sigma_z)
(a_x \sigma_x
+ a_y \sigma_y
+ a_z \sigma_z) \\
&=
a_x a_x \sigma_x \sigma_x
+a_x a_y \sigma_x \sigma_y
+a_x a_z \sigma_x \sigma_z \\
&\quad+a_y a_x \sigma_y \sigma_x
+a_y a_y \sigma_y \sigma_y
+a_y a_z \sigma_y \sigma_z \\
&\quad+a_z a_x \sigma_z \sigma_x
+a_z a_y \sigma_z \sigma_y
+a_z a_z \sigma_z \sigma_z \\
&= (a_x^2 + a_y^2 + a_z^2) I
+ a_x a_y ( \sigma_x \sigma_y + \sigma_y \sigma_x) \\
&\quad
+ a_y a_z ( \sigma_y \sigma_z + \sigma_z \sigma_y)
+ a_z a_x ( \sigma_z \sigma_x + \sigma_x \sigma_z) \\
&= \Abs{\Bx}^2 I
\end{aligned}
\end{equation}
%
So we have
\begin{equation}\label{eqn:PHY356HNov2:135}
\begin{aligned}
(\Bsigma \cdot \Ba)^2 = (\Ba \cdot \Ba) \BOne \equiv \Ba^2
\end{aligned}
\end{equation}
%
Where the matrix representations
\begin{equation}\label{eqn:PHY356HNov2:140}
\begin{aligned}
\sigma_x &\leftrightarrow \PauliX \\
\sigma_y &\leftrightarrow \PauliY \\
\sigma_z &\leftrightarrow \PauliZ,
\end{aligned}
\end{equation}
%
would be used to show that
\begin{equation}\label{eqn:PHY356HNov2:150}
\begin{aligned}
\sigma_x^2 = \sigma_y^2 = \sigma_z^2 = I
\end{aligned}
\end{equation}
%
and
%
\begin{equation}\label{eqn:PHY356HNov2:155}
\begin{aligned}
\sigma_x \sigma_y &= -\sigma_y \sigma_x \\
\sigma_y \sigma_z &= -\sigma_z \sigma_y \\
\sigma_z \sigma_x &= -\sigma_x \sigma_z
\end{aligned}
\end{equation}
%
} % example
