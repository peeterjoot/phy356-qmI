%
% Copyright © 2012 Peeter Joot.  All Rights Reserved.
% Licenced as described in the file LICENSE under the root directory of this GIT repository.
%
%\QMlecture{12 --- Rotations, Angular Momentum. --- December 6, 2010}

\section{Rotations (chapter 26)}
\index{rotation}

Why are we doing the math?  Because it applies to physical systems.  Slides of \href{http://www.almaden.ibm.com/vis/stm/images/stm.gif}{IBM's SEM quantum coral} and others shown and discussed.

PICTURE: Standard right handed coordinate system with point \((x,y,z)\). We would like to discuss how to represent this point in other coordinate systems, such as one with the \(x,y\) axes rotated to \(x',y'\) through an angle \(\phi\).

Our problem is to find in the rotated coordinate system from \((x,y,z)\) to \((x', y', z')\).

There is clearly a relationship between the representations.  That relationship between \(x', y', z'\) and \(x,y,z\) for a counter-clockwise rotation about the \(z\) axis is

\begin{equation}\label{eqn:PHY356FDec7:10}
\begin{aligned}
x' &= x \cos \phi - y \sin\phi \\
y' &= x \sin \phi + y \cos\phi \\
z' &= z
\end{aligned}
\end{equation}

Treat \((x,y,z)\) and \((x',y',z')\) like vectors and write

\begin{equation}\label{eqn:PHY356FDec7:20}
\begin{bmatrix}
x'  \\
y' \\
z'
\end{bmatrix}
=
\begin{bmatrix}
\cos \phi &- \sin\phi & 0 \\
\sin \phi & \cos\phi & 0 \\
0 & 0 & 1
\end{bmatrix}
\begin{bmatrix}
x  \\
y \\
z
\end{bmatrix}
\end{equation}

Or
\begin{equation}\label{eqn:PHY356FDec7:30}
\begin{bmatrix}
x'  \\
y' \\
z'
\end{bmatrix}
=
R_z(\phi)
\begin{bmatrix}
x  \\
y \\
z
\end{bmatrix}
\end{equation}

\paragraph{Q: Is \(R_z(\phi)\) a unitary operator?}

Definition \(U\) is unitary if \(U^\dagger U = \BOne\), where \(\BOne\) is the identity operator.  We take Hermitian conjugates, which in this case is just the transpose since all elements of the matrix are real, and multiply

\begin{equation}\label{eqn:lecture12rotationsAndAngularMomentum:240}
\begin{aligned}
(R_z(\phi))^\dagger R_z(\phi) &=
\begin{bmatrix}
\cos \phi & \sin\phi & 0 \\
-\sin \phi & \cos\phi & 0 \\
0 & 0 & 1
\end{bmatrix}
\begin{bmatrix}
\cos \phi &- \sin\phi & 0 \\
\sin \phi & \cos\phi & 0 \\
0 & 0 & 1
\end{bmatrix} \\
&=
\begin{bmatrix}
\cos^2 \phi + \sin^2\phi  & -\sin\phi \cos\phi  + \sin\phi \cos\phi  & 0 \\
-\cos\phi \sin\phi  + \cos\phi \sin\phi  & \cos^2\phi  + \sin^2 \phi & 0 \\
0 & 0 & 1 \\
\end{bmatrix} \\
&=
\begin{bmatrix}
1 & 0 & 0 \\
0 & 1 & 0 \\
0 & 0 & 1 \\
\end{bmatrix} \\
&= \BOne
\end{aligned}
\end{equation}

Apply the above to a vector \(\Bv = (v_x, v_y, v_z)\) and write \(\Bv' = (v_x', v_y', v_z')\).  These are related as

\begin{equation}\label{eqn:PHY356FDec7:40}
\Bv = R_z(\phi) \Bv
\end{equation}

Now we want to consider the infinitesimal case where we allow the rotation angle to get arbitrarily small.  Consider this specific \(z\) axis rotation case, and assume that \(\phi\) is very small.  Let \(\phi = \epsilon\) and write

\begin{equation}\label{eqn:PHY356FDec7:55}
\begin{aligned}
\Bv' &=
\begin{bmatrix}
v_x'  \\
v_y' \\
v_z'
\end{bmatrix}
=
R_z(\phi)
\begin{bmatrix}
v_x \\
v_y \\
v_z
\end{bmatrix}
=
\begin{bmatrix}
\cos \epsilon &- \sin\epsilon & 0 \\
\sin \epsilon & \cos\epsilon & 0 \\
0 & 0 & 1
\end{bmatrix}
\Bv \\
&\approx
\begin{bmatrix}
1 &- \epsilon & 0 \\
\epsilon & 1 & 0 \\
0 & 0 & 1
\end{bmatrix}
\Bv
=
\left(
\begin{bmatrix}
1 & 0 & 0 \\
0 & 1 & 0 \\
0 & 0 & 1
\end{bmatrix}
+
\begin{bmatrix}
0 &- \epsilon & 0 \\
\epsilon & 0 & 0 \\
0 & 0 & 0
\end{bmatrix}
\right)
\Bv
\end{aligned}
\end{equation}

Define
\begin{equation}\label{eqn:PHY356FDec7:60}
S_z = i \Hbar
\begin{bmatrix}
0 &- 1 & 0 \\
1 & 0 & 0 \\
0 & 0 & 0
\end{bmatrix}
\end{equation}

which is the generator of infinitesimal rotations about the \(z\) axis.

Our rotated coordinate vector becomes
\begin{equation}\label{eqn:lecture12rotationsAndAngularMomentum:260}
\begin{aligned}
\Bv' &=
\left(
\begin{bmatrix}
1 & 0 & 0 \\
0 & 1 & 0 \\
0 & 0 & 1
\end{bmatrix}
+
\frac{i \Hbar \epsilon}{i\Hbar}
\begin{bmatrix}
0 &- 1 & 0 \\
1 & 0 & 0 \\
0 & 0 & 0
\end{bmatrix}
\right)
\Bv \\
&=
\left(
\BOne + \frac{\epsilon}{i \Hbar} S_z
\right)
\Bv
\end{aligned}
\end{equation}

Or
\begin{equation}\label{eqn:PHY356FDec7:70}
\begin{aligned}
\Bv'
=
\left(
\BOne - \frac{i \epsilon}{\Hbar} S_z
\right)
\Bv
\end{aligned}
\end{equation}

Many infinitesimal rotations can be combined to create a finite rotation via
\index{rotation!infinitesimal}
\index{rotation!generator}
\begin{equation}\label{eqn:PHY356FDec7:80}
\begin{aligned}
\lim_{N \rightarrow \infty} \left( 1 + \frac{\alpha}{N} \right)^N = e^\alpha
\end{aligned}
\end{equation}

\begin{equation}\label{eqn:PHY356FDec7:85}
\alpha = -i \phi S_z/\Hbar
\end{equation}

For a finite rotation
\begin{equation}\label{eqn:PHY356FDec7:90}
\Bv'
=
e^{ -i \frac{\phi S_z}{\Hbar} }
\Bv
\end{equation}

Now think about transforming \(g(x,y,z)\), an arbitrary function.  Take \(\epsilon\) is very small so that

\begin{equation}\label{eqn:PHY356FDec7:100}
\begin{aligned}
x' &= x \cos \phi - y \sin\phi = x \cos \epsilon - y \sin\epsilon \approx x - y \epsilon \\
y' &= x \sin \phi + y \cos\phi = x \sin \epsilon + y \cos\epsilon \approx x \epsilon + y \\
z' &= z
\end{aligned}
\end{equation}

\paragraph{Question:} Why can we assume that \(\epsilon\) is small?

\paragraph{Answer:} We declare it to be small because it is simpler, and eventually build up to the general case where it is larger.  We want to master the easy task before moving on to the more difficult ones.

Our function is now transformed
\begin{equation}\label{eqn:lecture12rotationsAndAngularMomentum:280}
\begin{aligned}
g(x', y', z') \approx g( x - y \epsilon, y + x \epsilon, z) \\
&=
g( x , y , z) - \epsilon y \PD{x}{g} + \epsilon x \PD{y}{g} + \cdots \\
&=
\left( \BOne
- \epsilon y \PD{x}{}
+ \epsilon x \PD{y}{}
\right)
g( x, y ,z )
\end{aligned}
\end{equation}

Recall that the coordinate definition of the angular momentum operator is
\index{angular momentum operator}
\begin{equation}\label{eqn:PHY356FDec7:120}
L_z = -i \Hbar \left( x \PD{y}{} - y \PD{x}{} \right) = x p_y - y p_x
\end{equation}

We can now write
\begin{equation}\label{eqn:lecture12rotationsAndAngularMomentum:300}
\begin{aligned}
g(x', y', z')
&=
\left( \BOne
+
\frac{-i \Hbar \epsilon}{-i\Hbar} \left(
x \PD{y}{}
- y \PD{x}{}
\right)
\right)
g( x, y ,z ) \\
&=
\left( \BOne +
\frac{i \epsilon}{\Hbar} L_z
\right)
g( x, y ,z )
\end{aligned}
\end{equation}

For a finite rotation with angle \(\phi\) we have

\begin{equation}\label{eqn:PHY356FDec7:130}
g(x', y', z')
=
e^{i \frac{\phi L_z}{\Hbar}}
g( x, y ,z )
\end{equation}

\paragraph{Question:} Somebody says that the rotation is clockwise not counterclockwise?

I did not follow the reasoning briefly mentioned on the board since it looks right to me.  Perhaps this is the age old mixup between rotating the coordinates and the basis vectors.  Review what is in the text carefully.  Can also check by

If you rotate a ket, and examine how the state representation of that ket changes under rotation, we have

\begin{equation}\label{eqn:PHY356FDec7:140}
\ket{x', y', z'} = \ket{x - \epsilon y, y + \epsilon x, z}
\end{equation}

Or
\begin{equation}\label{eqn:lecture12rotationsAndAngularMomentum:320}
\begin{aligned}
\braket{\Psi}{x', y', z'}
&=
\Psi^\conj(x', y', z') \\
&=
\Psi^\conj(x - \epsilon y, y + \epsilon x, z) \\
&=
\Psi^\conj(x , y , z)
- \epsilon \PD{y}{\Psi^\conj}
+ \epsilon \PD{x}{\Psi^\conj} \\
&=
\left( \BOne + \frac{i \epsilon} {\Hbar} L_z \right) \Psi^\conj(x , y , z)
\end{aligned}
\end{equation}

Taking the complex conjugate we have
\begin{equation}\label{eqn:PHY356FDec7:150}
\Psi(x', y', z')
\left( \BOne - \frac{i \epsilon} {\Hbar} L_z \right) \Psi(x , y , z)
\end{equation}

For infinitesimal rotations about the \(z\) axis we have for functions
\begin{equation}\label{eqn:PHY356FDec7:160}
\Psi(x', y', z')
=
e^{ - \frac{i \epsilon} {\Hbar} L_z } \Psi(x , y , z)
\end{equation}

For finite rotations of a vector about the \(z\) axis we have
\begin{equation}\label{eqn:PHY356FDec7:170}
\Bv'
=
e^{ -\frac{i \phi S_z} {\Hbar} } \Bv
\end{equation}

and for functions
\begin{equation}\label{eqn:PHY356FDec7:180}
\Psi(x', y', z')
=
e^{ - \frac{i \phi L_z} {\Hbar} } \Psi(x , y , z)
\end{equation}

Vatche has mentioned \href{http://plato.stanford.edu/entries/qt-quantcomp/#2.2}{some devices being researched right now} where there is an attempt to isolate the spin orientation so that, say, only spin up or spin down electrons are allowed to flow.  There are some possible interesting applications here to Quantum computation.  Can we actually make a quantum computing device that is actually usable?  We can make NAND devices as mentioned in the article above.  Can this be scaled?  We do not know how to do this yet.

Recall that one description of a ``particle'' that has both a position and spin representation is

\begin{equation}\label{eqn:PHY356FDec7:190}
\ket{\Psi} = \ket{u} \directproduct \ket{s m}
\end{equation}

where we have a tensor product of kets.  One usually just writes the simpler
\begin{equation}\label{eqn:PHY356FDec7:195}
\ket{u} \directproduct \ket{s m} \equiv \ket{u} \ket{s m}
\end{equation}

An example of the above is
\begin{equation}\label{eqn:PHY356FDec7:200}
\begin{bmatrix}
u_1(\Br) \\
u_2(\Br) \\
u_3(\Br) \\
\end{bmatrix}
= \Bigl( \bra{\Br} \bra{ s m} \Bigr) \ket{\Psi}
\end{equation}

where \(u_1\) is spin component one.  For \(s=1\) this would be \(m=-1, 0, 1\).

Here we have also used
\begin{equation}\label{eqn:lecture12rotationsAndAngularMomentum:340}
\begin{aligned}
\ket{\Br}
=
\ket{x}
\directproduct
\ket{y}
\directproduct
\ket{z} \\
&=
\ket{x}
\ket{y}
\ket{z} \\
&=
\ket{x y z}
\end{aligned}
\end{equation}

We can now ask the question of how this thing transforms.  We transform each component of this as a vector.  The transformation of

\begin{equation}\label{eqn:lecture12rotationsAndAngularMomentum:360}
\begin{aligned}
\begin{bmatrix}
u_1(\Br) \\
u_2(\Br) \\
u_3(\Br)
\end{bmatrix}
\end{aligned}
\end{equation}

results in
\begin{equation}\label{eqn:PHY356FDec7:210}
{\begin{bmatrix}
u_1(\Br) \\
u_2(\Br) \\
u_3(\Br)
\end{bmatrix}}'
=
e^{ -i \phi (S_z + L_z)/\Hbar }
\begin{bmatrix}
u_1(\Br) \\
u_2(\Br) \\
u_3(\Br)
\end{bmatrix}
\end{equation}

Or with \(J_z = S_z + L_z\)
\begin{equation}\label{eqn:PHY356FDec7:220}
\ket{\Psi'} = e^{-i \phi J_z/\Hbar } \ket{\Psi}
\end{equation}

Observe that this separates out nicely with the \(S_z\) operation acting on the vector parts, and the \(L_z\) operator acting on the functional dependence.
