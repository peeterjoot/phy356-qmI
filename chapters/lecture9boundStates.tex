%
% Copyright © 2012 Peeter Joot.  All Rights Reserved.
% Licenced as described in the file LICENSE under the root directory of this GIT repository.
%
%\QMlecture{9 --- Bound-state problems. --- November 16, 2010}


\paragraph{Motivation.}  Motivation for today's physics is \href{http://physicsworld.com/cws/article/news/38046}{Solar Cell technology and quantum dots}.

\makeexample{}{example:lecture9boundStates:1}{
What are the eigenvalues and eigenvectors for an electron trapped in a 1D potential well?


\paragraph{MODEL}
\index{infinite potential}
Quantum state \(\ket{\Psi}\) describes the particle.  What \(V(X)\) should we choose?  Try a quantum well with infinite barriers first.

These spherical quantum dots are like quantum wells.  When you trap electrons in this scale you will get energy quantization.

\paragraph{VISUALIZE}
Draw a picture for \(V(X)\) with infinite spikes at \(\pm a\). (ie: figure 8.1 in the text).

\paragraph{SOLVE}
First task is to solve the time independent Schr\"{o}dinger equation.

\begin{equation}\label{eqn:PHY356Lecture9:1}
\begin{aligned}
H \ket{\Psi} = E \ket{\Psi}
\end{aligned}
\end{equation}

derivable from
\begin{equation}\label{eqn:PHY356Lecture9:2}
\begin{aligned}
H \ket{\Psi} = i \Hbar \PD{t}{} \ket{\Psi}
\end{aligned}
\end{equation}

In the position representation, we project \(\bra{x}\) onto \(H \ket{\Psi}\) and solve for \(\braket{x}{\Psi} = \Psi(x)\).  For the problems in Chapter 8,

\begin{equation}\label{eqn:PHY356Lecture9:3}
\begin{aligned}
H = \frac{\BP^2}{2m} + V(X,Y,Z),
\end{aligned}
\end{equation}

where
\begin{equation}\label{eqn:lecture9boundStates:60}
\begin{aligned}
P &= \text{momentum operator} \\
X &= \text{position operator} \\
m &= \text{electron mass}
\end{aligned}
\end{equation}

We should be careful to be strict about the notation, and not interchange the operators and their specific representations (ie: not interchanging ``little-x'' and ``big-x'') as we see in the text in this chapter.

Here the potential energy operator \(V(X,Y,Z)\) is time independent.

If \(i \Hbar \frac{d\ket{\Psi}}{dt} = H \ket{\Psi}\) and \(H\) is time independent then \(\ket{\Psi} = \ket{u} e^{-i E t/\Hbar}\) implies

\begin{equation}\label{eqn:lecture9boundStates:80}
\begin{aligned}
i \Hbar \frac{ -i E }{\Hbar} \ket{u} e^{-i E t/\Hbar} = H \ket{u} e^{-i E t/\Hbar}
\end{aligned}
\end{equation}

or
\begin{equation}\label{eqn:PHY356Lecture9:4}
\begin{aligned}
E \ket{u} = H \ket{u}
\end{aligned}
\end{equation}

Here \(E\) is the energy eigenvalue, and \(\ket{u}\) is the energy eigenstate.  Our differential equation now becomes

\begin{equation}\label{eqn:PHY356Lecture9:5}
\begin{aligned}
-\frac{\Hbar^2 }{2m} \frac{d^2 u(x)}{dx^2} + V(x) u(x) = E u(x)
\end{aligned}
\end{equation}

where \(V(x) = 0\) for \(\Abs{x} < a\).  We will not find anything like this for real, but this is our first approximation to the quantum dot.

Our differential equation in the well is now

\begin{equation}\label{eqn:PHY356Lecture9:6}
\begin{aligned}
-\frac{\Hbar^2 }{2m} \frac{d^2 u(x)}{dx^2} = E u(x)
\end{aligned}
\end{equation}

or with \(\alpha = \sqrt{2m E/\Hbar^2}\)

\begin{equation}\label{eqn:PHY356Lecture9:7}
\begin{aligned}
\frac{d^2 u(x)}{dx^2} u(x) = -\frac{2 m E}{\Hbar^2} u(x) = - \alpha^2 u(x)
\end{aligned}
\end{equation}

Our solution for \(\Abs{x} < a\) is then

\begin{equation}\label{eqn:PHY356Lecture9:8}
\begin{aligned}
u(x) = A \cos \alpha x + B \sin\alpha x
\end{aligned}
\end{equation}

and for \(\Abs{x} > a\) we have \(u(x) = 0\) since \(V(x) = \infty\).

Setting \(u(a) = u(-a) = 0\) we have
\begin{equation}\label{eqn:lecture9boundStates:100}
\begin{aligned}
A \cos \alpha a + B \sin\alpha a &= 0 \\
A \cos \alpha a - B \sin\alpha a &= 0
\end{aligned}
\end{equation}


\paragraph{Type I}
\(B=0\), \(A \cos\alpha a = 0\).  For \(A \ne 0\) we must have

\begin{equation}\label{eqn:lecture9boundStates:120}
\begin{aligned}
\cos \alpha a = 0
\end{aligned}
\end{equation}

or \(\alpha a = n \frac{\pi}{2}\), where \(n = 1, 3, 5, ...\), so our solution is

\begin{equation}\label{eqn:PHY356Lecture9:9}
\begin{aligned}
u(x) = A \cos \left( \frac{n \pi}{2 a} x \right)
\end{aligned}
\end{equation}

\paragraph{Type II}
\(A=0\), \(B \sin\alpha a = 0\).  For \(B \ne 0\) we must have

\begin{equation}\label{eqn:lecture9boundStates:140}
\begin{aligned}
\sin \alpha a = 0
\end{aligned}
\end{equation}

or \(\alpha a = n \frac{\pi}{2}\), where \(n = 1, 2, 4, ...\), so our solution is

\begin{equation}\label{eqn:PHY356Lecture9:10}
\begin{aligned}
u(x) = B \sin \left( \frac{n \pi}{2 a} x \right)
\end{aligned}
\end{equation}

\paragraph{Via determinant}

We could also write
\begin{equation}\label{eqn:lecture9boundStates:160}
\begin{aligned}
\begin{bmatrix}
\cos \alpha a & \sin\alpha a \\
\cos \alpha a & - \sin\alpha a
\end{bmatrix}
\begin{bmatrix}
A \\
B
\end{bmatrix}
= 0
\end{aligned}
\end{equation}

and then must have zero determinant, or

\begin{equation}\label{eqn:PHY356Lecture9:11}
\begin{aligned}
-2 \sin\alpha a \cos\alpha a = -\sin 2 \alpha a
\end{aligned}
\end{equation}

so we must have
\begin{equation}\label{eqn:lecture9boundStates:180}
\begin{aligned}
2 \alpha a = n \pi
\end{aligned}
\end{equation}

or
\begin{equation}\label{eqn:lecture9boundStates:200}
\begin{aligned}
\alpha = \frac{n \pi}{2a}
\end{aligned}
\end{equation}

regardless of \(A\) and \(B\).  We can then determine the solutions \eqnref{eqn:PHY356Lecture9:9}, and \eqnref{eqn:PHY356Lecture9:10} simply by noting that this value for \(\alpha\) kills off either the sine or cosine terms of \eqnref{eqn:PHY356Lecture9:8} depending on whether \(n\) is even or odd.

\paragraph{CHECK}

\begin{equation}\label{eqn:lecture9boundStates:220}
\begin{aligned}
u_n(x) &= A \cos \left( \frac{n \pi}{2 a} x \right) \\
u_n(x) &= B \sin \left( \frac{n \pi}{2 a} x \right)
\end{aligned}
\end{equation}

satisfy the time independent Schr\"{o}dinger equation, and the corresponding eigenvalues from from

\begin{equation}\label{eqn:lecture9boundStates:240}
\begin{aligned}
\alpha = \sqrt{\frac{2 m E}{\Hbar^2}},
\end{aligned}
\end{equation}

or

\begin{equation}\label{eqn:lecture9boundStates:260}
\begin{aligned}
E = \frac{\Hbar^2 \alpha^2}{2m} = \frac{\Hbar^2}{2m} \left( \frac{n \pi}{2a} \right)^2
\end{aligned}
\end{equation}

for \(n = 1, 2, 3, \cdots\).

\paragraph{On the derivative of \(u\) at the boundaries}

Integrating

\begin{equation}\label{eqn:PHY356Lecture9:20}
\begin{aligned}
-\frac{\Hbar^2 }{2m} \frac{d^2 u(x)}{dx^2} u(x) + V(x) u(x) = E u(x),
\end{aligned}
\end{equation}

over \([a-\epsilon,a+\epsilon]\) we have

\begin{equation}\label{eqn:PHY356Lecture9:21}
\begin{aligned}
-\frac{\Hbar^2 }{2m} &
\int_{a-\epsilon}^{a-\epsilon}
\frac{d^2 u(x)}{dx^2} dx
+
\int_{a-\epsilon}^{a-\epsilon}
V(x) u(x) dx =
\int_{a-\epsilon}^{a-\epsilon}
E u(x) dx \\
-\frac{\Hbar^2 }{2m} &
\left(
\left.\frac{du}{dx}\right\vert_{a-\epsilon}^{a+\epsilon} + 0 = 0
\right)
\end{aligned}
\end{equation}

which gives us

\begin{equation}\label{eqn:PHY356Lecture9:22}
\begin{aligned}
\left.\frac{du}{dx}\right\vert_{a + \epsilon}
-\left.\frac{du}{dx}\right\vert_{a - \epsilon} = 0
\end{aligned}
\end{equation}

or
\begin{equation}\label{eqn:PHY356Lecture9:23}
\begin{aligned}
\left.\frac{du}{dx}\right\vert_{a + \epsilon}
&=
\left.\frac{du}{dx}\right\vert_{a - \epsilon}
\end{aligned}
\end{equation}

We can infer how the derivative behaves over the potential discontinuity, so in the limit where \(\epsilon \rightarrow 0\) we must have wave function continuity at despite the potential discontinuity.

This sort of analysis, which is potential dependent, we see that for this \textAndIndex{infinite potential well}, our derivative must be continuous at the boundary.

} % example

\makeexample{Non-infinite step well potential.}{example:lecture9boundStates:2}{

\index{step well}
Given a zero potential in the well \(\Abs{x} < a\)
\begin{equation}\label{eqn:PHY356Lecture9:30}
\begin{aligned}
-\frac{\Hbar^2 }{2m} \frac{d^2 u(x)}{dx^2} u(x) + 0 = E u(x),
\end{aligned}
\end{equation}

and outside of the well \(\Abs{x} > a\)
\begin{equation}\label{eqn:PHY356Lecture9:31}
\begin{aligned}
-\frac{\Hbar^2 }{2m} \frac{d^2 u(x)}{dx^2} u(x) + V_0 u(x) = E u(x)
\end{aligned}
\end{equation}

Inside of the well, we have the solution worked previously, with \(\alpha = \sqrt{2m E/\Hbar^2}\)

\begin{equation}\label{eqn:PHY356Lecture9:32}
\begin{aligned}
u(x) &= A \cos\alpha x + B \sin\alpha x
\end{aligned}
\end{equation}

Then we have outside of the well the same form
\begin{equation}\label{eqn:PHY356Lecture9:33}
\begin{aligned}
-\frac{\Hbar^2 }{2m} \frac{d^2 u(x)}{dx^2} u(x) = (E - V_0 )u(x)
\end{aligned}
\end{equation}

With \(\beta = \sqrt{ 2m (V_0 - E)/\Hbar^2}\), this is

\begin{equation}\label{eqn:PHY356Lecture9:34}
\begin{aligned}
\frac{d^2 u(x)}{dx^2} u(x) = \beta^2 u(x)
\end{aligned}
\end{equation}

If \(V_0 - E > 0\), we have \(V_0 > E\), and the states are ``bound'' or ``localized'' in the well.

Our solutions for this \(V_0 > E\) case are then

\begin{equation}\label{eqn:PHY356Lecture9:35}
\begin{aligned}
u(x) &= D e^{\beta x} \\
u(x) &= C e^{-\beta x}
\end{aligned}
\end{equation}

for \(x \le a\), and \(x \ge a\) respectively.

\paragraph{Question:} Why can we not have

\begin{equation}\label{eqn:PHY356Lecture9:36}
\begin{aligned}
u(x) = D e^{\beta x} + C e^{-\beta x}
\end{aligned}
\end{equation}

for \(x \le -a\)?

\paragraph{Answer:} As \(x \rightarrow -\infty\) we would then have

\begin{equation}\label{eqn:lecture9boundStates:280}
\begin{aligned}
u(x) \rightarrow C e^{\beta \infty} \rightarrow \infty
\end{aligned}
\end{equation}

This is a non-physical solution, and we discard it based on our normalization requirement.

Our total solution, in regions \(x < -a\), \(\Abs{x} \le a\), and \(x > a\) respectively

\begin{equation}\label{eqn:lecture9boundStates:300}
\begin{aligned}
u_1(x) &= D e^{\beta x} \\
u_2(x) &= A \cos\alpha x + B \sin\alpha x \\
u_3(x) &= C e^{-\beta x}
\end{aligned}
\end{equation}

To find the coefficients, set \(u_1(-a) = u_2(-a)\), \(u_2(a) = u_3(a)\) \(u_1'(-a) = u_2'(-a)\), \(u_2'(a) = u_3'(a)\), and NORMALIZE \(u(x)\).

Now, how about in region 2 (\(x < -a\)), \(V_0 < E\) implies that our equation is

\begin{equation}\label{eqn:PHY356Lecture9:40}
\begin{aligned}
\frac{d^2 u(x)}{dx^2} u(x) = - \frac{2m}{\Hbar^2} (E - V_0) u(x) = - k^2 u(x)
\end{aligned}
\end{equation}

We no longer have quantized energy for such a solution.  These correspond to the ``unbound'' or ``continuum'' states.  Even though we do not have quantized energy we still have quantum effects.  Our solution becomes

\begin{equation}\label{eqn:lecture9boundStates:320}
\begin{aligned}
u_1(x) &=
C_2 e^{i k x}
+D_2 e^{-i k x}  \\
u_2(x) &=
A e^{i \alpha x}
+B e^{-i \alpha x}  \\
u_3(x) &=
C_3 e^{i k x}
\end{aligned}
\end{equation}

\paragraph{Question.}  Why no \(D_2 e^{-i k x}\), in the \(u_3(x)\) term?

Answer.  We can, but this is not physically relevant.  Why is because we associate \(e^{ikx}\) with an incoming wave, with reflection in the \(x < -a\) interval, and both \(e^{\pm i \alpha x}\) in the \(\Abs{x} < a\) interval, but just an outgoing wave \(e^{i k x}\) in the \(x > a\) region.

FIXME: scan picture: 9.1 in my notebook.

Observe that this is not normalizable as is.  We require ``delta-function'' normalization.  What we can do is ask about current densities.  How much passes through the barrier, and so forth.


\paragraph{Note to self.}
We probably really we want to consider a wave packet of states, something like:

\begin{equation}\label{eqn:lecture9boundStates:340}
\begin{aligned}
\Psi_1(x) &= \int dk f_1(k) e^{i k x} \\
\Psi_2(x) &= \int d\alpha f_2(\alpha) e^{i \alpha x} \\
\Psi_3(x) &= \int dk f_3(k) e^{i k x}
\end{aligned}
\end{equation}

Then we would have something that we can normalize.  Play with this later.

} % example
