%
% Copyright � 2012 Peeter Joot.  All Rights Reserved.
% Licenced as described in the file LICENSE under the root directory of this GIT repository.
%

%\chapter{Hydrogen like atom, and Laguerre polynomials}
\index{hydrogen atom}
\index{Laguerre polynomial}
\label{chap:hyrdogenLaguerre}
%\blogpage{http://sites.google.com/site/peeterjoot/math2010/hyrdogenLaguerre.pdf}
%\date{Nov 24, 2010}
%
%\section{Motivation}
%
For the hydrogen atom, after some variable substitutions the radial part of the Schr\"{o}dinger equation takes the form
%
\begin{equation}\label{eqn:hyrdogenLaguerre:10}
\frac{d^2 R_l}{d\rho^2} + \frac{2}{\rho} \frac{d R_l}{d\rho} + \left( \frac{\lambda}{\rho} - \frac{l(l+1)}{\rho^2} - \inv{4} \right) R_l = 0
\end{equation}
%
In \citep{desai2009quantum} \S 8.8 it is argued that the functions \(R_l\) are of the form
%
\begin{equation}\label{eqn:hyrdogenLaguerre:20}
R_l = \rho^s L(\rho) e^{-\rho/2}
\end{equation}
%
where \(L\) is a polynomial in \(\rho\), specifically Laguerre polynomials.  Let us look at some of those details a bit more closely.
%
%\section{Guts}
The first part of the argument comes from considering the \(\rho \rightarrow \infty\) case, where Schr\"{o}dinger's equation is approximately
%
\begin{equation}\label{eqn:hyrdogenLaguerre:10a}
\frac{d^2 R_l}{d\rho^2} - \inv{4} R_l \approx 0.
\end{equation}
%
This large \(\rho\) approximation has solutions \(e^{\pm \rho/2}\), and we take the negative sign case as physically meaningful in order for the wave function to be normalizable.

Next it is argued that polynomial multiples of this will also be approximate solutions.  Utilizing monomial multiple of the decreasing exponential as a trial solution, let us compute how this fits into the radial Schr\"{o}dinger's equation \eqnref{eqn:hyrdogenLaguerre:10} above.  Write
%
\begin{equation}\label{eqn:hyrdogenLaguerre:15}
R_l = \rho^s e^{-\rho/2}
\end{equation}
%
The derivatives are
%
\begin{equation}\label{eqn:hyrdogenLaguerre:200}
\begin{aligned}
R_l' &= \rho^{s-1} \left( s -\frac{\rho}{2}\right) e^{-\rho/2} \\
R_l'' &=
\rho^{s-2}
\left( s (s-1) -s \rho +\frac{1}{4} \rho^2
\right)
e^{-\rho/2}
\end{aligned}
\end{equation}
%
and substitution yields
%\rho^{s-2}
%e^{-\rho/2}
%\left(
%s (s-1)
%-s \rho
%+\frac{1}{4} \rho^2
%+2s
%-\rho
%+\lambda \rho
%- l(l+1)
%- \frac{\rho^2}{4}
%\right)
%&=
\begin{equation}\label{eqn:hyrdogenLaguerre:16}
\rho^{s-2}
e^{-\rho/2}
\left(
(s - \rho) (s+1)
+\lambda \rho
- l(l+1)
\right)
\end{equation}
%
There are two things that this can show.  The first is that for \(\rho \rightarrow \infty\) this produces a polynomial with degree \(s-2\) and \(s-1\) terms multiplied by the exponential, and we have approximately
%
\begin{equation}\label{eqn:hyrdogenLaguerre:17}
\rho^{s-1}
e^{-\rho/2}
(\lambda - s - 1)
\end{equation}
%
The \(s-1\) terms will dominate the polynomial, but the exponential dominate all, approaching zero for \(\rho \rightarrow \infty\), just as the non-polynomial multiplied \(e^{-\rho/2}\) approximate solution will.  This confirms that in the limit this polynomial multiplied exponential still has the desired behavior in the large \(\rho\) limit.  Also observe that in the limit of small \(\rho\) we have approximately
%
\begin{equation}\label{eqn:hyrdogenLaguerre:18}
\rho^{s-2}
e^{-\rho/2}
\left(
s (s+1) - l(l+1)
\right)
\end{equation}
%
Since \(\rho^{s-2} \rightarrow \infty\) as \(\rho \rightarrow 0\), we require either a different trial solution, or \(s=l\) to have a normalizable wavefunction.

Before settling on \(s=l\) let us compute the derivatives for a more general trial function, of the form \eqnref{eqn:hyrdogenLaguerre:20}, and substitute those.  After a bit of computation we find
%
\begin{equation}\label{eqn:hyrdogenLaguerre:19}
R_l' = \rho^{s-1} e^{-\rho/2} \left( \left( s - \frac{\rho}{2} \right) L + \rho L'
\right)
\end{equation}
\begin{equation}\label{eqn:hyrdogenLaguerre:19b}
R_l'' = \rho^{s-2} e^{-\rho/2} \left(
\left( s(s-1) - s \rho + \frac{\rho^2}{4} \right) L
+\left( 2 s \rho -\rho^2 \right) L'
+ \rho^2 L''
\right)
\end{equation}
%
Putting these together and substitution back into \eqnref{eqn:hyrdogenLaguerre:10} yields
%
\begin{equation}\label{eqn:hyrdogenLaguerre:19c}
0 = \rho^{s-2} e^{-\rho/2} \left(
L \left( (s-\rho)(s+1) + \rho \lambda -l (l+1)
\right)
+\rho L' \left( 2 (s+1) -\rho \right)
+ \rho^2 L''
\right)
\end{equation}
%
In the \(\rho \rightarrow 0\) limit where the \(\rho^{s-2}\) terms dominate \eqnref{eqn:hyrdogenLaguerre:50} becomes
\begin{equation}\label{eqn:hyrdogenLaguerre:50}
0 \approx
\rho^{s-2} L \left(
s(s+1) - l(l+1)
\right)
\end{equation}
%
Again, this provides the \(s=l\) or \(s = -(l+1)\) possibilities from the text, and we discard \(s=-(l+1)\) due to non-normalizability.  A side question.  How does one solve integer equations like this?
%
\paragraph{What remains?}
%
With \(s=l\) killing off the \(\rho^{s-2}\) terms, what is our differential equation for \(L\)?
%
\begin{equation}\label{eqn:hyrdogenLaguerre:100}
0 =
\rho L''
+L' \left( 2 (l+1) -\rho \right)
+L \left( \lambda - (l+1) \right)
\end{equation}
%
Comparing this to \citep{wiki:laguerre} we have something pretty close to the stated differential equation for the Laguerre polynomial.  Ours is of the form
%
\begin{equation}\label{eqn:hyrdogenLaguerre:101}
0 =
\rho L''
+L' \left( m + 1 -\rho \right)
+L n,
\end{equation}
%
where the differential equation in the wikipedia article has \(m=0\).  No change of variables involving a scalar multiplicative factor for \(\rho\) appears to be able to get it into that form, and I am guessing this is the differential equation for the associated Laguerre polynomial (something not stated in the wikipedia article).

Let us derive the recurrence relations for the coefficients, and work out the first few such polynomials to compare.  Plugging in a polynomial of the form
%
\begin{equation}\label{eqn:hyrdogenLaguerre:110}
L = \sum_{k=0}^r a_k \rho^{k},
\end{equation}
%
where \(a_r\) is assumed to be non-zero.  We also assume that this polynomial is not an infinite series (ruling out the infinite series with convergence arguments is covered nicely in the text).

we have for \eqnref{eqn:hyrdogenLaguerre:101}
\begin{equation}\label{eqn:hyrdogenLaguerre:220}
\begin{aligned}
0 &= \sum_{k=0}^r a_k
\left(
k (k-1) \rho^{k-1}
+ k (m+1) \rho^{k-1}
- k \rho^{k}
+ n \rho^{k}
\right) \\
&=
\sum_{{k'}=1}^r
\rho^{{k'}-1}
a_{k'}
{k'}
\left(
{k'}-1
+
(m +1)
\right)
+\sum_{k=0}^r
\rho^{k}
a_k
\left(
-
 k
+
 n
\right) \\
&=
% k'-1 = k
% k' = k + 1
\sum_{{k}=0}^{r-1}
\rho^{k}
a_{k+1}
(k+1)
\left(
k
+
(m+1)
\right)
+\sum_{k=0}^r
\rho^{k}
a_k
\left(
-
 k
+
 n
\right) \\
&=
\sum_{{k}=0}^{r-1}
\rho^{k}
\Bigl(
a_{k+1} (k+1) (k + m + 1)
+a_k (n -k)
\Bigr)
+a_r (n-r) \rho^{r}
\end{aligned}
\end{equation}
%
Observe first that since we have assumed \(a_r \ne 0\), we must have \(r=n\).  Requiring termwise equality with zero gives us the recurrence relation between the coefficients, for \(k \in [0,n-1]\)
%
\begin{equation}\label{eqn:hyrdogenLaguerre:120}
a_{k+1} = a_k \frac{k - n}{ (k+1) (k + m + 1) }.
\end{equation}
%
Repeated application shows the pattern for these coefficients, and with \(a_0=1\) we have
%
\begin{equation}\label{eqn:hyrdogenLaguerre:240}
\begin{aligned}
a_1 &= -\frac{n-0}{(1)(m+1)} \\
a_2 &= \frac{(n-1)(n-0)}{(2)(1)(m+2)(m+1)} \\
a_3 &= -\frac{(n-2)(n-1)(n-0)}{(3)(2)(1)(m+3)(m+2)(m+1)},
\end{aligned}
\end{equation}
%
With
\begin{equation}\label{eqn:hyrdogenLaguerre:260}
\begin{aligned}
a_k
&= \frac{(-1)^k (n-(k-1))\cdots(n-1)(n-0)}{k!(m+k)\cdots(m+2)(m+1)} \\
&= \frac{(-1)^k n! m!}{k!(m+k)!(n-(k-1) -1)!},
\end{aligned}
\end{equation}
%
Or
\begin{equation}\label{eqn:hyrdogenLaguerre:130}
a_k
= \frac{(-1)^k n! m!}{k!(m+k)!(n-k)!}.
\end{equation}
%
Forming the complete series, we can get at the form of the associated Laguerre polynomials in the wikipedia article without too much trouble
%
\begin{equation}\label{eqn:hyrdogenLaguerre:280}
\begin{aligned}
L_n^m(\rho)
&\propto 1 + \sum_{k=1}^n \frac{(-1)^k}{k!} \frac{n! m!}{(n-k)!(m+k)!} \rho^k \\
&\propto \frac{(n+m)!}{n!m!} + \sum_{k=1}^n \frac{(-1)^k}{k!} \frac{(n+m)!}{(n-k)!(m+k)!} \rho^k.
\end{aligned}
\end{equation}
%
Dropping the proportionality, this simplifies to just
%
\begin{equation}\label{eqn:hyrdogenLaguerre:140}
L_n^m(\rho) = \sum_{k=0}^n \frac{(-1)^k}{k!} \binom{n+m}{m+k} \rho^k
\end{equation}
%
This is not necessarily the form of the polynomials used in the text.  To see if that is the case, we need to check the normalization.

According to the wikipedia article we have for the associated Laguerre polynomials as defined above
\begin{equation}\label{eqn:hyrdogenLaguerre:150}
\int_0^{\infty}\rho^m e^{-\rho} L_n^{m}(\rho)L_{n'}^{m}(\rho)d\rho = \frac{(n+m)!}{n!}\delta_{n,{n'}}
\end{equation}
%
whereas in the text we have
\begin{equation}\label{eqn:hyrdogenLaguerre:150b}
\int_0^{\infty}\rho^{2l + 2} e^{-\rho} \left( L_{n+l}^{2l + 1}(\rho) \right)^2 d\rho = \frac{2n ((n+l)!)^3}{(n-l-1)!}.
\end{equation}
%
%with \(m=2l + 1\), and \(n \rightarrow n+l+1\), we have
%\begin{equation}\label{eqn:hyrdogenLaguerre:150}
%\int_0^{\infty}\rho^{2l+1} e^{-\rho} \left(L_{n+l+1}^{{2l+1}}(\rho) \right)^2 d\rho = \frac{(n+3l + 2)!}{n+ l-1!}
%\end{equation}
It seems clear that two different notations are being used.  In this physical context of wave functions we want the normalization defined by
\begin{equation}\label{eqn:hyrdogenLaguerre:160}
1 = \int_0^\infty \rho^2 R_l^2(\rho) d\rho = \int_0^\infty \rho^{2l + 2} e^{-\rho} L^2(\rho) d\rho
\end{equation}
%
Using the wikipedia notation, with
\begin{equation}\label{eqn:hyrdogenLaguerre:170}
L(\rho) = A L_n^{2l+1},
\end{equation}
%
we want
\begin{equation}\label{eqn:hyrdogenLaguerre:300}
\begin{aligned}
1
&= \int \rho^{2l + 2} e^{-\rho} L^2(\rho) d\rho \\
&=
A^2 \sum_{a,b=0}^n \frac{(-1)^{a+b}}{a!b!}
\binom{n+2l+1}{2l+1+a}
\binom{n+2l+1}{2l+1+b}
\int_0^\infty d\rho \rho^{2l + 2 + a + b} e^{-\rho}
\end{aligned}
\end{equation}
%
Since \(\int_0^\infty d\rho \rho^{a} e^{-\rho} = \Gamma(a+1) = a!\) we have
%
\begin{equation}\label{eqn:hyrdogenLaguerre:180}
1 = A^2 \sum_{a,b=0}^n \frac{(-1)^{a+b}}{a!b!}
\binom{n+m}{m+a}
\binom{n+m}{m+b}
(m + 1 + a + b)!
\end{equation}
%
It looks like there is probably some way to simplify this, and if so we would be able to map the notation used (without definition) used in the text, to the notation used in the wikipedia article.  If we do not care about that, nor the specifics of the normalization constant then there is not too much more to say.

This is an ugly kind of place to leave things, but that is enough for today.  It is too bad that the text is not just more explicit, and it is probably best to refer elsewhere for any more detail.  With no specifics about the functions themselves in any form, one has to do that anyways.
