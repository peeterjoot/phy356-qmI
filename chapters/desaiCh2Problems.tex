%
% Copyright � 2013 Peeter Joot.  All Rights Reserved.
% Licenced as described in the file LICENSE under the root directory of this GIT repository.
%
\makeoproblem{Cauchy-Schwartz identity}{problem:desaiCh2Problems:1}{\citep{desai2009quantum} pr 2.1}{
FIXME: describe.
\index{Cauchy-Schwartz identity}
} % problem
\makeanswer{problem:desaiCh2Problems:1}{
We wish to find the value of \(\lambda\) that is just right to come up with the desired identity.  The starting point is the expansion of the inner product
\begin{equation}\label{eqn:desaiCh2Problems:2020}
\braket{a + \lambda b}
{a + \lambda b}
= \braket{a}{a} + \lambda \lambda^\conj \braket{b}{b} + \lambda \braket{a}{b} + \lambda^\conj \braket{b}{a}.
\end{equation}
%
There is a trial and error approach to this problem, where one magically picks \(\lambda \propto \braket{b}{a}/\braket{b}{b}^n\), and figures out the proportionality constant and scale factor for the denominator to do the job.  A nicer way is to set up the problem as an extreme value exercise.  We can write this inner product as a function of \(\lambda\), and proceed with setting the derivative equal to zero
\begin{equation}\label{eqn:desaiCh2Problems:2040}
f(\lambda) =
\braket{a}{a} + \lambda \lambda^\conj \braket{b}{b} + \lambda \braket{a}{b} + \lambda^\conj \braket{b}{a}.
\end{equation}
The derivative is
\begin{equation}\label{eqn:desaiCh2Problems:2060}
\begin{aligned}
\frac{df}{d\lambda}
&=
\left(\lambda^\conj + \lambda \frac{d\lambda^\conj}{d\lambda}\right) \braket{b}{b} + \braket{a}{b} + \frac{d\lambda^\conj}{d\lambda} \braket{b}{a} \\
&=
\lambda^\conj \braket{b}{b} + \braket{a}{b}
+
\frac{d\lambda^\conj}{d\lambda} \Bigl(
\lambda \braket{b}{b} + \braket{b}{a} \Bigr)
\end{aligned}
\end{equation}
%
Now, we have a bit of a problem with \(d\lambda^\conj/d\lambda\), since that does not actually exist.  However, that problem can be side stepped if we insist that the factor that multiplies it is zero.  That provides a value for \(\lambda\) that also kills of the remainder of \(df/d\lambda\).  That value is
%
\begin{equation}\label{eqn:desaiCh2Problems:2080}
\lambda = - \frac{\braket{b}{a} }{ \braket{b}{b}  }.
\end{equation}
%
Back substitution yields
\begin{equation}\label{eqn:desaiCh2Problems:2100}
\braket{a + \lambda b}
{a + \lambda b}
= \braket{a}{a} - \braket{a}{b}\braket{b}{a}/\braket{b}{b} \ge 0.
\end{equation}
%
This is easily rearranged to obtain the desired result:
%
\begin{equation}\label{eqn:desaiCh2Problems:2120}
\braket{a}{a} \braket{b}{b} \ge \braket{b}{a}\braket{a}{b}.
\end{equation}
} % answer
\makeoproblem{Uncertainty relation}{problem:desaiCh2Problems:2}{\citep{desai2009quantum} pr 2.2}{
FIXME: describe.
} % problem
\makeanswer{problem:desaiCh2Problems:2}{
Using the Schwartz inequality of problem 1, and a symmetric and antisymmetric (anticommutator and commutator) sum of products that
%
\begin{equation}\label{eqn:desaiCh2Problems:200}
\Abs{\Delta A \Delta B}^2 \ge  \inv{4}\Abs{ \antisymmetric{A}{B}}^2,
\end{equation}
and that this result implies
\begin{equation}\label{eqn:desaiCh2Problems:201}
\Delta x \Delta p \ge \frac{\Hbar}{2}.
\end{equation}
%
\paragraph{The solution}
\index{Schwartz inequality}
%
This problem seems somewhat misleading, since the Schwartz inequality appears to have nothing to do with showing \eqnref{eqn:desaiCh2Problems:200}, but only with the split of the operator product into symmetric and antisymmetric parts.  Another possible tricky thing about this problem is that there is no mention of the anticommutator in the text at this point that I can find, so if one does not know what it is defined as, it must be figured out by context.

I have also had an interpretation problem with this since \(\Delta x \Delta p\) in \eqnref{eqn:desaiCh2Problems:201} cannot mean the operators as is the case of \eqnref{eqn:desaiCh2Problems:200}.  My assumption is that in \eqnref{eqn:desaiCh2Problems:201} these deltas are really absolute expectation values, and that we really want to show
%
\begin{equation}\label{eqn:desaiCh2Problems:202}
\Abs{\expectation{\Delta X}} \Abs{\expectation{\Delta P}} \ge \frac{\Hbar}{2}.
\end{equation}
%
However, I am unable to demonstrate this.  Instead I am able to show two things:
%
\begin{equation}\label{eqn:desaiCh2Problems:2140}
\begin{aligned}
\expectation{(\Delta X)^2 } \expectation{(\Delta P)^2 }
&\ge \frac{\Hbar^2}{4} \\
\Abs{\expectation{\Delta X \Delta P } }
&\ge
\frac{\Hbar}{2}
\end{aligned}
\end{equation}
%
Is one of these the result to be shown?  Note that only the first of these required the Schwartz inequality.  Also, it seems strange that we want the expectation of the operator \(\Delta X\Delta P\)?

Starting with the first part of the problem, note that we can factor any operator product into a linear combination of two Hermitian operators using the commutator and anticommutator.  That is
%
\begin{equation}\label{eqn:desaiCh2Problems:2160}
\begin{aligned}
C D
&= \inv{2}\left( C D + D C\right) + \inv{2}\left( C D - D C\right) \\
&= \inv{2}\left( C D + D C\right) + \inv{2i}\left( C D - D C\right) i \\
&\equiv
\inv{2}\symmetric{C}{D}
+\inv{2i} \antisymmetric{C}{D} i
\end{aligned}
\end{equation}
%
For Hermitian operators \(C\), and \(D\), using \((CD)^\dagger = D^\dagger C^\dagger = D C\), we can show that the two operator factors are Hermitian,
%
\begin{equation}\label{eqn:desaiCh2Problems:2180}
\begin{aligned}
\left(\inv{2}\symmetric{C}{D}\right)^\dagger
&= \inv{2}\left( C D + D C\right)^\dagger \\
&= \inv{2}\left( D^\dagger C^\dagger + C^\dagger D^\dagger\right) \\
&= \inv{2}\left( D C + C D \right) \\
&= \inv{2}\symmetric{C}{D},
\end{aligned}
\end{equation}
%
\begin{equation}\label{eqn:desaiCh2Problems:2200}
\begin{aligned}
\left(\inv{2}\antisymmetric{C}{D} i\right)^\dagger
&= -\frac{i}{2} \left( C D - D C\right)^\dagger \\
&= -\frac{i}{2}\left( D^\dagger C^\dagger - C^\dagger D^\dagger\right) \\
&= -\frac{i}{2}\left( D C - C D \right) \\
&=
\inv{2}\antisymmetric{C}{D} i
\end{aligned}
\end{equation}
%
So for the absolute squared value of the expectation of product of two operators we have
%
\begin{equation}\label{eqn:desaiCh2Problems:2220}
\begin{aligned}
\expectation{C D }^2
&=
\Abs{\expectation{\inv{2}\symmetric{C}{D} +\inv{2i} \antisymmetric{C}{D} i}}^2 \\
&=
\Abs{ \inv{2}\expectation{\symmetric{C}{D}} +\inv{2i} \expectation{\antisymmetric{C}{D} i} }^2.
\end{aligned}
\end{equation}
%
Now, these expectation values are real, given the fact that these operators are Hermitian.  Suppose we write \(a = \expectation{\symmetric{C}{D}}/2\), and \(b = \expectation{\antisymmetric{C}{D}i}/2\), then we have
%
\begin{equation}\label{eqn:desaiCh2Problems:2240}
\begin{aligned}
\Abs{ \inv{2}\expectation{\symmetric{C}{D}} +\inv{2i} \expectation{\antisymmetric{C}{D} i} }^2
&=
\Abs{ a - b i }^2 \\
&=
( a - b i ) ( a + b i ) \\
&=
a^2 + b^2
\end{aligned}
\end{equation}
%
So we have for the squared expectation value of the operator product \(C D\)
\begin{equation}\label{eqn:desaiCh2Problems:2260}
\begin{aligned}
\expectation{C D }^2
&=
\inv{4}\expectation{\symmetric{C}{D}}^2 +\inv{4} \expectation{\antisymmetric{C}{D} i}^2 \\
&=
\inv{4}\Abs{\expectation{\symmetric{C}{D}}}^2 +\inv{4} \Abs{\expectation{\antisymmetric{C}{D} i}}^2 \\
&=
\inv{4}\Abs{\expectation{\symmetric{C}{D}}}^2 +\inv{4} \Abs{\expectation{\antisymmetric{C}{D}}}^2 \\
&\ge
\inv{4} \Abs{\expectation{\antisymmetric{C}{D}}}^2.
\end{aligned}
\end{equation}
%
With \(C = \Delta A\), and \(D = \Delta B\), this almost completes the first part of the problem.  The remaining thing to note is that \(\antisymmetric{\Delta A}{\Delta B} = \antisymmetric{A}{B}\).  This last is straight forward to show
%
\begin{equation}\label{eqn:desaiCh2Problems:2280}
\begin{aligned}
\antisymmetric{\Delta A}{\Delta B}
&=
\antisymmetric{A - \expectation{A}}{B - \expectation{B}}  \\
&=
(A - \expectation{A})(B - \expectation{B})
-(B - \expectation{B})(A - \expectation{A}) \\
&=
\left( A B
- \expectation{A}B
- \expectation{B}A
+
\expectation{A}
\expectation{B} \right)
-
\left( B A
- \expectation{B}A
- \expectation{A}B
+
\expectation{B}
\expectation{A} \right) \\
&=
A B - B A  \\
&=
\antisymmetric{A}{B}.
\end{aligned}
\end{equation}
%
Putting the pieces together we have
%
\begin{equation}\label{eqn:desaiCh2Problems:204}
\expectation{\Delta A \Delta B }^2
\ge
\inv{4} \Abs{\expectation{\antisymmetric{A}{B}}}^2.
\end{equation}
%
With expectation value implied by the absolute squared, this reproduces relation \eqnref{eqn:desaiCh2Problems:200} as desired.

For the remaining part of the problem, with \(\ket{\alpha} = \Delta A \ket{\psi}\), and \(\ket{\beta} = \Delta B \ket{\psi}\), and noting that \((\Delta A)^\dagger = \Delta A\) for Hermitian operator \(A\) (or \(B\) too in this case), the Schwartz inequality
%
\begin{equation}\label{eqn:desaiCh2Problems:205}
\braket{\alpha}{\alpha}\braket{\beta}{\beta}
\ge \Abs{\braket{\beta}{\alpha}}^2,
\end{equation}
%
takes the following form
%
\begin{equation}\label{eqn:desaiCh2Problems:2300}
\bra{\psi}(\Delta A)^\dagger \Delta A \ket{\psi} \bra{\psi}(\Delta B)^\dagger B \ket{\psi}
\ge \Abs{\bra{\psi} (\Delta B)^\dagger A \ket{\psi}}^2.
\end{equation}
%
These are expectation values, and allow us to use \eqnref{eqn:desaiCh2Problems:204} to show
%
\begin{equation}\label{eqn:desaiCh2Problems:2320}
\begin{aligned}
\expectation{(\Delta A)^2 } \expectation{(\Delta B)^2 }
&\ge \Abs{ \expectation{\Delta B \Delta A } }^2 \\
&= \inv{4} \Abs{\expectation{\antisymmetric{B}{A}}}^2.
\end{aligned}
\end{equation}
%
For \(A = X\), and \(B = P\), this is
%
\begin{equation}\label{eqn:desaiCh2Problems:210}
\expectation{(\Delta X)^2 } \expectation{(\Delta P)^2 }
\ge \frac{\Hbar^2}{4}
\end{equation}
%
Hmm.  This does not look like it is quite the result that I expected?  We have \(\expectation{(\Delta X)^2 } \expectation{(\Delta P)^2 }\) instead of \(\expectation{\Delta X }^2 \expectation{\Delta P}^2\)?
%
\index{Schwartz inequality}
\index{uncertainty relation}
Let us step back slightly.  Without introducing the Schwartz inequality the result \eqnref{eqn:desaiCh2Problems:204} of the commutator manipulation, and \(\antisymmetric{X}{P} = i \Hbar\) gives us
%
\begin{equation}\label{eqn:desaiCh2Problems:2340}
\expectation{\Delta X \Delta P }^2
\ge
\frac{\Hbar^2}{4} ,
\end{equation}
%
and taking roots we have
\begin{equation}\label{eqn:desaiCh2Problems:206}
\Abs{\expectation{\Delta X \Delta P } }
\ge
\frac{\Hbar}{2}.
\end{equation}
%
Is this really what we were intended to show?

Attempting to answer this myself, I refer to \citep{liboff2003iqm}, where I find he uses a loose notation for this too, and writes in his equation 3.36
%
\begin{equation}\label{eqn:desaiCh2Problems:2360}
(\Delta C)^2 = \expectation{ (C - \expectation{C})^2 } = \expectation{C^2} - \expectation{C}^2
\end{equation}
%
This usage seems consistent with that, so I think that it is a reasonable assumption that uncertainty relation \(\Delta x \Delta p \ge \Hbar/2\) is really shorthand notation for the more cumbersome relation involving roots of the expectations of mean-square deviation operators
%
\begin{equation}\label{eqn:desaiCh2Problems:230}
\sqrt{\expectation{ (X - \expectation{X})^2 }}
\sqrt{\expectation{ (P - \expectation{P})^2 }} \ge \frac{\Hbar}{2}.
\end{equation}
%
This is in fact what was proved arriving at \eqnref{eqn:desaiCh2Problems:210}.

Ah ha!  Found it.  Referring to equation 2.93 in the text, I see that a lower case notation \(\Delta x = \sqrt{(\Delta X)^2}\), was introduced.  This explains what seemed like ambiguous notation ... it was just tricky notation, perfectly well explained, but done in passing in the text in a somewhat hidden seeming way.
} % answer
\makeoproblem{Hermitian radial differential operator}{problem:desaiCh2Problems:5}{\citep{desai2009quantum} pr 2.5}{
\index{operator!radial differential}
Show that the operator
\begin{equation}\label{eqn:desaiCh2Problems:2380}
R = -i \Hbar \PD{r}{},
\end{equation}
is not Hermitian, and find the constant \(a\) so that
\begin{equation}\label{eqn:desaiCh2Problems:2400}
T = -i \Hbar \left( \PD{r}{} + \frac{a}{r} \right),
\end{equation}
is Hermitian.
} % problem
\makeanswer{problem:desaiCh2Problems:5}{
For the first part of the problem we can show that
\begin{equation}\label{eqn:desaiCh2Problems:2420}
\left( \bra{\psicap} R \ket{\phicap} \right)^\conj \ne \bra{\phicap} R \ket{\psicap}.
\end{equation}
%
For the RHS we have
%
\begin{equation}\label{eqn:desaiCh2Problems:2440}
\bra{\phicap} R \ket{\psicap}
= -i \Hbar \iiint dr d\theta d\phi r^2 \sin\theta \phicap^\conj \PD{r}{\psicap}
\end{equation}
%
and for the LHS we have
%
\begin{equation}\label{eqn:desaiCh2Problems:2460}
\begin{aligned}
\left( \bra{\psicap} R \ket{\phicap} \right)^\conj
&= i \Hbar \iiint dr d\theta d\phi r^2 \sin\theta \psicap \PD{r}{\phicap^\conj} \\
&= -i \Hbar \iiint dr d\theta d\phi \sin\theta
\left( 2 r \psicap
+ r^2 \PD{\psicap}{r}
\right)
\phicap^\conj
\\
\end{aligned}
\end{equation}
%
So, unless \(r\psicap = 0\), the operator \(R\) is not Hermitian.

Moving on to finding the constant \(a\) such that \(T\) is Hermitian we calculate
%
\begin{equation}\label{eqn:desaiCh2Problems:2480}
\begin{aligned}
\left( \bra{\psicap} T \ket{\phicap} \right)^\conj
&= i \Hbar \iiint dr d\theta d\phi r^2 \sin\theta \psicap \left( \PD{r}{} + \frac{a}{r} \right) \phicap^\conj \\
&= i \Hbar \iiint dr d\theta d\phi \sin\theta \psicap \left( r^2 \PD{r}{} + a r \right) \phicap^\conj \\
&= -i \Hbar \iiint dr d\theta d\phi \sin\theta \left( r^2 \PD{r}{\psicap} + 2 r \psicap - a r \psicap \right) \phicap^\conj \\
\end{aligned}
\end{equation}
%
and
%
\begin{equation}\label{eqn:desaiCh2Problems:2500}
\bra{\phicap} T \ket{\psicap}
= -i \Hbar \iiint dr d\theta d\phi r^2 \sin\theta \phicap^\conj \left( r^2 \PD{r}{\psicap} + a r \psicap \right)
\end{equation}
%
So, for \(T\) to be Hermitian, we require
%
\begin{equation}\label{eqn:desaiCh2Problems:2520}
2 r - a r = a r.
\end{equation}
%
So \(a = 1\), and our Hermitian operator is
\begin{equation}\label{eqn:desaiCh2Problems:2540}
T = -i \Hbar \left( \PD{r}{} + \frac{1}{r} \right).
\end{equation}
} % answer
\makeoproblem{Radial directional derivative operator}{problem:desaiCh2Problems:6}{\citep{desai2009quantum} pr 2.6}{
\index{operator!radial directional derivative}
Show that
\begin{equation}\label{eqn:desaiCh2Problems:2560}
D = \Bp \cdot \rcap + \rcap \cdot \Bp,
\end{equation}
is Hermitian.  Expand this operator in spherical coordinates.  Compare result to problem 5.
} % problem
\makeanswer{problem:desaiCh2Problems:6}{
Tackling the spherical coordinates expression of the operator \(D\), we have
\begin{equation}\label{eqn:desaiCh2Problems:2580}
\begin{aligned}
\inv{-i\Hbar} D \Psi
&= \left( \spacegrad \cdot \rcap + \rcap \cdot \spacegrad \right) \Psi \\
&=
\left( \spacegrad \cdot \rcap \right) \Psi
+ \left( \spacegrad \Psi \right) \cdot \rcap
+ \rcap \cdot \left(\spacegrad \Psi\right) \\
&=
\left( \spacegrad \cdot \rcap \right) \Psi
+ 2 \rcap \cdot \left( \spacegrad \Psi \right).
\end{aligned}
\end{equation}
%
Here braces have been used to denote the extend of the operation of the gradient.  In spherical polar coordinates, our gradient is
%
\begin{equation}\label{eqn:desaiCh2Problems:2600}
\spacegrad \equiv
\rcap \PD{r}{}
+\thetacap \inv{r} \PD{\theta}{}
+\phicap \inv{r \sin\theta} \PD{\phi}{}.
\end{equation}
%
This gets us most of the way there, and we have
%
\begin{equation}\label{eqn:desaiCh2Problems:2620}
\inv{-i\Hbar} D \Psi
=
2 \PD{r}{\Psi}
+
\left(
\rcap \cdot \PD{r}{\rcap}
+\inv{r} \thetacap \cdot \PD{\theta}{\rcap}
+\inv{r \sin\theta} \phicap \cdot \PD{\phi}{\rcap}
\right) \Psi.
\end{equation}
%
Since \(\PDi{r}{\rcap} = 0\), we are left with evaluating \(\thetacap \cdot \PDi{\theta}{\rcap}\), and \(\phicap \cdot \PDi{\phi}{\rcap}\).  To do so I chose to employ the (Geometric Algebra) exponential form of the spherical unit vectors \citep{gabookI:sphericalPolarUnit}
%
\begin{equation}\label{eqn:desaiCh2Problems:2640}
\begin{aligned}
I &= \Be_1 \Be_2 \Be_3 \\
\phicap &= \Be_{2} \exp( I \Be_3 \phi ) \\
\rcap &= \Be_3 \exp( I \phicap \theta ) \\
\thetacap &= \Be_1 \Be_2 \phicap \exp( I \phicap \theta ).
\end{aligned}
\end{equation}
%
The partials of interest are then
%
\begin{equation}\label{eqn:desaiCh2Problems:2660}
\PD{\theta}{\rcap} = \Be_3 I \phicap \exp( I \phicap \theta ) = \thetacap,
\end{equation}
%
and
%
\begin{equation}\label{eqn:desaiCh2Problems:2680}
\begin{aligned}
\PD{\phi}{\rcap}
&= \PD{\phi}{} \Be_3 \left( \cos\theta + I \phicap \sin\theta \right) \\
&= \Be_1 \Be_2 \sin\theta \PD{\phi}{\phicap} \\
&= \Be_1 \Be_2 \sin\theta \Be_2 \Be_1 \Be_2 \exp( I \Be_3 \phi ) \\
&= \sin\theta \phicap.
\end{aligned}
\end{equation}
%
Only after computing these, did I find exactly these results for the partials of interest, in \href{http://mathworld.wolfram.com/SphericalCoordinates.html}{mathworld's Spherical Coordinates page}, which confirms these calculations.  Note that a different angle convention is used there, so one has to exchange \(\phi\), and \(\theta\) and the corresponding unit vector labels.

Substitution back into our expression for the operator we have
\begin{equation}\label{eqn:desaiCh2Problems:2700}
D = - 2 i \Hbar \left( \PD{r}{} + \inv{r} \right),
\end{equation}
%
an operator that is exactly twice the operator of problem 5, already shown to be Hermitian.  Since the constant numerical scaling of a Hermitian operator leaves it Hermitian, this shows that \(D\) is Hermitian as expected.
%
\paragraph{\(\thetacap\) directional momentum operator}
\index{momentum operator}
%
Let us try this for the other unit vector directions too.  We also want
%
\begin{equation}\label{eqn:desaiCh2Problems:2720}
\left( \spacegrad \cdot \thetacap + \thetacap \cdot \spacegrad \right) \Psi
=
2 \thetacap \cdot (\spacegrad \Psi) + \left( \spacegrad \cdot \thetacap \right) \Psi.
\end{equation}
%
The work consists of evaluating
%
\begin{equation}\label{eqn:desaiCh2Problems:2740}
\spacegrad \cdot \thetacap
= \rcap \cdot \PD{r}{\thetacap}
+ \inv{r} \thetacap \cdot \PD{\theta}{\thetacap}
+ \inv{r \sin\theta} \phicap \cdot \PD{\phi}{\thetacap}.
\end{equation}
%
This time we need the \(\PDi{\theta}{\thetacap}\), \(\PDi{\phi}{\thetacap}\) partials, which are
%
\begin{equation}\label{eqn:desaiCh2Problems:2760}
\begin{aligned}
\PD{\theta}{\thetacap}
&=
\Be_1 \Be_2 \phicap I \phicap \exp( I \phicap \theta) \\
&=
-\Be_3 \exp( I \phicap \theta) \\
&=
- \rcap.
\end{aligned}
\end{equation}
%
This has no \(\thetacap\) component, so does not contribute to \(\spacegrad \cdot \thetacap\).  Noting that
%
\begin{equation}\label{eqn:desaiCh2Problems:2780}
\PD{\phi}{\phicap} = -\Be_1 \exp( I \Be_3 \phi ) = \Be_2 \Be_1 \phicap,
\end{equation}
%
the \(\phi\) partial is
%
\begin{equation}\label{eqn:desaiCh2Problems:2800}
\begin{aligned}
\PD{\phi}{\thetacap} &=
\Be_1 \Be_2 \left(
\PD{\phi}{\phicap} \exp( I \phicap \theta )
+\phicap I \sin\theta \PD{\phi}{\phicap}
\right) \\
&=
\phicap
\left(
\exp( I \phicap \theta )
+I \sin\theta \Be_2 \Be_1 \phicap
\right),
\end{aligned}
\end{equation}
%
with \(\phicap\) component
\begin{equation}\label{eqn:desaiCh2Problems:2820}
\begin{aligned}
\phicap \cdot \PD{\phi}{\thetacap} &=
\gpgradezero{
\exp( I \phicap \theta )
+I \sin\theta \Be_2 \Be_1 \phicap } \\
&=
\cos\theta + \Be_3 \cdot \phicap \sin\theta \\
&=
\cos\theta.
\end{aligned}
\end{equation}
%
Assembling the results, and labeling this operator \(\Theta\) we have
%
\begin{equation}\label{eqn:desaiCh2Problems:2840}
\begin{aligned}
\Theta &\equiv \inv{2} \left( \Bp \cdot \thetacap + \thetacap \cdot \Bp \right)  \\
&=
-i \Hbar \inv{r} \left( \PD{\theta}{} + \inv{2} \cot\theta \right).
\end{aligned}
\end{equation}
%
It would be reasonable to expect this operator to also be Hermitian, and checking this explicitly by comparing
\(\bra{\Phi} \Theta \ket{\Psi}^\conj\) and \(\bra{\Psi} \Theta \ket{\Phi}\), shows that this is in fact the case.
%
\paragraph{\(\phicap\) directional momentum operator}
%
Let us try this for the other unit vector directions too.  We also want
%
\begin{equation}\label{eqn:desaiCh2Problems:2860}
\left( \spacegrad \cdot \phicap + \phicap \cdot \spacegrad \right) \Psi
=
2 \phicap \cdot (\spacegrad \Psi) + \left( \spacegrad \cdot \phicap \right) \Psi.
\end{equation}
%
The work consists of evaluating
%
\begin{equation}\label{eqn:desaiCh2Problems:2880}
\spacegrad \cdot \phicap
= \rcap \cdot \PD{r}{\phicap}
+ \inv{r} \thetacap \cdot \PD{\theta}{\phicap}
+ \inv{r \sin\theta} \phicap \cdot \PD{\phi}{\phicap}.
\end{equation}
%
This time we need the \(\PDi{\theta}{\phicap}\), \(\PDi{\phi}{\phicap} = \Be_2 \Be_1 \phicap\) partials.  The \(\theta\) partial is
%
\begin{equation}\label{eqn:desaiCh2Problems:2900}
\begin{aligned}
\PD{\theta}{\phicap}
&=
\PD{\theta}{} \Be_2 \exp( I \Be_3 \phi ) \\
&= 0.
\end{aligned}
\end{equation}
%
We conclude that \(\spacegrad \cdot \phicap = 0\), and expect that we have one more Hermitian operator
%
\begin{equation}\label{eqn:desaiCh2Problems:2920}
\begin{aligned}
\Phi &\equiv \inv{2} \left( \Bp \cdot \phicap + \phicap \cdot \Bp \right)  \\
&=
-i \Hbar \inv{r \sin\theta} \PD{\phi}{}.
\end{aligned}
\end{equation}
%
It is simple to confirm that this is Hermitian since the integration by parts does not involve any of the volume element.  In fact, any operator \(-i\Hbar f(r,\theta) \PDi{\phi}{}\) would also be Hermitian, including the simplest case \(-i\Hbar \PDi{\phi}{}\).  Have to dig out my Bohm text again, since I seem to recall that one used in the spherical Harmonics chapter.
%
\paragraph{A note on the Hermitian test and Dirac notation}
\index{Hermitian test}
\index{Dirac notation}
%
I have been a bit loose with my notation.  I have stated that my demonstrations of the Hermitian nature have been done by showing
%
\begin{equation}\label{eqn:desaiCh2Problems:2940}
\bra{\phi} A \ket{\psi}^\conj - \bra{\psi} A \ket{\phi} = 0.
\end{equation}
%
However, what I have actually done is show that
%
\begin{equation}\label{eqn:desaiCh2Problems:2960}
\left( \int d^3 \Bx \phi^\conj (\Bx) A(\Bx) \psi(\Bx) \right)^\conj - \int d^3 \Bx \psi^\conj (\Bx) A(\Bx) \phi(\Bx) = 0.
\end{equation}
%
To justify this note that
%
\begin{equation}\label{eqn:desaiCh2Problems:2980}
\begin{aligned}
\bra{\phi} A \ket{\psi}^\conj
&=
\left( \iint d^3 \Br d^3 \Bs \braket{\phi}{\Br} \bra{\Br} A \ket{\Bs} \braket{\Bs}{\psi} \right)^\conj \\
&=
\iint d^3 \Br d^3 \Bs \phi(\Br) \delta^3(\Br - \Bs) A^\conj(\Bs) \psi(\Bs) \\
&=
\int d^3 \Br \phi(\Br) A^\conj(\Br) \psi(\Br),
\end{aligned}
\end{equation}
%
and
\begin{equation}\label{eqn:desaiCh2Problems:3000}
\begin{aligned}
\bra{\phi} A \ket{\psi}^\conj
&=
\iint d^3 \Br d^3 \Bs \braket{\psi}{\Br} \bra{\Br} A \ket{\Bs} \braket{\Bs}{\phi} \\
&=
\iint d^3 \Br d^3 \Bs \bra{\Br} \psi(\Br) \delta^3(\Br - \Bs) A(\Bs) \phi(\Bs) \\
&=
\int d^3 \Br \psi(\Br) A(\Br) \phi(\Br).
\end{aligned}
\end{equation}
%
Working backwards one sees that the comparison of the wave function integrals in explicit inner product notation is sufficient to demonstrate the Hermitian property.
} % answer
\makeoproblem{Some commutators}{problem:desaiCh2Problems:7}{\citep{desai2009quantum} pr 2.7}{
For \(D\) in problem 6, obtain
\begin{itemize}
\item i) \([D, x_i]\)
\item ii) \([D, p_i]\)
\item iii) \([D, L_i]\), where \(L_i = \Be_i \cdot (\Br \cross \Bp)\).
\item iv) Show that \(e^{i\alpha D/\Hbar} x_i e^{-i\alpha D/\Hbar} = e^\alpha x_i\)
\end{itemize}
} % problem
\makeanswer{problem:desaiCh2Problems:7}{
\paragraph{Expansion of \(\antisymmetric{D}{x_i}\)}
While expressing the operator as \(D = -2 i \Hbar (1/r) (1 + \partial_r)\) has less complexity than the \(D = \Bp \cdot \rcap + \rcap \cdot \Bp\), since no operation on \(\rcap\) is required, this does not look particularly convenient for use with Cartesian coordinates.  Slightly better perhaps is
%
\begin{equation}\label{eqn:desaiCh2Problems:3020}
D = -2 i\Hbar \inv{r}( \Br \cdot \spacegrad + 1)
\end{equation}
%
\begin{equation}\label{eqn:desaiCh2Problems:3040}
\begin{aligned}
[D, x_i] \Psi
&=
D x_i \Psi - x_i D \Psi \\
&=
-2 i \Hbar \inv{r} \left( \Br \cdot \spacegrad + 1 \right) x_i \Psi
+2 i \Hbar x_i \inv{r} \left( \Br \cdot \spacegrad + 1 \right) \Psi \\
&=
-2 i \Hbar \inv{r} \Br \cdot \spacegrad x_i \Psi
+2 i \Hbar x_i \inv{r} \Br \cdot \spacegrad \Psi \\
&=
-2 i \Hbar \inv{r} \Br \cdot (\spacegrad x_i) \Psi
-2 i \Hbar x_i \inv{r} \Br \cdot \spacegrad \Psi
+2 i \Hbar x_i \inv{r} \Br \cdot \spacegrad \Psi \\
&=
-2 i \Hbar \inv{r} \Br \cdot \Be_i \Psi.
\end{aligned}
\end{equation}
%
So this first commutator is:
%
\begin{equation}\label{eqn:desaiCh2Problems:3060}
[D, x_i] = -2 i \Hbar \frac{x_i}{r}.
\end{equation}
%
\paragraph{Alternate expansion of \(\antisymmetric{D}{x_i}\)}
%
Let us try this instead completely in coordinate notation to verify.  I will use implicit summation for repeated indices, and write \(\partial_k = \partial/\partial x_k\).  A few intermediate results will be required
%
\begin{equation}\label{eqn:desaiCh2Problems:3080}
\begin{aligned}
\partial_k \inv{r}
&= \partial_k (x_m x_m)^{-1/2}  \\
&= -\inv{2} 2 x_k (x_m x_m)^{-3/2}  \\
\end{aligned}
\end{equation}
%
Or
\begin{equation}\label{eqn:desaiCh2Problems:1000}
\partial_k \inv{r}
= - \frac{x_k}{r^3}
\end{equation}
%
\begin{equation}\label{eqn:desaiCh2Problems:1001}
\partial_k \frac{x_i}{r}
=
\frac{\delta_{ik}}{r} - \frac{ x_i }{r^3}
\end{equation}
%
\begin{equation}\label{eqn:desaiCh2Problems:1002}
\partial_k \frac{x_k}{r}
=
\frac{3}{r} - \frac{ x_k }{r^3}
\end{equation}
%
%\begin{align*}
%p_k \frac{x_k}{r} \Psi
%&=
%-i \Hbar \partial_k \frac{x_k \Psi}{r} \\
%&=
%-i \Hbar \left(
%\frac{3}{r} - \frac{ x_k}{r^3} + \frac{x_k}{r} \partial_k
%\right)
%\Psi
%\end{align*}

The action of the momentum operators on the coordinates is
%
\begin{equation}\label{eqn:desaiCh2Problems:3100}
\begin{aligned}
p_k x_i \Psi
&=
-i \Hbar \partial_k x_i \Psi \\
&=
-i \Hbar \left( \delta_{ik} + x_i \partial_k \right) \Psi \\
&=
-i \Hbar \delta_{ik} + x_i p_k
\end{aligned}
\end{equation}
%
\begin{equation}\label{eqn:desaiCh2Problems:3120}
\begin{aligned}
p_k x_k \Psi
&=
-i \Hbar \partial_k x_k \Psi \\
&=
-i \Hbar \left( 3 + x_k \partial_k \right) \Psi
\end{aligned}
\end{equation}
%
Or
%
\begin{equation}\label{eqn:desaiCh2Problems:1003}
\begin{aligned}
p_k x_i &= -i \Hbar \delta_{ik} + x_i p_k \\
p_k x_k &= - 3 i \Hbar + x_k p_k
\end{aligned}
\end{equation}
%
And finally
%
\begin{equation}\label{eqn:desaiCh2Problems:3140}
\begin{aligned}
p_k \inv{r} \Psi
&=
(p_k \inv{r}) \Psi
+ \inv{r} p_k \Psi \\
&=
-i \Hbar \left(
-\frac{x_k}{r^3}
\right) \Psi
+ \inv{r} p_k \Psi \\
\end{aligned}
\end{equation}
%
So
\begin{equation}\label{eqn:desaiCh2Problems:1004}
p_k \inv{r} = i \Hbar \frac{x_k}{r^3} + \inv{r}p_k
\end{equation}
%
We can use these to rewrite \(D\)
%
\begin{equation}\label{eqn:desaiCh2Problems:3160}
\begin{aligned}
D
&= p_k \frac{x_k}{r} + \frac{x_k}{r} p_k \\
&= p_k x_k \inv{r} + \frac{x_k}{r} p_k \\
&= \left( - 3 i \Hbar + x_k p_k \right)\inv{r} + \frac{x_k}{r} p_k \\
&= - \frac{3 i \Hbar}{r} + x_k \left( i \Hbar \frac{x_k}{r^3} + \inv{r}p_k \right) + \frac{x_k}{r} p_k \\
\end{aligned}
\end{equation}
%
\begin{equation}\label{eqn:desaiCh2Problems:2000}
D = \frac{2}{r} ( -i \Hbar + x_k p_k )
\end{equation}
%
This leaves us in the position to compute the commutator
%
\begin{equation}\label{eqn:desaiCh2Problems:3180}
\begin{aligned}
\antisymmetric{D}{x_i}
&= \frac{2}{r} ( -i \Hbar + x_k p_k ) x_i
- \frac{2 x_i}{r} ( -i \Hbar + x_k p_k ) \\
&= \frac{2}{r} x_k ( -i \Hbar \delta_{ik} + x_i p_k )
- \frac{2 x_i}{r} x_k p_k \\
&= -\frac{2 i \Hbar x_i}{r}
\end{aligned}
\end{equation}
%
So, unless I am doing something fundamentally wrong, the same way in both methods, this appears to be the desired result.  I question my answer since utilizing this for the later computation of \(e^{i\alpha D/\Hbar} x_i e^{-i\alpha D/\Hbar}\) did not yield the expected answer.
%
\paragraph{\(\symmetric{D}{p_i}\)}
%
\begin{equation}\label{eqn:desaiCh2Problems:3200}
\begin{aligned}
\antisymmetric{D}{p_i}
&=
-\frac{2 i \Hbar }{r} ( 1 + x_k p_k ) p_i
+2 i \Hbar p_i \inv{r} ( 1 + x_k p_k )  \\
&=
-\frac{2 i \Hbar }{r}
\left(
p_i + x_k p_k p_i
-
\left( i \Hbar \frac{x_i}{r^2} + p_i \right) ( 1 + x_k p_k )
\right) \\
&=
-\frac{2 i \Hbar }{r}
\left(
x_k p_k p_i
- i \Hbar \frac{x_i}{r^2}
- i \Hbar \frac{x_i x_k}{r^2} p_k
-
\left( -i \Hbar \delta_{ki} + x_k p_i \right) p_k
\right) \\
&=
-\frac{2 i \Hbar }{r}
\left(
- i \Hbar \frac{x_i}{r^2}
- i \Hbar \frac{x_i x_k}{r^2} p_k
+ i \Hbar p_i
\right) \\
&=
-\frac{i \Hbar}{r}
\left(
\frac{x_i}{r} D
+ 2 i \Hbar p_i
\right)
\qedmarker
\end{aligned}
\end{equation}
%
If there is some significance to this expansion, other than to get a feel for operator manipulation, it escapes me.
%
\paragraph{\(\symmetric{D}{L_i}\)}
%
To expand \([D, L_i]\), it will be sufficient to consider any specific index \(i \in \{1,2,3\}\) and then utilize cyclic permutation of the indices in the result to generalize.  Let us pick \(i=1\), for which we have
%
\begin{equation}\label{eqn:desaiCh2Problems:3220}
L_1 = x_2 p_3 - x_3 p_2
\end{equation}
%
It appears we will want to know
%
\begin{equation}\label{eqn:desaiCh2Problems:3240}
\begin{aligned}
p_m D
&=
-2 i \Hbar p_m \inv{r} ( 1 + x_k p_k ) \\
&=
-2 i \Hbar
\left(
i \Hbar \frac{x_m}{r^3} + \inv{r}p_m
\right)
( 1 + x_k p_k ) \\
&=
-2 i \Hbar \left(
i \Hbar \frac{x_m}{r^3} + \inv{r}p_m
+i \Hbar \frac{x_m x_k }{r^3} p_k + \inv{r}p_m x_k p_k
\right) \\
&=
-\frac{2 i \Hbar}{r} \left(
i \Hbar \frac{x_m}{r^2} + p_m
+i \Hbar \frac{x_m x_k }{r^2} p_k -i \Hbar p_m + x_k p_m p_k
\right)
\end{aligned}
\end{equation}
%
and we also want
%
\begin{equation}\label{eqn:desaiCh2Problems:3260}
\begin{aligned}
D x_m
&=
- \frac{2 i \Hbar }{r} ( 1 + x_k p_k ) x_m  \\
&=
- \frac{2 i \Hbar }{r} ( x_m + x_k ( -i \Hbar \delta_{km} + x_m p_k ) ) \\
&=
- \frac{2 i \Hbar }{r} ( x_m - i \Hbar x_m + x_m x_k p_k ) \\
\end{aligned}
\end{equation}
%
This also happens to be \(D x_m = x_m D + \frac{2 (i \Hbar)^2 x_m }{r}\), but does that help at all?

Assembling these we have
%
\begin{equation}\label{eqn:desaiCh2Problems:3280}
\begin{aligned}
\antisymmetric{D}{L_1}
&=
D x_2 p_3 - D x_3 p_2 - x_2 p_3 D + x_3 p_2 D \\
&=
- \frac{2 i \Hbar }{r} ( x_2 - i \Hbar x_2 + x_2 x_k p_k ) p_3
+ \frac{2 i \Hbar }{r} ( x_3 - i \Hbar x_3 + x_3 x_k p_k ) p_2  \\
&+\frac{2 i \Hbar x_2 }{r} \left(
i \Hbar \frac{x_3}{r^2} + p_3
+i \Hbar \frac{x_3 x_k }{r^2} p_k -i \Hbar p_3 + x_k p_3 p_k
\right) \\
&-\frac{2 i \Hbar x_3 }{r} \left(
i \Hbar \frac{x_2}{r^2} + p_2
+i \Hbar \frac{x_2 x_k }{r^2} p_k -i \Hbar p_2 + x_k p_2 p_k
\right) \\
\end{aligned}
\end{equation}
%
With a bit of brute force it is simple enough to verify that all these terms mystically cancel out, leaving us zero
%
\begin{equation}\label{eqn:desaiCh2Problems:3300}
\antisymmetric{D}{L_1} = 0
\end{equation}
%
There surely must be an easier way to demonstrate this.  Likely utilizing the commutator relationships derived earlier.
%
\paragraph{\(e^{i\alpha D/\Hbar} x_i e^{-i\alpha D/\Hbar}\)}
%
We will need to evaluate \(D^k x_i\).  We have the first power from our commutator relation
%
\begin{equation}\label{eqn:desaiCh2Problems:3320}
D x_i = x_i \left( D - \frac{ 2 i \Hbar }{r} \right)
\end{equation}
%
A successive application of this operator therefore yields
%
\begin{equation}\label{eqn:desaiCh2Problems:3340}
\begin{aligned}
D^2 x_i
&= D x_i \left( D - \frac{ 2 i \Hbar }{r} \right) \\
&= x_i \left( D - \frac{ 2 i \Hbar }{r} \right)^2 \\
\end{aligned}
\end{equation}
%
So we have
%
\begin{equation}\label{eqn:desaiCh2Problems:3360}
D^k x_i
= x_i \left( D - \frac{ 2 i \Hbar }{r} \right)^k
\end{equation}
%
This now preps us to expand the first product in the desired exponential sandwich
%
\begin{equation}\label{eqn:desaiCh2Problems:3380}
\begin{aligned}
e^{i\alpha D/\Hbar} x_i
&=
x_i + \sum_{k=1}^\infty \inv{k!} \left( \frac{i D}{\Hbar} \right)^k x_i \\
&=
x_i + \sum_{k=1}^\infty \inv{k!} \left( \frac{i}{\Hbar} \right)^k D^k x_i \\
&=
x_i + \sum_{k=1}^\infty \inv{k!} \left( \frac{i}{\Hbar} \right)^k x_i  \\
&= x_i e^{ \frac{i \alpha }{\Hbar} \left( D - \frac{ 2 i \Hbar }{r} \right) } \\
&= x_i e^{ 2 \alpha /r } e^{ i \alpha D /\Hbar }.
\end{aligned}
\end{equation}
%
The exponential sandwich then produces
%
\begin{equation}\label{eqn:desaiCh2Problems:3400}
e^{i\alpha D/\Hbar} x_i e^{-i\alpha D/\Hbar} = e^{2 \alpha/r } x_i
\end{equation}
%
Note that this is not the value we are supposed to get.  Either my value for \(D x_i\) is off by a factor of \(2/r\) or the problem in the text contains a typo.

%%%%For the second power we have
%%%%\begin{align*}
%%%%D^2 x_i
%%%%&= D \left(x_i D - \frac{ 2 i \Hbar x_i }{r} \right) \\
%%%%&= ( D x_i ) D - 2 i \Hbar D \frac{ x_i }{r} \\
%%%%&= \left(x_i D - \frac{2 i \Hbar x_i}{r} \right) D - 2 i \Hbar \left( x_i D - \frac{2 i \Hbar x_i}{r}\right) \frac{ 1 }{r} \\
%%%%\end{align*}
%%%%
%%%%To procede we require \(D (1/r)\), which is
%%%%\begin{align*}
%%%%D \inv{r}
%%%%&=
%%%%- \frac{2 i \Hbar}{r} (1 + x_k p_k ) \inv{r} \\
%%%%&=
%%%%- \frac{2 i \Hbar}{r} \left(\inv{r} + x_k \left( \frac{i \Hbar x_k}{r^3} + \inv{r} p_k  \right) \right) \\
%%%%&=
%%%%- \frac{2 i \Hbar}{r^2} \left(1 + i \Hbar + x_k p_k \right) \\
%%%%&=
%%%%\inv{r} D - \frac{2 (i \Hbar)^2}{r^2}
%%%%\end{align*}
%%%%
%%%%Back subst we have:
%%%%\begin{align*}
%%%%D^2 x_i
%%%%&= x_i D^2 - 2 \frac{2 i \Hbar x_i}{r} D
%%%%- 2 i \Hbar \left(
%%%%- \frac{2 (i \Hbar)^2 x_i }{r^2}
%%%%- \frac{2 i \Hbar x_i}{r^2}
%%%%\right) \\
%%%%&= x_i D^2 - 2 \frac{2 i \Hbar x_i}{r} D
%%%%- 2 i \Hbar \left(
%%%%- \frac{2 (i \Hbar)^2 x_i }{r^2}
%%%%- \frac{2 i \Hbar x_i}{r^2}
%%%%\right) \\
%%%%&= x_i D^2 - 2 \frac{2 i \Hbar x_i}{r} D
%%%%+ (2) \frac{2 (i \Hbar)^2 x_i}{r^2} \left(
%%%%1 + i \Hbar
%%%%\right) \\
%%%%&= x_i \left( D^2 - 2 \frac{2 i \Hbar }{r} D
%%%%+ \frac{(2 i \Hbar)^2 }{r^2} \left(
%%%%1 + i \Hbar
%%%%\right) \right) \\
%%%%\end{align*}
} % answer
%\makeoproblem{Reduction of some commutators using the fundamental commutator relation}{problem:desaiCh2Problems:8}{\citep{desai2009quantum} pr 2.8}
\makeoproblem{Fundamental commutator relation.}{problem:desaiCh2Problems:8}{\citep{desai2009quantum} pr 2.8}{
Using the fundamental commutation relation
\begin{equation}\label{eqn:desaiCh2Problems:3420}
\begin{aligned}
\antisymmetric{p}{x} = -i \Hbar,
\end{aligned}
\end{equation}
which we can also write as
\begin{equation}\label{eqn:desaiCh2Problems:3440}
\begin{aligned}
p x = x p -i \Hbar,
\end{aligned}
\end{equation}
expand \(\antisymmetric{x}{p^2}\), \(\antisymmetric{x^2}{p}\), and \(\antisymmetric{x^2}{p^2}\).
} % problem
\makeanswer{problem:desaiCh2Problems:8}{
The first is
\begin{equation}\label{eqn:desaiCh2Problems:3460}
\begin{aligned}
\antisymmetric{x}{p^2}
&= x p^2 - p^2 x \\
&= x p^2 - p (p x) \\
&= x p^2 - p (x p -i \Hbar) \\
&= x p^2 - (x p -i \Hbar) p + i \Hbar p \\
&= 2 i \Hbar p \\
\end{aligned}
\end{equation}
%
The second is
\begin{equation}\label{eqn:desaiCh2Problems:3480}
\begin{aligned}
\antisymmetric{x^2}{p}
&= x^2 p - p x^2 \\
&= x^2 p - (x p - i\Hbar) x \\
&= x^2 p - x (x p - i\Hbar) + i \Hbar x \\
&= 2 i \Hbar x \\
\end{aligned}
\end{equation}
%
Note that it is helpful for the last reduction of this problem to observe that we can write this as
%
\begin{equation}\label{eqn:desaiCh2Problems:3500}
p x^2 = x^2 p - 2 i \Hbar x
\end{equation}
%
Finally for this last we have
%
\begin{equation}\label{eqn:desaiCh2Problems:3520}
\begin{aligned}
\antisymmetric{x^2}{p^2}
&= x^2 p^2 - p^2 x^2 \\
&= x^2 p^2 - p (x^2 p - 2 i \Hbar x) \\
&= x^2 p^2 - (x^2 p - 2 i \Hbar x) p + 2 i \Hbar (x p - i \Hbar) \\
&= 4 i \Hbar x p - 2 (i \Hbar)^2 \\
\end{aligned}
\end{equation}
%
That is about as reduced as this can be made, but it is not very tidy looking.  From this point we can simplify it a bit by factoring
%
\begin{equation}\label{eqn:desaiCh2Problems:3540}
\begin{aligned}
\antisymmetric{x^2}{p^2}
&= 4 i \Hbar x p - 2 (i \Hbar)^2 \\
&= 2 i \Hbar ( 2 x p - i \Hbar) \\
&= 2 i \Hbar ( x p + p x ) \\
&= 2 i \Hbar \symmetric{x}{p}
\end{aligned}
\end{equation}
%
} % answer
%
\makeoproblem{Finite displacement operator}{problem:desaiCh2Problems:9}{\citep{desai2009quantum} pr 2.9}{
Consider the operator which corresponds to finite displacement
\begin{equation}\label{eqn:desaiCh2Problems:3560a}
F(d) = e^{-i p d/\Hbar}.
\end{equation}
%
Show that
%
\begin{equation}\label{eqn:desaiCh2Problems:3560b}
\antisymmetric{x}{F(d)} = d F(d).
\end{equation}
%
If for a state \( \ket{\alpha_d} = F(d) \ket{\alpha} \), then show that the expectation value with respect to the two states satisfy
%
\begin{equation}\label{eqn:desaiCh2Problems:3560c}
\expectation{x}_d = \expectation{x} + d.
\end{equation}
} % problem
%
\makeanswer{problem:desaiCh2Problems:9}{
\paragraph{Part I}
%
For
%
\begin{equation}\label{eqn:desaiCh2Problems:3560}
F(d) = e^{-i p d/\Hbar},
\end{equation}
%
the first part of this problem is to show that
%
\begin{equation}\label{eqn:desaiCh2Problems:3580}
\antisymmetric{x}{F(d)} = x F(d) - F(d) x = d F(d)
\end{equation}
%
We need to evaluate
\begin{equation}\label{eqn:desaiCh2Problems:3600}
e^{-i p d/\Hbar} x = \sum_{k=0}^\infty \inv{k!} \left( \frac{-i p d}{\Hbar} \right)^k x.
\end{equation}
%
To do so requires a reduction of \(p^k x\).  For \(k=2\) we have
%
\begin{equation}\label{eqn:desaiCh2Problems:3620}
\begin{aligned}
p^2 x
&= p ( x p - i\Hbar ) \\
&= ( x p - i\Hbar ) p - i \Hbar p \\
&= x p^2 - 2 i\Hbar p.
\end{aligned}
\end{equation}
%
For the cube we get \(p^3 x = x p^3 - 3 i\Hbar p^2\), supplying confirmation of an induction hypothesis \(p^k x = x p^k - k i\Hbar p^{k-1}\), which can be verified
%
\begin{equation}\label{eqn:desaiCh2Problems:3640}
\begin{aligned}
p^{k+1} x
&= p ( x p^k - k i \Hbar p^{k-1}) \\
&= (x p - i\Hbar) p^k - k i \Hbar p^k \\
&= x p^{k+1} - (k+1) i \Hbar p^k \qedmarker
\end{aligned}
\end{equation}
%
For our exponential we then have
\begin{equation}\label{eqn:desaiCh2Problems:3660}
\begin{aligned}
e^{-i p d/\Hbar} x
&= x + \sum_{k=1}^\infty \inv{k!} \left( \frac{-i d}{\Hbar} \right)^k (x p^k - k i\Hbar p^{k-1}) \\
&= x e^{-i p d /\Hbar }
+ \sum_{k=1}^\infty \inv{(k-1)!} \left( \frac{-i p d}{\Hbar} \right)^{k-1} (-i d/\Hbar)(- i\Hbar) \\
&= ( x - d ) e^{-i p d /\Hbar }.
\end{aligned}
\end{equation}
%
Put back into our commutator we have
\begin{equation}\label{eqn:desaiCh2Problems:3680}
\antisymmetric{x}{e^{-i p d/\Hbar}} = d e^{-ip d/\Hbar},
\end{equation}
%
completing the proof.
%
\paragraph{Part II}
%
For state \(\ket{\alpha}\) with \(\ket{\alpha_d} = F(d) \ket{\alpha}\), show that the expectation values satisfy
%
\begin{equation}\label{eqn:desaiCh2Problems:3700}
\expectation{X}_d = \expectation{X} + d
\end{equation}
%
\begin{equation}\label{eqn:desaiCh2Problems:3720}
\begin{aligned}
\expectation{X}_d
&=
\bra{\alpha_d} X \ket{\alpha_d} \\
&=
\iint dx' dx'' \braket{\alpha_d}{x'} \bra{x'} X \ket{x''} \braket{x''}{\alpha_d} \\
&=
\iint dx' dx'' \alpha_d^\conj{x'} \delta(x' -x'') x' \alpha_d(x'') \\
&=
\int dx' \alpha_d^\conj(x') x' \alpha_d(x') \\
\end{aligned}
\end{equation}
%
But
\begin{equation}\label{eqn:desaiCh2Problems:3740}
\begin{aligned}
\alpha_d(x')
&= \exp\left( -\frac{i d }{\Hbar} (-i\Hbar) \frac{\partial}{\partial x'} \right) \alpha(x') \\
&= e^{- d \frac{\partial}{\partial x'} } \alpha(x') \\
&= \alpha(x' - d),
\end{aligned}
\end{equation}
%
so our position expectation is
\begin{equation}\label{eqn:desaiCh2Problems:3760}
\expectation{X}_d
=
\int dx' \alpha^\conj(x' -d) x' \alpha(x'- d).
\end{equation}
%
A change of variables \(x = x' -d\) gives us
\begin{equation}\label{eqn:desaiCh2Problems:3780}
\begin{aligned}
\expectation{X}_d
&=
\int dx \alpha^\conj(x) (x + d) \alpha(x) \\
%&=
%\int dx \alpha^\conj(x) x \alpha(x)
%+ d \int dx \alpha^\conj{x} \alpha(x) \\
%&=
\expectation{X} + d
+ d \int dx \alpha^\conj{x} \alpha(x) \qedmarker
\end{aligned}
\end{equation}
} % answer
\makeoproblem{Hamiltonian commutators}{problem:desaiCh2Problems:10}{\citep{desai2009quantum} pr 2.10}{
\index{commutator!Hamiltonian}
For
\begin{equation}\label{eqn:desaiCh2Problems:3800}
H = \inv{2m} p^2 + V(x)
\end{equation}
calculate \(\antisymmetric{H}{x}\), and \(\antisymmetric{\antisymmetric{H}{x}}{x}\).
} % problem
\makeanswer{problem:desaiCh2Problems:10}{
These are
\begin{equation}\label{eqn:desaiCh2Problems:3820}
\begin{aligned}
\antisymmetric{H}{x}
&=
\inv{2m} p^2 x + V(x) x -\inv{2m} x p^2 - x V(x)  \\
&=
\inv{2m} p ( x p - i \Hbar) -\inv{2m} x p^2 \\
&=
\inv{2m} \left( ( x p - i \Hbar) p -i \Hbar p \right) -\inv{2m} x p^2 \\
&=
-\frac{i\Hbar p}{m} \\
\end{aligned}
\end{equation}
and
\begin{equation}\label{eqn:desaiCh2Problems:3840}
\begin{aligned}
\antisymmetric{\antisymmetric{H}{x}}{x}
&=
-\frac{i\Hbar }{m} \antisymmetric{p}{x} \\
&=
\frac{(-i\Hbar)^2 }{m} \\
&=
-\frac{\Hbar^2 }{m} \\
\end{aligned}
\end{equation}
%
We also have to show that
%
\begin{equation}\label{eqn:desaiCh2Problems:3860}
\sum_k (E_k -E_n) \Abs{ \bra{k} x \ket{n} }^2 = \frac{\Hbar^2}{2m}
\end{equation}
%
Expanding the absolute value in terms of conjugates we have
%
\begin{equation}\label{eqn:desaiCh2Problems:3880}
\begin{aligned}
\sum_k (E_k -E_n) \Abs{ \bra{k} x \ket{n} }^2
&=
\sum_k (E_k -E_n) \bra{k} x \ket{n} \bra{n} x \ket{k} \\
&=
\sum_k \bra{k} x \ket{n} \bra{n} x E_k \ket{k}
-\bra{k} x E_n \ket{n} \bra{n} x \ket{k} \\
&=
\sum_k
\bra{n} x H \ket{k} \bra{k} x \ket{n}
- \bra{n} x \ket{k} \bra{k} x H \ket{n} \\
&=
\bra{n} x H x \ket{n}
- \bra{n} x x H \ket{n} \\
&=
\bra{n} x \antisymmetric{H}{x} \ket{n}  \\
&=
-\frac{i \Hbar}{m} \bra{n} x p \ket{n}  \\
\end{aligned}
\end{equation}
%
It is not obvious where to go from here.  Taking the clue from the problem that the result involves the double commutator, we have
%
\begin{equation}\label{eqn:desaiCh2Problems:3900}
\begin{aligned}
- \frac{\Hbar^2}{m}
&=
\bra{n}
\antisymmetric{\antisymmetric{H}{x}}{x} \ket{n} \\
&=
\bra{n}
H x^2 - 2 x H x + x^2 H
\ket{n} \\
&=
2 E_n \bra{n} x^2 \ket{ n} - 2 \bra{n} x H x \ket{n} \\
&=
2 E_n \bra{n} x^2 \ket{ n} - 2 \bra{n} ( -\antisymmetric{H}{x} + H x) x \ket{n} \\
&=
2 \bra{n} \antisymmetric{H}{x} x \ket{n} \\
&=
-\frac{2 i \Hbar}{m} \bra{n} p x \ket{n} \\
&=
-\frac{2 i \Hbar}{m} \bra{n} x p - i \Hbar \ket{n} \\
&=
-\frac{2 i \Hbar}{m} \bra{n} x p \ket{n} +\frac{2 (i \Hbar)^2}{m}
\end{aligned}
\end{equation}
%
So, somewhat flukily by working backwards, with a last rearrangement, we now have
%
\begin{equation}\label{eqn:desaiCh2Problems:3920}
\begin{aligned}
-\frac{i \Hbar}{m} \bra{n} x p \ket{n}
&= \frac{\Hbar^2}{m} -\frac{\Hbar^2}{2 m} \\
&= \frac{\Hbar^2}{2 m}
\end{aligned}
\end{equation}
%
Substitution above gives the desired result.  This is extremely ugly, and does not follow any sort of logical progression.  Is there a good way to sequence this proof?
} % answer

%
\makeoproblem{Another double commutator.}{problem:desaiCh2Problems:11}{\citep{desai2009quantum} pr 2.11}{
FIXME: describe.
} % problem
%
\makeanswer{problem:desaiCh2Problems:11}{
%
\paragraph{Attempt 1.  Incomplete}
%
\begin{equation}\label{eqn:desaiCh2Problems:3940}
H = \frac{\Bp^2}{2m} + V(\Br)
\end{equation}
%
use \(\antisymmetric{\antisymmetric{H}{e^{i \Bk \cdot \Br}}}{e^{-i \Bk \cdot \Br}}\) to obtain
%
\begin{equation}\label{eqn:desaiCh2Problems:3960}
\sum_n (E_n - E_s) \Abs{\bra{n} e^{i\Bk \cdot \Br} \ket{s}}^2
\end{equation}
%
First evaluate the commutators.  The first is
%
\begin{equation}\label{eqn:desaiCh2Problems:3980}
\antisymmetric{H}{ e^{i \Bk \cdot \Br}}
= \inv{2m} \antisymmetric{\Bp^2}{e^{i\Bk \cdot \Br}}
\end{equation}
%
The Laplacian applied to this exponential is
%
\begin{equation}\label{eqn:desaiCh2Problems:4000}
\begin{aligned}
\spacegrad^2 e^{i \Bk \cdot \Br } \Psi
&=
\partial_m \partial_m e^{i k_n x_n } \Psi \\
&=
\partial_m (i k_m e^{i \Bk\cdot \Br} \Psi + e^{i \Bk \cdot \Br } \partial_m \Psi ) \\
&=
- \Bk^2 e^{i \Bk\cdot \Br} \Psi + i e^{i \Bk \cdot \Br } \Bk \cdot \spacegrad \Psi
+ i e^{i \Bk \cdot \Br} \Bk \cdot \spacegrad \Psi
+ e^{i \Bk \cdot \Br} \spacegrad^2 \Psi
\end{aligned}
\end{equation}
%
Factoring out the exponentials this is
%
\begin{equation}\label{eqn:desaiCh2Problems:4020}
\begin{aligned}
\spacegrad^2 e^{i \Bk \cdot \Br } &=
e^{i \Bk\cdot \Br}
\left(
- \Bk^2 + 2 i \Bk \cdot \spacegrad + \spacegrad^2
\right),
\end{aligned}
\end{equation}
%
and in terms of \(\Bp\), we have
%
\begin{equation}\label{eqn:desaiCh2Problems:4040}
\Bp^2
e^{i \Bk\cdot \Br}
=
e^{i \Bk\cdot \Br}
\left(
(\Hbar\Bk)^2 + 2 (\Hbar \Bk) \cdot \Bp + \Bp^2
\right)
=
e^{i \Bk\cdot \Br} (\Hbar \Bk + \Bp)^2
\end{equation}
%
So, finally, our first commutator is
%
\begin{equation}\label{eqn:desaiCh2Problems:1}
\antisymmetric{H}{ e^{i \Bk \cdot \Br}}
%= \inv{2m} \antisymmetric{\Bp^2}{e^{i\Bk \cdot \Br}}
= \inv{2m}
e^{i \Bk\cdot \Br}
\left(
(\Hbar\Bk)^2 + 2 (\Hbar \Bk) \cdot \Bp
\right)
\end{equation}
%
The double commutator is then
%
\begin{equation}\label{eqn:desaiCh2Problems:4060}
\antisymmetric{\antisymmetric{H}{e^{i \Bk \cdot \Br}}}{e^{-i \Bk \cdot \Br}}
=
e^{i\Bk \cdot \Br} \frac{\Hbar \Bk}{m} \cdot \left( \Bp e^{-i \Bk \cdot \Br} - e^{-i \Bk \cdot \Br} \Bp \right)
\end{equation}
%
To simplify this we want
%
\begin{equation}\label{eqn:desaiCh2Problems:4080}
\begin{aligned}
\Bk \cdot \spacegrad e^{-i \Bk \cdot \Br} \Psi
&=
k_n \partial_n e^{-i k_m x_m } \Psi \\
&=
e^{-i \Bk \cdot \Br }
\left(
k_n (-i k_n) \Psi + k_n \partial_n \Psi \right) \\
&=
e^{-i \Bk \cdot \Br } \left( -i \Bk^2 + \Bk \cdot \spacegrad \right) \Psi
\end{aligned}
\end{equation}
%
The double commutator is then left with just
\begin{equation}\label{eqn:desaiCh2Problems:2}
\antisymmetric{\antisymmetric{H}{e^{i \Bk \cdot \Br}}}{e^{-i \Bk \cdot \Br}}
=
- \inv{m} (\Hbar \Bk)^2
\end{equation}
%
Now, returning to the energy expression
%
\begin{equation}\label{eqn:desaiCh2Problems:4100}
\begin{aligned}
\sum_n (E_n - E_s) \Abs{\bra{n} e^{i\Bk \cdot \Br} \ket{s}}^2
&=
\sum_n (E_n - E_s)
\bra{s} e^{-i\Bk \cdot \Br} \ket{n} \bra{n} e^{i\Bk \cdot \Br} \ket{s} \\
&=
\sum_n
\bra{s} e^{-i\Bk \cdot \Br} H \ket{n} \bra{n} e^{i\Bk \cdot \Br} \ket{s}
-\bra{s} e^{-i\Bk \cdot \Br} \ket{n} \bra{n} e^{i\Bk \cdot \Br} H \ket{s} \\
&=
\bra{s} e^{-i\Bk \cdot \Br} H e^{i\Bk \cdot \Br} \ket{s}
-\bra{s} e^{-i\Bk \cdot \Br} e^{i\Bk \cdot \Br} H \ket{s} \\
&=
\bra{s} e^{-i\Bk \cdot \Br} \antisymmetric{H}{e^{i\Bk \cdot \Br}} \ket{s} \\
&=
\inv{2m} \bra{s} e^{-i\Bk \cdot \Br}
e^{i \Bk\cdot \Br}
\left(
(\Hbar\Bk)^2 + 2 (\Hbar \Bk) \cdot \Bp
\right)
\ket{s} \\
&=
\inv{2m} \bra{s} (\Hbar\Bk)^2 + 2 (\Hbar \Bk) \cdot \Bp \ket{s} \\
&=
\frac{(\Hbar\Bk)^2}{2m} + \inv{m} \bra{s} (\Hbar \Bk) \cdot \Bp \ket{s} \\
\end{aligned}
\end{equation}
%
I can not figure out what to do with the \(\Hbar \Bk \cdot \Bp\) expectation, and keep going around in circles.

I figure there is some trick related to the double commutator, so expanding the expectation of that seems appropriate
%
\begin{equation}\label{eqn:desaiCh2Problems:4120}
\begin{aligned}
-\inv{m} (\Hbar \Bk)^2
&=
\bra{s} \antisymmetric{\antisymmetric{H}{e^{i \Bk \cdot \Br}}}{e^{-i \Bk \cdot \Br}} \ket{s} \\
&=
\bra{s}
\antisymmetric{H}{e^{i \Bk \cdot \Br}} e^{-i \Bk \cdot \Br}
-
e^{-i \Bk \cdot \Br}
\antisymmetric{H}{e^{i \Bk \cdot \Br}}
 \ket{s} \\
%&=
%\bra{s}
%2 H - e^{i \Bk \cdot \Br} H e^{-i \Bk \cdot \Br} - e^{-i \Bk \cdot \Br} H e^{i \Bk \cdot \Br}
% \ket{s} \\
&=
\inv{2m } \bra{s}
e^{ i \Bk \cdot \Br} ( (\Hbar \Bk)^2 + 2 \Hbar \Bk \cdot \Bp) e^{-i \Bk \cdot \Br}
-
e^{-i \Bk \cdot \Br} e^{ i \Bk \cdot \Br} ( (\Hbar \Bk)^2 + 2 \Hbar \Bk \cdot \Bp)
 \ket{s} \\
&=
\inv{m} \bra{s}
e^{ i \Bk \cdot \Br} (\Hbar \Bk \cdot \Bp) e^{-i \Bk \cdot \Br}
-
\Hbar \Bk \cdot \Bp
 \ket{s} \\
\end{aligned}
\end{equation}
%
\paragraph{Attempt 2}
I was going in circles above.  With the help of betel on
%\href{http://www.physicsforums.com/showthread.php?t=432923}{physicsforums}
\href{https://www.physicsforums.com/threads/double-commutator-used-to-obtain-energy-relationship-summed-over-energy-differences.432923/}{physicsforums}
, I got pointed in the right direction.  Here is a rework of this problem from scratch, also benefiting from hindsight.

Our starting point is the same, with the evaluation of the first commutator
%
\begin{equation}\label{eqn:desaiCh2Problems:110}
\antisymmetric{H}{ e^{i \Bk \cdot \Br}}
= \inv{2m} \antisymmetric{\Bp^2}{e^{i\Bk \cdot \Br}}.
\end{equation}
%
To continue we need to know how the momentum operator acts on an exponential of this form
%
\begin{equation}\label{eqn:desaiCh2Problems:4140}
\begin{aligned}
\Bp e^{\pm i \Bk \cdot \Br} \Psi
&=
-i \Hbar \Be_m \partial_m e^{\pm i k_n x_n } \Psi \\
&=
e^{\pm i \Bk \cdot \Br} \left(
-i \Hbar (\pm i \Be_m k_m ) \Psi -i \Hbar \Be_m \partial_m \Psi
\right).
\end{aligned}
\end{equation}
%
This gives us the helpful relationship
%
\begin{equation}\label{eqn:desaiCh2Problems:111}
\Bp e^{\pm i \Bk \cdot \Br} = e^{\pm i \Bk \cdot \Br} (\Bp \pm \Hbar \Bk).
\end{equation}
%
Squared application of the momentum operator on the positive exponential found in the first commutator \eqnref{eqn:desaiCh2Problems:110}, gives us
%
\begin{equation}\label{eqn:desaiCh2Problems:112}
\Bp^2 e^{i \Bk \cdot \Br} = e^{i \Bk \cdot \Br} (\Hbar \Bk + \Bp)^2 = e^{i \Bk \cdot \Br} ((\Hbar \Bk)^2 + 2 \Hbar \Bk \cdot \Bp + \Bp^2),
\end{equation}
%
with which we can evaluate this first commutator.
%
\begin{equation}\label{eqn:desaiCh2Problems:113}
\antisymmetric{H}{ e^{i \Bk \cdot \Br}}
= \inv{2m} e^{i \Bk \cdot \Br} ((\Hbar \Bk)^2 + 2 \Hbar \Bk \cdot \Bp).
\end{equation}
%
For the double commutator we have
%
\begin{equation}\label{eqn:desaiCh2Problems:4160}
\begin{aligned}
2m \antisymmetric{\antisymmetric{H}{e^{i \Bk \cdot \Br}}}{e^{-i \Bk \cdot \Br}}
&=
e^{i \Bk \cdot \Br} ((\Hbar \Bk)^2 + 2 \Hbar \Bk \cdot \Bp) e^{-i \Bk \cdot \Br}
-((\Hbar \Bk)^2 + 2 \Hbar \Bk \cdot \Bp)  \\
&=
e^{i \Bk \cdot \Br} 2 (\Hbar \Bk \cdot \Bp) e^{-i \Bk \cdot \Br}
-2 \Hbar \Bk \cdot \Bp \\
&=
2 \Hbar \Bk \cdot (\Bp - \Hbar \Bk)
-2 \Hbar \Bk \cdot \Bp,
\end{aligned}
\end{equation}
so for the double commutator we have just a scalar
\begin{equation}\label{eqn:desaiCh2Problems:114}
\antisymmetric{\antisymmetric{H}{e^{i \Bk \cdot \Br}}}{e^{-i \Bk \cdot \Br}}
= -\frac{(\Hbar \Bk)^2}{m}.
\end{equation}
%
Now consider the expectation of this double commutator, expanded with some unintuitive steps that have been motivated by working backwards
\begin{equation}\label{eqn:desaiCh2Problems:4180}
\begin{aligned}
\bra{s} &\antisymmetric{\antisymmetric{H}{e^{i \Bk \cdot \Br}}}{e^{-i \Bk \cdot \Br}} \ket{s} \\
&=
\bra{s} 2 H
- e^{i \Bk \cdot \Br} H e^{-i \Bk \cdot \Br}
- e^{-i \Bk \cdot \Br} H e^{i \Bk \cdot \Br} \ket{s} \\
&=
\bra{s} 2
e^{-i \Bk \cdot \Br} e^{i \Bk \cdot \Br} H
- 2 e^{-i \Bk \cdot \Br} H e^{i \Bk \cdot \Br} \ket{s} \\
&=
2 \sum_n \bra{s}
e^{-i \Bk \cdot \Br} \ket{n} \bra{n} e^{i \Bk \cdot \Br} H \ket{s}
- \bra{s} e^{-i \Bk \cdot \Br} H \ket{n} \bra{n} e^{i \Bk \cdot \Br} \ket{s} \\
&=
2 \sum_n
E_s
\bra{s}
e^{-i \Bk \cdot \Br} \ket{n} \bra{n} e^{i \Bk \cdot \Br} \ket{s}
- E_n \bra{s} e^{-i \Bk \cdot \Br} \ket{n} \bra{n} e^{i \Bk \cdot \Br} \ket{s} \\
&=
2 \sum_n
(E_s - E_n)
\Abs{\bra{n} e^{i \Bk \cdot \Br} \ket{s}}^2
\end{aligned}
\end{equation}
%
By \eqnref{eqn:desaiCh2Problems:114}, we have completed the problem
%
\begin{equation}\label{eqn:desaiCh2Problems:115}
\sum_n (E_n - E_s) \Abs{\bra{n} e^{i \Bk \cdot \Br} \ket{s}}^2 = \frac{(\Hbar \Bk)^2}{2m}.
\end{equation}
%
There is one subtlety above that is worth explicit mention before moving on, in particular, I did not find it intuitive that the following was true
%
\begin{equation}\label{eqn:desaiCh2Problems:117}
\bra{s} e^{i \Bk \cdot \Br} H e^{-i \Bk \cdot \Br} + e^{-i \Bk \cdot \Br} H e^{i \Bk \cdot \Br} \ket{s}
=
\bra{s} 2 e^{-i \Bk \cdot \Br} H e^{i \Bk \cdot \Br} \ket{s}.
\end{equation}
%
However, observe that both of these exponential sandwich operators \(e^{i \Bk \cdot \Br} H e^{-i \Bk \cdot \Br}\), and \(e^{-i \Bk \cdot \Br} H e^{i \Bk \cdot \Br}\) are Hermitian, since we have for example
%
\begin{equation}\label{eqn:desaiCh2Problems:4200}
\begin{aligned}
(e^{i \Bk \cdot \Br} H e^{-i \Bk \cdot \Br})^\dagger
&= (e^{-i \Bk \cdot \Br})^\dagger H^\dagger (e^{i \Bk \cdot \Br})^\dagger \\
&= e^{i \Bk \cdot \Br} H e^{-i \Bk \cdot \Br}
\end{aligned}
\end{equation}
%
Also observe that these operators are both complex conjugates of each other, and with \(\Bk \cdot \Br = \alpha\) for short, can be written
%
\begin{equation}\label{eqn:desaiCh2Problems:116}
\begin{aligned}
e^{i \alpha} H e^{-i \alpha}
&= \cos\alpha H \cos \alpha + \sin\alpha H \sin\alpha
+ i\sin\alpha H \cos \alpha -i \cos\alpha H \sin\alpha \\
e^{-i \alpha} H e^{i \alpha}
&= \cos\alpha H \cos \alpha + \sin\alpha H \sin\alpha
- i\sin\alpha H \cos \alpha +i \cos\alpha H \sin\alpha
\end{aligned}
\end{equation}
%
Because \(H\) is real valued, and the expectation value of a Hermitian operator is real valued, none of the imaginary terms can contribute to the expectation values, and in the summation of \eqnref{eqn:desaiCh2Problems:117} we can thus pick and double either of the exponential sandwich terms, as desired.
} % answer
