%
% Copyright © 2012 Peeter Joot.  All Rights Reserved.
% Licenced as described in the file LICENSE under the root directory of this GIT repository.
%
%\QMlecture{7 --- Stern Gerlach --- October 26, 2010}

Short class today since 43 minutes was wasted since the feedback given the Prof was so harsh that he wants to cancel the mid-term because students have said they are not prepared.  How ironic that this wastes more time that could be getting us prepared!
%
\section{Why do this (Dirac notation) math?}
\index{Dirac notation}
%
Because of the \textAndIndex{Stern-Gerlach} Experiment.  Explaining the Stern-Gerlach experiment is just not possible with wave functions and the ``old style'' Schr\"{o}dinger equation that operates on wave functions
%
\begin{equation}\label{eqn:PHY356Foct26:100}
\begin{aligned}
- \frac{\Hbar^2}{2m} \spacegrad^2 \Psi(\Bx,t) + V(\Bx) \Psi(\Bx,t) = i \Hbar \PD{t}{\Psi(\Bx,t)}.
\end{aligned}
\end{equation}
%
Review all of Chapter I so that you understand the idea of a Hermitian operator and its associated eigenvalues and eigenvectors.

Hermitian operation \(A\) is associated with a measurable quantity.

\makeexample{Spin-up}{example:lecture7sternGerlach:1}{
%
\index{spin up}
\(S_z\) is associated with the measurement of ``spin-up'' \(\ket{z+}\) or ``spin-down'' \(\ket{z-}\) states in silver atoms in the Stern-Gerlach experiment.

Each operator has associated with it a set of eigenvalues, and those eigenvalues are the outcomes of possible measurements.

\(S_z\) can be represented as
%
\begin{equation}\label{eqn:PHY356Foct26:101}
\begin{aligned}
S_z = \frac{\Hbar}{2}
\begin{bmatrix}
1 & 0 \\
0 & -1
\end{bmatrix},
\end{aligned}
\end{equation}
%
or
%
\begin{equation}\label{eqn:PHY356Foct26:102}
\begin{aligned}
S_z = \frac{\Hbar}{2}
\left( \ket{z+}\bra{z+} - \ket{z-}\bra{z-} \right).
\end{aligned}
\end{equation}
%
Find the eigenvalues of \(S_z\) in order to establish the possible outcomes of measurements of the z-component of the intrinsic angular momentum.

This is the point of the course.  It is to find the possible outcomes.  You have to appreciate that the measurement in the Stern-Gerlach experiment are trying to find the possible outcomes of the z-component measurement.  The eigenvalues of this operator give us those possible outcomes.
} % example

\makeexample{What if you put a brick in the experiment?}{example:lecture7sternGerlach:2}{

In the Stern-Gerlach experiment the ``spin down'' along the z-direction are atoms are blocked.  Diagram: silver going through a hole, with a brick between the detector and the spin-down location on the screen:

FIXME: scan it.  Oct 26, Fig 1.

What is the probability of measuring an outcome of \(+\Hbar/2\) along the x-direction?

Recall from Chapter I
%
\begin{equation}\label{eqn:PHY356Foct26:103}
\begin{aligned}
\ket{\phi} = \sum_n c_n \ket{a_n}
\end{aligned}
\end{equation}
%
We can express an arbitrary state \(\ket{\phi}\) in terms of basis vectors (could be eigenstates of an operator \(A\), but could be for example the eigenstates of the operator \(B\), say.)  Note that here in physics we will only work with orthonormal basis sets.  The generality.  To calculate the \(c_n's\) we take inner products
%
\begin{equation}\label{eqn:PHY356Foct26:104}
\begin{aligned}
\braket{a_m}{\phi} = \sum_n c_n \braket{a_m}{a_n} = \sum_n c_n \delta_{mn} = c_m
\end{aligned}
\end{equation}
%
The probability for measured outcome \(a_m\) is
%
\begin{equation}\label{eqn:PHY356Foct26:105}
\begin{aligned}
\Abs{c_m}^2 = \Abs{ \braket{a_m}{\phi} }^2
\end{aligned}
\end{equation}
%
In the end we have to appreciate that part of QM is figuring out what the possible outcomes are and the probabilities of those outcomes.

Appreciate that \(\ket{\phi} = \ket{z+}\) in this case.  This is a superposition of eigenstates of \(S_z\).  Why is it a superposition?  Because one of the coefficients is 1, and the other is 0.
%
\begin{equation}\label{eqn:PHY356Foct26:106}
\begin{aligned}
\ket{\phi}
=
 c_1 \ket{z+}
+ c_2 \ket{z-}
=
 c_1 \ket{z+}
+ 0 \ket{z-}
\end{aligned}
\end{equation}
%
So
\begin{equation}\label{eqn:PHY356Foct26:107}
\begin{aligned}
c_1 = 1
\end{aligned}
\end{equation}
%
recall that
\begin{equation}\label{eqn:PHY356Foct26:108}
\begin{aligned}
S_z &= \frac{\Hbar}{2}
\begin{bmatrix}
1 & 0 \\
0 & -1
\end{bmatrix} \\
\ket{z+} &\leftrightarrow
\begin{bmatrix}
1 \\
0
\end{bmatrix} \\
\ket{z+} &\leftrightarrow
\begin{bmatrix}
0 \\
1
\end{bmatrix}
\end{aligned}
\end{equation}
%
Also recall that
\begin{equation}\label{eqn:PHY356Foct26:109}
\begin{aligned}
S_x &= \frac{\Hbar}{2}
\begin{bmatrix}
0 & 1 \\
1 & 0
\end{bmatrix} \\
\ket{x+} &\leftrightarrow
\inv{\sqrt{2}}
\begin{bmatrix}
1 \\
1
\end{bmatrix} \\
\ket{x-} &\leftrightarrow
\inv{\sqrt{2}}
\begin{bmatrix}
1 \\
-1
\end{bmatrix}
\end{aligned}
\end{equation}
%
(with eigenvalues \(\pm\Hbar/2\)).

These eigenvectors are expressed in terms of \(\ket{z+}\) and \(\ket{z-}\), so that
%
\begin{equation}\label{eqn:PHY356Foct26:110}
\begin{aligned}
\ket{x+}
&=
\inv{\sqrt{2}} \left( \ket{z+} + \ket{z-} \right) \\
\ket{x-}
&=
\inv{\sqrt{2}} \left( \ket{z+} - \ket{z-} \right) .
\end{aligned}
\end{equation}
%
Outcome \(+\Hbar/2\) along the x-direction has an associated state \(\ket{x+}\).  That probability is
%
\begin{equation}\label{eqn:lecture7sternGerlach:220}
\begin{aligned}
\Abs{\braket{x+}{\phi}}^2
&=
\Abs{
\inv{\sqrt{2}} \left( \bra{z+} + \bra{z-} \right) \ket{\phi}
}^2 \\
&=
\inv{2}
\Abs{
\braket{z+}{\phi} + \braket{z-}{\phi}
}^2 \\
&=
\inv{2}
\Abs{
\braket{z+}{z+} + \braket{z-}{z+}
}^2 \\
&=
\inv{2}
\Abs{
1 + 0
}^2 \\
&=
\inv{2}
\end{aligned}
\end{equation}
} % example

\makeexample{Variation.  With a third splitter (SGZ)}{example:lecture7sternGerlach:3}{

The probability for outcome \(+\Hbar/2\) along z after the second SGZ magnets is
%
\begin{equation}\label{eqn:lecture7sternGerlach:240}
\begin{aligned}
\Abs{\braket{z+}{\phi'}}^2
&=
\Abs{\braket{z+}{x+}}^2 \\
&=
\Abs{\bra{z+} \inv{\sqrt{2}} \left( \ket{z+} + \ket{z-} \right) }^2 \\
&=
\inv{2}
\Abs{\braket{z+}{z+} + \braket{z+}{z-} }^2 \\
&=
\inv{2}
\end{aligned}
\end{equation}
%
My question: what is the point of the brick when the second splitter is already only being fed by the ``spin up'' stream.  Answer: just to ensure that the states are prepared in the expected way.  If the beams are two close together, without the brick perhaps we end up with some spin up in the upper stream.  Note that the beam separation here is on the order of centimeters.  ie: imagine that it is hard to redirect just one of the beams to the next stage splitter without blocking one of the beams or else the next splitter inevitably gets fed with some of both.  Might be nice to see a real picture of the Stern-Gerlach apparatus complete with scale.

Why silver?  Silver has 47 electrons, all of which but one are in spin pairs.  Only the ``outermost'' electron is free to have independent spin.

Aside: Note that we have the term ``Collapse'' to describe the now-known state after measurement.  There is some debate about the applicability of this term, and the interpretation that this imposes.  Will not be discussed here.

} % example
%
\section{On section 5.11, the complete wavefunction}
%
Aside: section 5.12 (Pauli exclusion principle and Fermi energy) excluded.
%
The complete wavefunction
%
\begin{equation}\label{eqn:lecture7sternGerlach:260}
\begin{aligned}
\ket{\phi} &= \text{the complete state of an atomic in the Stern-Gerlach experiment} \\
&=
\ket{u} \directproduct \ket{\chi}
\end{aligned}
\end{equation}
%
We also write
%
\begin{equation}\label{eqn:PHY356Foct26:200}
\begin{aligned}
\ket{u} \directproduct \ket{\chi}  &= \ket{u} \ket{\chi}
\end{aligned}
\end{equation}
%
where \(\ket{u}\) is associate with translation, and \(\ket{\chi}\) is associated with spin.  This is a product state where the \(\directproduct\) is a symbol for states in two or more different spaces.
