%
% Copyright � 2012 Peeter Joot.  All Rights Reserved.
% Licenced as described in the file LICENSE under the root directory of this GIT repository.
%
\label{chap:desaiCh4}
%\blogpage{http://sites.google.com/site/peeterjoot/math2010/desaiCh4.pdf}
%\date{Oct 10, 2010}
%
% Chapter IV notes and problems for \citep{desai2009quantum}.
%
% There is a lot of magic related to the spherical Harmonics in this chapter, with identities pulled out of the Author's butt.  It would be nice to work through that, but need a better reference to work from (or skip ahead to chapter 26 where some of this is apparently derived).
%
% Other stuff pending background derivation and verification are
\section{Antisymmetric tensor summation identity}
\index{antisymmetric tensor}
\begin{equation}\label{eqn:desaiCh4:1}
\sum_i \epsilon_{ijk} \epsilon_{iab} = \delta_{ja} \delta_{kb} - \delta_{jb}\delta_{ka}
\end{equation}

This is obviously the coordinate equivalent of the dot product of two bivectors
\begin{equation}\label{eqn:desaiCh4:2}
(\Be_j \wedge \Be_k) \cdot (\Be_a \wedge \Be_b) =
( (\Be_j \wedge \Be_k) \cdot \Be_a ) \cdot \Be_b)
=
\delta_{ka}\delta_{jb} - \delta_{ja}\delta_{kb}
\end{equation}

We can prove \eqnref{eqn:desaiCh4:1} by expanding the LHS of \eqnref{eqn:desaiCh4:2} in coordinates
\begin{equation}\label{eqn:desaiCh4:1021}
\begin{aligned}
(\Be_j \wedge \Be_k) \cdot (\Be_a \wedge \Be_b)
&= \sum_{ie} \gpgradezero{
\epsilon_{ijk} \Be_j \Be_k \epsilon_{eab} \Be_a \Be_b
} \\
&=
\sum_{ie}
\epsilon_{ijk} \epsilon_{eab}
\gpgradezero{
(\Be_i \Be_i) \Be_j \Be_k (\Be_e \Be_e) \Be_a \Be_b
} \\
&=
\sum_{ie}
\epsilon_{ijk} \epsilon_{eab}
\gpgradezero{
\Be_i \Be_e I^2
} \\
&=
-\sum_{ie} \epsilon_{ijk} \epsilon_{eab} \delta_{ie} \\
&=
-
\sum_i
\epsilon_{ijk} \epsilon_{iab}
\qedmarker
\end{aligned}
\end{equation}
\section{Question on raising and lowering arguments}
\index{raising}
\index{lowering}
How equation (4.240) was arrived at is not clear.  In (4.239) he writes
\begin{equation}\label{eqn:desaiCh4:1041}
\int_0^{2\pi} \int_0^{\pi} d\theta d\phi
(L_{-} Y_{lm})^\dagger
L_{-} Y_{lm} \sin\theta
\end{equation}

Should not that Hermitian conjugation be just complex conjugation? if so one would have
\begin{equation}\label{eqn:desaiCh4:1061}
\int_0^{2\pi} \int_0^{\pi} d\theta d\phi
L_{-}^\conj Y_{lm}^\conj
L_{-} Y_{lm} \sin\theta.
\end{equation}
How does he end up with the \(L_{-}\) and the \(Y_{lm}^\conj\) interchanged.  What justifies this commutation?

A much clearer discussion of this can be found in \href{http://quantummechanics.ucsd.edu/ph130a/130_notes/node217.html}{The operators \(L_{\pm}\)}, where Dirac notation is used for the normalization discussion.
\paragraph{Vatche's explanation}
Asked Vatche about this and had it explained nicely.  He also used the braket notation, and wrote
\begin{equation}\label{eqn:desaiCh4:1000}
\braket{\theta,\phi}{l,m} \equiv Y_{lm}(\theta,\phi)
\end{equation}
and introduces the identity
\begin{equation}\label{eqn:desaiCh4:1001}
I = \int_{0}^\pi d\theta \sin\theta \int_0^{2\pi} d\phi \ket{\theta,\phi}\bra{\theta,\phi}
\end{equation}

Now, if we want to normalize the state \(L_{-} \ket{l,m}\) we write
\begin{equation}
\begin{aligned}
&\bra{l,m} L_{-}^\dagger L_{-} \ket{l,m} \\
&=
\bra{l,m} L_{-}^\dagger L_{-} \ket{l,m} \\
&=
\int_{0}^\pi d\theta \sin\theta \int_0^{2\pi} d\phi
\int_{0}^\pi d\theta' \sin\theta' \int_0^{2\pi} d\phi'
\braket{l,m}{\theta,\phi}\bra{\theta,\phi}
L_{+} L_{-}
\ket{\theta',\phi'}\braket{\theta',\phi'}{l,m} \\
&=
\int_{0}^\pi d\theta \sin\theta \int_0^{2\pi} d\phi
\int_{0}^\pi d\theta' \sin\theta' \int_0^{2\pi} d\phi'
Y_{lm}^\conj(\theta, \phi)
\bra{\theta,\phi} L_{+} L_{-} \ket{\theta',\phi'}
Y_{lm}(\theta', \phi')
\end{aligned}
\end{equation}

Now he points out that the matrix element has both the differential operator portion, as well as a delta function portion, so we would have
\begin{equation}\label{eqn:desaiCh4:1081}
\bra{\theta,\phi} L_{+} L_{-} \ket{\theta',\phi'}
=
\inv{\sin\theta} \delta(\theta-\theta') \delta(\phi - \phi')
L_{+}(\theta,\phi) L_{-} (\theta, \phi)
\end{equation}
where the raising and lowering operators are now in their differential form
\begin{equation}\label{eqn:desaiCh4:1101}
L_{+} (\theta, \phi)
L_{-} (\theta, \phi)
=
\Hbar e^{i\theta} \left( \partial_\theta + i\cot\theta \partial_\phi \right)
\Hbar e^{- i\theta} \left( - \partial_\theta + i\cot\theta \partial_\phi \right)
\end{equation}

This now gives us
\begin{equation}
\begin{aligned}
&\bra{l,m} L_{-}^\dagger L_{-} \ket{l,m} \\
&=
\int_{0}^\pi d\theta \int_0^{2\pi} d\phi
\int_{0}^\pi d\theta' \sin\theta' \int_0^{2\pi} d\phi'
Y_{lm}^\conj(\theta, \phi)
L_{+}(\theta, \phi) L_{-}(\theta, \phi)
Y_{lm}(\theta', \phi')
\delta(\theta-\theta') \delta(\phi - \phi') \\
&=
\int_{0}^\pi d\theta \sin\theta \int_0^{2\pi} d\phi
Y_{lm}^\conj(\theta, \phi)
L_{+}(\theta, \phi) L_{-}(\theta, \phi)
Y_{lm}(\theta, \phi)
\end{aligned}
\end{equation}

This now fills in the reasoning (and notational) gap that the text has between (4.239) and (4.240).  It is now clear that in 4.239 (where Hermitian conjugation seemed out of place), that it should just have been regular complex number conjugation.  In the context of the normalization integral, Hermitian conjugation plays no role.  Here the \(L_{-} Y_{lm}\) used in the text are just functions.
\section{Another question on raising and lowering arguments}
The reasoning leading to (4.238) is not clear to me.  I fail to see how the \(L_{-}\) commutation with \(\BL^2\) implies this?
