%
% Copyright � 2012 Peeter Joot.  All Rights Reserved.
% Licenced as described in the file LICENSE under the root directory of this GIT repository.
%

%\chapter{Notes and problems for Desai Chapter VI}
\label{chap:desaiCh6}
%\blogpage{http://sites.google.com/site/peeterjoot/math2010/desaiCh6.pdf}
%\date{Oct 18, 2010}
%
%\section{Motivation}
%
%Chapter VI notes for \citep{desai2009quantum}.
%
%\section{Notes}
%
%\subsection{section 6.5, interaction with orbital angular momentum}
%
\index{gauge invariance}
In \S 6.5 it is stated that we take
\begin{equation}\label{eqn:desaiCh6:1}
\begin{aligned}
\BA = \inv{2} (\BB \cross \Br)
\end{aligned}
\end{equation}
%
and that this reproduces the gauge condition \(\spacegrad \cdot \BA = 0\), and the requirement \(\spacegrad \cross \BA = \BB\).

These seem to imply that \(\BB\) is constant, which also accounts for the fact that he writes \(\Bmu \cdot \BL = \BL \cdot \Bmu\).

Consider the gauge condition first, by expanding the divergence of a cross product
%
\begin{equation}\label{eqn:desaiCh6:27}
\begin{aligned}
\spacegrad \cdot (\BF \cross \BG)
&=
\gpgradezero{ \spacegrad -I \frac{ \BF \BG - \BG \BF }{2} } \\
&=
-\inv{2} \gpgradezero{ I \spacegrad \BF \BG - I \spacegrad \BG \BF } \\
&=
-\inv{2} \gpgradezero{
I \BG(\rspacegrad \BF)  - I \BF (\rspacegrad \BG)
+I (\BG \lspacegrad) \BF - I (\BF \lspacegrad) \BG
} \\
&=
-\inv{2} \gpgradezero{
I \BG(\rspacegrad \wedge \BF)  - I \BF (\rspacegrad \wedge \BG)
+I (\BG \wedge \lspacegrad) \BF - I (\BF \wedge \lspacegrad) \BG
} \\
&=
\inv{2} \gpgradezero{
\BG (\rspacegrad \cross \BF)  - \BF (\rspacegrad \cross \BG)
+(\BG \cross \lspacegrad) \BF - (\BF \cross \lspacegrad) \BG
} \\
&=
\inv{2} \left(
\BG \cdot (\spacegrad \cross \BF)  - \BF \cdot (\spacegrad \cross \BG)
-\BF \cdot (\spacegrad \cross \BG)  + \BG \cdot (\spacegrad \cross \BF )
\right) \\
\end{aligned}
\end{equation}
%
This gives us
%
\begin{equation}\label{eqn:desaiCh6:2}
\begin{aligned}
\spacegrad \cdot (\BF \cross \BG)
&=
\BG \cdot (\spacegrad \cross \BF)  - \BF \cdot (\spacegrad \cross \BG)
\end{aligned}
\end{equation}
%
With \(\BA = (\BB \cross \Br)/2\) we then have
%
\begin{equation}\label{eqn:desaiCh6:3}
\begin{aligned}
\spacegrad \cdot \BA =
\inv{2} \Br \cdot (\spacegrad \cross \BB)  - \inv{2} \BB \cdot (\spacegrad \cross \Br)
=
\inv{2} \Br \cdot (\spacegrad \cross \BB)
\end{aligned}
\end{equation}
%
Unless \(\spacegrad \cross \BB\) is always perpendicular to \(\Br\) we can only have a zero divergence when \(\BB\) is constant.

Now, let us look at \(\spacegrad \cross \BA\).  We need another auxiliary identity
%
\begin{equation}\label{eqn:desaiCh6:47}
\begin{aligned}
\spacegrad \cross (\BF \cross \BG)
&=
-I \spacegrad \wedge (\BF \cross \BG) \\
&=
-\inv{2} \gpgradeone{
I \rspacegrad (\BF \cross \BG)
- I (\BF \cross \BG) \lspacegrad
} \\
&=
\inv{2} \left(
-\rspacegrad \cdot (\BF \wedge \BG)
+ (\BF \wedge \BG) \cdot \lspacegrad
\right) \\
&=
\inv{2} \left(
-(\rspacegrad \cdot \BF) \BG
+(\rspacegrad \cdot \BG) \BF
+ \BF (\BG \cdot \lspacegrad )
- \BG (\BF \cdot \lspacegrad )
\right)
\\
&=
\inv{2} \left(
-(\spacegrad \cdot \BF) \BG
+(\spacegrad \cdot \BG) \BF
+ (\spacegrad \cdot \BG ) \BF
- (\spacegrad \cdot \BF ) \BG
\right)
\end{aligned}
\end{equation}
%
Here the gradients are all still acting on both \(\BF\) and \(\BG\).  Expanding this out by chain rule we have
%
\begin{equation}\label{eqn:desaiCh6:67}
\begin{aligned}
2 \spacegrad \cross (\BF \cross \BG)
=
&-(\BF \cdot \spacegrad) \BG
-\BG (\spacegrad \cdot \BF)
+\BF (\spacegrad \cdot \BG)
+(\BG \cdot \spacegrad ) \BF  \\
\quad&+\BF (\spacegrad \cdot \BG )
+ (\BG \cdot \spacegrad ) \BF
- (\BF \cdot \spacegrad ) \BG
- \BG (\spacegrad \cdot \BF )
\end{aligned}
\end{equation}
%
or
\begin{equation}\label{eqn:desaiCh6:4}
\begin{aligned}
\spacegrad \cross (\BF \cross \BG)
&=
\BF (\spacegrad \cdot \BG) -(\BF \cdot \spacegrad) \BG
+(\BG \cdot \spacegrad ) \BF  -\BG (\spacegrad \cdot \BF)
\end{aligned}
\end{equation}
%
With \(\BF = \BB/2\), and \(\BG = \Br\), we have
%
\begin{equation}\label{eqn:desaiCh6:87}
\begin{aligned}
\spacegrad \cross \BA
&=
\inv{2}
\BB (\spacegrad \cdot \Br) -\inv{2}(\BB \cdot \spacegrad) \Br
+\inv{2}(\Br \cdot \spacegrad ) \BB  -\inv{2}\Br (\spacegrad \cdot \BB)
\end{aligned}
\end{equation}
%
We note that \(\spacegrad \cdot \Br = 3\), and
%
\begin{equation}\label{eqn:desaiCh6:107}
\begin{aligned}
(\BB \cdot \spacegrad ) \Br
&=
B_k \partial_k x_m \Be_m \\
&=
B_k \delta_{km} \Be_m \\
%&=
%B_m \Be_m
&=
\BB
\end{aligned}
\end{equation}
%
If \(\BB\) is constant, we have
%
\begin{equation}\label{eqn:desaiCh6:7}
\begin{aligned}
\spacegrad \cross \BA = \frac{3\BB}{2} - \frac{\BB}{2} = \BB,
\end{aligned}
\end{equation}
%
as desired.  Now this would all likely be a lot more intuitive if one started with constant \(\BB\) and derived from that what the vector potential was.  That is probably worth also thinking about.

