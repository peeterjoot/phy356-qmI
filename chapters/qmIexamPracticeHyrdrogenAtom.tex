%
% Copyright © 2012 Peeter Joot.  All Rights Reserved.
% Licenced as described in the file LICENSE under the root directory of this GIT repository.
%
%
\makeproblem{Hydrogen atom (2007 PHY355H1F 4)}{problem:qmIexamPractice2007Dec:4}{
%
This problem deals with the hydrogen atom, with an initial ket
%
\begin{equation}\label{eqn:qmIexamPractice2007Dec:4:10}
\ket{\psi(0)} =
\inv{\sqrt{3}} \ket{100}
+\inv{\sqrt{3}} \ket{210}
+\inv{\sqrt{3}} \ket{211},
\end{equation}

where
%
\begin{equation}\label{eqn:qmIexamPractice2007Dec:4:20}
\braket{\Br}{100} = \Phi_{100}(\Br),
\end{equation}

etc.
} % problem
%
\makeanswer{problem:qmIexamPractice2007Dec:4}{
%
If no measurement is made until time \(t = t_0\),
%
\begin{equation}\label{eqn:qmIexamPractice2007Dec:4:30}
t_0 = \frac{\pi \Hbar}{ \frac{3}{4} (13.6 \text{eV}) } = \frac{ 4 \pi \Hbar }{ 3 E_I},
\end{equation}

what is the ket \(\ket{\psi(t)}\) just before the measurement is made?

\paragraph{A:}

Our time evolved state is
%
\begin{equation}\label{eqn:qmIexamPractice2007Dec:4:35}
\ket{\psi{t_0}} =
\inv{\sqrt{3}} e^{-i E_1 t_0 /\Hbar } \ket{100}
+\inv{\sqrt{3}} e^{- i E_2 t_0/\Hbar }
(\ket{210} + \ket{211}).
\end{equation}

Also observe that this initial time was picked to make the exponential values come out nicely, and we have
%
\begin{equation}\label{eqn:qmIexamPractice2007Dec:490}
\begin{aligned}
\frac{E_n t_0 }{\Hbar}
&= - \frac{E_I \pi \Hbar }{\frac{3}{4} E_I n^2 \Hbar} \\
&= - \frac{4 \pi }{ 3 n^2 },
\end{aligned}
\end{equation}

so our time evolved state is just
%
\begin{equation}\label{eqn:qmIexamPractice2007Dec:4:100}
\ket{\psi(t_0)} =
\inv{\sqrt{3}} e^{-i 4 \pi / 3} \ket{100}
+\inv{\sqrt{3}} e^{- i \pi / 3 }
(\ket{210} + \ket{211}).
\end{equation}

\paragraph{Q: (b)}

Suppose that at time \(t_0\) an \(L_z\) measurement is made, and the outcome 0 is recorded.  What is the appropriate ket \(\psi_{\text{after}}(t_0)\) right after the measurement?

\paragraph{A:}

A measurement with outcome 0, means that the \(L_z\) operator measurement found the state at that point to be the eigenstate for \(L_z\) eigenvalue 0.  Recall that  if \(\ket{\phi}\) is an eigenstate of \(L_z\) we have
%
\begin{equation}\label{eqn:qmIexamPractice2007Dec:4:200}
L_z \ket{\phi} = m \Hbar \ket{\phi},
\end{equation}

so a measurement of \(L_z\) with outcome zero means that we have \(m=0\).  Our measurement of \(L_z\) at time \(t_0\) therefore filters out all but the \(m=0\) states and our new state is proportional to the projection over all \(m=0\) states as follows
%
\begin{equation}\label{eqn:qmIexamPractice2007Dec:510}
\begin{aligned}
\ket{\psi_{\text{after}}(t_0)}
&\propto \left( \sum_{n l} \ket{n l 0}\bra{n l 0} \right) \ket{\psi(t_0)}  \\
&\propto \left(
\ket{1 0 0}\bra{1 0 0}
+\ket{2 1 0}\bra{2 1 0}
\right) \ket{\psi(t_0)}  \\
&=
\inv{\sqrt{3}} e^{-i 4 \pi / 3} \ket{100}
+\inv{\sqrt{3}} e^{- i \pi / 3 } \ket{210}
\end{aligned}
\end{equation}

A final normalization yields
\begin{equation}\label{eqn:qmIexamPractice2007Dec:4:210}
\ket{\psi_{\text{after}}(t_0)}
= \inv{\sqrt{2}} (\ket{210} - \ket{100})
\end{equation}

\paragraph{Q: (c)}

Right after this \(L_z\) measurement, what is \(\Abs{\psi_{\text{after}}(t_0)}^2\)?

\paragraph{A:}

Our amplitude is
%
\begin{equation}\label{eqn:qmIexamPractice2007Dec:530}
\begin{aligned}
\braket{\Br}{\psi_{\text{after}}(t_0)}
&= \inv{\sqrt{2}} (\braket{\Br}{210} - \braket{\Br}{100}) \\
&= \inv{\sqrt{2 \pi a_0^3}}
\left(
\frac{r}{4\sqrt{2} a_0} e^{-r/2a_0} \cos\theta
-e^{-r/a_0}
\right) \\
&= \inv{\sqrt{2 \pi a_0^3}}
e^{-r/2 a_0}
\left(
\frac{r}{4\sqrt{2} a_0} \cos\theta
-e^{-r/2 a_0}
\right),
\end{aligned}
\end{equation}

so the probability density is
\begin{equation}\label{eqn:qmIexamPractice2007Dec:4:300}
\Abs{\braket{\Br}{\psi_{\text{after}}(t_0)}}^2
= \inv{2 \pi a_0^3}
e^{-r/a_0}
\left(
\frac{r}{4\sqrt{2} a_0} \cos\theta
-e^{-r/2 a_0}
\right)^2
\end{equation}

\paragraph{Q: (d)}

If then a position measurement is made immediately, which if any components of the expectation value of \(\BR\) will be non-vanishing?  Justify your answer.

\paragraph{A:}

The expectation value of this vector valued operator with respect to a radial state \(\ket{\psi} = \sum_{nlm} a_{nlm} \ket{nlm}\) can be expressed as
%
\begin{equation}\label{eqn:qmIexamPractice2007Dec:4:400}
\expectation{\BR} = \sum_{i=1}^3 \Be_i \sum_{nlm, n'l'm'}
a_{nlm}^\conj a_{n'l'm'}
\bra{nlm} X_i
\ket{n'l'm'},
\end{equation}

where \(X_1 = X = R \sin\Theta \cos\Phi, X_2 = Y = R \sin\Theta \sin\Phi, X_3 = Z = R \cos\Phi\).

Consider one of the matrix elements, and expand this by introducing an identity twice
\begin{equation}\label{eqn:qmIexamPractice2007Dec:550}
\begin{aligned}
&\bra{nlm} X_i \ket{n'l'm'} \\
&=
\int
r^2 \sin\theta dr d\theta d\phi
{r'}^2 \sin\theta' dr' d\theta' d\phi'
\braket{nlm}{r \theta \phi} \bra{r \theta \phi} X_i \ket{r' \theta' \phi' }\braket{r' \theta' \phi'}{n'l'm'} \\
&=
\int
r^2 \sin\theta dr d\theta d\phi
{r'}^2 \sin\theta' dr' d\theta' d\phi'
R_{nl}(r) Y_{lm}^\conj(\theta,\phi)
\delta^3(\Bx - \Bx') x_i
R_{n'l'}(r') Y_{l'm'}(\theta',\phi')
\\
&=
\int
r^2 \sin\theta dr d\theta d\phi
{r'}^2 \sin\theta' dr' d\theta' d\phi'
R_{nl}(r) Y_{lm}^\conj(\theta,\phi) \\
&\qquad{r'}^2 \sin\theta' \delta(r-r') \delta(\theta - \theta') \delta(\phi-\phi')
x_i
R_{n'l'}(r') Y_{l'm'}(\theta',\phi')
\\
&=
\int
r^2 \sin\theta dr d\theta d\phi
dr' d\theta' d\phi'
R_{nl}(r) Y_{lm}^\conj(\theta,\phi)
\delta(r-r') \delta(\theta - \theta') \delta(\phi-\phi')
x_i
R_{n'l'}(r') Y_{l'm'}(\theta',\phi')
\\
&=
\int
r^2 \sin\theta dr d\theta d\phi
R_{nl}(r) R_{n'l'}(r)
Y_{lm}^\conj(\theta,\phi) Y_{l'm'}(\theta,\phi)
x_i
\\
\end{aligned}
\end{equation}

Because our state has only \(m=0\) contributions, the only \(\phi\) dependence for the \(X\) and \(Y\) components of \(\BR\) come from those components themselves.  For \(X\), we therefore integrate \(\int_0^{2\pi} \cos\phi d\phi = 0\), and for \(Y\) we integrate \(\int_0^{2\pi} \sin\phi d\phi = 0\), and these terms vanish.  Our expectation value for \(\BR\) for this state, therefore lies completely on the \(z\) axis.
} % answer
