%
% Copyright © 2012 Peeter Joot.  All Rights Reserved.
% Licenced as described in the file LICENSE under the root directory of this GIT repository.
%
Deal with operators that have continuous eigenvalues and eigenvectors.
\index{continuous eigenvalues}
\index{continuous eigenvectors}
We now express
%
\begin{equation}\label{eqn:lecture5BigReviewCh2Operators:20}
\ket{\phi} = \int dk
\mathLabelBox
[
   labelstyle={xshift=2cm},
   linestyle={out=270,in=90, latex-}
]
{f(k)}{coefficients analogous to \(c_n\)}
\ket{k}
\end{equation}
%
Now if we project onto \(k'\)
%
\begin{equation}\label{eqn:lecture5BigReviewCh2Operators:40}
\begin{aligned}
\braket{k'}{\phi}
&= \int dk f(k)
\mathLabelBox{\braket{k'}{k}}{Dirac delta}
\\
&= \int dk f(k) \delta(k' -k) \\
&= f(k')
\end{aligned}
\end{equation}
\index{Dirac delta}
%
\index{probability density}
Unlike the discrete case, this is not a probability.  Probability density for obtaining outcome \(k'\) is \(\Abs{f(k')}^2\).

Example 2.
\begin{equation}\label{eqn:lecture5BigReviewCh2Operators:60}
\ket{\phi} = \int dk f(k) \ket{k}
\end{equation}
%
Now if we project x onto both sides
%
\begin{equation}\label{eqn:lecture5BigReviewCh2Operators:80}
\braket{x}{\phi}
= \int dk f(k) \braket{x}{k}
\end{equation}
%
With \(\braket{x}{k} = u_k(x)\)
%
\begin{equation}\label{eqn:lecture5BigReviewCh2Operators:100}
\begin{aligned}
\phi(x)
&\equiv \braket{x}{\phi} \\
&= \int dk f(k) u_k(x)  \\
&= \int dk f(k) \inv{\sqrt{L}} e^{ikx}
\end{aligned}
\end{equation}
%
This is with periodic boundary value conditions for the normalization.  The infinite normalization is also possible.
%
\begin{equation}\label{eqn:lecture5BigReviewCh2Operators:120}
\phi(x)
= \inv{\sqrt{L}} \int dk f(k) e^{ikx}
\end{equation}
%
Multiply both sides by \(e^{-ik'x}/\sqrt{L}\) and integrate.  This is analogous to multiplying \(\ket{\phi} = \int f(k) \ket{k} dk\) by \(\bra{k'}\).  We get
%
\begin{equation}\label{eqn:lecture5BigReviewCh2Operators:140}
\begin{aligned}
\int \phi(x) \inv{\sqrt{L}} e^{-ik'x} dx
&= \inv{L} \iint dk f(k) e^{i(k-k')x} dx \\
&= \int dk f(k) \Bigl( \inv{L} \int e^{i(k-k')x} \Bigr) \\
&= \int dk f(k) \delta(k-k') \\
&= f(k')
\end{aligned}
\end{equation}
%
\begin{equation}\label{eqn:lecture5BigReviewCh2Operators:160}
f(k') = \int \phi(x) \inv{\sqrt{L}} e^{-ik'x} dx
\end{equation}
%
We can talk about the state vector in terms of its position basis \(\phi(x)\) or in the momentum space via Fourier transformation.  This is the equivalent thing, but just expressed different.  The question of interpretation in terms of probabilities works out the same.  Either way we look at the probability density.

The quantity
%
\begin{equation}\label{eqn:lecture5BigReviewCh2Operators:180}
\ket{\phi} = \int dk f(k) \ket{k}
\end{equation}
%
is also called a wave packet state since it involves a superposition of many stats \(\ket{k}\).  Example: See Fig 4.1 (Gaussian wave packet, with \(\Abs{\phi}^2\) as the height).  This wave packet is a snapshot of the wave function amplitude at one specific time instant.  The evolution of this wave packet is governed by the Hamiltonian, which brings us to chapter 3.

