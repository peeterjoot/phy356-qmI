%
% Copyright � 2012 Peeter Joot.  All Rights Reserved.
% Licenced as described in the file LICENSE under the root directory of this GIT repository.
%

%\chapter{Desai Chapter 10 notes and problems}
\label{chap:desaiCh10}
%\blogpage{http://sites.google.com/site/peeterjoot/math2010/desaiCh10.pdf}
%\date{Nov 20, 2010}

%\section{Motivation}
%
%Chapter 10 notes for \citep{desai2009quantum}.
%
%\section{Notes}
%
In \S 10.3 (interaction with a electric field), Green's functions are introduced to solve the first order differential equation

\begin{equation}\label{eqn:desaiCh10:1}
\begin{aligned}
\frac{da}{dt} + i \omega_0 a = - i \omega_0 \lambda(t)
\end{aligned}
\end{equation}

A simpler way is to use the usual trick of assuming that we can take the constant term in the homogeneous solution and allow it to vary with time.

Since our homogeneous solution is of the form

\begin{equation}\label{eqn:desaiCh10:2}
\begin{aligned}
a_H(t) = a_H(0) e^{-i\omega_0 t},
\end{aligned}
\end{equation}

we can look for a specific solution to the forcing term equation of the form

\begin{equation}\label{eqn:desaiCh10:3}
\begin{aligned}
a_S(t) = f(t) e^{-i\omega_0 t}
\end{aligned}
\end{equation}

We get
\begin{equation}\label{eqn:desaiCh10:4}
\begin{aligned}
f' = -i \omega_0 \lambda(t) e^{i \omega_0 t}
\end{aligned}
\end{equation}

which can be integrate directly to find the non-homogeneous solution

\begin{equation}\label{eqn:desaiCh10:5}
\begin{aligned}
a_S(t) = a_S(t_0) e^{-i \omega_0 (t - t_0)} - i \omega_0 \int_{t_0}^t \lambda(t') e^{-i \omega_0 (t-t')} dt'
\end{aligned}
\end{equation}

Setting \(t_0 = -\infty\), with a requirement that \(a_S(-\infty) = 0\) and adding in a general homogeneous solution one then has 10.92 without the complications of Green's functions or the associated contour integrals.  I suppose the author wanted to introduce this as a general purpose tool and this was a simple way to do so.

His introduction of Green's functions this way I did not personally find very clear.  Specifically, he does not actually define what a Green's function is, and the Appendix 20.13 he refers to only discusses the subtleties of the associated Contour integration.  I did not understand where equation 10.83 came from in the first place.

Something like the following would have been helpful (the type of argument found in \citep{wiki:greens})
\index{Green's function}

Given a linear operator \(L\), such that \(L u(x) = f(x)\), we search for the Green's function \(G(x,s)\) such that \(L G(x,s) = \delta(x-s)\).  For such a function we have

\begin{equation}\label{eqn:desaiCh10:25}
\begin{aligned}
\int L G(x,s) f(s) ds
&= \int \delta(x-s) f(s) ds \\
&= f(x)
\end{aligned}
\end{equation}

and by linearity we also have
\begin{equation}\label{eqn:desaiCh10:45}
\begin{aligned}
f(x)
&=
\int L G(x,s) f(s) ds \\
&= L \int G(x,s) f(s) ds \\
\end{aligned}
\end{equation}

and can therefore identify \(u(x) = \int G(x,s) f(s) ds\) as the desired solution to \(L u(x) = f(x)\) once the Green's function \(G(x,s)\) associated with operator \(L\) has been determined.

