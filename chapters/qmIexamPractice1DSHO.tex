%
% Copyright © 2016 Peeter Joot.  All Rights Reserved.
% Licenced as described in the file LICENSE under the root directory of this GIT repository.
%
\makeproblem{One dimensional harmonic oscillator (2008 PHY355H1F final 3.)}{problem:qmIexamPractice:5}{

Consider a one-dimensional harmonic oscillator with the Hamiltonian

\begin{equation}\label{eqn:qmIexamPractice2008Dec:3:11}
H = \inv{2m}P^2 + \inv{2} m \omega^2 X^2
\end{equation}

Denote the ground state of the system by \(\ket{0}\), the first excited state by \(\ket{1}\) and so on.

\makesubproblem{}{problem:qmIexamPractice:5:a}

Evaluate \(\bra{n} X \ket{n}\) and \(\bra{n} X^2 \ket{n}\) for arbitrary \(\ket{n}\).

\makesubproblem{}{problem:qmIexamPractice:5:b}

Suppose that at \(t=0\) the system is prepared in the state

\begin{equation}\label{eqn:qmIexamPractice2008Dec:3:20}
\ket{\psi(0)} = \inv{\sqrt{2}} ( \ket{0} + i \ket{1} ).
\end{equation}

If a measurement of position \(X\) were performed immediately, sketch the probability distribution \(P(x)\) that a particle would be found within \(dx\) of \(x\).  Justify how you construct the sketch.

\makesubproblem{}{problem:qmIexamPractice:5:c}

Now suppose the state given in (b) above were allowed to evolve for a time \(t\), determine the expectation value of \(X\) and \(\Delta X\) at that time.

\makesubproblem{}{problem:qmIexamPractice:5:d}

Now suppose that initially the system were prepared in the ground state \(\ket{0}\), and then the resonance frequency is changed abruptly from \(\omega\) to \(\omega'\) so that the Hamiltonian becomes

\begin{equation}\label{eqn:qmIexamPractice2008Dec:3:10}
H = \inv{2m}P^2 + \inv{2} m {\omega'}^2 X^2.
\end{equation}

Immediately, an energy measurement is performed ; what is the probability of obtaining the result \(E = \Hbar \omega' (3/2)\)?

} % problem

\makeanswer{problem:qmIexamPractice:5}{

\makeSubAnswer{}{problem:qmIexamPractice:5:a}

Writing \(X\) in terms of the raising and lowering operators we have

\begin{equation}\label{eqn:qmIexamPractice2008Dec:3:100}
X = \frac{\alpha}{\sqrt{2}} (a^\dagger + a),
\end{equation}

so \(\expectation{X}\) is proportional to

\begin{equation}\label{eqn:qmIexamPractice2008Dec:3:110}
\bra{n} a^\dagger + a \ket{n} = \sqrt{n+1} \braket{n}{n+1} + \sqrt{n} \braket{n}{n-1} = 0.
\end{equation}

For \(\expectation{X^2}\) we have

\begin{equation}\label{eqn:qmIexamPractice:730}
\begin{aligned}
\expectation{X^2}
&=
\frac{\alpha^2}{2}
\bra{n} (a^\dagger + a)(a^\dagger + a) \ket{n} \\
&=
\frac{\alpha^2}{2}
\bra{n} (a^\dagger + a) \left(
\sqrt{n+1} \ket{n+1} + \sqrt{n-1} \ket{n-1}
\right)  \\
&=
\frac{\alpha^2}{2}
\bra{n}
\Bigl( (n+1) \ket{n} + \sqrt{n(n-1)} \ket{n-2}
+ \sqrt{(n+1)(n+2)} \ket{n+2} + n \ket{n} \Bigr).
\end{aligned}
\end{equation}

We are left with just

\begin{equation}\label{eqn:qmIexamPractice2008Dec:3:140}
\expectation{X^2} = \frac{\Hbar}{2 m \omega} (2n + 1).
\end{equation}

\makeSubAnswer{}{problem:qmIexamPractice:5:b}

The probability that we started in state \(\ket{\psi(0)}\) and ended up in position \(x\) is governed by the amplitude \(\braket{x}{\psi(0)}\), and the probability of being within an interval \(\Delta x\), surrounding the point \(x\) is given by

\begin{equation}\label{eqn:qmIexamPractice2008Dec:3:200}
\int_{x'=x-\Delta x/2}^{x+\Delta x/2} \Abs{ \braket{x'}{\psi(0)} }^2 dx'.
\end{equation}

In the limit as \(\Delta x \rightarrow 0\), this is just the squared amplitude itself evaluated at the point \(x\), so we are interested in the quantity

\begin{equation}\label{eqn:qmIexamPractice2008Dec:3:210}
\Abs{ \braket{x}{\psi(0)} }^2  = \inv{2} \Abs{ \braket{x}{0} + i \braket{x}{1} }^2.
\end{equation}

We are given these wave functions in the supplemental formulas.  Namely,

\begin{equation}\label{eqn:qmIexamPractice2008Dec:3:220}
\begin{aligned}
\braket{x}{0} &= \psi_0(x) = \frac{e^{-x^2/2\alpha^2}}{ \sqrt{\alpha \sqrt{\pi}}} \\
\braket{x}{1} &= \psi_1(x) = \frac{e^{-x^2/2\alpha^2} 2 x }{ \alpha \sqrt{2 \alpha \sqrt{\pi}}}.
\end{aligned}
\end{equation}

Substituting these into \eqnref{eqn:qmIexamPractice2008Dec:3:210} we have

\begin{equation}\label{eqn:qmIexamPractice2008Dec:3:230}
\Abs{ \braket{x}{\psi(0)} }^2
=
\inv{2}
e^{-x^2/\alpha^2}
\inv{
\alpha \sqrt{\pi}}
\Abs{ 1 + \frac{2 i x}{\alpha \sqrt{2} } }^2
=
\frac{e^{-x^2/\alpha^2}}{ 2
\alpha \sqrt{\pi}}
\left( 1 + \frac{2 x^2}{\alpha^2 } \right).
\end{equation}

This \href{http://www.wolframalpha.com/input/?i=graph+e^(-x^2)+(1+\%2B+2x^2)}{is parabolic near the origin and then quickly tapers off}.

\makeSubAnswer{}{problem:qmIexamPractice:5:c}
Our time evolved state is
\begin{equation}\label{eqn:qmIexamPractice2008Dec:3:300}
\begin{aligned}
U(t) \ket{\psi(0)} 
&= \inv{\sqrt{2}}
\left(
e^{-i \Hbar \omega \left( 0 + \inv{2} \right) t/\Hbar } \ket{0}
+ i e^{-i \Hbar \omega \left( 1 + \inv{2} \right) t/\Hbar } \ket{0}
\right) \\
&=
\inv{\sqrt{2}}
\left(
e^{-i \omega t/2 } \ket{0}
+ i e^{- 3 i \omega t/2 } \ket{1}
\right).
\end{aligned}
\end{equation}

The position expectation is therefore
\begin{equation}\label{eqn:qmIexamPractice:750}
\begin{aligned}
\bra{\psi(t)} X \ket{\psi(t)}
&=
\frac{\alpha}{2 \sqrt{2}}
\left(
e^{i \omega t/2 } \bra{0}
- i e^{ 3 i \omega t/2 } \bra{1}
\right)
(a^\dagger + a)
\left(
e^{-i \omega t/2 } \ket{0}
+ i e^{- 3 i \omega t/2 } \ket{1}
\right) \\
\end{aligned}
\end{equation}

We have already demonstrated that \(\bra{n} X \ket{n} = 0\), so we must only expand the cross terms, but those are just \(\bra{0} a^\dagger + a \ket{1} = 1\).  This leaves

\begin{equation}\label{eqn:qmIexamPractice2008Dec:3:310}
\bra{\psi(t)} X \ket{\psi(t)}
=
\frac{\alpha}{2 \sqrt{2}}
\left( -i e^{i \omega t} + i e^{-i \omega t} \right)
=
\sqrt{\frac{\Hbar}{2 m \omega}} \cos(\omega t)
\end{equation}

For the squared position expectation
\begin{equation}\label{eqn:qmIexamPractice:770}
\begin{aligned}
&\bra{\psi(t)} X^2 \ket{\psi(t)} \\
&=
\frac{\alpha^2}{4 (2)}
\left(
e^{i \omega t/2 } \bra{0}
- i e^{ 3 i \omega t/2 } \bra{1}
\right)
(a^\dagger + a)^2
\left(
e^{-i \omega t/2 } \ket{0}
+ i e^{- 3 i \omega t/2 } \ket{1}
\right) \\
&=
\inv{2} ( \bra{0} X^2 \ket{0} + \bra{1} X^2 \ket{1} )
+ i \frac{\alpha^2 }{8} (
- e^{ i \omega t} \bra{1} (a^\dagger + a)^2 \ket{0}
+ e^{ -i \omega t} \bra{0} (a^\dagger + a)^2 \ket{1}
)
\end{aligned}
\end{equation}

Noting that \((a^\dagger + a) \ket{0} = \ket{1}\), and \((a^\dagger + a)^2 \ket{0} = (a^\dagger + a)\ket{1} = \sqrt{2} \ket{2} + \ket{0}\), so we see the last two terms are zero.  The first two we can evaluate using our previous result \eqnref{eqn:qmIexamPractice2008Dec:3:140} which was \(\expectation{X^2} = \frac{\alpha^2}{2} (2n + 1)\).  This leaves

\begin{equation}\label{eqn:qmIexamPractice2008Dec:3:330}
\bra{\psi(t)} X^2 \ket{\psi(t)} = \alpha^2
\end{equation}

Since \(\expectation{X}^2 = \alpha^2 \cos^2(\omega t)/2\), we have

\begin{equation}\label{eqn:qmIexamPractice2008Dec:3:340}
(\Delta X)^2 = \expectation{X^2} - \expectation{X}^2 = \alpha^2 \left(1 - \inv{2} \cos^2(\omega t) \right)
\end{equation}

\makeSubAnswer{}{problem:qmIexamPractice:5:d}

This energy measurement \(E = \Hbar \omega' (3/2) = \Hbar \omega' (1 + 1/2)\), corresponds to an observation of state \(\ket{1'}\), after an initial observation of \(\ket{0}\).  The probability of such a measurement is

\begin{equation}\label{eqn:qmIexamPractice2008Dec:3:400}
\Abs{ \braket{1'}{0} }^2
\end{equation}

Note that

\begin{equation}\label{eqn:qmIexamPractice:790}
\begin{aligned}
\braket{1'}{0}
&=
\int dx \braket{1'}{x}\braket{x}{0} \\
&=
\int dx \psi_{1'}^\conj \psi_0(x) \\
\end{aligned}
\end{equation}

The wave functions above are
\begin{equation}\label{eqn:qmIexamPractice2008Dec:3:410}
\begin{aligned}
\phi_{1'}(x) &= \frac{ 2 x e^{-x^2/2 {\alpha'}^2 }}{ \alpha' \sqrt{ 2 \alpha' \sqrt{\pi} } } \\
\phi_{0}(x) &= \frac{ e^{-x^2/2 {\alpha}^2 } } { \sqrt{ \alpha \sqrt{\pi} } }
\end{aligned}
\end{equation}

Putting the pieces together we have

\begin{equation}\label{eqn:qmIexamPractice2008Dec:3:810}
\begin{aligned}
\braket{1'}{0}
&=
\frac{2 }{ \alpha' \sqrt{ 2 \alpha' \alpha \pi } }
\int dx
x e^{-\frac{x^2}{2}\left( \inv{{\alpha'}^2} + \inv{\alpha^2} \right) }
\end{aligned}
\end{equation}

Since this is an odd integral kernel over an even range, this evaluates to zero, and we conclude that the probability of measuring the specified energy is zero when the system is initially prepared in the ground state associated with the original Hamiltonian.  Intuitively this makes some sense, if one thinks of the Fourier coefficient problem: one cannot construct an even function from linear combinations of purely odd functions.
} % answer
