%
% Copyright � 2012 Peeter Joot.  All Rights Reserved.
% Licenced as described in the file LICENSE under the root directory of this GIT repository.
%

%\chapter{PHY356 Problem Set 4}
\label{chap:qmIproblemSet4}
%\blogpage{http://sites.google.com/site/peeterjoot/math2010/qmIproblemSet4.pdf}
%\date{Nov 16, 2010}
%
\makeproblem{ps 4, p1.}{problem:qmIproblemSet4:1}{
Is it possible to derive the eigenvalues and eigenvectors presented in Section 8.2 from those in Section 8.1.2?  What does this say about the potential energy operator in these two situations?

For reference 8.1.2 was a finite potential barrier, \(V(x) = V_0, \Abs{x} > a\), and zero in the interior of the well.  This had trigonometric solutions in the interior, and died off exponentially past the boundary of the well.

On the other hand, 8.2 was a delta function potential \(V(x) = -g \delta(x)\), which had the solution \(u(x) = \sqrt{\beta} e^{-\beta \Abs{x}}\), where \(\beta = m g/\Hbar^2\).

} % problem
%
\makeanswer{problem:qmIproblemSet4:1}{
%
The pair of figures in the text \citep{desai2009quantum} for these potentials does not make it clear that there are possibly any similarities.  The attractive delta function potential is not illustrated (although the delta function is, but with opposite sign), and the scaling and the reference energy levels are different.  Let us illustrate these using the same reference energy level and sign conventions to make the similarities more obvious.

\imageFigure{../figures/phy356-qmI/FiniteWellPotential}{8.1.2 Finite Well potential (with energy shifted downwards by \(V_0\))}{fig:FiniteWellPotential}{0.4}

\imageFigure{../figures/phy356-qmI/deltaFunctionPotential}{8.2 Delta function potential}{fig:deltaFunctionPotential}{0.4}

The physics is not changed by picking a different point for the reference energy level, so let us compare the two potentials, and their solutions using \(V(x) = 0\) outside of the well for both cases.  The method used to solve the finite well problem in the text is hard to follow, so re-doing this from scratch in a slightly tidier way does not hurt.

Schr\"{o}dinger's equation for the finite well, in the \(\Abs{x} > a\) region is
%
\begin{equation}\label{eqn:qmIproblemSet4:110}
\begin{aligned}
-\frac{\Hbar^2}{2m} u'' = E u = - E_B u,
\end{aligned}
\end{equation}
%
where a positive bound state energy \(E_B = -E > 0\) has been introduced.

Writing
\begin{equation}\label{eqn:qmIproblemSet4:115}
\begin{aligned}
\beta = \sqrt{\frac{2 m E_B}{\Hbar^2}},
\end{aligned}
\end{equation}
%
the wave functions outside of the well are
\begin{equation}\label{eqn:qmIproblemSet4:120}
\begin{aligned}
u(x) =
\left\{
\begin{array}{l l}
u(-a) e^{\beta(x+a)} &\quad \mbox{\(x < -a\)} \\
u(a) e^{-\beta(x-a)} &\quad \mbox{\(x > a\)} \\
\end{array}
\right.
\end{aligned}
\end{equation}
%
Within the well Schr\"{o}dinger's equation is
\begin{equation}\label{eqn:qmIproblemSet4:125}
\begin{aligned}
-\frac{\Hbar^2}{2m} u'' - V_0 u = E u = - E_B u,
\end{aligned}
\end{equation}
%
or
\begin{equation}\label{eqn:qmIproblemSet4:126}
\begin{aligned}
\frac{\Hbar^2}{2m} u'' = - \frac{2m}{\Hbar^2} (V_0 - E_B) u,
\end{aligned}
\end{equation}
%
Noting that the bound state energies are the \(E_B < V_0\) values, let \(\alpha^2 = 2m (V_0 - E_B)/\Hbar^2\), so that the solutions are of the form
\begin{equation}\label{eqn:qmIproblemSet4:130}
\begin{aligned}
u(x) = A e^{i\alpha x} + B e^{-i\alpha x}.
\end{aligned}
\end{equation}
%
As was done for the wave functions outside of the well, the normalization constants can be expressed in terms of the values of the wave functions on the boundary.  That provides a pair of equations to solve
%
\begin{equation}\label{eqn:qmIproblemSet4:135}
\begin{aligned}
\begin{bmatrix}
u(a) \\
u(-a)
\end{bmatrix}
=
\begin{bmatrix}
e^{i \alpha a} & e^{-i \alpha a} \\
e^{-i \alpha a} & e^{i \alpha a}
\end{bmatrix}
\begin{bmatrix}
A \\
B
\end{bmatrix}.
\end{aligned}
\end{equation}
%
Inverting this and substitution back into \eqnref{eqn:qmIproblemSet4:130} yields
\begin{equation}\label{eqn:qmIproblemSet4:175}
\begin{aligned}
u(x)
&=
\begin{bmatrix}
e^{i\alpha x} & e^{-i\alpha x}
\end{bmatrix}
\begin{bmatrix}
A \\
B
\end{bmatrix} \\
&=
\begin{bmatrix}
e^{i\alpha x} & e^{-i\alpha x}
\end{bmatrix}
\inv{e^{2 i \alpha a} - e^{-2 i \alpha a}}
\begin{bmatrix}
e^{i \alpha a} & -e^{-i \alpha a} \\
-e^{-i \alpha a} & e^{i \alpha a}
\end{bmatrix}
\begin{bmatrix}
u(a) \\
u(-a)
\end{bmatrix} \\
&=
\begin{bmatrix}
\frac{\sin(\alpha (a + x))}{\sin(2 \alpha a)} &
\frac{\sin(\alpha (a - x))}{\sin(2 \alpha a)}
\end{bmatrix}
\begin{bmatrix}
u(a) \\
u(-a)
\end{bmatrix}.
\end{aligned}
\end{equation}
%
Expanding the last of these matrix products the wave function is close to completely specified.
%
\begin{equation}\label{eqn:qmIproblemSet4:140}
u(x) =
\left\{
\begin{array}{l l}
u(-a) e^{\beta(x+a)}
 & \quad \mbox{\(x < -a\)} \\
u(a) \frac{\sin(\alpha (a + x))}{\sin(2 \alpha a)} +
u(-a) \frac{\sin(\alpha (a - x))}{\sin(2 \alpha a)}
 & \quad \mbox{\(\Abs{x} < a\)} \\
u(a) e^{-\beta(x-a)}
 & \quad \mbox{\(x > a\)} \\
\end{array}
\right.
\end{equation}
%
There are still two unspecified constants \(u(\pm a)\) and the constraints on \(E_B\) have not been determined (both \(\alpha\) and \(\beta\) are functions of that energy level).  It should be possible to eliminate at least one of the \(u(\pm a)\) by computing the wavefunction normalization, and since the well is being narrowed the \(\alpha\) term will not be relevant.  Since only the vanishingly narrow case where \(a \rightarrow 0, x \in [-a,a]\) is of interest, the wave function in that interval approaches
%
\begin{equation}\label{eqn:qmIproblemSet4:145}
u(x) \rightarrow \inv{2} (u(a) + u(-a)) + \frac{x}{2} ( u(a) - u(-a) ) \rightarrow \inv{2} (u(a) + u(-a)).
\end{equation}
%
Since no discontinuity is expected this is just \(u(a) = u(-a)\).  Let us write \(\lim_{a\rightarrow 0} u(a) = A\) for short, and the limited width well wave function becomes
%
\begin{equation}\label{eqn:qmIproblemSet4:150}
u(x) =
\left\{
\begin{array}{l l}
A e^{\beta x}
 & \quad \mbox{\(x < 0\)} \\
A e^{-\beta x}
 & \quad \mbox{\(x > 0\)} \\
\end{array}
\right.
\end{equation}
%
This is now the same form as the delta function potential, and normalization also gives \(A = \sqrt{\beta}\).

One task remains before the attractive delta function potential can be considered a limiting case for the finite well, since the relation between \(a, V_0\), and \(g\) has not been established.  To do so integrate the Schr\"{o}dinger equation over the infinitesimal range \([-a,a]\).  This was done in the text for the delta function potential, and that provided the relation
%
\begin{equation}\label{eqn:qmIproblemSet4:155a}
\beta = \frac{mg}{\Hbar^2}
\end{equation}
%
For the finite well this is
%
\begin{equation}\label{eqn:qmIproblemSet4:151}
\int_{-a}^a -\frac{\Hbar^2}{2m} u'' - V_0 \int_{-a}^a u = -E_B \int_{-a}^a u \\
\end{equation}
%
In the limit as \(a \rightarrow 0\) this is
\begin{equation}\label{eqn:qmIproblemSet4:152}
\frac{\Hbar^2}{2m} (u'(a) - u'(-a)) + V_0 2 a u(0) = 2 E_B a u(0).
\end{equation}
%
Some care is required with the \(V_0 a\) term since \(a \rightarrow 0\) as \(V_0 \rightarrow \infty\), but the \(E_B\) term is unambiguously killed, leaving
\begin{equation}\label{eqn:qmIproblemSet4:153}
\frac{\Hbar^2}{2m} u(0) (-2\beta e^{-\beta a}) = -V_0 2 a u(0).
\end{equation}
%
The exponential vanishes in the limit and leaves
%
\begin{equation}\label{eqn:qmIproblemSet4:155}
\beta = \frac{m (2 a) V_0}{\Hbar^2}
\end{equation}
%
Comparing to \eqnref{eqn:qmIproblemSet4:155a} from the attractive delta function completes the problem.  The conclusion is that when the finite well is narrowed with \(a \rightarrow 0\), also letting \(V_0 \rightarrow \infty\) such that the absolute area of the well \(g = (2 a) V_0\) is maintained, the finite potential well produces exactly the attractive delta function wave function and associated bound state energy.

\paragraph{Grading notes}

Lost \(3/20\) marks, all in the first question.

I did not show that \(u(a) = u(-a)\).

I did not explain why the odd terms disappear in \eqnref{eqn:qmIproblemSet4:145}.

I also did not get agreement with my statement that ``but the \(E_B\) term is unambiguously killed'', where I have assumed that it remains finite.  Since \(V_0 \rightarrow \infty\), \(E_B\) could tend to infinity too.

\paragraph{Some references}

Some references that I found helpful to provide some of the context for WHY to consider the delta function potential in the first place are \citep{wiki:DeltaPotential}, \citep{deltaFunctionModelOfACrystal}, \citep{deltaFunctionPotentials}, \citep{theDeltaFunctionPotential}.

% can no longer access:
%\href{http://www.phys.ufl.edu/~rfield/PHY4604/images/Chapter2_20.pdf}{ufl}.

} % answer
