%
% Copyright © 2012 Peeter Joot.  All Rights Reserved.
% Licenced as described in the file LICENSE under the root directory of this GIT repository.
%
%\QMlecture{10 --- Hydrogen atom. --- November 23, 2010}
%
\paragraph{Introduce the center of mass coordinates}
\index{center of mass}

We will want to solve this using the formalism we have discussed.  The general problem is a proton, positively charged, with a nearby negative charge (the electron).

Our equation to solve is

\begin{align}\label{eqn:PHY356Lecture10:50}
\left(
-\frac{\Hbar^2}{2 m_1} \spacegrad_1^2
-\frac{\Hbar^2}{2 m_2} \spacegrad_2^2
\right)
\overbar{u}(\Br_1, \Br_2) +
V(\Br_1, \Br_2)
\overbar{u}(\Br_1, \Br_2)
=
E \overbar{u}(\Br_1, \Br_2).
\end{align}

Here \(\left( -\frac{\Hbar^2}{2 m_1} \spacegrad_1^2 -\frac{\Hbar^2}{2 m_2} \spacegrad_2^2 \right)\) is the total kinetic energy term.
For hydrogen we can consider the potential to be the Coulomb potential energy function that depends only on \(\Br_1 - \Br_2\).  We can transform this using a center of mass transformation.  Introduce the center of mass coordinate and relative coordinate vectors

\begin{align}\label{eqn:PHY356Lecture10:51}
\BR &= \frac{m_1 \Br_1 + m_2 \Br_2}{ m_1 + m_2 } \\
\Br &= \Br_1 - \Br_2.
\end{align}

The notation \(\spacegrad_k^2\) represents the Laplacian for the positions of the k'th particle, so that if \(\Br_1 = (x_1, x_2, x_3)\) is the position of the first particle, the Laplacian for this is:
%
\begin{equation}\label{eqn:PHY356FLecture10:52}
\spacegrad_1^2
=
\frac{\partial^2}{\partial x_1^2}
+\frac{\partial^2}{\partial y_1^2}
+\frac{\partial^2}{\partial z_1^2}
\end{equation}
%
Here \(\BR\) is the center of mass coordinate, and \(\Br\) is the relative coordinate.  With this transformation we can reduce the problem to a single coordinate PDE.

We set \(\overbar{u}(\Br_1, \Br_2) = u(\Br) U(\BR)\) and \(E = E_{rel} + E_{cm}\), and get
%
\begin{equation}\label{eqn:PHY356FLecture10:10}
-\frac{\Hbar^2}{2\mu} {\spacegrad_{\Br}}^2 u(\Br) + V(\Br) u(\Br) = E_{rel} u(\Br)
\end{equation}
and
\begin{equation}\label{eqn:PHY356FLecture10:20}
-\frac{\Hbar^2}{2M} {\spacegrad_{\BR}}^2 U(\BR) = E_{cm} U(\BR)
\end{equation}
%
where \(M = m_1 + m_2\) is the total mass, and \(\mu = m_1 m_2/M\) is the reduced mass.

Aside: WHY do we care (slide of Hydrogen line spectrum shown)?  This all started because when people looked at the spectrum for the hydrogen atom, a continuous spectrum was not found.  Instead what was found was quantized frequencies.  All this abstract Hilbert space notation with its bras and kets is a way of representing observable phenomena.

Also note that we have the same sort of problems in electrodynamics and mechanics, so we are able to recycle this sort of work, either applying it in those problems later, or using those techniques here.

In Electromagnetism these are the problems involving the solution to
%
\begin{equation}\label{eqn:PHY356FLecture10:30}
\spacegrad \cdot \BE = 0
\end{equation}
%
or for
\begin{equation}\label{eqn:PHY356FLecture10:40}
\BE = - \spacegrad \Phi
\end{equation}
%
%
\begin{equation}\label{eqn:PHY356FLecture10:60}
\spacegrad^2 \Phi = 0,
\end{equation}
%
where \(\BE\) is the electric field and \(\Phi\) is the electric potential.

% REQUEST FOR HELP FROM SOMEBODY WHO HAS NOTE TAKING PROBLEMS:
% accessiblity services: as.notetaking@utoronto.ca PHY356

We need sol solve \eqnref{eqn:PHY356FLecture10:10} for \(u(\Br)\).  In spherical coordinates
%
\begin{equation}\label{eqn:PHY356FLecture10:70}
-\frac{\Hbar^2}{2m} \inv{r} \frac{d^2}{dr^2} ( r R_l ) + \left( V(\Br) + \frac{\Hbar^2 }{2m} l(l+1) \right) R_l = E R_l
\end{equation}
%
where
\begin{equation}\label{eqn:PHY356FLecture10:80}
u(\Br) = R_l(\Br) Y_{lm}(\theta, \phi)
\end{equation}
%
This all follows by the separation of variables technique that we will use here, in E and M, in PDEs, and so forth.

FIXME: picture drawn.  Theta measured down from \(\Be_3\) axis to the position \(\Br\) and \(\phi\) measured in the \(x,y\) plane measured in the \(\Be_1\) to \(\Be_2\) orientation.

For the hydrogen atom, we have
%
\begin{equation}\label{eqn:PHY356FLecture10:90}
V(\Br) = - \frac{Z e^2}{r}
\end{equation}
%
\href{http://www.wolframalpha.com/input/?i=graph+-1/r}{Here is what this looks like}.

We introduce

\begin{align}\label{eqn:PHY356FLecture10:100}
\rho &= \alpha r \\
\alpha &= \sqrt{\frac{-8 m E}{\Hbar^2}} \\
\lambda &= \frac{2 m Z e^2}{\Hbar^2 \alpha} \\
\frac{2 m (- E) }{\Hbar^2 \alpha^2 } &= \inv{4}
\end{align}

and write
\begin{equation}\label{eqn:PHY356FLecture10:110}
\frac{d^2 R_l}{d\rho^2} + \frac{2}{\rho} \frac{d R_l}{d\rho} + \left( \frac{\lambda}{\rho} - \frac{l(l+1)}{\rho^2} - \inv{4} \right) R_l = 0
\end{equation}
%
\paragraph{Large \(\rho\) limit}
%
For \(\rho \rightarrow \infty\), \eqnref{eqn:PHY356FLecture10:110} becomes
\begin{equation}\label{eqn:PHY356FLecture10:120}
\frac{d^2 R_l}{d\rho^2} - \inv{4} R_l = 0
\end{equation}
%
which implies solutions of the form
\begin{equation}\label{eqn:PHY356FLecture10:130}
R_l(\rho) = e^{\pm \rho/2}
\end{equation}
%
but keep \(R_l(\rho) = e^{-\rho/2}\) and note that \(R_l(\rho) = F(\rho)e^{-\rho/2}\) is also a solution in the limit of \(\rho \rightarrow \infty\), where \(F(\rho)\) is a polynomial.

Let \(F(\rho) = \rho^s L(\rho)\) where \(L(\rho) = a_0 + a_1 \rho + \cdots a_\nu \rho^\nu + \cdots\).
%
\paragraph{Small \(\rho\) limit}
%
We also want to consider the small \(\rho\) limit, and piece together the information that we find.  Think about the following.  The small \(\rho \rightarrow 0\) or \(r \rightarrow 0\) limit gives
%
\begin{equation}\label{eqn:PHY356FLecture10:140}
\frac{d^2 R_l}{d\rho^2} - \frac{l(l+1)}{\rho^2} R_l = 0
\end{equation}
%
\paragraph{Question:} Is this correct?
%
Not always.  Also: we will also think about the \(l=0\) case later (where \(\lambda/\rho\) would probably need to be retained.)

We need:
\begin{equation}\label{eqn:PHY356FLecture10:140b}
\frac{d^2 R_l}{d\rho^2} + \frac{2}{\rho} \frac{d R_l}{d\rho} - \frac{l(l+1)}{\rho^2} R_l = 0
\end{equation}
%
Instead of using \eqnref{eqn:PHY356FLecture10:140} as in the text, we must substitute \(R_l = \rho^s\) into the above to find

\begin{align}\label{eqn:PHY356FLecture10:150}
s(s-1) \rho^{s-2} + 2 s \rho^{s-2} - l(l+1) \rho^{s-2} &= 0 \\
\left( s(s-1) + 2 s - l(l+1) \right) \rho^{s-2} &=
\end{align}

for this equality for all \(\rho\) we need
%
\begin{equation}\label{eqn:PHY356FLecture10:160}
s(s-1) + 2 s - l(l+1) = 0
\end{equation}
%
Solutions \(s = l\) and \(s = -(l+1)\) can be found to this, and we need s positive for normalizability, which implies
%
\begin{equation}\label{eqn:PHY356FLecture10:170}
R_l(\rho) = \rho^l L(\rho) e^{-\rho/2}.
\end{equation}
%
Now we need to find what restrictions we must have on \(L(\rho)\).  Recall that we have \(L(\rho) = \sum a_\nu \rho^\nu\).  Substitution into \eqnref{eqn:PHY356FLecture10:140} gives
%
\begin{equation}\label{eqn:PHY356FLecture10:180}
\rho \frac{d^2 L}{d\rho} + \left( 2(l+1) - \rho \right) \frac{d L}{d \rho} + (\lambda - l - 1) L = 0
\end{equation}
%
We get
\begin{equation}\label{eqn:PHY356FLecture10:190}
A_0 + A_1 \rho + \cdots A_\nu \rho^\nu + \cdots = 0
\end{equation}
%
For this to be valid for all \(\rho\),
%
\begin{equation}\label{eqn:PHY356FLecture10:200}
a_{\nu+1} \left(
(\nu+1)(\nu+ 2l + 2)
\right)
-
a_{\nu} \left(
\nu - \lambda + l + 1
\right)
=0
\end{equation}
%
or
\begin{equation}\label{eqn:PHY356FLecture10:210}
\frac{a_{\nu+1}}{ a_{\nu} }
=
\frac{ \nu - \lambda + l + 1 }{ (\nu+1)(\nu+ 2l + 2) }
\end{equation}
%
For large \(\nu\) we have
\begin{equation}\label{eqn:PHY356FLecture10:220}
\frac{a_{\nu+1}}{ a_{\nu} }
=
\inv{\nu+1}
\rightarrow \inv{\nu}
\end{equation}
%
Recall that for the exponential Taylor series we have
\begin{equation}\label{eqn:PHY356FLecture10:230}
e^\rho = 1 + \rho + \frac{\rho^2}{2!} + \cdots
\end{equation}
%
for which we have
\begin{equation}\label{eqn:PHY356FLecture10:240}
\frac{a_{\nu+1}}{a_\nu} \rightarrow \inv{\nu}
\end{equation}
%
\(L(\rho)\) is behaving like \(e^\rho\), and if we had that
%
\begin{equation}\label{eqn:PHY356FLecture10:250}
R_l(\rho) = \rho^l L(\rho) e^{-\rho/2} \rightarrow \rho^l e^\rho e^{-\rho/2} = \rho^l e^{\rho/2}
\end{equation}
%
This is divergent, so for normalizable solutions we require \(L(\rho)\) to be a polynomial of a finite number of terms.

The polynomial \(L(\rho)\) must stop at \(\nu = n'\), and we must have
%
\begin{equation}\label{eqn:PHY356FLecture10:260}
a_{\nu+1} = a_{n' +1} = 0
\end{equation}
\begin{equation}\label{eqn:PHY356FLecture10:270}
a_{n'} \ne 0
\end{equation}
%
From \eqnref{eqn:PHY356FLecture10:200} we have
%
\begin{equation}\label{eqn:PHY356FLecture10:200a}
a_{n'} \left(
n' - \lambda + l + 1
\right)
=0
\end{equation}
%
so we require
\begin{equation}\label{eqn:PHY356FLecture10:280}
n' = \lambda - l - 1
\end{equation}
%
Let \(\lambda = n\), an integer and \(n' = 0, 1, 2, \cdots\) so that \(n' + l + 1 = n\) says for \(n= 1,2, \cdots\)
%
\begin{equation}\label{eqn:PHY356FLecture10:290}
l \le n-1
\end{equation}
%
If
%
\begin{equation}\label{eqn:PHY356FLecture10:300}
\lambda = n = \frac{2 m Z e^2 }{\Hbar^2 \alpha}
\end{equation}
%
we have
\begin{equation}\label{eqn:PHY356FLecture10:310}
E = E_n = - \frac{Z^2 e^2 }{2 a_0} \inv{n^2}
\end{equation}
%
where \(a_0 = \Hbar^2/m e^2\) is the Bohr radius, and \(\alpha = \sqrt{-8 m E/\Hbar^2}\).  In the lecture \(m\) was originally used for the reduced mass.  I have switched to \(\mu\) earlier so that this cannot be mixed up with this use of \(m\) for the azimuthal quantum number associated with \(L_z Y_{lm} = m \Hbar Y_{lm}\).

PICTURE ON BOARD.  Energy level transitions on \(1/n^2\) graph with differences between \(n=2\) to \(n=1\) shown, and photon emitted as a result of the \(n=2\) to \(n=1\) transition.

From Chapter 4 and the story of the spherical harmonics, for a given \(l\), the quantum number \(m\) varies between \(-l\) and \(l\) in integer steps.  The radial part of the solution of this separation of variables problem becomes
%
\begin{equation}\label{eqn:PHY356FLecture10:320}
R_l = \rho^l L(\rho) e^{-\rho/2}
\end{equation}
%
where the functions \(L(\rho)\) are the Laguerre polynomials, and our complete wavefunction is
\begin{equation}\label{eqn:PHY356FLecture10:330}
u_{nlm}(r, \theta, \phi) = R_l(\rho) Y_{lm}(\theta, \phi)
\end{equation}
%
\begin{align}\label{eqn:PHY356FLecture10:340}
n &= 1, 2, \cdots \\
l &= 0, 1, 2, \cdots, n-1 \\
m &= -l, -l+1, \cdots 0, 1, 2, \cdots, l-1, l
\end{align}

Note that for \(n=1, l=0\), \(R_{10} \propto e^{-r/a_0}\), as \href{http://www.wolframalpha.com/input/?i=graph+e^{-r}}{graphed here}.

