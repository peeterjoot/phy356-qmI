%
% Copyright � 2012 Peeter Joot.  All Rights Reserved.
% Licenced as described in the file LICENSE under the root directory of this GIT repository.
%

%\chapter{Unitary exponential sandwich}
\label{chap:exponentialSandwichCommutator}
%\blogpage{http://sites.google.com/site/peeterjoot/math2010/exponentialSandwichCommutator.pdf}
%\date{Sept 27, 2010}
%
%\section{Motivation}
%
One of the chapter II exercises in \citep{desai2009quantum} involves a commutator exponential sandwich of the form

\index{exponential sandwich}
\begin{equation}\label{eqn:exponentialSandwichCommutator:1}
e^{i F} B e^{-iF}
\end{equation}
%
where \(F\) is Hermitian.  Asking about commutators on physicsforums I was told that such sandwiches (my term) preserve expectation values, and also have a Taylor series like expansion involving the repeated commutators.  Let us derive the commutator relationship.

It turns out that the solution of this sandwich expansion is also known as the ``Baker-Campbell-Hausdorff'' formula
\citep{wiki:bakercampbellHausdorff}
.
%
%\section{Guts}
%
Let us expand a sandwich of this form in series, and shuffle the summation order so that we sum over all the index plane diagonals \(k + m = \text{constant}\).  That is
%
\begin{equation}\label{eqn:exponentialSandwichCommutator:24}
\begin{aligned}
e^{A} B e^{-A}
&=
\sum_{k,m=0}^\infty \inv{k!m!} A^k B (-A)^m \\
&=
\sum_{r=0}^\infty \sum_{m=0}^r \inv{(r-m)!m!} A^{r-m} B (-A)^m \\
&=
\sum_{r=0}^\infty \inv{r!} \sum_{m=0}^r \frac{r!}{(r-m)!m!} A^{r-m} B (-A)^m \\
&=
\sum_{r=0}^\infty \inv{r!} \sum_{m=0}^r \binom{r}{m} A^{r-m} B (-A)^m.
\end{aligned}
\end{equation}
%
Assuming that these interior sums can be written as commutators, we will shortly have an induction exercise.  Let us write these out for a couple values of \(r\) to get a feel for things.

\begin{itemize}
\item \(r=1\)
%
\begin{equation}\label{eqn:exponentialSandwichCommutator:44}
\binom{1}{0} A B + \binom{1}{1} B (-A) = \antisymmetric{A}{B}
\end{equation}
%
\item \(r=2\)
%
\begin{equation}\label{eqn:exponentialSandwichCommutator:64}
\binom{2}{0} A^2 B + \binom{2}{1} A B (-A) + \binom{2}{2} B (-A)^2 = A^2 B - 2 A B A + B A
\end{equation}
%
This compares exactly to the double commutator:
\begin{equation}\label{eqn:exponentialSandwichCommutator:84}
\begin{aligned}
\antisymmetric{A}{\antisymmetric{A}{B}}
&=
A(A B - B A) -(A B - B A)A \\
&=
A^2 B - A B A - A B A + B A^2 \\
&=
A^2 B - 2 A B A + B A^2 \\
\end{aligned}
\end{equation}
%
\item \(r=3\)
\begin{equation}\label{eqn:exponentialSandwichCommutator:104}
\begin{aligned}
\binom{3}{0} &A^3 B + \binom{3}{1} A^2 B (-A) + \binom{3}{2} A B (-A)^2 + \binom{3}{3} B (-A)^3 \\
&=
A^3 B - 3 A^2 B A + 3 A B A^2 - B A^3.
\end{aligned}
\end{equation}
%
And this compares exactly to the triple commutator
\begin{equation}\label{eqn:exponentialSandwichCommutator:124}
\begin{aligned}
\antisymmetric{A}{\antisymmetric{A}{\antisymmetric{A}{B}}}
&=
A^3 B - 2 A^2 B A + A B A^2 -(A^2 B A - 2 A B A^2 + B A^3) \\
&=
A^3 B - 3 A^2 B A + 3 A B A^2 -B A^3 \\
\end{aligned}
\end{equation}
\end{itemize}

The induction pattern is clear.  Let us write the \(r\) fold commutator as
%
\begin{equation}\label{eqn:exponentialSandwichCommutator:2}
C_r(A,B) \equiv
\mathLabelBox{[A, [A, \cdots, [A,}{\(r\) times}
B]] \cdots ]
= \sum_{m=0}^r \binom{r}{m} A^{r-m} B (-A)^m,
\end{equation}
%
and calculate this for the \(r+1\) case to verify the induction hypothesis.  We have
%
\begin{equation}\label{eqn:exponentialSandwichCommutator:144}
\begin{aligned}
C_{r+1}&(A,B) \\
&= \sum_{m=0}^r \binom{r}{m}
\left( A^{r-m+1} B (-A)^m
-A^{r-m} B (-A)^{m} A \right) \\
&= \sum_{m=0}^r \binom{r}{m}
\left( A^{r-m+1} B (-A)^m
+A^{r-m} B (-A)^{m+1} \right) \\
&=
A^{r+1} B
+ \sum_{m=1}^r \binom{r}{m}
A^{r-m+1} B (-A)^m
+ \sum_{m=0}^{r-1} \binom{r}{m}
A^{r-m} B (-A)^{m+1}
+ B (-A)^{r+1} \\
&=
A^{r+1} B
% k=m-1
% m = k+1
+ \sum_{k=0}^{r-1} \binom{r}{k+1}
A^{r-k} B (-A)^{k+1}
+ \sum_{m=0}^{r-1} \binom{r}{m}
A^{r-m} B (-A)^{m+1}
+ B (-A)^{r+1} \\
&=
A^{r+1} B
+ \sum_{k=0}^{r-1} \left( \binom{r}{k+1} + \binom{r}{k} \right) A^{r-k} B (-A)^{k+1}
+ B (-A)^{r+1} \\
\end{aligned}
\end{equation}
%
We now have to sum those binomial coefficients.  I like the search and replace technique for this, picking two visibly distinct numbers for \(r\), and \(k\) that are easy to manipulate without abstract confusion.  How about \(r=7\), and \(k=3\).  Using those we have
%
\begin{equation}\label{eqn:exponentialSandwichCommutator:164}
\begin{aligned}
\binom{7}{3+1} + \binom{7}{3}
&=
\frac{7!}{(3+1)!(7-3-1)!}
+\frac{7!}{3!(7-3)!} \\
&=
\frac{7!(7-3)}{(3+1)!(7-3)!}
+\frac{7!(3+1)}{(3+1)!(7-3)!} \\
&=
\frac{7! \left( 7-3 + 3 + 1 \right) }{(3+1)!(7-3)!} \\
&=
\frac{(7+1)! }{(3+1)!((7+1)-(3+1))!}.
\end{aligned}
\end{equation}
%
Straight text replacement of \(7\) and \(3\) with \(r\) and \(k\) respectively now gives the harder to follow, but more general identity
%
\begin{equation}\label{eqn:exponentialSandwichCommutator:184}
\begin{aligned}
\binom{r}{k+1} + \binom{r}{k}
&=
\frac{r!}{(k+1)!(r-k-1)!}
+\frac{r!}{k!(r-k)!} \\
&=
\frac{r!(r-k)}{(k+1)!(r-k)!}
+\frac{r!(k+1)}{(k+1)!(r-k)!} \\
&=
\frac{r! \left( r-k + k + 1 \right) }{(k+1)!(r-k)!} \\
&=
\frac{(r+1)! }{(k+1)!((r+1)-(k+1))!} \\
&=
\binom{r+1}{k+1}
\end{aligned}
\end{equation}
%
For our commutator we now have
%
\begin{equation}\label{eqn:exponentialSandwichCommutator:204}
\begin{aligned}
C_{r+1}(A,B)
&=
A^{r+1} B
+ \sum_{k=0}^{r-1} \binom{r+1}{k+1} A^{r-k} B (-A)^{k+1}
+ B (-A)^{r+1} \\
&=
A^{r+1} B
% k+1=s
% k=s-1
+ \sum_{s=1}^{r} \binom{r+1}{s} A^{r+1-s} B (-A)^{s}
+ B (-A)^{r+1} \\
&= \sum_{s=0}^{r+1} \binom{r+1}{s} A^{r+1-s} B (-A)^{s}
\qedmarker
\end{aligned}
\end{equation}
%
That completes the inductive proof and allows us to write
%
\begin{equation}\label{eqn:exponentialSandwichCommutator:3}
\begin{aligned}
e^A B e^{-A}
&=
\sum_{r=0}^\infty \inv{r!} C_{r}(A,B),
\end{aligned}
\end{equation}
%
Or, in explicit form
\begin{equation}\label{eqn:exponentialSandwichCommutator:4}
\begin{aligned}
e^A B e^{-A}
&=
B
+ \inv{1!} \antisymmetric{A}{B}
+ \inv{2!}
\antisymmetric{A}{\antisymmetric{A}{B}}
+ \cdots
\end{aligned}
\end{equation}
