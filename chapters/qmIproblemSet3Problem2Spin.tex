%
% Copyright � 2013 Peeter Joot.  All Rights Reserved.
% Licenced as described in the file LICENSE under the root directory of this GIT repository.
%
\makeproblem{ps III, p2.}{problem:qmIproblemSet3Problem2Spin:2}{
%
A particle with intrinsic angular momentum or spin \(s=1/2\) is prepared in the spin-up with respect to the z-direction state \(\ket{f}=\ket{z+}\). Determine
%
\begin{equation}\label{eqn:qmIproblemSet3:3}
\begin{aligned}
\left(\bra{f} \left( S_z - \bra{f} S_z \ket{f} \BOne \right)^2 \ket{f} \right)^{1/2}
\end{aligned}
\end{equation}
%
and
%
\begin{equation}\label{eqn:qmIproblemSet3:4}
\begin{aligned}
\left(\bra{f} \left( S_x - \bra{f} S_x \ket{f} \BOne \right)^2 \ket{f} \right)^{1/2}
\end{aligned}
\end{equation}
%
and explain what these relations say about the system.

} % problem
%
\makeanswer{problem:qmIproblemSet3Problem2Spin:2}{
\paragraph{Solution:  Uncertainty of \(S_z\) with respect to \(\ket{z+}\)}
\index{uncertainty}

%%%%To start, we note that we have the matrix representations
%%%%
%%%%\begin{align}\label{eqn:qmIproblemSet3:5}
%%%%S_z &=
%%%%\frac{\Hbar}{2}
%%%%\PauliZ \\
%%%%\ket{f} = \ket{z+} &=
%%%%\begin{bmatrix}
%%%%1 \\
%%%%0
%%%%\end{bmatrix}.
%%%%\end{align}
%%%%
%%%%In the matrix representation, the expectation values are straightforward to calculate.  For the expectation of \(S_z\) with respect to this state we have
%%%%
%%%%\begin{align*}
%%%%\bra{f} S_z \ket{f}
%%%%&=
%%%%\frac{\Hbar}{2}
%%%%\begin{bmatrix}
%%%%1 & 0
%%%%\end{bmatrix}
%%%%\PauliZ
%%%%\begin{bmatrix}
%%%%1 \\
%%%%0
%%%%\end{bmatrix} \\
%%%%&=
%%%%\frac{\Hbar}{2}
%%%%\begin{bmatrix}
%%%%1 & 0
%%%%\end{bmatrix}
%%%%\begin{bmatrix}
%%%%1 \\
%%%%0
%%%%\end{bmatrix} \\
%%%%&=
%%%%\frac{\Hbar}{2}
%%%%\end{align*}
%%%%
%%%% SNIP.
%%%%
%%%%We can next compute \(S_z - \bra{f} S_z \ket{f} \BOne\)
%%%%
%%%%\begin{align*}
%%%%S_z - \bra{f} S_z \ket{f} \BOne
%%%%&=
%%%%\frac{\Hbar}{2} \PauliZ - \frac{\Hbar}{2}
%%%%\begin{bmatrix}
%%%%1 & 0 \\
%%%%0 & 1
%%%%\end{bmatrix}
%%%%&=
%%%%\Hbar
%%%%\begin{bmatrix}
%%%%0 & 0 \\
%%%%0 & 1
%%%%\end{bmatrix}.
%%%%\end{align*}
%%%%
%%%%The matrix factor is a projector, squaring to itself, so we have
%%%%
%%%%\begin{align}\label{eqn:qmIproblemSet3:6}
%%%%\bra{f} \left( S_z - \bra{f} S_z \ket{f} \BOne \right)^2 \ket{f}
%%%%&=
%%%%\Hbar^2
%%%%\begin{bmatrix}
%%%%1 & 0
%%%%\end{bmatrix}
%%%%\begin{bmatrix}
%%%%0 & 0 \\
%%%%0 & 1
%%%%\end{bmatrix}
%%%%\begin{bmatrix}
%%%%1 \\
%%%%0
%%%%\end{bmatrix}
%%%%= 0
%%%%\end{align}
%%%%XX

Noting that \(S_z \ket{f} = S_z \ket{z+} = \Hbar/2 \ket{z+}\) we have
%
\begin{equation}\label{eqn:qmIproblemSet3:10}
\begin{aligned}
\bra{f} S_z \ket{f} = \frac{\Hbar}{2}
\end{aligned}
\end{equation}
%
The average outcome for many measurements of the physical quantity associated with the operator \(S_z\) when the system has been prepared in the state \(\ket{f} = \ket{z+}\) is \(\Hbar/2\).
%
\begin{equation}\label{eqn:qmIproblemSet3:11}
\begin{aligned}
\Bigl(S_z - \bra{f} S_z \ket{f} \BOne \Bigr) \ket{f}
&=
\frac{\Hbar}{2} \ket{f}
-\frac{\Hbar}{2} \ket{f} = 0
\end{aligned}
\end{equation}
%
We could also compute this from the matrix representations, but it is slightly more work.

Operating once more with \(S_z - \bra{f} S_z \ket{f} \BOne\) on the zero ket vector still gives us zero, so we have zero in the root for \eqnref{eqn:qmIproblemSet3:3}
\begin{equation}\label{eqn:qmIproblemSet3:3b}
\begin{aligned}
\left(\bra{f} \left( S_z - \bra{f} S_z \ket{f} \BOne \right)^2 \ket{f} \right)^{1/2} = 0
\end{aligned}
\end{equation}
%

%In the variance calculation above we have the operator \(D = S_z - \bra{f} S_z \ket{S_z} \BOne\), and its square.  Both \(D\) and \(D^2\) commute with \(S_z\), and thus have the same eigenstates as \(S_z\).  In particular when the system is prepared in the state \(\ket{z+}\), no measurement of the physical quantity We have a physical quantity associated with the operator \(D\)
%Each of the operators involved in the variance calculation above

%Observe that we are looking at the expectation value of a new operator, say, \(V = (S_z - \bra{f} S_z \ket{f})^2\).  Also observe that \(V\) commutes with \(S_z\), and thus shares the same eigenstates.  A measurement

What does \eqnref{eqn:qmIproblemSet3:3b} say about the state of the system?  Given many measurements of the physical quantity associated with the operator \(V = (S_z - \bra{f} S_z \ket{f} \BOne)^2\), where the initial state of the system is always \(\ket{f} = \ket{z+}\), then the average of the measurements of the physical quantity associated with \(V\) is zero.  We can think of the operator \(V^{1/2} = S_z - \bra{f} S_z \ket{f} \BOne\) as a representation of the observable, ``how different is the measured result from the average \(\bra{f} S_z \ket{f}\)''.

So, given a system prepared in state \(\ket{f} = \ket{z+}\), and performance of repeated measurements capable of only examining spin-up, we find that the system is never any different than its initial spin-up state.  We have no uncertainty that we will measure any difference from spin-up on average, when the system is prepared in the spin-up state.
%
\paragraph{Solution:  Uncertainty of \(S_x\) with respect to \(\ket{z+}\)}
%
For this second part of the problem, we note that we can write
%
\begin{equation}\label{eqn:qmIproblemSet3:20}
\begin{aligned}
\ket{f} = \ket{z+} = \inv{\sqrt{2}} ( \ket{x+} + \ket{x-} ).
\end{aligned}
\end{equation}
%
So the expectation value of \(S_x\) with respect to this state is
\begin{equation}\label{eqn:qmIproblemSet3Problem2Spin:40}
\begin{aligned}
\bra{f} S_x \ket{f}
&=
\inv{2}
( \ket{x+} + \ket{x-} ) S_x ( \ket{x+} + \ket{x-} ) \\
&=
\Hbar
( \ket{x+} + \ket{x-} ) ( \ket{x+} - \ket{x-} ) \\
&=
\Hbar
( 1 + 0 + 0 -1 ) \\
&= 0
\end{aligned}
\end{equation}
%
After repeated preparation of the system in state \(\ket{f}\), the average measurement of the physical quantity associated with operator \(S_x\) is zero.  In terms of the eigenstates for that operator \(\ket{x+}\) and \(\ket{x-}\) we have equal probability of measuring either given this particular initial system state.

For the variance calculation, this reduces our problem to the calculation of \(\bra{f} S_x^2 \ket{f}\), which is
%
\begin{equation}\label{eqn:qmIproblemSet3Problem2Spin:60}
\begin{aligned}
\bra{f} S_x^2 \ket{f}
&=
\inv{2} \left( \frac{\Hbar}{2} \right)^2
( \ket{x+} + \ket{x-} ) ( (+1)^2 \ket{x+} + (-1)^2 \ket{x-} ) \\
&=
\left( \frac{\Hbar}{2} \right)^2,
\end{aligned}
\end{equation}
%
so for \eqnref{eqn:qmIproblemSet3:4b} we have
%
\begin{equation}\label{eqn:qmIproblemSet3:4b}
\begin{aligned}
\left(\bra{f} \left( S_x - \bra{f} S_x \ket{f} \BOne \right)^2 \ket{f} \right)^{1/2} = \frac{\Hbar}{2}
\end{aligned}
\end{equation}
%
The average of the absolute magnitude of the physical quantity associated with operator \(S_x\) is found to be \(\Hbar/2\) when repeated measurements are performed given a system initially prepared in state \(\ket{f} = \ket{z+}\).  We saw that the average value for the measurement of that physical quantity itself was zero, showing that we have equal probabilities of measuring either \(\pm \Hbar/2\) for this experiment.  A measurement that would show the system was in the x-direction spin-up or spin-down states would find that these states are equi-probable.
} % answer
