%
% Copyright © 2012 Peeter Joot.  All Rights Reserved.
% Licenced as described in the file LICENSE under the root directory of this GIT repository.
%
HOMEWORK: go through the steps to understand how to formulate \(\spacegrad^2\) in spherical polar coordinates.  This is a lot of work, but is good practice and background for dealing with the Hydrogen atom, something with spherical symmetry that is most naturally analyzed in the spherical polar coordinates.

\index{spin 1/2}
\index{spherical polar coordinates}
\index{Laplacian}
In spherical coordinates (We will not go through this here, but it is good practice) with
%
\begin{equation}\label{eqn:lecture5BigReviewAngularMomentum:5020}
\begin{aligned}
x &= r \sin\theta \cos\phi \\
y &= r \sin\theta \sin\phi \\
z &= r \cos\theta
\end{aligned}
\end{equation}

we have with \(u = u(r,\theta, \phi)\)
%
\begin{equation}\label{eqn:lecture5BigReviewAngularMomentum:5040}
\begin{aligned}
-\frac{\Hbar^2}{2m} \left(
\inv{r} \partial_{rr} (r u) +  \inv{r^2 \sin\theta} \partial_\theta (\sin\theta \partial_\theta u)
+ \inv{r^2 \sin^2\theta} \partial_{\phi\phi} u
 \right)
&= E u
\end{aligned}
\end{equation}

We see the start of a separation of variables attack with \(u = R(r) Y(\theta, \phi)\).  We end up with
%
\begin{equation}\label{eqn:lecture5BigReviewAngularMomentum:5060}
\begin{aligned}
-\frac{\Hbar^2}{2m} &\left(
\frac{r}{R} (r R')' +  \inv{Y \sin\theta} \partial_\theta (\sin\theta \partial_\theta Y)
+ \inv{Y \sin^2\theta} \partial_{\phi\phi} Y
 \right) \\
\end{aligned}
\end{equation}
%
\begin{equation}\label{eqn:lecture5BigReviewAngularMomentum:5080}
\begin{aligned}
r (r R')' + \left( \frac{2m E}{\Hbar^2} r^2 - \lambda \right) R &= 0
\end{aligned}
\end{equation}
\begin{equation}\label{eqn:lecture5BigReviewAngularMomentum:5100}
\begin{aligned}
\inv{Y \sin\theta} \partial_\theta (\sin\theta \partial_\theta Y) + \inv{Y \sin^2\theta} \partial_{\phi\phi} Y &= -\lambda
\end{aligned}
\end{equation}

Application of separation of variables again, with \(Y = P(\theta) Q(\phi)\) gives us
%
\begin{equation}\label{eqn:lecture5BigReviewAngularMomentum:5120}
\begin{aligned}
\inv{P \sin\theta} \partial_\theta (\sin\theta \partial_\theta P) + \inv{Q \sin^2\theta} \partial_{\phi\phi} Q &= -\lambda
\end{aligned}
\end{equation}
%
\begin{equation}\label{eqn:lecture5BigReviewAngularMomentum:5140}
\begin{aligned}
\frac{\sin\theta}{P } \partial_\theta (\sin\theta \partial_\theta P)
+\lambda  \sin^2\theta
+ \inv{Q } \partial_{\phi\phi} Q &= 0
\end{aligned}
\end{equation}
%
\begin{equation}\label{eqn:lecture5BigReviewAngularMomentum:5160}
\begin{aligned}
\frac{\sin\theta}{P } \partial_\theta (\sin\theta \partial_\theta P) + \lambda \sin^2\theta - \mu = 0
\inv{Q } \partial_{\phi\phi} Q &= -\mu
\end{aligned}
\end{equation}

or
\begin{equation}\label{eqn:PHY356F:1000}
\begin{aligned}
\frac{1}{P \sin\theta} \partial_\theta (\sin\theta \partial_\theta P) +\lambda -\frac{\mu}{\sin^2\theta} &= 0
\end{aligned}
\end{equation}
\begin{equation}\label{eqn:PHY356F:2000}
\begin{aligned}
\partial_{\phi\phi} Q &= -\mu Q
\end{aligned}
\end{equation}

The equation for \(P\) can be solved using the \textAndIndex{Legendre function} \(P_l^m(\cos\theta)\) where \(\lambda = l(l+1)\) and \(l\) is an integer

Replacing \(\mu\) with \(m^2\), where \(m\) is an integer
%
\begin{equation}\label{eqn:lecture5BigReviewAngularMomentum:5180}
\begin{aligned}
\frac{d^2 Q}{d\phi^2} &= -m^2 Q
\end{aligned}
\end{equation}

Imposing a periodic boundary condition \(Q(\phi) = Q(\phi + 2\pi)\), where (\(m = 0, \pm 1, \pm 2, \cdots\)) we have
%
\begin{equation}\label{eqn:lecture5BigReviewAngularMomentum:5200}
\begin{aligned}
Q &= \inv{\sqrt{2\pi}} e^{im\phi}
\end{aligned}
\end{equation}

There is the overall solution \(r(r,\theta,\phi) = R(r) Y(\theta, \phi)\) for a free particle.  The functions \(Y(\theta, \phi)\) are
%
\begin{equation}\label{eqn:lecture5BigReviewAngularMomentum:5220}
\begin{aligned}
Y_{lm}(\theta, \phi)
&= N \left( \inv{\sqrt{2\pi}} e^{im\phi} \right)
\mathLabelBox{ P_l^m(\cos\theta) }{\( -l \le m \le l\)}
\end{aligned}
\end{equation}

where \(N\) is a normalization constant, and \(m = 0, \pm 1, \pm 2, \cdots\).  \(Y_{lm}\) is an eigenstate of the \(\BL^2\) operator and \(L_z\) (two for the price of one).  There is no specific reason for the direction \(z\), but it is the direction picked out of convention.

\index{angular momentum}
\index{L squared}
\index{Y lm}
Angular momentum is given by
%
\begin{equation}\label{eqn:lecture5BigReviewAngularMomentum:5240}
\begin{aligned}
\BL = \Br \cross \Bp
\end{aligned}
\end{equation}

where
%
\begin{equation}\label{eqn:lecture5BigReviewAngularMomentum:5260}
\begin{aligned}
\BR = x \xcap + y\ycap + z\zcap
\end{aligned}
\end{equation}

and
\begin{equation}\label{eqn:lecture5BigReviewAngularMomentum:5280}
\begin{aligned}
\Bp = p_x \xcap + p_y\ycap + p_z\zcap
\end{aligned}
\end{equation}

The important thing to remember is that the aim of following all the math is to show that
%
\begin{equation}\label{eqn:lecture5BigReviewAngularMomentum:5300}
\begin{aligned}
\BL^2 Y_{lm} = \Hbar^2 l (l+1) Y_{lm}
\end{aligned}
\end{equation}

and simultaneously
%
\begin{equation}\label{eqn:lecture5BigReviewAngularMomentum:5320}
\begin{aligned}
\BL_z Y_{lm} = \Hbar m Y_{lm}
\end{aligned}
\end{equation}

Part of the solution involves working with \(\antisymmetric{L_z}{L_{+}}\), and \(\antisymmetric{L_z}{L_{-}}\), where
%
\begin{equation}\label{eqn:lecture5BigReviewAngularMomentum:5340}
\begin{aligned}
L_{+} &= L_x + i L_y \\
L_{-} &= L_x - i L_y
\end{aligned}
\end{equation}

An exercise (not in the book) is to evaluate
\begin{equation}\label{eqn:PHY356F:4000}
\begin{aligned}
\antisymmetric{L_z}{L_{+}}
&= L_z L_x + i L_z L_y - L_x L_z - i L_y L_z
\end{aligned}
\end{equation}

where
\begin{equation}\label{eqn:PHY356F:5000}
\begin{aligned}
\antisymmetric{L_x}{L_y}  &= i \Hbar L_z \\
\antisymmetric{L_y}{L_z}  &= i \Hbar L_x \\
\antisymmetric{L_z}{L_x}  &= i \Hbar L_y
\end{aligned}
\end{equation}

Substitution back in \eqnref{eqn:PHY356F:4000} we have
%
\begin{equation}\label{eqn:lecture5BigReviewAngularMomentum:5360}
\begin{aligned}
\antisymmetric{L_z}{L_{+}}
&=
\antisymmetric{L_z}{L_x}
+ i \antisymmetric{L_z}{L_y}  \\
&=
i \Hbar ( L_y - i L_x ) \\
&=
\Hbar ( i L_y +  L_x ) \\
&=
\Hbar L_{+}
\end{aligned}
\end{equation}
