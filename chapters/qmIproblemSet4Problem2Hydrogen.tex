%
% Copyright � 2013 Peeter Joot.  All Rights Reserved.
% Licenced as described in the file LICENSE under the root directory of this GIT repository.
%
%
\makeproblem{ps4, p2.}{problem:qmIproblemSet4Problem2Hydrogen:2}{
%
For the hydrogen atom, determine \(\bra{nlm}(1/R)\ket{nlm}\) and \(1/\bra{nlm}R\ket{nlm}\) such that \((nlm)=(211)\) and \(R\) is the radial position operator \((X^2+Y^2+Z^2)^{1/2}\). What do these quantities represent physically and are they the same?
} % problem
%
\makeanswer{problem:qmIproblemSet4Problem2Hydrogen:2}{
%
Both of the computation tasks for the hydrogen like atom require expansion of a braket of the following form
%
\begin{equation}\label{eqn:qmIproblemSet4:200}
\bra{nlm} A(R) \ket{nlm},
\end{equation}
%
where \(A(R) = R = (X^2 + Y^2 + Z^2)^{1/2}\) or \(A(R) = 1/R\).
%
\index{identity!spherical}
The spherical representation of the identity resolution is required to convert this braket into integral form
%
\begin{equation}\label{eqn:qmIproblemSet4:202}
\BOne = \int r^2 \sin\theta dr d\theta d\phi
\ket{ r \theta \phi}
\bra{ r \theta \phi},
\end{equation}
%
where the spherical wave function is given by the braket \(\braket{ r \theta \phi}{nlm} = R_{nl}(r) Y_{lm}(\theta,\phi)\).

Additionally, the radial form of the delta function will be required, which is
\begin{equation}\label{eqn:qmIproblemSet4:204}
\delta(\Bx - \Bx') = \inv{r^2 \sin\theta} \delta(r - r') \delta(\theta - \theta') \delta(\phi - \phi')
\end{equation}
%
Two applications of the identity operator to the braket yield
\begin{equation}\label{eqn:qmIproblemSet4Problem2Hydrogen:246}
\begin{aligned}
&\bra{nlm} A(R) \ket{nlm} = \bra{nlm} \BOne A(R) \BOne \ket{nlm} \\
&=
\int
dr d\theta d\phi
dr' d\theta' d\phi'
r^2 \sin\theta
{r'}^2 \sin\theta' \\
&\quad
\braket{nlm}{ r \theta \phi}
\bra{ r \theta \phi} A(R)
\ket{ r' \theta' \phi'}
\braket{ r' \theta' \phi'}{nlm} \\
&=
\int
dr d\theta d\phi
dr' d\theta' d\phi'
r^2 \sin\theta
{r'}^2 \sin\theta' \\
&\quad
R_{nl}(r) Y_{lm}^\conj(\theta, \phi)
\bra{ r \theta \phi} A(R) \ket{ r' \theta' \phi'}
R_{nl}(r') Y_{lm}(\theta', \phi') \\
\end{aligned}
\end{equation}
%
To continue an assumption about the matrix element \(\bra{ r \theta \phi} A(R) \ket{ r' \theta' \phi'}\) is required.  It seems reasonable that this would be
\begin{equation}\label{eqn:qmIproblemSet4:206}
\bra{ r \theta \phi} A(R) \ket{ r' \theta' \phi'} = \\
\delta(\Bx - \Bx') A(r) = \inv{r^2 \sin\theta} \delta(r-r') \delta(\theta -\theta')\delta(\phi-\phi') A(r).
\end{equation}
%
The braket can now be written completely in integral form as
\begin{equation}\label{eqn:qmIproblemSet4Problem2Hydrogen:266}
\begin{aligned}
&\bra{nlm} A(R) \ket{nlm} \\
&=
\int
dr d\theta d\phi
dr' d\theta' d\phi'
r^2 \sin\theta
{r'}^2 \sin\theta'  \\
&\quad R_{nl}(r) Y_{lm}^\conj(\theta, \phi)
\inv{r^2 \sin\theta} \delta(r-r') \delta(\theta -\theta')\delta(\phi-\phi') A(r)
R_{nl}(r') Y_{lm}(\theta', \phi') \\
&=
\int
dr d\theta d\phi
{r'}^2 \sin\theta' dr' d\theta' d\phi'
R_{nl}(r) Y_{lm}^\conj(\theta, \phi) \\
&\quad
\delta(r-r') \delta(\theta -\theta')\delta(\phi-\phi') A(r)
R_{nl}(r') Y_{lm}(\theta', \phi') \\
\end{aligned}
\end{equation}
%
Application of the delta functions then reduces the integral, since the only \(\theta\), and \(\phi\) dependence is in the (orthonormal) \(Y_{lm}\) terms they are found to drop out
\begin{equation}\label{eqn:qmIproblemSet4Problem2Hydrogen:286}
\begin{aligned}
\bra{nlm} A(R) \ket{nlm}
&=
\int
dr d\theta d\phi
r^2 \sin\theta
R_{nl}(r) Y_{lm}^\conj(\theta, \phi)
A(r)
R_{nl}(r) Y_{lm}(\theta, \phi) \\
&=
\int
dr
r^2
R_{nl}(r)
A(r)
R_{nl}(r)
\mathLabelBox{\int
\sin\theta d\theta d\phi
Y_{lm}^\conj(\theta, \phi)
Y_{lm}(\theta, \phi) }{\(=1\)}
\\
\end{aligned}
\end{equation}
%
This leaves just the radial wave functions in the integral
\begin{equation}\label{eqn:qmIproblemSet4:208}
\bra{nlm} A(R) \ket{nlm}
=
\int
dr
r^2
R_{nl}^2(r)
A(r)
\end{equation}
%
As a consistency check, observe that with \(A(r) = 1\), this integral evaluates to 1 according to equation (8.274) in the text, so we can think of \((r R_{nl}(r))^2\) as the radial probability density for functions of \(r\).

The problem asks specifically for these expectation values for the \(\ket{211}\) state.  For that state the radial wavefunction is found in (8.277) as
%
\begin{equation}\label{eqn:qmIproblemSet4:210}
R_{21}(r) =
\left(\frac{Z}{2a_0}\right)^{3/2} \frac{ Z r }{a_0 \sqrt{3}} e^{-Z r/2 a_0}
\end{equation}
%
The braket can now be written explicitly
\begin{equation}\label{eqn:qmIproblemSet4:212}
\bra{21m} A(R) \ket{21m}
=
%\left(\frac{Z}{2a_0}\right)^{3} \frac{ Z^2 }{3 a_0^2 }
\inv{24} \left(\frac{ Z }{a_0 } \right)^5
\int_0^\infty
dr
r^4
e^{-Z r/ a_0}
A(r)
\end{equation}
%
Now, let us consider the two functions \(A(r)\) separately.  First for \(A(r) = r\) we have
\begin{equation}\label{eqn:qmIproblemSet4Problem2Hydrogen:306}
\begin{aligned}
\bra{21m} R \ket{21m}
&=
\inv{24} \left(\frac{ Z }{a_0 } \right)^5
\int_0^\infty
dr
r^5
e^{-Z r/ a_0}
\\
&=
\frac{ a_0 }{ 24 Z }
\int_0^\infty
du
u^5
e^{-u}
\\
\end{aligned}
\end{equation}
%
% 5 a_0 / Z
The last integral evaluates to \(120\), leaving
\begin{equation}\label{eqn:qmIproblemSet4:220a}
\bra{21m} R \ket{21m}
=
\frac{ 5 a_0 }{ Z }.
\end{equation}
%
The expectation value associated with this \(\ket{21m}\) state for the radial position is found to be proportional to the Bohr radius.  For the hydrogen atom where \(Z=1\) this average value for repeated measurements of the physical quantity associated with the operator \(R\) is found to be 5 times the Bohr radius for \(n=2, l=1\) states.

Our problem actually asks for the inverse of this expectation value, and for reference this is
\begin{equation}\label{eqn:qmIproblemSet4:220}
1/ \bra{21m} R \ket{21m}
=
\frac{ Z }{ 5 a_0 } %= n^2 \frac{a_0}{Z}
\end{equation}
%
Performing the same task for \(A(R) = 1/R\)
\begin{equation}\label{eqn:qmIproblemSet4Problem2Hydrogen:326}
\begin{aligned}
\bra{21m} 1/R \ket{21m}
&=
\inv{24} \left(\frac{ Z }{a_0 } \right)^5
\int_0^\infty
dr
r^3
e^{-Z r/ a_0}
\\
&=
\inv{24} \frac{ Z }{ a_0 }
\int_0^\infty
du
u^3
e^{-u}.
\end{aligned}
\end{equation}
%
This last integral has value \(6\), and we have the second part of the computational task complete
\begin{equation}\label{eqn:qmIproblemSet4:225}
\bra{21m} 1/R \ket{21m} = \inv{4} \frac{ Z }{ a_0 }
\end{equation}
%
The question of whether or not \eqnref{eqn:qmIproblemSet4:220}, and \eqnref{eqn:qmIproblemSet4:225} are equal is answered.  They are not.

Still remaining for this problem is the question of the what these quantities represent physically.

The quantity \(\bra{nlm} R \ket{nlm}\) is the expectation value for the radial position of the particle measured from the center of mass of the system.  This is the average outcome for many measurements of this radial distance when the system is prepared in the state \(\ket{nlm}\) prior to each measurement.

Interestingly, the physical quantity that we associate with the operator \(R\) has a different measurable value than the inverse of the expectation value for the inverted operator \(1/R\).  Regardless, we have a physical (observable) quantity associated with the operator \(1/R\), and when the system is prepared in state \(\ket{21m}\) prior to each measurement, the average outcome of many measurements of this physical quantity produces this value \(\bra{21m} 1/R \ket{21m} = Z/n^2 a_0\), a quantity inversely proportional to the Bohr radius.
%
\paragraph{ASIDE: Comparing to the general case}
%
As a confirmation of the results obtained, we can check \eqnref{eqn:qmIproblemSet4:220}, and \eqnref{eqn:qmIproblemSet4:225} against the general form of the expectation values \(\expectation{R^s}\) for various powers \(s\) of the radial position operator.  These can be found in locations such as \href{http://farside.ph.utexas.edu/teaching/qmech/lectures/node81.html}{farside.ph.utexas.edu} which gives for \(Z=1\) (without proof), and in \citep{liboff2003iqm} (where these and harder looking ones expectation values are left as an exercise for the reader to prove).  Both of those give:
\begin{equation}\label{eqn:qmIproblemSet4:226}
\begin{aligned}
\expectation{R} &= \frac{a_0}{2} ( 3 n^2 -l (l+1) ) \\
\expectation{1/R} &= \frac{1}{n^2 a_0}
\end{aligned}
\end{equation}
%
It is curious to me that the general expectation values noted in \eqnref{eqn:qmIproblemSet4:226} we have a \(l\) quantum number dependence for \(\expectation{R}\), but only the \(n\) quantum number dependence for \(\expectation{1/R}\).  It is not obvious to me why this would be the case.
} % answer
%%%\subsection{Suggested non-submission problem 1}
%%%\subsubsection{Statement}
%%%In Section 8.1.1, why are boundary conditions imposed on \(u(x)\) at \(x=\pm a\)?
%%%\subsubsection{Solution}
%%%
%%%\subsection{Suggested non-submission problem 2}
%%%\subsubsection{Statement}
%%%
%%%In Section 8.1.2, what is the probability current density for \(u(x)\). Does this make physical sense?
%%%
%%%\subsubsection{Solution}
%%%
%%%\subsection{Suggested non-submission problem 3}
%%%\subsubsection{Statement}
%%%
%%%Why are the states described in Sections 8.6.1 and 8.8 degenerate?
%%%
%%%\subsubsection{Solution}
%
%\subsubsection{Sanity check}
%\begin{align*}
%\braket{nlm}{nlm}
%&=
%\inv{24} \left(\frac{ Z }{a_0 } \right)^5
%\int_0^\infty
%dr
%r^4
%e^{-Z r/ a_0} \\
%&=
%\inv{24}
%\int_0^\infty
%du
%u^4
%e^{-u} \\
%&= 1
%\end{align*}
