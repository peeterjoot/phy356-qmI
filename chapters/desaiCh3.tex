%
% Copyright � 2012 Peeter Joot.  All Rights Reserved.
% Licenced as described in the file LICENSE under the root directory of this GIT repository.
%

%\chapter{Notes and problems for Desai chapter III}
\label{chap:desaiCh3}
%\blogpage{http://sites.google.com/site/peeterjoot/math2010/desaiCh3.pdf}
%\date{Oct 1, 2010}
%\section{Notes}
%
%Chapter III notes and problems for \citep{desai2009quantum}.
%
%FIXME:
%Some puzzling stuff in the interaction section and superposition of time-dependent states sections.  Work through those here.
\makeoproblem{Virial Theorem}{problem:desaiCh3:1}{\citep{desai2009quantum} pr 3.1}{
\index{virial theorem}
With the assumption that \(\expectation{\Br \cdot \Bp}\) is independent of time, and
\begin{equation}\label{eqn:desaiCh3:100}
H = \frac{\Bp^2}{2m} + V(\Br) = T + V,
\end{equation}
show that
\begin{equation}\label{eqn:desaiCh3:102}
2 \expectation{T} = \expectation{ \Br \cdot \spacegrad V}.
\end{equation}
} % problem
\makeanswer{problem:desaiCh3:1}{
I floundered with this a bit, but found the required hint in
%\href{htts://www.physicsforums.com/showthread.php?t=164682}{physicsforums}
\href{https://www.physicsforums.com/threads/three-dimensional-virial-theorem-quantum-mechanics.164682/}{physicsforums}
.  We can start with the Hamiltonian time derivative relation
\begin{equation}\label{eqn:desaiCh3:103}
i\Hbar \frac{d A_H}{dt} = \antisymmetric{A_H}{H}.
\end{equation}
So, with the assumption that \(\expectation{\Br \cdot \Bp}\) is independent of time, and the use of a stationary state \(\ket{\psi}\) for the expectation calculation we have
\begin{equation}\label{eqn:desaiCh3:1121}
\begin{aligned}
0 &=
\frac{d}{dt} \expectation{\Br \cdot \Bp}  \\
&=
\frac{d}{dt} \bra{\psi} \Br \cdot \Bp \ket{\psi} \\
&=
\bra{\psi}
\frac{d}{dt} ( \Br \cdot \Bp ) \ket{\psi} \\
&=
\inv{i\Hbar} \expectation{ \antisymmetric{ \Br \cdot \Bp }{H} } \\
&=
-\expectation{ \antisymmetric{ \Br \cdot \spacegrad }{\frac{\Bp^2}{2m}} }
-\expectation{ \antisymmetric{ \Br \cdot \spacegrad }{V(\Br)} }.
\end{aligned}
\end{equation}

The exercise now becomes one of evaluating the remaining commutators.  For the Laplacian commutator we have

\begin{equation}\label{eqn:desaiCh3:1141}
\begin{aligned}
\antisymmetric{ \Br \cdot \spacegrad }{\spacegrad^2} \psi
&=
x_m \partial_m \partial_n \partial_n \psi
- \partial_n \partial_n x_m \partial_m \psi \\
&=
x_m \partial_m \partial_n \partial_n \psi
- \partial_n \partial_n \psi
- \partial_n x_m \partial_n \partial_m \psi \\
&=
x_m \partial_m \partial_n \partial_n \psi
- \partial_n \partial_n \psi
- \partial_n \partial_n \psi
- x_m \partial_n \partial_n \partial_m \psi \\
&=
- 2 \spacegrad^2 \psi
\end{aligned}
\end{equation}

For the potential commutator we have

\begin{equation}\label{eqn:desaiCh3:1161}
\begin{aligned}
\antisymmetric{ \Br \cdot \spacegrad }{V(\Br)} \psi
&=
x_m \partial_m V \psi
-V x_m \partial_m \psi  \\
&=
x_m (\partial_m V) \psi
x_m V \partial_m \psi
-V x_m \partial_m \psi  \\
&=
\Bigl( \Br \cdot (\spacegrad V) \Bigr) \psi
\end{aligned}
\end{equation}

Putting all the \(\Hbar\) factors back in, we get

\begin{equation}\label{eqn:desaiCh3:104}
2 \expectation{ \frac{\Bp^2}{2m} } = \expectation{ \Br \cdot (\spacegrad V) },
\end{equation}

which is the desired result.

Followup: why assume \(\expectation{\Br \cdot \Bp}\) is independent of time?

%\href{http://www.caelestis.de/dateien/UEA05_2.pdf}{http://www.caelestis.de/dateien/UEA05_2.pdf} (ie: we need the Hamiltonian commutator).
} % answer


\makeoproblem{Application of virial theorem}{problem:desaiCh3:2}{\citep{desai2009quantum} pr 3.2}{

Calculate \(\expectation{T}\) with \(V = \lambda \ln(r/a)\).

} % problem

\makeanswer{problem:desaiCh3:2}{
\begin{equation}\label{eqn:desaiCh3:1181}
\begin{aligned}
\Br \cdot \spacegrad V
&= r \rcap \cdot \rcap \lambda \PD{r}{\ln(r/a)} \\
&= \lambda r \inv{a} \frac{a}{r} \\
&= \lambda  \\
\implies \\
\expectation{T} &= \lambda/2
\end{aligned}
\end{equation}

} % answer



\makeoproblem{Heisenberg Position operator representation}{problem:desaiCh3:3}{\citep{desai2009quantum} pr 3.3}{
\index{Heisenberg picture!position operator}
} % problem

\makeanswer{problem:desaiCh3:3}{
\paragraph{Part I}
Express \(x\) as an operator \(x_H\) for \(H = \Bp^2/2m\).

With

\begin{equation}\label{eqn:desaiCh3:1201}
\bra{\psi} x \ket{\psi} = \bra{\psi_0} U^\dagger x U \ket{\psi_0}
\end{equation}

We want to expand
\begin{equation}\label{eqn:desaiCh3:1221}
\begin{aligned}
x_H
&= U^\dagger x U \\
&= e^{i H t/\Hbar} x e^{-iH t/\Hbar} \\
&= \sum_{k,l = 0}^\infty \inv{k!} \inv{l!}
\left(\frac{i H t}{\Hbar}\right)^k
x
\left(\frac{-i H t}{\Hbar}\right)^l .
\end{aligned}
\end{equation}

We to evaluate \(H^k x H^l\) to proceed.  Using \(p^n x = -i \Hbar n p^{n-1} + x p^n\), we have

\begin{equation}\label{eqn:desaiCh3:1241}
\begin{aligned}
H^k x
&= \inv{(2m)^k} p^2k x \\
&= \inv{(2m)^k} \left( -i \Hbar (2k) p^{2k -1} + x p^2k \right) \\
&= x H^k + \inv{2m} (-i \Hbar) (2k) p p^{2(k-1)}/(2m)^{k-1} \\
&= x H^k - \frac{i \Hbar k}{m} p H^{k-1}.
\end{aligned}
\end{equation}

This gives us
\begin{equation}\label{eqn:desaiCh3:1261}
\begin{aligned}
x_H
&= x - \frac{i \Hbar p }{m} \sum_{k,l=0}^\infty \frac{k}{k!} \inv{l!}
\left(\frac{i t}{\Hbar}\right)^k H^{k-1 + l}
\left(\frac{-i t}{\Hbar}\right)^l  \\
&= x - \frac{i \Hbar p i t }{m \Hbar}
\end{aligned}
\end{equation}

Or
\begin{equation}\label{eqn:desaiCh3:303}
x_H
= x + \frac{p t }{m}
\end{equation}

\paragraph{Part II}
Express \(x\) as an operator \(x_H\) for \(H = \Bp^2/2m + V\) with \(V = \lambda x^m\).

In retrospect, for the first part of this problem, it would have been better to use the series expansion for this exponential sandwich

Or, in explicit form
\begin{equation}\label{eqn:desaiCh3:304}
e^A B e^{-A}
=
B
+ \inv{1!} \antisymmetric{A}{B}
+ \inv{2!}
\antisymmetric{A}{\antisymmetric{A}{B}}
+ \cdots
\end{equation}

Doing so, we would find for the first commutator

\begin{equation}\label{eqn:desaiCh3:305}
\frac{i t}{2m \Hbar} \antisymmetric{\Bp^2}{x} = \frac{t p}{m},
\end{equation}

so that the series has only the first two terms, and we would obtain the same result.  That seems like a logical approach to try here too.  For the first commutator, we get the same \(tp/m\) result since \(\antisymmetric{V}{x} = 0\).

Employing
\begin{equation}\label{eqn:desaiCh3:306}
x^n p = i \Hbar n x^{n-1} + p x^n,
\end{equation}

I find

\begin{equation}\label{eqn:desaiCh3:1281}
\begin{aligned}
\left( \frac{i t}{\Hbar} \right)^2 \antisymmetric{H}{\antisymmetric{H}{x}}
&= \frac{i \lambda t^2}{\Hbar m } \antisymmetric{x^n}{p}  \\
&= - \frac{n t^2 \lambda}{m} x^{n-1} \\
&= - \frac{n t^2 V}{m x} \\
\end{aligned}
\end{equation}

The triple commutator gets no prettier, and I get

\begin{equation}\label{eqn:desaiCh3:1301}
\begin{aligned}
\left( \frac{i t}{\Hbar} \right)^3 \antisymmetric{H}{\antisymmetric{H}{\antisymmetric{H}{x}}}
&=
\frac{it}{\Hbar} \antisymmetric{ \frac{\Bp^2}{2m} + \lambda x^n}{ - \frac{n t^2 V}{m x} } \\
&=
-\frac{it}{\Hbar} \frac{n t^2 }{m } \frac{\lambda}{2m} \antisymmetric{\Bp^2}{ x^{n-1}} \\
&= \cdots \\
&= \frac{n(n-1)t^3 V}{ 2 m^2 x^3 } (i \Hbar n + 2 p x).
\end{aligned}
\end{equation}

Putting all the pieces together this gives

\begin{equation}\label{eqn:desaiCh3:307}
x_H =
e^{iH t/\Hbar} x e^{-iH t/\Hbar}  =
x + \frac{tp}{m} - \frac{n t^2 V}{ 2 m x}
+ \frac{n(n-1)t^3 V}{ 12 m^2 x^3 } (i \Hbar n + 2 p x) + \cdots
\end{equation}

If there is a closed form for this it is not obvious to me.  Would a fixed lower degree potential function shed any more light on this.  How about the Harmonic oscillator Hamiltonian

\begin{equation}\label{eqn:desaiCh3:308}
H = \frac{p^2}{2m} + \frac{m \omega^2 }{2} x^2
\end{equation}

... this one works out nicely since there is an even-odd alternation.

Get

\begin{equation}\label{eqn:desaiCh3:309}
x_H = x \cos (\omega^2 t^2 /2) +
\frac{ p t }{m} \frac{\sin( \omega^2 t^2/2)}{ \omega^2 t^2/2 }
\end{equation}

I had not expect such a tidy result for an arbitrary \(V(x) = \lambda x^n\) potential.

} % answer

\makeoproblem{Feynman-Hellman relation}{problem:desaiCh3:4}{\citep{desai2009quantum} pr 3.4}{
\index{Feynman-Hellman relation}
} % problem

\makeanswer{problem:desaiCh3:4}{
For continuously parameterized eigenstate, eigenvalue and Hamiltonian \(\ket{\psi(\lambda)}\), \(E(\lambda)\) and \(H(\lambda)\) respectively, we can relate the derivatives

\begin{equation}\label{eqn:desaiCh3:1321}
\begin{aligned}
\PD{\lambda}{} ( H \ket{\psi} ) &= \PD{\lambda}{} ( E \ket{\psi} ) \\
\implies \\
\PD{\lambda}{H} \ket{\psi} +H \PD{\lambda}{\ket{\psi}} &= \PD{\lambda}{E} \ket{\psi} + E \PD{\lambda}{\ket{\psi} }
\end{aligned}
\end{equation}

Left multiplication by \(\bra{\psi}\) gives

\begin{equation}\label{eqn:desaiCh3:1341}
\begin{aligned}
\bra{\psi}\PD{\lambda}{H} \ket{\psi} +\bra{\psi}H \PD{\lambda}{\ket{\psi}} &= \bra{\psi}\PD{\lambda}{E} \ket{\psi} +  E \bra{\psi}\PD{\lambda}{\ket{\psi} } \\
\implies \\
\bra{\psi}\PD{\lambda}{H} \ket{\psi} +(\bra{\psi}E) \PD{\lambda}{\ket{\psi}} &= \bra{\psi}\PD{\lambda}{E} \ket{\psi} +  E \bra{\psi}\PD{\lambda}{\ket{\psi} } \\
\implies \\
\bra{\psi}\PD{\lambda}{H} \ket{\psi} &= \PD{\lambda}{E} \braket{\psi}{\psi},
\end{aligned}
\end{equation}

which provides the desired identity
\begin{equation}\label{eqn:desaiCh3:4}
\PD{\lambda}{E}
= \bra{\psi(\lambda)}\PD{\lambda}{H} \ket{\psi(\lambda)}
%= \expectation{ \psi(\lambda) \PD{\lambda}{H} \psi(\lambda) }
\end{equation}
} % answer


\makeoproblem{}{problem:desaiCh3:5}{\citep{desai2009quantum} pr 3.5}{

With eigenstates \(\ket{\phi_1}\) and \(\ket{\phi_2}\), of \(H\) with eigenvalues \(E_1\) and \(E_2\), respectively, and

\begin{equation}\label{eqn:desaiCh3:1361}
\begin{aligned}
\ket{\chi_1} &= \inv{\sqrt{2}}( \ket{\phi_1} +\ket{\phi_2}) \\
\ket{\chi_2} &= \inv{\sqrt{2}}( \ket{\phi_1} -\ket{\phi_2})
\end{aligned}
\end{equation}

and \(\ket{\psi(0)} = \ket{\chi_1}\), determine \(\ket{\psi(t)}\) in terms of \(\ket{\phi_1}\) and \(\ket{\phi_2}\).

} % problem

\makeanswer{problem:desaiCh3:5}{

\begin{equation}\label{eqn:desaiCh3:1381}
\begin{aligned}
\ket{\psi(t)}
&= e^{-i H t /\Hbar} \ket{\psi(0)} \\
&= e^{-i H t /\Hbar} \ket{\chi_1} \\
&= \inv{\sqrt{2}} e^{-i H t /\Hbar} ( \ket{\phi_1} -\ket{\phi_2}) \\
&= \inv{\sqrt{2}} (e^{-i E_1 t /\Hbar} \ket{\phi_1} -e^{-i E_2 t /\Hbar} \ket{\phi_2} )
\qedmarker
\end{aligned}
\end{equation}

} % answer



\makeoproblem{}{problem:desaiCh3:6}{\citep{desai2009quantum} pr 3.6}{

Consider a Coulomb like potential \(-\lambda/r\) with angular momentum \(l=0\).  If the eigenfunction is

\begin{equation}\label{eqn:desaiCh3:700}
u(r) = u_0 e^{-\beta r}
\end{equation}

determine \(u_0\), \(\beta\), and the energy eigenvalue \(E\) in terms of \(\lambda\), and \(m\).

} % problem

\makeanswer{problem:desaiCh3:6}{

We can start with the normalization constant \(u_0\) by integrating

\begin{equation}\label{eqn:desaiCh3:1401}
\begin{aligned}
1 &= u_0^2 \int_0^\infty dr e^{-\beta r} e^{-\beta r}  \\
&=
u_0^2 \left. \frac{e^{-2 \beta r}}{-2 \beta} \right\vert_{0^\infty} \\
&= u_0^2 \inv{2\beta} \\
\end{aligned}
\end{equation}

\begin{equation}\label{eqn:desaiCh3:701}
u_0 = \sqrt{2\beta}
\end{equation}

To go further, we need the Hamiltonian.  Note that we can write the Laplacian with the angular momentum operator factored out using

\begin{equation}\label{eqn:desaiCh3:702}
\spacegrad^2 = \inv{\Bx^2} \left( (\Bx \cdot \spacegrad)^2 + \Bx \cdot \spacegrad + (\Bx \cross \spacegrad)^2 \right)
\end{equation}

%ddFor constant \(\Ba\), This can be motivated by the factorization $\gpgradezero{ (\Ba \spacegrad)^2 } = (\Ba \cdot \spacegrad)^2

With zero for the angular momentum operator \(\Bx \cross \spacegrad\), and switching to spherical coordinates, we have

\begin{equation}\label{eqn:desaiCh3:1421}
\begin{aligned}
\spacegrad^2
&= \inv{r} \partial_r + \inv{r} \partial_r r \partial_r \\
&= \inv{r} \partial_r
+ \inv{r} \partial_r
+ \inv{r} r \partial_{rr} \\
&= \frac{2}{r} \partial_r
+ \partial_{rr} \\
\end{aligned}
\end{equation}

We can now write the Hamiltonian for the zero angular momentum case
\begin{equation}\label{eqn:desaiCh3:703}
H
=
-\frac{\Hbar^2}{2m} \left( \frac{2}{r} \partial_r + \partial_{rr} \right) - \frac{\lambda}{r}
\end{equation}

With application of this Hamiltonian to the eigenfunction we have

\begin{equation}\label{eqn:desaiCh3:1441}
\begin{aligned}
E u_0 e^{-\beta r} &=
\left( -\frac{\Hbar^2}{2m} \left( \frac{2}{r} \partial_r + \partial_{rr} \right) - \frac{\lambda}{r} \right) u_0 e^{-\beta r}  \\
&=
\left( -\frac{\Hbar^2}{2m} \left( \frac{2}{r} (-\beta) + \beta^2 \right) - \frac{\lambda}{r} \right) u_0 e^{-\beta r} .
\end{aligned}
\end{equation}

In particular for \(r = \infty\) we have
\begin{equation}\label{eqn:desaiCh3:704}
-\frac{\Hbar^2 \beta^2 }{2m} = E
\end{equation}

\begin{equation}\label{eqn:desaiCh3:1461}
\begin{aligned}
-\frac{\Hbar^2 \beta^2 }{2m} &=
\left( -\frac{\Hbar^2}{2m} \left( \frac{2}{r} (-\beta) + \beta^2 \right) - \frac{\lambda}{r} \right)  \\
\implies \\
\frac{\Hbar^2}{2m} \frac{2}{r} \beta &= \frac{\lambda}{r}
\end{aligned}
\end{equation}

Collecting all the results we have

\begin{equation}\label{eqn:desaiCh3:705}
\begin{aligned}
\beta &= \frac{\lambda m}{\Hbar^2} \\
E &= -\frac{\lambda^2 m}{2 \Hbar^2} \\
u_0 &= \frac{\sqrt{2 \lambda m}}{\Hbar}
\end{aligned}
\end{equation}

} % answer

\makeoproblem{}{problem:desaiCh3:7}{\citep{desai2009quantum} pr 3.7}{
\index{expectation!acceleration}
A particle in a uniform field \(\BE_0\).  Show that the expectation value of the position operator \(\expectation{\Br}\) satisfies

\begin{equation}\label{eqn:desaiCh3:600}
m \frac{d^2 \expectation{\Br} }{dt^2} = e \BE_0.
\end{equation}

} % problem

\makeanswer{problem:desaiCh3:7}{

This follows from \textAndIndex{Ehrenfest's theorem} once we formulate the force \(e \BE_0 = -\spacegrad \phi\), in terms of a potential \(\phi\).  That potential is

\begin{equation}\label{eqn:desaiCh3:601}
\phi = - e \BE_0 \cdot (x,y,z)
\end{equation}

The Hamiltonian is therefore

\begin{equation}\label{eqn:desaiCh3:602}
H = \frac{\Bp^2}{2m} - e \BE_0 \cdot (x,y,z).
\end{equation}

Ehrenfest's theorem gives us
\begin{equation}\label{eqn:desaiCh3:1481}
\begin{aligned}
\frac{d}{dt} \expectation{x_k} &= \inv{m} \expectation{p_k} \\
\frac{d}{dt} \expectation{p_k} &= -\expectation{ \PD{x_k}{V} },
\end{aligned}
\end{equation}

or
\begin{equation}\label{eqn:desaiCh3:603}
\frac{d^2}{dt^2} \expectation{x_k} = -\inv{m} \expectation{ \PD{x_k}{V} }.
\end{equation}

\begin{equation}\label{eqn:desaiCh3:1501}
\PD{x_k}{V} = - e (\BE_0)_k
\end{equation}

Putting all the last bits together, and summing over the directions \(\Be_k\) we have
\begin{equation}\label{eqn:desaiCh3:1521}
m \frac{d^2}{dt^2} \Be_k \expectation{x_k}
= \Be_k \expectation{ e (\BE_0)_k }
= e \BE_0
\qedmarker
\end{equation}

} % answer

\makeoproblem{}{problem:desaiCh3:8}{\citep{desai2009quantum} pr 3.8}{

For Hamiltonian eigenstates \(\ket{E_n}\), \(C = AB\), \(A = \antisymmetric{B}{H}\), obtain the matrix element \(\bra{E_m} C \ket{E_n}\) in terms of the matrix element of \(A\).

} % problem

\makeanswer{problem:desaiCh3:8}{

I was able to get most of what was asked for here, with a small exception.  I started with the matrix element for \(A\), which is

\begin{equation}\label{eqn:desaiCh3:800}
\bra{E_m} A \ket{E_n}
=
\bra{E_m} BH - HB \ket{E_n}
=
(E_n - E_m)\bra{E_m} B \ket{E_n}
\end{equation}

Next, computing the matrix element for \(C\) we have

\begin{equation}\label{eqn:desaiCh3:1541}
\begin{aligned}
\bra{E_m} C \ket{E_n}
&=
\bra{E_m} BHB - HB^2 \ket{E_n} \\
&=
\sum_a \bra{E_m} BH \ket{E_a}\bra{E_a} B \ket{E_n} - E_m \bra{E_m} B \ket{E_a} \bra{E_a} B \ket{E_n} \\
&=
\sum_a E_a \bra{E_m} B \ket{E_a}\bra{E_a} B \ket{E_n} -E_m \bra{E_m} B \ket{E_a} \bra{E_a} B \ket{E_n} \\
&=
\sum_a (E_a - E_m)\bra{E_m} B \ket{E_a}\bra{E_a} B \ket{E_n} \\
&=
\sum_a \bra{E_m} A \ket{E_a} \bra{E_a} B \ket{E_n} \\
&=
\bra{E_m} A \ket{E_n} \bra{E_n} B \ket{E_n}
+\sum_{a \ne n} \bra{E_m} A \ket{E_a} \bra{E_a} B \ket{E_n} \\
&=
\bra{E_m} A \ket{E_n}
\bra{E_n} B \ket{E_n}
+\sum_{a \ne n} \bra{E_m} A \ket{E_a}
\frac{\bra{E_a} A \ket{E_n}}{E_n - E_a}
\end{aligned}
\end{equation}

Except for the \(\bra{E_n} B \ket{E_n}\) part of this expression, the problem as stated is complete.  The relationship \eqnref{eqn:desaiCh3:800} is no help for with \(n = m\), so I see no choice but to leave that small part of the expansion in terms of \(B\).

} % answer

\makeoproblem{}{problem:desaiCh3:9}{\citep{desai2009quantum} pr 3.9}{

Operator \(A\) has eigenstates \(\ket{a_i}\), with a unitary change of basis operation \(U \ket{a_i} = \ket{b_i}\).  Determine in terms of \(U\), and \(A\) the operator \(B\) and its eigenvalues for which \(\ket{b_i}\) are eigenstates.

} % problem

\makeanswer{problem:desaiCh3:9}{

Consider for motivation the matrix element of \(A\) in terms of \(\ket{b_i}\).  We will also let \(A \ket{a_i} = \alpha_i \ket{a_i}\).  We then have

\begin{equation}\label{eqn:desaiCh3:1561}
\bra{a_i} A \ket{a_j}
=
\bra{b_i} U A U^\dagger \ket{b_j}
\end{equation}

We also have
\begin{equation}\label{eqn:desaiCh3:1581}
\begin{aligned}
\bra{a_i} A \ket{a_j}
&=
a_j \bra{a_i} \ket{a_j} \\
&=
a_j \delta_{ij}
\end{aligned}
\end{equation}

So it appears that the operator \(U A U^\dagger\) has the orthonormality relation required.  In terms of action on the basis \(\{\ket{b_i}\}\), let us see how it behaves.  We have

\begin{equation}\label{eqn:desaiCh3:1601}
\begin{aligned}
U A U^\dagger \ket{b_i}
&= U A \ket{a_i} \\
&= U \alpha_i \ket{a_i} \\
&= \alpha_i \ket{b_i} \\
\end{aligned}
\end{equation}

So we see that the operators \(A\) and \(B = U A U^\dagger\) have common eigenvalues.

} % answer

\makeoproblem{}{problem:desaiCh3:10}{\citep{desai2009quantum} pr 3.10}{


With \(H \ket{n} = E_n \ket{n}\), \(A = \antisymmetric{H}{F}\) and \(\bra{0} F \ket{0} = 0\), show that

\begin{equation}\label{eqn:desaiCh3:1000}
\sum_{n\ne 0} \frac{
\bra{0} A \ket{n} \bra{n} A \ket{0} }{E_n - E_0} =
\bra{0} AF \ket{0}
\end{equation}

} % problem

\makeanswer{problem:desaiCh3:10}{

\begin{equation}\label{eqn:desaiCh3:1621}
\begin{aligned}
\bra{0} AF \ket{0}
&=
\bra{0} HF F - FH F\ket{0} \\
&=
\sum_n E_0 \bra{0} F \ket{n}\bra{n} F \ket{0} - E_n \bra{0} F \ket{n} \bra{n} F\ket{0} \\
&=
\sum_n (E_0 -E_n) \bra{0} F \ket{n}\bra{n} F \ket{0}  \\
&=
\sum_{n\ne0} (E_0 -E_n) \bra{0} F \ket{n}\bra{n} F \ket{0}  \\
\end{aligned}
\end{equation}

We also have

\begin{equation}\label{eqn:desaiCh3:1641}
\begin{aligned}
\bra{0} A \ket{n} \bra{n} A \ket{0}
&=
\bra{0} HF -F H \ket{n} \bra{n} A \ket{0} \\
&=
(E_0 - E_n) \bra{0} F \ket{n} \bra{n} HF - FH \ket{0} \\
&=
-(E_0 - E_n)^2 \bra{0} F \ket{n} \bra{n} F \ket{0} \\
\end{aligned}
\end{equation}

Or, for \(n \ne 0\),
\begin{equation}\label{eqn:desaiCh3:1661}
\bra{0} F \ket{n} \bra{n} F \ket{0} =
-\frac{\bra{0} A \ket{n} \bra{n} A \ket{0}}{(E_0 - E_n)^2 }.
\end{equation}

This gives

\begin{equation}\label{eqn:desaiCh3:1681}
\begin{aligned}
\bra{0} AF \ket{0}
&=
-\sum_{n\ne0} (E_0 -E_n) \frac{\bra{0} A \ket{n} \bra{n} A \ket{0}}{(E_0 - E_n)^2 } \\
&=
\sum_{n\ne0} \frac{\bra{0} A \ket{n} \bra{n} A \ket{0}}{E_n - E_0 }
\qedmarker
\end{aligned}
\end{equation}

} % answer

\makeoproblem{Commutator of angular momentum with Hamiltonian}{problem:desaiCh3:11}{\citep{desai2009quantum} pr 3.11}{
\index{angular momentum}
Show that \(\antisymmetric{\BL}{H} = 0\), where \(H = \Bp^2/2m + V(r)\).

} % problem

\makeanswer{problem:desaiCh3:11}{
This follows by considering \(\antisymmetric{\BL}{\Bp^2}\), and \(\antisymmetric{\BL}{V(r)}\).  Let

\begin{equation}\label{eqn:desaiCh3:1100}
L_{jk} = x_j p_k - x_k p_j,
\end{equation}

so that

\begin{equation}\label{eqn:desaiCh3:1101}
\BL = \Be_i \epsilon_{ijk} L_{jk}.
\end{equation}

We now need to consider the commutators of the operators \(L_{jk}\) with \(\Bp^2\) and \(V(r)\).

Let us start with \(p^2\).  In particular

\begin{equation}\label{eqn:desaiCh3:1701}
\begin{aligned}
\Bp^2 x_m p_n
&=
p_k p_k x_m p_n \\
&=
p_k (p_k x_m) p_n \\
&=
p_k (-i\Hbar \delta_{km} + x_m p_k) p_n \\
&=
-i\Hbar p_m p_n + (p_k x_m) p_k p_n \\
&=
-i\Hbar p_m p_n + (-i \Hbar \delta_{km} + x_m p_k ) p_k p_n \\
&=
-2 i\Hbar p_m p_n + x_m p_n \Bp^2.
\end{aligned}
\end{equation}

So our commutator with \(\Bp^2\) is

\begin{equation}\label{eqn:desaiCh3:1721}
\antisymmetric{L_{jk}}{\Bp^2}
=
(x_j p_k - x_j p_k) \Bp^2
-( -2 i\Hbar p_j p_k + x_j p_k \Bp^2 +2 i\Hbar p_k p_j - x_k p_j \Bp^2 ).
\end{equation}

Since \(p_j p_k = p_k p_j\), all terms cancel out, and the problem is reduced to showing that

\begin{equation}\label{eqn:desaiCh3:1741}
\antisymmetric{\BL}{H} = \antisymmetric{\BL}{V(r)} = 0.
\end{equation}

Now assume that \(V(r)\) has a series representation

\begin{equation}\label{eqn:desaiCh3:1761}
V(r) = \sum_j a_j r^j = \sum_j a_j (x_k x_k)^{j/2}
\end{equation}

We would like to consider the action of \(x_m p_n\) on this function

\begin{equation}\label{eqn:desaiCh3:1781}
\begin{aligned}
x_m p_n V(r) \Psi
&= -i \Hbar x_m \sum_j a_j \partial_n (x_k x_k)^{j/2} \Psi \\
&= -i \Hbar x_m \sum_j a_j (j x_n (x_k x_k)^{j/2-1} + r^j \partial_n \Psi) \\
&= -\frac{i \Hbar x_m x_n}{r^2} \sum_j a_j j r^j +
x_m V(r) p_n \Psi
\end{aligned}
\end{equation}

\begin{equation}\label{eqn:desaiCh3:1801}
\begin{aligned}
L_{mn} V(r)
&=
(x_m p_n - x_n p_m) V(r) \\
&=
-\frac{i \Hbar x_m x_n}{r^2} \sum_j a_j j r^j
+\frac{i \Hbar x_n x_m}{r^2} \sum_j a_j j r^j
+
V(r) (x_m p_n - x_n p_m )
\\
&=
V(r) L_{mn}
\end{aligned}
\end{equation}

Thus \(\antisymmetric{L_{mn}}{V(r)} = 0\) as expected, implying \(\antisymmetric{\BL}{H} = 0\).
} % answer
