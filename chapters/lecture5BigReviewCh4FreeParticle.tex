%
% Copyright © 2012 Peeter Joot.  All Rights Reserved.
% Licenced as described in the file LICENSE under the root directory of this GIT repository.
%
For three dimensions with \(V(x,y,z) = 0\)
%
\begin{equation}\label{eqn:lecture5BigReviewCh4FreeParticle:20}
\begin{aligned}
H &= \inv{2m} \Bp^2 \\
\Bp &= \sum_i p_i \Be_i \\
\end{aligned}
\end{equation}
%
In the position representation, where
%
\begin{equation}\label{eqn:lecture5BigReviewCh4FreeParticle:40}
\begin{aligned}
p_i &= -i \Hbar \frac{d}{dx_i}
\end{aligned}
\end{equation}
%
the Schr\"{o}dinger equation is
\begin{equation}\label{eqn:lecture5BigReviewCh4FreeParticle:60}
\begin{aligned}
H u(x,y,z) &= E u(x,y,z) \\
H &= -\frac{\Hbar^2}{2m} \spacegrad^2 \\
&= -\frac{\Hbar^2}{2m} \left(
\frac{\partial^2}{\partial {x}^2}
+\frac{\partial^2}{\partial {y}^2}
+\frac{\partial^2}{\partial {z}^2}
\right)
\end{aligned}
\end{equation}
%
Separation of variables assumes it is possible to let
%
\begin{equation}\label{eqn:lecture5BigReviewCh4FreeParticle:80}
\begin{aligned}
u(x,y,z) = X(x) Y(y) Z(z)
\end{aligned}
\end{equation}
%
(these capital letters are functions, not operators).
%
\begin{equation}\label{eqn:lecture5BigReviewCh4FreeParticle:100}
\begin{aligned}
-\frac{\Hbar^2}{2m} \left(
YZ \frac{\partial^2 X}{\partial {x}^2}
+ XZ \frac{\partial^2 Y}{\partial {y}^2}
+ YZ \frac{\partial^2 Z}{\partial {z}^2}\right)
&= E X Y Z
\end{aligned}
\end{equation}
%
Dividing as usual by \(XYZ\) we have
%
\begin{equation}\label{eqn:lecture5BigReviewCh4FreeParticle:120}
\begin{aligned}
-\frac{\Hbar^2}{2m} \left(
\inv{X} \frac{\partial^2 X}{\partial {x}^2}
+ \inv{Y} \frac{\partial^2 Y}{\partial {y}^2}
+ \inv{Z} \frac{\partial^2 Z}{\partial {z}^2} \right)
&= E
\end{aligned}
\end{equation}
%
The curious thing is that we have these three derivatives, which is supposed to be related to an Energy, which is independent of any \(x,y,z\), so it must be that each of these is separately constant.  We can separate these into three individual equations
%
\begin{equation}\label{eqn:lecture5BigReviewCh4FreeParticle:140}
\begin{aligned}
-\frac{\Hbar^2}{2m} \inv{X} \frac{\partial^2 X}{\partial {x}^2} &= E_1 \\
-\frac{\Hbar^2}{2m} \inv{Y} \frac{\partial^2 Y}{\partial {x}^2} &= E_2 \\
-\frac{\Hbar^2}{2m} \inv{Z} \frac{\partial^2 Z}{\partial {x}^2} &= E_3
\end{aligned}
\end{equation}
%
or
\begin{equation}\label{eqn:lecture5BigReviewCh4FreeParticle:160}
\begin{aligned}
\frac{\partial^2 X}{\partial {x}^2} &= \left( - \frac{2m E_1}{\Hbar^2} \right) X  \\
\frac{\partial^2 Y}{\partial {x}^2} &= \left( - \frac{2m E_2}{\Hbar^2} \right) Y  \\
\frac{\partial^2 Z}{\partial {x}^2} &= \left( - \frac{2m E_3}{\Hbar^2} \right) Z
\end{aligned}
\end{equation}
%
We have then
%
\begin{equation}\label{eqn:lecture5BigReviewCh4FreeParticle:180}
\begin{aligned}
X(x) = C_1 e^{i k x}
\end{aligned}
\end{equation}
%
with
\begin{equation}\label{eqn:lecture5BigReviewCh4FreeParticle:200}
\begin{aligned}
E_1 &= \frac{\Hbar^2 k_1^2 }{2m} = \frac{p_1^2}{2m} \\
E_2 &= \frac{\Hbar^2 k_2^2 }{2m} = \frac{p_2^2}{2m} \\
E_3 &= \frac{\Hbar^2 k_3^2 }{2m} = \frac{p_3^2}{2m}
\end{aligned}
\end{equation}
%
We are free to use any sort of normalization procedure we wish (periodic boundary conditions, infinite Dirac, ...)

