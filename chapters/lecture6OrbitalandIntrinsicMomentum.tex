%
% Copyright © 2012 Peeter Joot.  All Rights Reserved.
% Licenced as described in the file LICENSE under the root directory of this GIT repository.
%
%\QMlecture{6 --- Orbital and Intrinsic Momentum --- October 19, 2010}

Last time, we started thinking about angular momentum.  This time, we will examine orbital and intrinsic angular momentum.

Orbital angular momentum in classical physics and quantum physics is expressed as

\begin{equation}\label{eqn:PHY356Foct19:1000}
\BL = \Br \cross \Bp,
\end{equation}

and
\begin{equation}\label{eqn:PHY356Foct19:1001}
\BL = \BR \cross \BP,
\end{equation}

where \(\BR\) and \(\BP\) are quantum mechanical operators corresponding to position and momentum

\begin{equation}\label{eqn:PHY356Foct19:1002}
\begin{aligned}
\BR &= X \xcap + Y \ycap + Z \zcap \\
\BP &= P_x \xcap + P_y \ycap + P_z \zcap \\
\BL &= L_x \xcap + L_y \ycap + L_z \zcap
\end{aligned}
\end{equation}


\makeexample{Angular momentum commutators}{example:lecture6OrbitalandIntrinsicMomentum:1}{

Determine the commutators \(\antisymmetric{L_x}{L_y}, \antisymmetric{L_y}{L_z}, \antisymmetric{L_z}{L_x}\) and

\begin{equation}\label{eqn:lecture6OrbitalandIntrinsicMomentum:2040}
\begin{aligned}
\antisymmetric{L_x}{L_y}
&=
(r_y p_z -r_z p_y)
(r_z p_x -r_x p_z)
-
(r_z p_x -r_x p_z)
(r_y p_z -r_z p_y) \\
&=
r_y p_z (r_z p_x -r_x p_z)
-r_z p_y (r_z p_x -r_x p_z)
- r_z p_x (r_y p_z -r_z p_y)
+ r_x p_z (r_y p_z -r_z p_y) \\
&=
r_y p_z r_z p_x
-r_y p_z r_x p_z
-r_z p_y r_z p_x
+r_z p_y r_x p_z \\
&- r_z p_x r_y p_z
+ r_z p_x r_z p_y
+ r_x p_z r_y p_z
- r_x p_z r_z p_y \\
\end{aligned}
\end{equation}

With \(p_i r_j = r_j p_i - i \Hbar \delta_{ij}\), we have

\begin{equation*}
\antisymmetric{L_x}{L_y}
=
r_y r_z p_z p_x
-r_y r_z p_x p_z
-r_z r_y p_z p_x
+r_z r_y p_x p_z
- r_z r_x p_y p_z
+ r_z r_x p_z p_y
+ r_x r_z p_y p_z
- r_x r_z p_z p_y
+
-i \Hbar
\left(
%r_y p_z
%+r_z p_x
%-r_z p_z
%+ r_z p_z
%- r_z p_y
%- r_x p_z
r_y p_x
- r_x p_y
\right)
\end{equation*}

Since the \(p_i p_j\) operators commute, all the first terms cancel, leaving just
\begin{equation}\label{eqn:lecture6OrbitalandIntrinsicMomentum:2060}
\antisymmetric{L_x}{L_y}
=i \Hbar L_z
\end{equation}

%Oops.  Mistake above to fix.  Much easier to do this problem in abstract index notation.

} % example

\makeexample{\(L_z\) in spherical coordinates}{example:lecture6OrbitalandIntrinsicMomentum:2}{

The answer is
\begin{equation}\label{eqn:PHY356Foct19:1003}
L_z \leftrightarrow -i \Hbar \PD{\phi}{}
\end{equation}

FIXME: Work through this.
} % example

Part of the task in this intro QM treatment is to carefully determine the eigenfunctions for these operators.

The spherical harmonics are given by \(Y_{lm}(\theta, \phi)\) such that

\begin{equation}\label{eqn:PHY356Foct19:1004}
Y_{lm}(\theta, \phi) \propto e^{i m \phi}
\end{equation}

\begin{equation}\label{eqn:lecture6OrbitalandIntrinsicMomentum:2080}
\begin{aligned}
L_z Y_{lm}(\theta, \phi)
&= -i \Hbar \PD{\phi}{} Y_{lm}(\theta, \phi) \\
&= -i \Hbar \PD{\phi}{} \text{constants} (e^{im \phi}) \\
&= \Hbar m \text{constants} e^{i m \phi} \\
&= \Hbar m Y_{lm}(\theta, \phi)
\end{aligned}
\end{equation}

The z-component is quantized since, \(m\) is an integer \(m = 0, \pm 1, \pm 2, ...\).  The total angular momentum

\begin{equation}\label{eqn:PHY356Foct19:1005}
\BL^2 = \BL \cdot \BL = L_x^2 + L_y^2 + L_z^2
\end{equation}

is also quantized (details in the book).

The eigenvalue properties here represent very important physical features.  This is also important in the hydrogen atom problem.  In the hydrogen atom problem, the particle is effectively free in the angular components, having only \(r\) dependence.  This allows us to apply the work for the free particle to our subsequent potential bounded solution.

Note that for the above, we also have the alternate, abstract ket notation, method of writing the eigenvalue behavior.
\begin{equation}\label{eqn:PHY356Foct19:1006}
L_z \ket{lm} = \Hbar m \ket{lm}
\end{equation}

Just like
\begin{equation}\label{eqn:PHY356Foct19:1007}
\begin{aligned}
X \ket{x} &= x \ket{x} \\
P \ket{p} &= p \ket{p}
\end{aligned}
\end{equation}

For the total angular momentum our spherical harmonic eigenfunctions have the property

\begin{equation}\label{eqn:PHY356Foct19:1008}
\BL^2 \ket{lm} = \Hbar^2 l (l + 1)\ket{l m}
\end{equation}

with \(l = 0, 1, 2, \cdots\).

Alternately in plain old non-abstract notation we can write this as
\begin{equation}\label{eqn:PHY356Foct19:1009}
\BL^2 Y_{lm}(\theta, \phi) = \Hbar^2 l (l + 1) Y_{lm}(\theta, \phi)
\end{equation}

Now we can introduce the Raising and Lowering Operators, which are

\begin{equation}\label{eqn:PHY356Foct19:1010}
\begin{aligned}
L_{+} &= L_x + i L_y \\
L_{-} &= L_x - i L_y,
\end{aligned}
\end{equation}

respectively.  These are abstract quantities, but also physically important since they relate quantum levels of the angular momentum.  How do we show this?

Last time, we saw that
\begin{equation}\label{eqn:PHY356Foct19:1011}
\begin{aligned}
\antisymmetric{L_z}{L_{+}} &= +\Hbar L_{+} \\
\antisymmetric{L_z}{L_{-}} &= -\Hbar L_{-}
\end{aligned}
\end{equation}

Note that it is implied that we are operating on ket vectors

\begin{equation}\label{eqn:lecture6OrbitalandIntrinsicMomentum:2100}
L_z (L_{-} \ket{lm} )
\end{equation}

with
\begin{equation}\label{eqn:PHY356Foct19:1012}
\ket{lm} \leftrightarrow Y_{lm}(\theta, \phi)
\end{equation}

Question: What is \(L_{-} \ket{lm}\)?

Substitute
\begin{equation}\label{eqn:lecture6OrbitalandIntrinsicMomentum:2120}
\begin{aligned}
L_z L_{-} - L_{-} L_z &= - \Hbar L_{-} \\
\implies \\
L_z L_{-} &= L_{-} L_z - \Hbar L_{-}
\end{aligned}
\end{equation}

\begin{equation}\label{eqn:lecture6OrbitalandIntrinsicMomentum:2140}
\begin{aligned}
L_z \left( L_{-} \ket{lm} \right)
&=
L_{-} L_z \ket{lm} - \Hbar L_{-} \ket{lm} \\
&=
L_{-} m \Hbar \ket{lm} - L_{-} \ket{lm} \\
&=
\Hbar \left( m L_{-} \ket{lm} - L_{-} \ket{lm} \right) \\
&=
\Hbar (m-1) \left( L_{-} \ket{lm} \right)
\end{aligned}
\end{equation}

So \(L_{-} \ket{lm} = \ket{\psi}\) is another spherical harmonic, and we have

\begin{equation}\label{eqn:PHY356Foct19:1013}
L_z \ket{\psi} = \Hbar (m-1) \ket{\psi}
\end{equation}

This lowering operator quantity causes a physical change in the state of the system, lowering the observable state (ie: the eigenvalue) by \(\Hbar\).

Now we want to normalize \(\ket{\psi} = L_{-} \ket{lm}\), via \(\braket{\psi}{\psi} = 1\).

\begin{equation}\label{eqn:lecture6OrbitalandIntrinsicMomentum:2160}
\begin{aligned}
1
&= \braket{\psi}{\psi} \\
&= \bra{lm} L_{-}^\dagger L_{-} \ket{\psi} \\
&= \bra{lm} L_{+} L_{-} \ket{\psi}
\end{aligned}
\end{equation}

We can use
\begin{equation}\label{eqn:PHY356Foct19:1014}
\begin{aligned}
L_{+} L_{-} = \BL^2 - L_z^2 + \Hbar L_z,
\end{aligned}
\end{equation}

So, knowing (how exactly?) that

\begin{equation}\label{eqn:PHY356Foct19:1015}
\begin{aligned}
L_{-} \ket{lm} = C \ket{l,m-1}
\end{aligned}
\end{equation}

we have from \eqnref{eqn:PHY356Foct19:1014}

\begin{equation}\label{eqn:lecture6OrbitalandIntrinsicMomentum:2180}
\begin{aligned}
\Abs{C}^2
&= \bra{lm} (\BL^2 - L_z^2 + \Hbar L_z ) \ket{\psi}  \\
&=
\mathLabelBox
[
   labelstyle={xshift=6cm},
   labelstyle={yshift=1cm},
   linestyle={out=270,in=90, latex-}
]
{\braket{lm}{lm}}{\(=1\)}
\left(\Hbar^2 l(l+1) - (\Hbar m)^2 + \Hbar^2 m \right)  \\
&= \Hbar^2 \left(l(l+1) - m^2 + m \right).
\end{aligned}
\end{equation}

we have
\begin{equation}\label{eqn:PHY356Foct19:1016}
\Abs{C}^2
\mathLabelBox
[
   labelstyle={xshift=1cm},
   linestyle={out=270,in=90, latex-}
]
{\braket{l,m-1}{l,m-1}}{\(1\)}
= \Hbar^2 \left(l(l+1) - m^2 + m \right).
\end{equation}

and can normalize the functions \(\ket{\psi}\) as

\begin{equation}\label{eqn:PHY356Foct19:1017}
L_{-} \ket{lm} = \Hbar \left(l(l+1) - m^2 + m \right)^{1/2} \ket{l, m-1}
\end{equation}

Abstract notation side note:

\begin{equation}\label{eqn:PHY356Foct19:1018}
\braket{\theta,\phi}{lm} = Y_{lm}(\theta, \phi)
\end{equation}

\paragraph{Generalizing orbital angular momentum}
\index{orbital angular momentum}

To explain the results of the \textAndIndex{Stern-Gerlach} experiment, assume that there is an intrinsic angular momentum \(\BS\) that has most of the same properties as \(\BL\).  But \(\BS\) has no classical counterpart such as \(\Br \cross \Bp\).

This experiment is the classic QM experiment because the silver atoms not only have the orbital angular momentum, but also have an additional observed intrinsic spin in the outermost electron.  In turns out that if you combine relativity and QM, you can get out something that looks like the \(\BS\) operator.  That marriage produces the Dirac electron theory.

We assume the commutation relations

\begin{equation}\label{eqn:PHY356Foct19:2000}
\begin{aligned}
\antisymmetric{S_x}{S_y} &= i \Hbar S_z \\
\antisymmetric{S_y}{S_z} &= i \Hbar S_x \\
\antisymmetric{S_z}{S_x} &= i \Hbar S_y
\end{aligned}
\end{equation}

Where we have the analogous eigenproperties

\begin{equation}\label{eqn:PHY356Foct19:2001}
\begin{aligned}
\BS^2 \ket{sm} &= \Hbar^2 s(s+1) \ket{sm} \\
S_z \ket{sm} &= \Hbar m \ket{sm}
\end{aligned}
\end{equation}

with \(s = 0, 1/2, 1, 3/2, ...\)

Electrons and protons are examples of particles that have spin one half.

Note that there is no position representation of \(\ket{sm}\).  We cannot project these states.

This basic quantum mechanics end up applying to quantum computing and cryptography as well, when we apply the mathematics we are learning here to explain the Stern-Gerlach experiment to photon spin states.

(DRAWS Stern-Gerlach picture with spin up and down labeled \(\ket{z+}\), and \(\ket{z-}\) with the magnetic field oriented in along the \(z\) axis.)

Silver atoms have \(s = 1/2\) and \(m= \pm 1/2\), where \(m\) is the quantum number associated with the z-direction intrinsic angular momentum.  The angular momentum that is being acted on in the Stern-Gerlach experiment is primarily due to the outermost electron.

\begin{equation}\label{eqn:PHY356Foct19:2005}
\begin{aligned}
S_z \ket{z+} &= \frac{\Hbar}{2} \ket{z+} \\
S_z \ket{z-} &= -\frac{\Hbar}{2} \ket{z-} \\
\BS^2 \ket{z\pm} &= \inv{2} \left( \inv{2} + 1 \right) \Hbar^2 \ket{z\pm}
\end{aligned}
\end{equation}

where
\begin{equation}\label{eqn:PHY356Foct19:2006}
\begin{aligned}
\ket{z+} &= \ket{ \inv{2} \inv{2} } \\
\ket{z-} &= \ket{ \inv{2} -\inv{2} }
\end{aligned}
\end{equation}

%You can imagine p

What about \(S_x\)?  We can leave the detector in the \(x,z\) plane, but rotate the magnet so that it lies in the \(x\) direction.

We have the correspondence

\begin{equation}\label{eqn:PHY356Foct19:2007}
S_z \leftrightarrow \frac{\Hbar}{2} \PauliX,
\end{equation}

but this is perhaps more properly viewed as the matrix representation of the less specific form

\begin{equation}\label{eqn:PHY356Foct19:2008}
S_z = \frac{\Hbar}{2} \left(
\ket{z+} \bra{z+}
-\ket{z-} \bra{z-}
\right).
\end{equation}

Where the translation to the form of \eqnref{eqn:PHY356Foct19:2007} is via the matrix elements

\begin{equation}\label{eqn:PHY356Foct19:2020}
\begin{aligned}
&\bra{z+} S_z \ket{z+} \\
&\bra{z+} S_z \ket{z-} \\
&\bra{z-} S_z \ket{z+} \\
&\bra{z-} S_z \ket{z-}.
\end{aligned}
\end{equation}

We can work out the same for \(S_x\) using \(S_{+}\) and \(S_{-}\), or equivalently for \(\sigma_x\) using \(\sigma_{+}\) and \(\sigma_{-}\), where

\begin{equation}\label{eqn:PHY356Foct19:2009}
\begin{aligned}
S_x &= \frac{\Hbar}{2} \sigma_x \\
S_y &= \frac{\Hbar}{2} \sigma_y \\
S_z &= \frac{\Hbar}{2} \sigma_z
\end{aligned}
\end{equation}

The operators \(\sigma_x, \sigma_y, \sigma_z\) are the \textAndIndex{Pauli operators}, and avoid the pesky \(\Hbar/2\) factors.

We find

\begin{equation}\label{eqn:PHY356Foct19:2010}
\begin{aligned}
\sigma_x &= \PauliX \\
\sigma_y &= \PauliY \\
\sigma_z &= \PauliZ
\end{aligned}
\end{equation}

And from \(\Abs{\sigma_x - \lambda I} = (-\lambda)^2 -1\), we have eigenvalues \(\lambda = \pm 1\) for the \(\sigma_x\) operator.

The corresponding eigenkets in column matrix notation are found

\begin{equation}\label{eqn:lecture6OrbitalandIntrinsicMomentum:2200}
\begin{aligned}
\begin{bmatrix}
\mp 1 & 1 \\
1 & \mp 1
\end{bmatrix}
\begin{bmatrix}
a_1 \\
a_2
\end{bmatrix}
&= 0 \\
\implies
\mp a_1 + a_2 &= 0 \\
\implies
a_2 &= \pm a_1
\end{aligned}
\end{equation}

Or
\begin{equation}\label{eqn:lecture6OrbitalandIntrinsicMomentum:2220}
\ket{x\pm} \propto
\begin{bmatrix}
a_1 \\
a_2
\end{bmatrix}
=
a_1
\begin{bmatrix}
1 \\
\pm 1
\end{bmatrix}
\end{equation}

which can be normalized as
\begin{equation}\label{eqn:PHY356Foct19:2011}
\ket{x\pm} =
\inv{\sqrt{2}}
\begin{bmatrix}
1 \\
\pm 1
\end{bmatrix}
\end{equation}

We see that this is different from

\begin{equation}\label{eqn:PHY356Foct19:2012}
\ket{z+} =
\begin{bmatrix}
1 \\
0
\end{bmatrix}
\end{equation}

We will still end up with two spots, but there has been a projection of spin in a different fashion?  Does this mean the measurement will be different.  There is still a lot more to learn before understanding exactly how to relate the spin operators to a real physical system.

